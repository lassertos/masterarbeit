% === Dokumentklasse ==========================================================
\documentclass[%
    a4paper,							% A4
    12pt,								% Schriftgröße 12
    twoside,							% zweiseitig
    fleqn								% Formeln linksbündig ausrichten
]{book}


% === Grundlegende Pakete =====================================================
\usepackage{etoolbox}					% portiert viele nützliche Sachen aus e-TeX (z.B. booleans, ifs)
\usepackage[							% erweiterte Angabe von Farben
    pdftex,								% Farbtreiber auswählen
    dvipsnames							% vordefinierte Farben laden
]{xcolor}								% erweiterte Angabe von Farben
\usepackage{xparse}						% high-level Interface für Dokumentbefehle wie \NewDocument(Command,Environment)

\usepackage[utf8]{inputenc}				% Kodierung der *.tex Dateien ist UTF-8 nicht ISO-8859-1
\usepackage{cmap}						% character map tables -> pdf Inhalt wird besser durchsuch- und kopierbar
\usepackage[T1]{fontenc}				% OT1 encoding für deutsche Sonderzeichen -> pdf Inhalt kann Sonderzeichen durchsucht werden
\usepackage{lmodern}					% Latin-Modern-Schriftart (Computer Modern in Verbindung mit OT1 führt zu Bitmap-Fonts auf Windows)
\usepackage[nopatch=eqnum]{microtype}					% mikrotypografische Einstellungen, die den gesetzten Text nochmals wesentlich verbessern (weniger badboxes)


% === Benutzerdefinierte Einstellungen ========================================
% === Art der Arbeit ==========================================================
% von den nachfolgenden Blöcken bitte den richtigen auswählen und die anderen auskommentieren/löschen

% --- Diplom ----------
%\newcommand{\settingsDegree}{{Diplom}}
%\newcommand{\settingsDegreeName}{{Diplominformatiker}}
%\newcommand{\settingsDegreeName}{{Diplomingenieur}}

% --- Bachelor ----------
% \newcommand{\settingsDegree}{{Bachelor}}
% \newcommand{\settingsDegreeName}{{Bachelor of Science}}

% --- Master ----------
\newcommand{\settingsDegree}{{Master}}
\newcommand{\settingsDegreeName}{{Master of Science}}

% === Name, Abgabedatum und Sprache der Arbeit ================================
\newcommand{\settingsName}{{Pierre Helbing}}
\newcommand{\settingsFinishDate}{{DD.MM.YYYY}}
\newcommand{\settingsLanguage}{german}   		% german / american

% === Weitere Einstellungen ===================================================
% --- Suchpfad (Unterverzeichnis) für eingebundene Grafiken ----------
\newcommand{\settingsGraphicsPath}{image/}

% --- Hinweiskapitel ----------
\newbool{settingsWithHints}
\setbool{settingsWithHints}{false}				% true / false

% --- Zeilennummern ----------
\newbool{settingsWithLineNumbers}
\setbool{settingsWithLineNumbers}{false}			% true / false

% --- Todos ----------
\newbool{settingsWithTodos}
\setbool{settingsWithTodos}{true}				% true / false

% --- Anzahl an Nummerierungsebenen im Text und Inhaltsverzeichnis ----------
% 1: \section
% 2: \section + \subsection
% Achtung: 3 oder 4 nur nach Absprache mit Betreuer !
% 3: \section + \subsection + \subsubsection
% 4: \section + \subsection + \subsubsection + \paragraph
\setcounter{secnumdepth}{2}
\setcounter{tocdepth}{2}

% --- Anforderungen ----------

\usepackage{tabularx}
\usepackage{multicol}
\newenvironment{myreq}[1]{%
    \setlist[description]{font=\normalfont\color{darkgray}}%
    \begin{tcolorbox}[colframe=black,colback=white, sharp corners, boxrule=1pt]%
        \bfseries\color{blue}%
        \begin{description}#1}%
            {\end{description}\end{tcolorbox}}

\newcommand{\threeinline}[3]{\begin{multicols}{3}#1 #2 #3\end{multicols}}
\newcommand{\twoinline}[2]{\begin{multicols}{2}#1 #2\end{multicols}}

\newcommand{\reqno}{\item[Requirement \#:]}
\newcommand{\reqtype}{\item[Requirement Type:]}
\newcommand{\reqevent}{\item[Event/BUC/PUC \#:]}
\newcommand{\reqdesc}{\item[Description:]}
\newcommand{\reqrat}{\item[Rationale:]}
\newcommand{\reqorig}{\item[Originator:]}
\newcommand{\reqfit}{\item[Fit Criterion:]}
\newcommand{\reqsatis}{\item[Customer Satisfaction:]}
\newcommand{\reqdissat}{\item[Customer Dissatisfaction:]}
\newcommand{\reqdep}{\item[Dependencies:]}
\newcommand{\reqconf}{\item[Conflicts:]}
\newcommand{\reqmater}{\item[Materials:]}
\newcommand{\reqhist}{\item[History:]}
\usepackage{multicol}
\usepackage{multirow}
\usepackage[dvipsnames]{xcolor}
\usepackage{enumitem}
\usepackage{tcolorbox}

% === Übersetzungen ===========================================================
% Definitionen je nach \mylanguage (siehe NIKR_settings.tex)
\ifdefstring{\settingsLanguage}{german}{%
    \newcommand{\acroname}{Abkürzungsverzeichnis}	% Name für Abkürzungsverzeichnis
    \newcommand{\todoname}{Todo Liste}	% Name für Todo-Liste
    \newcommand{\pagename}{Seite}		% Name für Seite
}%
{%
    \newcommand{\acroname}{Acronyms}	% Name für Abkürzungsverzeichnis
    \newcommand{\todoname}{Todo list}	% Name für Todo-Liste
    \newcommand{\pagename}{page}		% Name für Seite
}%


% === Wichtige Pakete und Einstellungen =======================================
\usepackage{calc}						% ermöglicht Arithmetik in den Argumenten von Befehlen

% Einstellungen je nach \mylanguage (siehe NIKR_settings.tex)
\ifdefstring{\settingsLanguage}{german}{%
    \usepackage[ngerman]{babel}			% Spracheinstellungen (für deutsch z.B. Contents -> Inhaltsverzeichnis, etc.)
    \usepackage{bibgerm}            	% Stylefile für deutsche Literaturstellenangabe
    \bibliographystyle{gerapali}		% Stil für Literaturangaben festlegen
}%
{%
    \usepackage[american]{babel}		% Spracheinstellungen (für deutsch z.B. Contents -> Inhaltsverzeichnis, etc.)
    \bibliographystyle{apalike}			% Stil für Literaturangaben festlegen
    \frenchspacing						% einfaches Leerzeiches nach Satzende (für deutsch bereits Standard)
}%
\usepackage[%
    noadjust							% noadjust verhindert automatische Leerzeichen um die Referenz, was am Zeilenanfang zu Problemen führt
]{cite}     							% erlaubt Zeilenumbruch innerhalb von Zitierungen

\usepackage{graphicx}					% erweiterte Argumente in \in­clude­graph­ics
\graphicspath{{\settingsGraphicsPath}}	% Standard-Pfad für Bilder siehe NIKR_settings.tex

\usepackage[bf]{caption}				% erlaubt erweiterte Formatierungen in \caption (siehe unten)
\usepackage{subcaption}					% mehrere Abbildungen nebeneinander

\usepackage{amsmath}					% ermöglicht \DeclareMathOperator
\usepackage{amssymb}					% mathmatische Symbole und Sonderzeichen
\usepackage{nicefrac}					% für \nicefrac
\usepackage{nccmath}					% für \mfrac

\usepackage{icomma}						% intelligentes Komma (macht Verwendung von {,} überflüssig)
\usepackage{siunitx}					% für einheitliche Angabe von Einheiten

\usepackage{fancyhdr}					% Kopf- und Fußzeilen (siehe unten)

\usepackage{hhline}						% erweitere Rahmengestaltung in Tabellen

\usepackage[%							% Einbettung von Links im Dokument und erlaubt die Nutzugn von \url
    hyperfootnotes=false,				% keine Fußnoten als Link im Dokument (geht nicht mit footmisc)
    pagebackref=true					% Backrefs im Literaturverzeichnis
]{hyperref}           					% Einbettung von Links im Dokument und erlaubt die Nutzugn von \url
\renewcommand*{\backref}[1]{\textit{(\pagename:~#1)}}   % Format für backrefs

\usepackage[%							% erlaubt erweiterte Formatierungen von Fußnoten (siehe unten)
    multiple,							% mehrere mit Komma abtrennen
    hang								% linksbündig, \footnotemargin entscheidet über Einrückung
]{footmisc}								% erlaubt erweiterte Formatierungen von Fußnoten (siehe unten)
\patchcmd{\footref}{\ref}{\ref*}{}{}	% Hyperlink in \footref entfernen

\usepackage[nohyperlinks]{acronym}		% Abkürzungsverzeichnis

\usepackage{setspace}					% für \setstretch (ändern Zeilenabstand, aber nicht floating Umgebungen)

\usepackage[%							% Todos
    colorinlistoftodos,					% farbige Markierungen in Todo-Liste
    prependcaption,						% caption=val
    textsize=tiny,						% Schriftgröße
    linecolor=red,						% Standard-Linienfarbe für \todo
    backgroundcolor=red!25,				% Standard-Hintergrundfarbe für \todo
    bordercolor=red,					% Standard-Rahmenfarbe für \todo
    textwidth=2cm,						% Standard-Breite für \todo
]{todonotes}							% Todos
\ifbool{settingsWithTodos}{%
    \setlength{\marginparwidth}{2cm}	% sonst werden Notes am Rand nicht richtig angezeigt
    \NewDocumentCommand{\todoaddref}{O{} m}{%
        \todo[linecolor=blue,backgroundcolor=blue!25,bordercolor=blue,#1]{#2}%
    }
    \NewDocumentCommand{\todouncertain}{O{} m}{%
        \todo[linecolor=green,backgroundcolor=green!25,bordercolor=green,#1]{#2}%
    }
    \NewDocumentCommand{\todooptional}{O{} m}{%
        \todo[linecolor=cyan,backgroundcolor=cyan!25,bordercolor=cyan,#1]{#2}%
    }
    \pretocmd{\mainmatter}{\listoftodos[\todoname]{\markboth{\MakeUppercase{\todoname}}{\MakeUppercase{\todoname}}}}{}{}
}{
    \presetkeys{todonotes}{disable}{}	% disable \todo
    \NewDocumentCommand{\todoaddref}{O{} m}{}
    \NewDocumentCommand{\todouncertain}{O{} m}{}
    \NewDocumentCommand{\todooptional}{O{} m}{}
}

\usepackage[switch*,pagewise]{lineno}	% Zeilennummern
\ifbool{settingsWithLineNumbers}{%
    \renewcommand\linenumberfont{\textbf\sffamily\color{black!50}\footnotesize}
    \apptocmd{\mainmatter}{\linenumbers}{}{}
    \pretocmd{\backmatter}{\nolinenumbers}{}{}
}{}

\usepackage{placeins}					% FloatBarriers

\usepackage{enumitem}					% erweiterte Formatierung von \enumerate, \itemize und \description

\usepackage[linewidth=0.5pt]{mdframed}	% für Boxen in Hinweisen

\usepackage[ddmmyyyy]{datetime}			% Datumsangabe
\renewcommand{\dateseparator}{.}		% Punkt als Trennzeichen in Datumsangabe


% === Längen und Abstände =====================================================
% horizontales Layout
\setlength{\oddsidemargin}{0.2in}
\setlength{\evensidemargin}{0.0in}
\setlength{\textwidth}{\paperwidth - 2.2in}

% vertikales Layout
%\setlength{\topskip}{0.0cm}
\setlength{\headheight}{15.1pt}
%\setlength{\headsep}{0.0cm}
\setlength{\topmargin}{0.0cm}
\setlength{\footskip}{0.6in}
\setlength{\textheight}{\paperheight - 2.0in}
\addtolength{\textheight}{-1.0\headheight}
\addtolength{\textheight}{-1.0\headsep}
\addtolength{\textheight}{-1.0\footskip}

% Zeilenabstand
\setstretch{1.3}
\AtBeginEnvironment{tabular}{\setstretch{1.3}}

% Einrückung von Formeln
\setlength{\mathindent}{1.0cm}

% Absätze
\setlength{\parindent}{0.0cm}

% Fußnoten
\renewcommand{\footnotelayout}{\setstretch{1.2}}
\setlength\footnotemargin{10pt}

% Listen (noitemsep, nosep, ...)
\setlist{noitemsep}

\setlength{\topsep}{0.3cm}


% === Bild- und Tabellenunterschrift ==========================================
\renewcommand{\captionfont}{\small \setstretch{1.3}}
\newcommand{\NIcaption}[2]{\caption[#1]{#1\protect\\ \emph{#2}}}
\setcaptionmargin{0.75cm}


% === Abkürzungsverzeichnis ===================================================
% Verwendung vor jedem Kapitel zurücksetzen
\pretocmd{\chapter}{\acresetall}{}{}


% === Seitenstil ==============================================================
% Pagestyle plain überschreiben
\pagestyle{fancy}
% Kapitel- und Abschnittangaben ohne Punkt
\renewcommand{\sectionmark}[1]{\markright{\uppercase{\thesection~~#1}}}
\renewcommand{\chaptermark}[1]{\markboth{\uppercase{\chaptername\ \thechapter~~#1}}{}}
\fancypagestyle{plain}{%
    \fancyhead[ER]{\itshape\leftmark}%
    \fancyhead[OL]{\itshape\rightmark}%
    \fancyhead[EL,OR]{\thepage}%
    \fancyfoot[EL,OL]{}%
    \fancyfoot[EC,OC]{}%
    \renewcommand{\headrulewidth}{0.4pt}%
    \renewcommand{\footrulewidth}{0.4pt}%
}


% === Verweise ================================================================
% Klammern in Formel-Referenzen entfernen
\makeatletter
\renewcommand\tagform@[1]{\maketag@@@{\ignorespaces#1\unskip\@@italiccorr}}
\makeatother


% === Angabe von Einheiten ====================================================
\ifdefstring{\settingsLanguage}{german}{%
    \sisetup{locale=DE}		% deutsche Lokalisierung (konvertiert 1.00 automatisch zu 1,00)
}%
{%
    \sisetup{locale=US}		% englische Lokalisierung (konvertiert 1,00 automatisch zu 1.00)
}%


% === Mathematische Definitionen ==============================================
% Darstellung von Vektoren und Matrizen
\renewcommand{\vec}[1]{\underline{\mathbf{\MakeLowercase{#1}}}}		% Vektoren
\newcommand{\veci}[1]{\underline{\MakeLowercase{#1}}}				% Vektoren als Indizes
\newcommand{\mat}[1]{\underline{\mathbf{\MakeUppercase{#1}}}}		% Matrizen
\newcommand{\mati}[1]{\underline{\MakeUppercase{#1}}}				% Matrizen als Indizes
% zusätzliche mathematische Operatoren (damit sie nicht als Formelzeichen geschrieben werden)
\DeclareMathOperator{\step}{step}									% Stufenfunktion
\DeclareMathOperator{\sign}{sign}									% Vorzeichen


% === Manuelle Definition von Worttrennungen ==================================
\hyphenation{
    Convolutional
    Neural
    Net-work
    Net-works
    Ko-ef-fi-zi-ent
    Ko-ef-fi-zi-ent-en
    Drop-out
    pixel-genaue
    Patch
    Patch-größen
}


% === Pseudocode-Darstellung ==================================================
% Import nicht oben, weil \parindent zuvor gesetzt werden muss!
% siehe: https://ctan.org/pkg/algorithm2e?lang=de
\ifdefstring{\settingsLanguage}{german}{%
    \usepackage[%						% Pseudocode
        linesnumbered,					% mit Zeilennummern
        noend,							% Ende von Befehlen, wie etwa While, unterdrücken
        ruled,							% Layout
        german,							% deutsche Bezeichnung und deutsches Verzeichnis
        %onelanguage,					% Keywords übersetzen
        algochapter						% Nummerierung analog zu Abbildungen
    ]{algorithm2e}						% Pseudocode
}{%
    \usepackage[%						% Pseudocode
        linesnumbered,					% mit Zeilennummern
        noend,							% Ende von Befehlen, wie etwa While, unterdrücken
        ruled,							% Layout
        algochapter						% Nummerierung analog zu Abbildungen
    ]{algorithm2e}						% Pseudocode
}%

\newenvironment{NIalgorithm}{%
    % Algorithmus um 1.5em einrücken, damit Zeilennummern nicht im Rand sind
    \setlength{\algomargin}{1.5em}%
    % Padding oben und unten für Caption
    \setlength{\interspacetitleruled}{\smallskipamount}%
    % Padding oben und unten für Algorithmus
    \SetAlgoInsideSkip{smallskip}%
    % Kommentarstyle ändern
    \newcommand\NIcommentstyle[1]{\ttfamily\textcolor{black!60}{##1}}
    \SetCommentSty{NIcommentstyle}%
    % zweiten Teil in NIcaption unterdrücken (falls NIcaption genutzt wird)
    \renewcommand{\NIcaption}[2]{\caption[##1]{##1}}%
    \begin{algorithm}%
        % Zeilenabstand minimal vergrößern
        \setstretch{1.1}%
        % kleine Schrift
        \small%
        % Semikolon unterdrücken
        \DontPrintSemicolon%
        % korrekt ausgerichtete mehrzeilige Input- bzw. Outputdefinitionen mittels \Input und \Output
        \SetKwInOut{Input}{Input}%
        \SetKwInOut{Output}{Output}%
        % Abschnitt für In- und Outputs zurückrücken
        \pretocmd{\Input}{\Indentp{-1.5em}}{}{}%
        \apptocmd{\Output}{\Indentp{1.5em}}{}{}%
        }{%
    \end{algorithm}%
}

\begin{document}
% --- Titelseite, Danksagung, Einverständniserklärung --------------------------
\pagestyle{empty}
% Buchstaben für Seitennummerierung verwenden 
% (verhindert "destination with the same identifier (name{page.X})" Warnung)
\pagenumbering{alph}

% === Titelblatt ==============================================================
\begin{titlepage}
	\hspace{0.2cm}
	\begin{minipage}{3.5cm}
		\includegraphics[width=0.8\textwidth]{images/logo}
	\end{minipage}
	\hspace{0.2cm}
	\begin{minipage}{11cm}
		\vspace{0.7cm}
		\large
		{\bf Technische Universität Ilmenau}\newline
		Fakultät für Informatik und Automatisierung\newline
		Fachgebiet Neuroinformatik und Kognitive Robotik
	\end{minipage}
	\begin{center}
		\vspace{0.8cm}
		{\Large\bfseries Entwicklung einer CrossLab-kompatiblen integrierten Entwicklungsumgebung für das GOLDi-Remotelab\\}
		\vspace{0.8cm}
		\settingsDegree arbeit zur Erlangung des akademischen Grades \settingsDegreeName\\[0.5cm]
		{\Large \bfseries \settingsName\\[1.0cm]}
		\begin{table}[ht]
			\centering
			\begin{tabular}{ll}
				Betreuer: & Dr. Detlef Streitferdt                    \\[2mm]
				\multicolumn{2}{l}{Verantwortlicher Hochschullehrer:} \\
				          & Prof. Dr.-Ing. habil. Daniel Ziener       \\[2cm]
				\multicolumn{2}{p{13cm}}{Die \settingsDegree arbeit wurde am \settingsFinishDate \ bei der Fakultät für Informatik und Automatisierung der Technischen Universität Ilmenau eingereicht.}
			\end{tabular}
		\end{table}
		% Hinweis für Entwurfsversion ausgeben (Variablen aus NIKR_settings.tex prüfen)
		\ifboolexpr{bool{settingsWithTodos} or bool{settingsWithLineNumbers} or bool{settingsWithHints}}{%
		{\color{red} Entwurf: \today \\[2mm]
		\textbf{Dies ist nicht die finale Druckvorlage.}\\[2mm]
		{\small%
		\ifbool{settingsWithHints}{%
			Hinweiskapitel aktiviert (siehe \texttt{settingsWithHints} in \texttt{NIKR\_settings.tex})\\%
		}{}%					
		\ifbool{settingsWithLineNumbers}{%
			Zeilennummern aktiviert (siehe \texttt{settingsWithLineNumbers} in \texttt{NIKR\_settings.tex})\\%
		}{}%
		\ifbool{settingsWithTodos}{%
			Todo-Markierungen aktiviert (siehe \texttt{settingsWithTodos} in \texttt{NIKR\_settings.tex})\\%
		}{}%
		}%
		}%
		}{}
	\end{center}
\end{titlepage}

\cleardoublepage

% === Danksagung ==============================================================
% Sollten Sie diesen Abschnitt nicht nutzen wollen, kommentieren Sie alle Zeilen
% bis zur Einverstaendniserklaerung (inkl. \cleardoublepage) aus
\vspace*{5cm}

Danksagung

Dieser Abschnitt {\bf kann} genutzt werden, um denjenigen Personen Dank auszusprechen, die Sie bei der Erstellung der Arbeit unterstützt haben.

\cleardoublepage

% === Einverstaendniserklaerung ================================================
\vspace*{16cm}

\begin{tabular}{lp{12.5cm}}
	{Erklärung:} & {"`Hiermit versichere ich, dass ich diese wissenschaftliche Arbeit selbständig verfasst und nur die angegebenen

			
			
			
			
			
			
			
			
			
			
			Quellen und Hilfsmittel verwendet habe. Alle von mir aus anderen
			Veröffentlichungen übernommenen Passagen sind als solche gekennzeichnet."'}
\end{tabular}
\vspace*{1.5cm}

\begin{tabular}{l}
	Ilmenau, \settingsFinishDate \\
	\\
\end{tabular}
\hfill
\begin{tabular}{c}
	{\makebox[6.0cm]{\dotfill}} \\
	\settingsName               \\
\end{tabular}

\cleardoublepage


% --- Inhalts- und Abkürzungsverzeichnis ---------------------------------------
\frontmatter
\pagestyle{plain}
% Inhaltsverzeichnis
\tableofcontents
% Abkürzungsverzeichnis
\chapter*{\acroname}
\markboth{\MakeUppercase{\acroname}}{\MakeUppercase{\acroname}}
\acrodefplural{LMS}[LMS]{Lernmanagementsysteme}

\begin{acronym}[XXXXXX] % durch XXXXXX kann der Einzug bestimmt wird
	\setlength{\itemsep}{-\parsep}
	\acro{GOLDi}{Grid of Online Laboratory Devices Ilmenau}
	\acro{BEAST}{Block Diagram Editing and Simulating Tool}
	\acro{WIDE}{Web Integrated Development Environment}
	\acro{FPGA}{Field Programmable Gate Array}
	\acro{IDE}{Integrated Development Environment}
	\acro{LSP}{Language Server Protocol}
	\acro{DAP}{Debug Adapter Protocol}
	\acro{VSCode}{Visual Studio Code}
	\acro{OT}{Operational Transformation}
	\acro{CRDT}{Conflict-free Replicated Data Type}
	\acro{LMS}{Lernmanagementsystem}
	\acro{LTI}{Learning Tools Interoperability}
\end{acronym}



% --- Inhalt -------------------------------------------------------------------
\mainmatter

% Hinweiskapitel
\ifbool{settingsWithHints}{\input{content/hinweise}}{}

% Kapitel
\chapter{Einleitung}\label{section:einleitung}

% \begin{note}
%     \textbf{Notizen:}
%     \begin{itemize}
%         \item Remote Labore (z.B. GOLDi)
%         \item IDEs in Remote Laboren (z.B. GOLDi und WIDE)
%         \item Änderungen und Möglichkeiten durch CrossLab
%         \item Weiterer Verlauf der Arbeit
%     \end{itemize}
% \end{note}

Praktische Versuche sind in der Lehre verschiedenster Fachrichtungen, wie z.B. in der Informatik, Elektrotechnik und Chemie essentiell. Dabei können Studierende die aus der Vorlesung bekannten theoretischen Grundlagen in der Praxis anwenden und somit ein tieferes Verständnis für diese erlangen. Ein Beispiel für einen praktischen Versuch ist die Steuerung eines elektromechanischen Hardwaremodells über die Programmierung eines Microcontroller oder das Erstellen einer entsprechenden Schaltung. Diese praktischen Versuche werden meistens in Laboren durchgeführt, welche die benötigte Hardware bereitstellen. Dabei gibt es auch sogenannte \textit{online Labore}. Diese erlauben die Durchführung der Versuche über eine entsprechende Webanwendung. Dadurch kann es Nutzern ermöglicht werden die Versuche unabhängig von den Zugangszeiten eines normalen Labors durchzuführen.

Ein Beispiel für ein derartiges online Labor ist das an der Technischen Universität Ilmenau \cite{noauthor_tu-ilmenau_2025} entwickelte \ac{GOLDi} \cite{sitepoint_goldi_nodate}, welches im Folgenden als GOLDi-Remotelab bezeichnet wird. Dieses erlaubt es Nutzern verschiedene Versuche bestehend aus einem elektromechanischen Hardwaremodell, z.B. ein Modell eines 3-Achsen-Portalkrans oder eines Aufzugs, sowie einer Steuereinheit, z.B. ein Microcontroller oder ein \ac{FPGA}, zusammenzustellen. Hierbei besitzt das GOLDi-Remotelab die Besonderheit, dass für jedes reale elektromechanischen Hardwaremodelle auch eine Simulation dessen bereitgestellt wird, die statt dem realen Modell innerhalb eines Versuchs verwendet werden kann. Somit können Nutzer auch Versuche für ein entsprechendes Modell durchführen, selbst wenn das reale Modell nicht verfügbar ist. Daher wird das GOLDi-Remotelab auch als ein \textit{hybrides online Labor} bezeichnet.

Eine weitere zentrale Komponente des GOLDi-Remotelab ist die integrierte Entwicklungsumgebung, im Englischen \ac{IDE}, \ac{WIDE} \cite{henke_hidden_2021}. Dabei handelt es sich um eine sogenannte \textit{online IDE}, da WIDE im Browser des Nutzers ausgeführt wird. WIDE ermöglicht es Nutzern ihre Programme für die in einem Versuch verwendete Steuereinheit direkt in dem bereitgestellten Web-Interface des Versuchs zu erstellen. Dabei können Nutzer zudem ihre verschiedenen Programme verwalten, diese kompilieren und das Kompilat für die Programmierung der verwendeten Steuereinheit nutzen.

Ein aktuelles Projekt welches sich mit online Laboren befasst ist das Verbundprojekt \textit{CrossLab} \cite{aubel_adaptable_2022} der Technischen Universität Bergakademie Freiberg \cite{noauthor_tu-freiberg_nodate}, der Technischen Universität Dortmund \cite{dortmund_tu-dortmund_nodate}, der NORDAKADEMIE \cite{noauthor_nordakademie_nodate} und der Technischen Universität Ilmenau, mit der folgenden Zielsetzung:

\begin{quote}
    ,,CrossLab zielt auf die Etablierung eines hochschulübergreifenden, interdisziplinären Netzwerkes von digitalisierten Labormodulen, die vergleichbar mit den Konzepten der Industrie 4.0, bedarfsbezogen in einer Lernumgebung für studierenden-zentrierte Lehre kombiniert werden können. Dafür werden durch die Partner TU Bergakademie Freiberg, TU Ilmenau, TU Dortmund und der NORDAKADEMIE sowohl auf didaktischer, technischer und organisatorischer Ebene Lösungen entwickelt und evaluiert.`` \cite{noauthor_crosslab_nodate}
\end{quote}

Im Rahmen dieses Verbundprojekts wurde eine neue Architektur für online Labore entwickelt \cite{nau_new_2022}. Diese basiert auf dem Konzept sogenannter \textit{Laborgeräte}. Diese können verschiedene \textit{Services} anbieten und durch die Verbindung dieser zu einem \textit{Experiment} zusammengestellt werden. Dadurch wird die Wiederverwendbarkeit und Austauschbarkeit von Laborgeräten in unterschiedlichen Experimenten ermöglicht. Weiterhin erlaubt diese Architektur die gemeinsame Nutzung von Laborgeräten über Institutionsgrenzen hinweg. Somit können z.B. Studierende der NORDAKADEMIE auf die in Ilmenau vorhandenen elektromechanischen Hardwaremodelle zugreifen.

Um diese neuen Möglichkeiten nutzen zu können wurde eine Umstellung des GOLDi-Remotelab auf diese neue Architektur vorgenommen. Um diese Umstellung fertigzustellen muss u.a. eine neue CrossLab-kompatible online IDE entwickelt werden. Dafür soll ein entsprechendes Laborgerät samt Services entwickelt werden. Dabei soll auch die, durch die neue Architektur unterstützte, Möglichkeit betrachtet werden einzelne Funktionen mithilfe weiterer Laborgeräte zu implementieren. Ein Beispiel Hierfür wäre die Auslagerung eines Compilers auf ein entsprechendes Laborgerät, dass einen Service für die Nutzung dessen anbietet. Dieser Service könnte dann wiederum von dem Laborgerät der IDE verwendet werden um die Kompilierung für Nutzer bereitzustellen. Die entwickelten Laborgeräte und Services sollen in allen online Laboren nutzbar sein, welche die CrossLab-Architektur verwenden.

Zunächst werden in \autoref{section:grundlagen} grundlegende Begriffe für diese Arbeit vorgestellt, darunter integrierte Entwicklungsumgebungen und die CrossLab-Architektur. Danach wird in \autoref{section:stand-der-technik} der aktuelle Stand der Technik im Bezug auf online IDEs erarbeitet und vorgestellt. Anschließend wird in \autoref{section:anforderungsanalyse} die Anforderungsanalyse für die zu entwickelnde online IDE dargelegt. Daraufhin werden in \autoref{section:konzeption} Konzepte für die Bereitstellung der verschiedenen Funktionen der zu entwickelnden IDE vorgestellt. Darauf aufbauend wird in \autoref{section:prototypische-implementierung} die prototypische Implementierung der IDE beschrieben. In \autoref{section:diskussion} werden die Erfüllung der gestellten Anforderungen sowie offene Probleme betrachtet. Abschließend wird in \autoref{section:zusammenfassung-und-ausblick} ein Fazit gegeben.

\chapter{Grundlagen}\label{section:grundlagen}

In diesem Kapitel werden zunächst integrierte Entwicklungsumgebungen in \autoref{section:grundlagen:integrierte-entwicklungsumgebung} erläutert. Daraufhin werden das Verbundprojekt CrossLab und die dazugehörige Architektur für online Labore in \autoref{section:grundlagen:crosslab} vorgestellt.

\section{Integrierte Entwicklungsumgebung}\label{section:grundlagen:integrierte-entwicklungsumgebung}
Eine integrierte Entwicklungsumgebung bzw. \ac{IDE} ist meistens auf einen speziellen Anwendungsfall ausgelegt und besteht aus einem Code Editor sowie weiteren benötigten Softwarewerkzeugen, wie z.B. Compiler, Debugger und Language Server \cite{noauthor_language-server-protocol_nodate}. Oftmals werden alle diese Komponenten direkt mit der \ac{IDE} ausgeliefert, wodurch der Nutzer direkt mit der Programmierung beginnen kann. Somit besitzen \acp{IDE} mehr Features als Code Editoren, da diese nur die Bearbeitung von Code erlauben, während \acp{IDE} u.a. auch die Kompilierung und das Debuggen von Programmen ermöglichen. Language Server erweitern die Funktionen eines Code Editors, indem sie u.a. Code-Vervollständigung, Code-Navigation und Refactoring ermöglichen.

\section{CrossLab}\label{section:grundlagen:crosslab}
CrossLab \cite{aubel_adaptable_2022} ist ein Verbundprojekt der Technischen Universität Bergakademie Freiberg, der Technischen Universität Dortmund, der NORDAKADEMIE und der Technischen Universität Ilmenau. Im Rahmen des Projekts wurde eine neue Architektur für online Labore erarbeitet \cite{nau_new_2022}.

Die CrossLab-Architektur basiert auf dem Konzept von sogenannten \textit{Laborgeräten}. Diese können verschiedene \textit{Services} anbieten bzw. konsumieren. Beispiele für Services sind der \textit{Electrical Connection Service}, welcher den Austausch von Sensor- und Aktorwerten ermöglicht, und der \textit{Webcam Service}, welcher bspw. die Übertragung der Webcamaufnahmen von einem elektromechanischen Hardwaremodell ermöglicht. Services besitzen immer einen \textit{Producer}, der die Funktionen des Services bereitstellt, sowie einen \textit{Consumer}, der diese nutzen kann. Dabei können auch sogenannte \textit{Prosumer} entwickelt werden, die beide Rollen erfüllen können. Durch die Verbindung der Services von verschiedenen Laborgeräten kann ein \textit{Experiment} erstellt werden. Ein Vorteil dieser Architektur ist die einfache Wiederverwendbarkeit und Austauschbarkeit von einzelnen Laborgeräten in Experimenten. So können z.B. Laborgeräte mit gleichen Services in einer entsprechenden \textit{Laborgerätegruppe} hinterlegt werden, welche dann statt eines konkreten Laborgeräts zur Erstellung eines Experiments genutzt werden kann. Beim Start des Experiments wird dann ein verfügbares Laborgerät aus der Laborgerätegruppe ausgewählt. Weiterhin gibt es noch \textit{cloud-instanziierbare} und \textit{edge-instanziierbare} Laborgeräte. Beim Start eines Experiments mit cloud- oder edge-instanziierbaren Laborgeräten wird eine entsprechende Instanz des Laborgeräts erstellt. Dabei wird für cloud-instanziierbare Laborgeräte eine Nachricht an die hinterlegte Instanziierungs-URL geschickt, wodurch die Instanz erstellt wird. Für edge-instanziierbare Laborgeräte wird mithilfe der hinterlegten Code-URL eine URL erstellt, die vom Nutzer aufgerufen werden muss, um die Instanz zu erstellen.

Das Backend von CrossLab besteht aus mehreren verschiedenen Diensten, welche zusammen eine \textit{CrossLab-Instanz} bilden. Diese Dienste sind im Folgenden aufgelistet:
\begin{itemize}
    \item \textbf{Authentication Service} \\ Dieser Dienst ist für die Authentifizierung der Nutzer verantwortlich.
    \item \textbf{Authorization Service} \\ Dieser Dienst ist für die Autorisierung der Nutzer verantwortlich.
    \item \textbf{Device Service} \\ Dieser Dienst verwaltet die Laborgeräte der CrossLab-Instanz.
    \item \textbf{Experiment Service} \\ Dieser Dienst ist für die Erstellung und Verwaltung der Experimente der CrossLab-Instanz verantwortlich.
    \item \textbf{Federation Service} \\ Dieser Dienst ist für das Teilen von Laborgeräten und Experimenten mit anderen CrossLab-Instanzen verantwortlich.
\end{itemize}
Um ein Experiment in einer CrossLab-Instanz starten zu können benötigen Nutzer ein entsprechendes Nutzerkonto für diese CrossLab-Instanz.
\chapter{Anforderungsanalyse} \label{anforderungsanalyse}

In diesem Kapitel sollen die Anforderung an die zu entwickelnde integrierte Entwicklungsumgebung dargelegt werden. Diese werden in Gesprächen mit Experten des GOLDi-Remotelab sowie Experten des CrossLab Projekts erhoben. Dazu wird in Abschnitt \ref{beispielszenario} zunächst ein Beispielszenario für die Nutzung des GOLDi Remotelab beschrieben, anhand dessen die in Abschnitt \ref{anforderungen} aufgeführten Anforderungen ermittelt werden.

\section{Beispielszenario: Praktikumsversuch} \label{beispielszenario}

\begin{itemize}
    \item Designer
    \item Entwickler
    \item Vorbereiter
    \item Betreuer
    \item Lernende
    \item Lehrende
    \item Administrator
\end{itemize}

\subsection{Designer}

\subsection{Entwickler}

\subsection{Vorbereiter}

\subsection{Betreuer}

\subsection{Lernende}

\subsection{Lehrende}

\subsection{Administrator}
\section{Anforderungen} \label{anforderungen}

\paragraph{Komplett im Browser nutzbar} \mbox{} \\
Die IDE soll komplett im Browser nutzbar sein. Dies vereinfacht die Einbindung der IDE in das GOLDi Remotelab. Zudem wird der Aufwand für die Studierenden verringert, weil diese keine zusätzliche Software herunterladen und installieren müssen.

\paragraph{Erweiterbarkeit} \mbox{} \\
Die IDE soll erweiterbar sein. Das bedeutet, dass z.B. die Unterstützung weiterer Programmiersprachen oder das Hinzufügen neuer CrossLab Services ermöglicht werden soll.

\paragraph{Kollaboratives Editieren von Dateien} \mbox{} \\
In der modernen Arbeitswelt ist Teamwork eine wichtige Fähigkeit und daher auch ein bedeutendes Lernziel. Daher soll es den Studierenden ermöglicht werden kollaborativ an Projekten zu arbeiten, indem sie gleichzeitig Änderungen an Dateien vornehmen können. Diese Änderungen sollen dann in Echtzeit zwischen den am Experiment teilnehmenden Studierenden synchronisiert werden.

\paragraph{Anbindbarkeit von Compilern} \mbox{} \\
Die IDE soll die Anbindung von Compilern unterstützen. Weiterhin sollen die Studierenden über ein entsprechendes UI-Element die Kompilierung starten bzw. abbrechen können.

\paragraph{Anbindbarkeit von Debuggern} \mbox{} \\
Die IDE soll die Anbindung von Debuggern unterstützen. Weiterhin sollen die Studierenden über ein entsprechendes UI-Element eine Debugging-Sitzung starten können. 

\paragraph{Hochladen von kompilierten Programmen auf Microcontroller} \mbox{} \\
Die IDE soll das Hochladen von kompilierten Programmen auf Microcontroller ermöglichen. Das bedeutet, dass das hochgeladene Programm auf das entsprechende Zielsystem aufgespielt werden soll.

\paragraph{Debuggen von Programmen auf Microcontrollern} \mbox{} \\
Die IDE soll das Debuggen von Programmen auf Microcontrollern ermöglichen. Dabei ist die im GOLDi Remotelab vorhandene Hardware zu beachten. Den Studierenden soll auch ein entsprechendes Nutzer Interface zur Interaktion mit der Debug Sitzung bereitgestellt werden. Weiterhin soll eine aktive Debug-Sitzung allen Teilnehmern eines Experiments angezeigt werden.

\paragraph{CrossLab-Kompatibilität} \mbox{} \\
Die IDE soll als Laborgerät in CrossLab einbindbar sein. Dazu müssen die benötigten Services definiert und implementiert werden.

\paragraph{Anbindbarkeit von Language Servern} \mbox{} \\
Die IDE soll die Anbindung von Language Servern unterstützen.

\paragraph{Kostenlos nutzbar} \mbox{} \\
Das GOLDi Remotelab soll für Nutzer kostenlos nutzbar sein. Daher gilt dasselbe auch für die zu entwickelnde IDE.

\paragraph{Nur ein Nutzerkonto} \mbox{} \\
Nutzer des GOLDi Remotelab sollten nur ein Nutzerkonto zur Ausführung von Experimenten benötigen.

\paragraph{Standalone nutzbar} \mbox{} \\
Die IDE soll auch Standalone nutzbar sein. Dies bedeutet, dass die IDE auch ohne weitere Laborgeräte verwendbar sein soll. Dies ermöglicht es den Studierenden die IDE nutzen zu können ohne dafür ggf. auf die Verfügbarkeit anderer Geräte zu warten bzw. diese zu blockieren.

\paragraph{Einbindbarkeit in Lernplattformen (z.B. über LTI)} \mbox{} \\
Die IDE bzw. Experimente, welche diese enthalten, sollen in Lernplattformen einbindbar sein. Dies vereinfacht den Zugriff für Studierende und erlaubt auch eine einfachere Bewertung der Aufgaben, da Ergebnisse direkt an die Lernplattform gesendet werden können.

\paragraph{Anzeigen der Seriellen Ausgaben} \mbox{} \\
Die IDE sollte in der Lage sein, die Ausgaben der seriellen Schnittstellen eines Microcontrollers darstellen zu können.

\paragraph{Hinterlegen von Testfällen} \mbox{} \\
Es soll ermöglicht werden experimentspezifische Testfälle in der IDE zu hinterlegen. Dies ermöglicht es Studierenden ihre erstellten Lösungen selbst testen zu können. Weiterhin soll die Weitermeldung erfolgreicher Testdurchläufe an die Lernplattform erfolgen können.

\paragraph{Hinterlegen von Beispielcode} \mbox{} \\
Es soll ermöglicht werden experimentspezifischen Beispielcode in der IDE zu hinterlegen. Dies ermöglicht die einfache Bereitstellung leichterer Programmieraufgaben bzw. die Vorgabe einer gewissen Programmstruktur.

\paragraph{Herunterladen von Code} \mbox{} \\
Der Nutzer sollte in der Lage sein seinen erstellten Code herunterladen zu können. 

\paragraph{Hochladen von Code} \mbox{} \\
Der Nutzer sollte in der Lage sein seinen lokal erstellten Code in die IDE hochladen zu können.

\chapter{Systematische Literaturrecherche} \label{systematische_literaturrecherche}

Um einen Überblick über den aktuellen Stand der Forschung zu bekommen wird zunächst eine Literaturrecherche vorgenommen. Dabei sollen vorhandene online IDEs gefunden sowie die folgenden Fragen beantwortet werden:

\begin{itemize}
    \item Was sind die Anforderungen an die online IDEs?
    \item Welche Vor- und Nachteile haben die online IDEs?
    \item Wie sind die online IDEs aufgebaut?
\end{itemize}

\section{Vorgehensbeschreibung}

Die folgenden Datenbanken wurden für die Literaturrecherche ausgewählt:

\begin{itemize}
    \item ACM Digital Library
    \item IEEE Xplore
    \item Scopus
    \item Web of Science
\end{itemize}

Zunächst wurde eine allgemeine Suche nach online IDEs in den genannten Datenbanken vorgenommen. Dazu werden zunächst die in Tabelle \ref{table:search-terms} genannten Stichwörter jeweils mit ihren Synonymen mit einer OR-Operation verknüpft. Danach werden die daraus resultierenden Terme mit einer AND-Operation verbunden. Die so entstehende Suchanfrage werden dann für die Suche in den Datenbanken verwendet. Dabei werden die Titel, Abstracts und Keywords der Publikationen durchsucht.

In Tabelle \ref{table:amount-search-results} ist die Anzahl der Hits für den einzelnen Datenbanken dargelegt. Um die Anzahl der zu betrachtenden Publikationen zu verringern wird eine weitere Filterung der Ergebnisse vorgenommen. Dafür werden nur Publikationen betrachtet, die IDE oder ein entsprechendes Synonym in ihrem Titel oder ihren Keywords enthalten. Dadurch sinkt die Anzahl der Hits auf insgesamt $1136$. Um eine handhabbare Anzahl an Publikationen zu erhalten werden in einem weiteren Schritt die Titel und Abstracts der Publikationen genauer betrachtet. Dabei werden unter anderem Arbeiten herausgefiltert, deren Titel und Abstracts keinen Bezug zu den Forschungsfragen besitzen. Weiterhin werden Publikationen bevorzugt, die sich zudem mit textbasierten Programmiersprachen, Kollaboration oder Lehre auf Universitätsniveau befassen. Aus dieser Filterung resultieren $110$ Publikationen zur weiteren Betrachtung.

\begin{table}[]
    \centering
    \begin{tabularx}{\textwidth}{| >{\hsize=.6\hsize\linewidth=\hsize}X |
            >{\hsize=1.4\hsize\linewidth=\hsize}X |}
        \hline
        Stichwort                           & Synonyme                                                                                               \\
        \hline
        integrated development environments & code editors, development environments, development tools, programming tools, programming environments \\
        \hline
        web                                 & browser, online, cloud                                                                                 \\
        \hline
    \end{tabularx}
    \caption{Suchbegriffe}
    \label{table:search-terms}
\end{table}


\begin{table}[]
    \centering
    \begin{tabular}{|c|c|c|c|c|c|}
        \hline
        ACM & DBLP & IEEE & Scopus & Web of Science \\
        \hline
        776 & 649  & 1438 & 4594   & 1025           \\
        \hline
    \end{tabular}
    \caption{Anzahl Suchergebnisse}
    \label{table:amount-search-results}
\end{table}

\section{Ergebnisse}

\begin{itemize}
    \item CS50
    \item PyodideU
    \item RIDE
    \item
\end{itemize}
\chapter{Stand der Technik} \label{stand_der_technik}

\begin{itemize}
    \item Standalone IDEs / Code Editoren
          \begin{itemize}
              \item VSCode
              \item Brackets
              \item Phoenix
              \item IntelliJ IDEs
              \item JupyterLab
          \end{itemize}
    \item IDE Frameworks
          \begin{itemize}
              \item Theia
              \item OpenSumi
          \end{itemize}
    \item Remote Development Platforms
          \begin{itemize}
              \item Eclipse Che
              \item Coder
              \item Gitpod
              \item Replit
              \item Codeanywhere
              \item Stackblitz
              \item Github Codespaces
              \item Amazon Cloud9
              \item CodeOcean?
          \end{itemize}
    \item Educational
          \begin{itemize}
              \item Codeboard
              \item Coding Rooms
              \item CodeHS
              \item JDoodle
              \item KIRA
              \item Codio
              \item Online IDE
              \item Codegrade
          \end{itemize}
\end{itemize}

Im Bereich der IDEs gibt es bereits viele bestehende Softwarelösungen. Für die Zwecke dieser Arbeit ergeben sich drei Kategorien:

\begin{itemize}
    \item \textbf{Standalone IDEs / IDE-Frameworks:} \\ Diese Kategorie beinhaltet Softwarelösungen wie VSCode, Theia und OpenSumi.
    \item \textbf{Educational IDE Plattformen:} \\ Diese Kategorie beinhaltet Softwarelösungen wie CodeHS, Codegrade und KIRA.
    \item \textbf{Workspace Management:} \\ Diese Kategorie beinhaltet Softwarelösungen wie Eclipse Che, Coder und Gitpod.
\end{itemize}

Im folgenden werden die einzelnen Kategorien genauer betrachtet.

\section{Standalone IDEs / IDE-Frameworks}
In der Kategorie Standalone IDEs / IDE-Frameworks befinden sich unter anderem Visual Studio Code, Theia und OpenSumi.

\section{Educational IDE Plattormen}


\section{Workspace Management}
\chapter{Konzeption} \label{konzeption}

Im Fokus dieser Arbeit liegt die Programmierung von Mikrocontrollern im Rahmen des GOLDi Remotelab. Bei den verwendeten Microcontrollern handelt es sich um ATmega2560. Damit diese mit der CrossLab Infrastruktur kommunizieren können sind sie über einen FPGA mit einem Raspberry Pi Compute Module 4 verbunden. Dabei übernimmt der FPGA die Kommunikation zwischen dem CM und dem Microcontroller, während das CM die Kommunikationsschnittstelle zur CrossLab Infrastruktur übernimmt. Als Beispiel für ein steuerbares elektromechanisches Modell wird das 3-Achs-Portal verwendet. Neben den realen Systemen sollen auch die virtuellen Versionen in Betracht gezogen werden. Daraus folgen vier verschiedene Experiment-Konfigurationen, die für die Konzeption betrachtet werden:

\begin{enumerate}
    \item Realer Microcontroller und reales 3-Achs-Portal
    \item Realer Microcontroller und virtuelles 3-Achs-Portal
    \item Virtueller Microcontroller und reales 3-Achs-Portal
    \item Virtueller Microcontroller und virtuelles 3-Achs-Portal
\end{enumerate}

Weiterhin ist zu beachten, dass auch mehrere Steuereinheiten und Modelle in einem Experiment enthalten sein können.

\input{content/06_konzeption/01_datenspeicherung.tex}
\section{Kompilierung} \label{kompilierung}

Die Kompilierung des Quellcodes kann vom CM übernommen werden. Alternativ steht auch hier die Möglichkeit eines externen Servers zur Verfügung. Weiterhin kann auch die Möglichkeit der Kompilierung innerhalb des Browsers des Nutzers erforscht werden. Hierbei gibt es bereits eine Version des C-Compilers clang in WebAssembly. Diese ermöglicht jedoch in der aktuellen Form nur die Kompilierung von C/C++ Code zu WebAssembly. Im Falle des GOLDi Remotelab müsste allerdings eine Kompilierung des Quellcodes für die verwendeten Microcontroller erfolgen. Diese wird normalerweise mit Hilfe des Compilers avr-gcc durchgeführt. Allerdings bietet auch der clang Compiler ein avr Ziel an, welches jedoch einen experimentellen Status hat.
\section{Language Server} \label{language_server}

\begin{itemize}
    \item Language Server über Cloud-instanziierbares Gerät oder über clangd im Browser
\end{itemize}

Eine Möglichkeit zur Einbindung von Language Servern ist es, diese auf dem CM zu installieren und sich dann zu diesen zu verbinden. Allerdings ist das Language Server Protokoll nur auf einen einzigen Client pro Server ausgelegt. Daher müsste in einem Experiment mit mehreren Nutzern für jeden dieser Nutzer eine eigene Instanz des Language Server erstellt werden. Eine spezielle Alternative könnte die Verwendung des zu WebAssembly kompilerten Language Servers clangd darstellen. Dieser könnte im Browser des jeweiligen Nutzers instanziiert werden, wodurch keine zusätzliche Last auf dem CM entsteht. Allerdings muss hierbei die neu entstandene Last auf dem Rechner des Nutzers in Betracht gezogen werden. Eine weitere Möglichkeit ist die Bereitstellung weiterer Server zur Bereitstellung der Language Server.
\input{content/06_konzeption/04_debugging.tex}
\section{Kollaboratives Zusammenarbeiten} \label{kollaboration}

\begin{itemize}
    \item Ein Nutzer teilt sein Projekt mit den anderen Nutzern
    \item Alle Nutzer greifen auf ein geteiltes Projekt zu
    \item Kompilierung, Upload und Debugging sollen auch geteilt werden
\end{itemize}

Um den Nutzern des GOLDi Remotelab ein kollaboratives Zusammenarbeiten zu ermöglichen muss eine entsprechende Extension konzipiert werden. Dazu ist es zunächst wichtig, unterschiedliche Ansätze von Kollaboration zu vergleichen. Durch den Vergleich soll ermittelt werden, welche Ansätze sich für den Einsatz im GOLDi Remotelab eignen. Dabei werden die folgenden Ansätze betrachtet:

\begin{enumerate}
    \item Ein Nutzer teilt sein Projekt mit anderen Nutzern innerhalb eines Experiments
    \item Nutzer arbeiten zusammen an einem geteilten Projekt
\end{enumerate}

\subsection{Conflict-free Replicated Data Types}

Durch die Nutzung von Conflict-free Replicated Data Types kann es den Nutzern ermöglicht werden gemeinsam an Projekten zu arbeiten. Dazu

\subsection{Operational Transformation}

\subsection{Notizen}

Wenn ein Nutzer ein Experiment startet wird entweder ein neues Projekt angelegt (ggf. mit einem entsprechenden Template) oder es wird ein Projekt geladen, dass zu dem entsprechenden Experiment passt.
\input{content/07_implementierung.tex}
\input{content/08_auswertung.tex}
\input{content/09_zusammenfassung_und_fazit.tex}

% opt: zusätzliche Anhänge
\begin{appendix}
    % \input{content/anhang1}
\end{appendix}

% --- Abbildungs-, Tabellen-, Algorithmen- und Literaturverzeichnis ------------
\backmatter
% Abbildungsverzeichnis
\listoffigures
% Tabellenverzeichnis
\listoftables
% Algorithmenverzeichnis
\listofalgorithms
% Literaturverzeichnis
\cleardoublepage	% Literaturverzeichnis immer auf ungerader Seite
\phantomsection		% Anker für Sprungmarke im Inhaltsverzeichnis korrigieren
\addcontentsline{toc}{chapter}{\bibname}
\bibliography{NIKR_bibliography}
\end{document}
