\chapter{Systematische Literaturrecherche} \label{systematische_literaturrecherche}

Um einen Überblick über den aktuellen Stand der Forschung zu bekommen wird zunächst eine Literaturrecherche vorgenommen. Dabei sollen vorhandene online IDEs gefunden sowie die folgenden Fragen beantwortet werden:

\begin{itemize}
    \item Was sind die Anforderungen an die online IDEs?
    \item Welche Vor- und Nachteile haben die online IDEs?
    \item Wie sind die online IDEs aufgebaut?
\end{itemize}

\section{Vorgehensbeschreibung}

Die folgenden Datenbanken wurden für die Literaturrecherche ausgewählt:

\begin{itemize}
    \item ACM Digital Library
    \item IEEE Xplore
    \item Scopus
    \item Web of Science
\end{itemize}

Zunächst wurde eine allgemeine Suche nach online IDEs in den genannten Datenbanken vorgenommen. Dazu werden zunächst die in Tabelle \ref{table:search-terms} genannten Stichwörter jeweils mit ihren Synonymen mit einer OR-Operation verknüpft. Danach werden die daraus resultierenden Terme mit einer AND-Operation verbunden. Die so entstehende Suchanfrage werden dann für die Suche in den Datenbanken verwendet. Dabei werden die Titel, Abstracts und Keywords der Publikationen durchsucht.

In Tabelle \ref{table:amount-search-results} ist die Anzahl der Hits für den einzelnen Datenbanken dargelegt. Um die Anzahl der zu betrachtenden Publikationen zu verringern wird eine weitere Filterung der Ergebnisse vorgenommen. Dafür werden nur Publikationen betrachtet, die IDE oder ein entsprechendes Synonym in ihrem Titel oder ihren Keywords enthalten. Dadurch sinkt die Anzahl der Hits auf insgesamt $1136$. Um eine handhabbare Anzahl an Publikationen zu erhalten werden in einem weiteren Schritt die Titel und Abstracts der Publikationen genauer betrachtet. Dabei werden unter anderem Arbeiten herausgefiltert, deren Titel und Abstracts keinen Bezug zu den Forschungsfragen besitzen. Weiterhin werden Publikationen bevorzugt, die sich zudem mit textbasierten Programmiersprachen, Kollaboration oder Lehre auf Universitätsniveau befassen. Aus dieser Filterung resultieren $110$ Publikationen zur weiteren Betrachtung.

\begin{table}[]
    \centering
    \begin{tabularx}{\textwidth}{| >{\hsize=.6\hsize\linewidth=\hsize}X |
            >{\hsize=1.4\hsize\linewidth=\hsize}X |}
        \hline
        Stichwort                           & Synonyme                                                                                               \\
        \hline
        integrated development environments & code editors, development environments, development tools, programming tools, programming environments \\
        \hline
        web                                 & browser, online, cloud                                                                                 \\
        \hline
    \end{tabularx}
    \caption{Suchbegriffe}
    \label{table:search-terms}
\end{table}


\begin{table}[]
    \centering
    \begin{tabular}{|c|c|c|c|c|c|}
        \hline
        ACM & DBLP & IEEE & Scopus & Web of Science \\
        \hline
        776 & 649  & 1438 & 4594   & 1025           \\
        \hline
    \end{tabular}
    \caption{Anzahl Suchergebnisse}
    \label{table:amount-search-results}
\end{table}

\section{Ergebnisse}

\begin{itemize}
    \item CS50
    \item PyodideU
    \item RIDE
    \item
\end{itemize}