\chapter{Systematische Literaturrecherche} \label{systematische_literaturrecherche}

Um einen Überblick über den aktuellen Stand der Forschung zu bekommen wird zunächst eine Literaturrecherche vorgenommen. Dabei sollen vorhandene online IDEs gefunden sowie die folgenden Fragen beantwortet werden:

\begin{itemize}
    \item Welche Anforderungen werden an online IDEs gestellt?
    \item Welchen Architekturmustern folgen online IDEs?
    \item Welche Vor- und Nachteile haben die online IDEs?
\end{itemize}

\section{Vorgehensbeschreibung}

Die folgenden Datenbanken wurden für die Literaturrecherche ausgewählt:

\begin{itemize}
    \item ACM Digital Library
    \item IEEE Xplore
    \item Scopus
    \item Web of Science
\end{itemize}

Zunächst wurde eine allgemeine Suche nach online IDEs in den genannten Datenbanken vorgenommen. Dazu werden zunächst die in Tabelle \ref{table:search-terms} genannten Stichwörter jeweils mit ihren Synonymen mit einer OR-Operation verknüpft. Danach werden die daraus resultierenden Terme mit einer AND-Operation verbunden. Die so entstehende Suchanfrage werden dann für die Suche in den Datenbanken verwendet. Dabei werden die Titel, Abstracts und Keywords der Publikationen durchsucht.

In Tabelle \ref{table:amount-search-results} ist die Anzahl der Hits für den einzelnen Datenbanken dargelegt. Um die Anzahl der zu betrachtenden Publikationen zu verringern wird eine weitere Filterung der Ergebnisse vorgenommen. Dafür werden nur Publikationen betrachtet, die IDE oder ein entsprechendes Synonym in ihrem Titel oder ihren Keywords enthalten. Dadurch sinkt die Anzahl der Hits auf insgesamt $1154$. Um eine handhabbare Anzahl an Publikationen zu erhalten werden in einem weiteren Schritt die Titel und Abstracts der Publikationen genauer betrachtet. Dabei werden unter anderem Arbeiten herausgefiltert, deren Titel und Abstracts keinen Bezug zu den Forschungsfragen besitzen. Weiterhin werden Publikationen bevorzugt, die sich zudem mit textbasierten Programmiersprachen, Kollaboration und Lehre auf Universitätsniveau befassen. Aus dieser Filterung resultieren $97$ Publikationen zur weiteren Betrachtung.

\begin{table}[tbp]
    \centering
    \begin{tabularx}{\textwidth}{| >{\hsize=.6\hsize\linewidth=\hsize}X |
            >{\hsize=1.4\hsize\linewidth=\hsize}X |}
        \hline
        Stichwort                           & Synonyme                                                                                                     \\
        \hline
        integrated development environments & IDEs, code editors, development environments, development tools, programming tools, programming environments \\
        \hline
        web                                 & browser, online, cloud                                                                                       \\
        \hline
    \end{tabularx}
    \caption{Suchbegriffe}
    \label{table:search-terms}
\end{table}


\begin{table}[tbp]
    \centering
    \begin{tabular}{|c|c|c|c|c|c|}
        \hline
        ACM & IEEE & Scopus & Web of Science \\
        \hline
        785 & 1472 & 4661   & 1044           \\
        \hline
    \end{tabular}
    \caption{Anzahl Suchergebnisse}
    \label{table:amount-search-results}
\end{table}

\section{Ergebnisse}

Die Publikationen beschreiben eine Vielzahl an verschiedenen webbasierten integrierten Entwicklungsumgebungen. Dabei kann eine grobe Unterteilung in die folgenden Kategorien erfolgen:

\begin{itemize}
    \item \textbf{Client-Server-basierte Lösungen} \hfill (Abschnitt \ref{client-server-based-approaches}) \\
          Systeme dieser Art zeichnen sich dadurch aus, dass sie eine Client-Server-Archi-tektur verwenden, wobei alle Nutzer auf denselben Server zugreifen.
    \item \textbf{Browser-basierte Lösungen} \hfill (Abschnitt \ref{browser-based-approaches}) \\
          Systeme dieser Art zeichnen sich dadurch aus, dass alle Features im Browser ausgeführt werden können, ohne die Hilfe eines separaten Servers.
    \item \textbf{Cloud-/Container-basierte Lösungen} \hfill (Abschnitt \ref{cloud-container-based-approaches}) \\
          Systeme dieser Art zeichnen sich dadurch aus, dass sie Cloud-basierte Systeme nutzen oder ihren Nutzern eigene Container zur Ausführung serverseitiger Features bereitstellen.
\end{itemize}

Zusätzlich zu den oben genannten Kategorien sollen in Abschnitt \ref{collaborative-approaches} auch unterschiedliche kollaborative Ansätze betrachtet werden. Schließlich werden in Abschnitt \ref{advantages-online-ides} die Vorteile und in Abschnitt \ref{challenges-online-ides} die Herausforderungen von online IDEs betrachtet.

\subsection{Client-Server-basierte Lösungen} \label{client-server-based-approaches}

Als Beispiele für Client-Server-basierte Lösungen werden IDEOL \cite{IDEOL-2013}, Collabode \cite{Collabode-2011} und JaguarCode \cite{JaguarCode-2018} betrachtet.

\begin{itemize}
    \item IDEOL
    \item collabode
    \item JaguarCode
    \item WIDE
    \item (MiDebug)
\end{itemize}

\subsection{Browser-basierte Lösungen} \label{browser-based-approaches}

\begin{itemize}
    \item PyodideU
    \item WebLinux
    \item LearnCS!
\end{itemize}

Jefferson et al. (2024) \cite{PyodideU-2024} beschreiben eine IDE, die es Nutzern ermöglicht Python Code im Browser zu schreiben und auszuführen. Dabei wird das Programm des Nutzers lokal in dessem Browser durchgeführt. Dies wird durch den Einsatz von PyodideU erreicht, einer erweiterten Version der WebAssembly-basierten Python Distribution Pyodide \cite{Pyodide}. Zusätzlich wird den Nutzern auch eine Grafikbibliothek angeboten samt eines Debuggers, der es ermöglicht Zeile für Zeile und auch rückwärts durch das Programm zu gehen und die entsprechenden Änderungen an der Grafik zu sehen. Weiterhin wird durch PyodideU auch die synchrone Eingabe von Daten unterstützt, während Python im Main-Thread des Browsers läuft. Zudem wird auch ein Dateisystem bereitgestellt. Insgesamt wurde die IDE sowohl von Studenten als auch von Lehrenden als hilfreich wahrgenommen.

\subsection{Cloud-/Container-basierte Lösungen} \label{cloud-container-based-approaches}

\begin{itemize}
    \item CS50
    \item RIDE
    \item MOCSIDE
\end{itemize}

\subsection{Kollaborative Lösungen} \label{collaborative-approaches}

\begin{itemize}
    \item ideol
    \item collabode
    \item RIDE
    \item maybe one of the social media inspired systems
\end{itemize}

\subsection{Vorteile von online IDEs} \label{advantages-online-ides}

\begin{itemize}
    \item Einfacher für Studierende
    \item Platformunabhängig
    \item Gleiche Entwicklungsumgebung für alle
    \item Ggf. einfacher für Lehrende
    \item Ggf. können hier auch Vorteile der einzelnen Lösungskategorien angesprochen werden
\end{itemize}

\subsection{Herausforderungen von online IDEs} \label{challenges-online-ides}

\begin{itemize}
    \item Verschiedene Angriffsvektoren
    \item Skalierbarkeit
    \item Realitätsnähe
    \item Onlinezwang
    \item Ggf. können hier auch Herausforderungen der einzelnen Lösungskategorien angesprochen werden
\end{itemize}

\subsection{Wichtige Aspekte bei der Konzipierung einer online IDE}

\begin{itemize}
    \item TODO
\end{itemize}