\chapter{Systematische Literaturrecherche} \label{systematische_literaturrecherche}

\section{Vorgehensbeschreibung}

Eine systematische Literarturrecherche wird in die folgenden Schritte aufgeteilt:

\begin{enumerate}
    \item Definition der Forschungsfragen
    \item Auswahl der Datenbanken und weiterer Forschungsquellen
    \item Definition der Suchbegriffe
    \item Zusammenführen der Treffer der unterschiedlichen Datenbanken
    \item Anwendung der Aufnahme- und Ausschlusskriterien
    \item Durchführung der Beurteilung
    \item Synthetisieren der Ergebnisse
\end{enumerate}

\section{Ergebnisse}

\begin{enumerate}
    \item Defining research questions
    \item Selecting databases and other research sources
    \item Defining search terms
    \item Merging hits from different databases
    \item Applying inclusion and exclusion criteria
    \item Perform the review
    \item Synthesizing results
\end{enumerate}

Um einen Überblick über den aktuellen Stand der Forschung zu bekommen wird zunächst eine systematische Literaturrecherche vorgenommen. Dabei sollen die folgenden Fragen beantwortet werden:

\begin{enumerate}
    \item Was sind Probleme von integrierten Entwicklungsumgebungen in der Lehre? /\\
          What are problems of integrated development environments in education?
    \item Was sind Vorteile von integrierten Entwicklungsumgebungen in der Lehre? /\\
          What are advantages of integrated development environments in education?
    \item Was sind Nachteile von integrierten Entwicklungsumgebungen in der Lehre? /\\
          What are disadvantages of integrated development environments in education?
    \item Was sind Anforderungen an integrierte Entwicklungsumgebungen in der Lehre? /\\
          What are requirements of integrated development environments in education?
\end{enumerate}

Die folgenden Datenbanken wurden für die Literaturrecherche ausgewählt:

\begin{itemize}
    \item ACM Digital Library
    \item IEEE Xplore
    \item Web of Science
    \item Scopus
\end{itemize}

Um die Datenbanken effizient durchsuchen zu können müssen nun die Suchbegriffe aus den zu beantwortenden Fragen hergeleitet werden. Dafür werden zunächst Synonyme für die einzelnen Stichwörter der Fragen gesucht. Weiterhin werden dabei auch die englischen Varianten der Fragen betrachtet, um ein möglichst breites Spektrum an möglichen Publikationen abzudecken.

\begin{table}[htbp]
    \centering
    \begin{tabularx}{\textwidth}{| >{\hsize=.6\hsize\linewidth=\hsize}X |
            >{\hsize=1.4\hsize\linewidth=\hsize}X |}
        \hline
        Stichwort                           & Synonyme                                                                                                  \\
        \hline
        Probleme                            & Hindernisse, Hürden, Schwierigkeiten, Barrieren, Komplikationen                                           \\ problems                            & obstacles, barriers, hurdles, impediments, obstructions, difficulties \\
        \hline
        Vorteile                            & Nutzen, positive Aspekte, Stärken                                                                         \\ advantages                          & gains, positive aspects, benefits, strengths \\
        \hline
        Nachteile                           & negative Aspekte, Kehrseite, Schwächen                                                                    \\ disadvantages                       & negative aspects, drawbacks, detriments, downsides, weaknesses \\
        \hline
        Anforderungen                       & Erwartungen, Ziele, Rahmenbedingungen                                                                     \\ requirements                        & expectations, goals, condition, precondition \\
        \hline
        integrierte Entwicklungs-umgebungen & Code Editoren, Entwicklungsumgebungen, Entwicklungswerkzeuge, Programmierwerkzeuge, Programmierumgebungen \\ integrated development environments & code editors, development environments, development tools, programming tools, programming environments \\
        \hline
        Lehre                               & Ausbildung, Studium, Fortbildung, Weiterbildung, Universität, Hochschule                                  \\ education                           & studies, training, university, college, academic setting, educational setting  \\
        \hline
        web                                 & browser, online, cloud
    \end{tabularx}
    \caption{Tabelle der Suchbegriffe}
    \label{table:search-terms}
\end{table}

Mithilfe dieser Suchbegriffe werden nun die Datenbanken durchsucht. Dafür werden zunächst für alle Fragen die entsprechenden Stichworte samt ihrer Synonyme mit OR-Operatoren verknüpft. Die auf diese Art entstandenen Gruppen für die Stichwörter werden dann für die jeweilige Frage mit AND-Operatoren verbunden. Dabei werden sowohl Singular als auch Plural der einzelnen Suchbegriffe verwendet. Die auf diese Weise gebildeten Suchanfragen werden dann in allen Datenbanken abgefragt.

\begin{table}[htbp]
    \centering
    \begin{tabular}{|c|c|c|c|c|c|}
        \hline
        ACM & DBLP & IEEE & Web of Science & Scopus \\
        \hline
        776 & 649  & 1438 & 1025           & 4594   \\
        \hline
    \end{tabular}
    \caption{Anzahl Ergebnisse der einzelnen Suchanfragen}
    \label{table:amount-search-results}
\end{table}

Die gefundenen Quellen werden nun gesammelt und analysiert. Dafür werden zunächst die obersten X Ergebnisse der Datenbanken ausgewählt und zusammengeführt. Daraufhin werden diese auf ihre Übereinstimmung mit den gestellten Forschungsfragen überprüft. Dafür werden zunächst die Überschriften, Abstracts und Keywords betrachtet. Die nach dieser Filterung verbleibenden Arbeiten werden dann zur Aufarbeitung des aktuellen Forschungsstands verwendet. Sollte die Anzahl der verbleibenden Arbeiten zu gering sein, so kann eine Vorwärts- bzw. Rückwartssuche mit Hilfe dieser erfolgen.