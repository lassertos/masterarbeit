\section{Anforderungen} \label{anforderungen}

\paragraph{Komplett im Browser nutzbar} \mbox{} \\
Die IDE soll komplett im Browser nutzbar sein. Dies vereinfacht die Einbindung der IDE in das GOLDi Remotelab. Zudem wird der Aufwand für die Studierenden verringert, weil diese keine zusätzliche Software herunterladen und installieren müssen.

\paragraph{Erweiterbarkeit} \mbox{} \\
Die IDE soll erweiterbar sein. Das bedeutet, dass z.B. die Unterstützung weiterer Programmiersprachen oder das Hinzufügen neuer CrossLab Services ermöglicht werden soll.

\paragraph{Kollaboratives Editieren von Dateien} \mbox{} \\
In der modernen Arbeitswelt ist Teamwork eine wichtige Fähigkeit und daher auch ein bedeutendes Lernziel. Daher soll es den Studierenden ermöglicht werden kollaborativ an Projekten zu arbeiten, indem sie gleichzeitig Änderungen an Dateien vornehmen können. Diese Änderungen sollen dann in Echtzeit zwischen den am Experiment teilnehmenden Studierenden synchronisiert werden.

\paragraph{Anbindbarkeit von Compilern} \mbox{} \\
Die IDE soll die Anbindung von Compilern unterstützen. Weiterhin sollen die Studierenden über ein entsprechendes UI-Element die Kompilierung starten bzw. abbrechen können.

\paragraph{Anbindbarkeit von Debuggern} \mbox{} \\
Die IDE soll die Anbindung von Debuggern unterstützen. Weiterhin sollen die Studierenden über ein entsprechendes UI-Element eine Debugging-Sitzung starten können. 

\paragraph{Hochladen von kompilierten Programmen auf Microcontroller} \mbox{} \\
Die IDE soll das Hochladen von kompilierten Programmen auf Microcontroller ermöglichen. Das bedeutet, dass das hochgeladene Programm auf das entsprechende Zielsystem aufgespielt werden soll.

\paragraph{Debuggen von Programmen auf Microcontrollern} \mbox{} \\
Die IDE soll das Debuggen von Programmen auf Microcontrollern ermöglichen. Dabei ist die im GOLDi Remotelab vorhandene Hardware zu beachten. Den Studierenden soll auch ein entsprechendes Nutzer Interface zur Interaktion mit der Debug Sitzung bereitgestellt werden. Weiterhin soll eine aktive Debug-Sitzung allen Teilnehmern eines Experiments angezeigt werden.

\paragraph{CrossLab-Kompatibilität} \mbox{} \\
Die IDE soll als Laborgerät in CrossLab einbindbar sein. Dazu müssen die benötigten Services definiert und implementiert werden.

\paragraph{Anbindbarkeit von Language Servern} \mbox{} \\
Die IDE soll die Anbindung von Language Servern unterstützen.

\paragraph{Kostenlos nutzbar} \mbox{} \\
Das GOLDi Remotelab soll für Nutzer kostenlos nutzbar sein. Daher gilt dasselbe auch für die zu entwickelnde IDE.

\paragraph{Nur ein Nutzerkonto} \mbox{} \\
Nutzer des GOLDi Remotelab sollten nur ein Nutzerkonto zur Ausführung von Experimenten benötigen.

\paragraph{Standalone nutzbar} \mbox{} \\
Die IDE soll auch Standalone nutzbar sein. Dies bedeutet, dass die IDE auch ohne weitere Laborgeräte verwendbar sein soll. Dies ermöglicht es den Studierenden die IDE nutzen zu können ohne dafür ggf. auf die Verfügbarkeit anderer Geräte zu warten bzw. diese zu blockieren.

\paragraph{Einbindbarkeit in Lernplattformen (z.B. über LTI)} \mbox{} \\
Die IDE bzw. Experimente, welche diese enthalten, sollen in Lernplattformen einbindbar sein. Dies vereinfacht den Zugriff für Studierende und erlaubt auch eine einfachere Bewertung der Aufgaben, da Ergebnisse direkt an die Lernplattform gesendet werden können.

\paragraph{Anzeigen der Seriellen Ausgaben} \mbox{} \\
Die IDE sollte in der Lage sein, die Ausgaben der seriellen Schnittstellen eines Microcontrollers darstellen zu können.

\paragraph{Hinterlegen von Testfällen} \mbox{} \\
Es soll ermöglicht werden experimentspezifische Testfälle in der IDE zu hinterlegen. Dies ermöglicht es Studierenden ihre erstellten Lösungen selbst testen zu können. Weiterhin soll die Weitermeldung erfolgreicher Testdurchläufe an die Lernplattform erfolgen können.

\paragraph{Hinterlegen von Beispielcode} \mbox{} \\
Es soll ermöglicht werden experimentspezifischen Beispielcode in der IDE zu hinterlegen. Dies ermöglicht die einfache Bereitstellung leichterer Programmieraufgaben bzw. die Vorgabe einer gewissen Programmstruktur.

\paragraph{Herunterladen von Code} \mbox{} \\
Der Nutzer sollte in der Lage sein seinen erstellten Code herunterladen zu können. 

\paragraph{Hochladen von Code} \mbox{} \\
Der Nutzer sollte in der Lage sein seinen lokal erstellten Code in die IDE hochladen zu können.