\section{Beispielszenario: Praktikumsversuch} \label{beispielszenario}

Als Beispielszenario wird ein Praktikumsversuch beschrieben. Bei einem typischen Praktikumsversuch gibt es die folgenden Rollen, die mit dem zu entwickelnden Code Editor interagieren:

\begin{itemize}
    \item Dozent / Übungsleiter
    \item Versuch-Entwickler
    \item Versuch-Vorbereiter
    \item Versuch-Betreuer
    \item Studierende
    \item Administrator
\end{itemize}

\subsection{Dozent / Übungsleiter}

Der Dozent bzw. der Übungsleiter sollten den Studierenden den Praktikumsversuch vorstellen. Dafür sollten beide in der Lage sein auf das zum Versuch gehörige Experiment ausführen und den Studierenden demonstrieren zu können. Gegebenenfalls können auch Beispielaufgaben als Teil einer Übung mit den Studierenden durchgeführt werden.

\subsection{Versuch-Entwickler}

Der Versuch-Entwickler hat die Aufgabe, den Praktikumsversuch zu entwerfen. Dabei muss er beachten, wie die Aufgabe der Studierenden aufgebaut sein soll und welche Hilfsmittel sie erhalten sollen. So sollte es dem Versuch-Entwickler ermöglicht werden den Code Editor um neue Features zu erweitern, um den Praktikumsversuch bestmöglich gestalten zu können. Dazu zählen unter anderem das Hinzufügen von Compilern, Language Servern, Debuggern, speziellen Editoren etc. Weiterhin sollte der Versuch-Entwickler die Möglichkeit besitzen Testfälle zu einem Experiment hinzufügen zu können, welche die Studierenden nutzen können, um ihre Lösungen zu überprüfen.

\subsection{Versuch-Vorbereiter}

Zu den Aufgaben des Versuch-Vorbereiters zählen das Testen der Versuchshardware sowie des Experiments als auch ggf. die Einbindung der verschiedenen Hilfsmittel in eine Lernplattform wie z.B. Moodle. Weiterhin kann der Labor-Vorbereiter auch Teile der Implementierung- bzw. Ausarbeitung des Praktikumsversuchs übernehmen.

\subsection{Versuch-Betreuer}

Die Aufgabe des Versuch-Betreuers ist es den Versuch zu überwachen und ggf. den Studierenden zu helfen. Beispielsweise könnte es dem Versuch-Betreuer erlaubt sein, einem laufenden Experiment beizutreten und den Studierenden bei einem Problem zu helfen. Angenommen ein Studierender überprüft seine Lösung mit den bereitgestellten Test-Cases und bekommt die Rückmeldung, dass noch Fehler vorliegen. Allerdings vermutet der Studierende einen Hardware-Defekt als Fehlerursache. Der Versuch-Betreuer könnte in diesem Fall am laufenden Experiment des Studierenden teilnehmen und testweise eine Musterlösung auf die Steuereinheit aufspielen. Bei erfolgreichen Durchlaufen der Tests kann der Studierende mit der Bearbeitung des Experiments fortfahren. Ansonsten kann der Versuch-Betreuer versuchen den Defekt zu beheben.

\subsection{Studierende}

Die Studierenden müssen die gegebene Aufgabenstellung bearbeiten. Dafür können sie die bereitgestellten Hilfsmittel und Features des Code Editors nutzen. Zum erfolgreichen Abschluss des Praktikumsversuchs müssen die Testfälle erfolgreich ausgeführt werden. Das Ergebnis sollte in diesem Fall ggf. auch an die Lernplattform weitergeleitet werden. Um Teamwork zu fördern könnte es den Studierenden ermöglicht werden gleichzeitig Änderungen an Dateien vorzunehmen. In diesem Fall könnten auch Compiler und Debugger geteilt werden.

\subsection{Administrator}

Der Administrator ist dafür zuständig, dass die für den Praktikumsversuch benötigte Software vorhanden ist und von den jeweiligen Nutzern verwendet werden kann.