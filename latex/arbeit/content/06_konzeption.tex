\chapter{Konzeption} \label{konzeption}

Im Fokus dieser Arbeit liegt die Programmierung von Mikrocontrollern im Rahmen des GOLDi Remotelab. Bei den verwendeten Microcontrollern handelt es sich um ATmega2560. Damit diese mit der CrossLab Infrastruktur kommunizieren können sind sie über einen FPGA mit einem Raspberry Pi Compute Module 4 verbunden. Dabei übernimmt der FPGA die Kommunikation zwischen dem CM und dem Microcontroller, während das CM die Kommunikationsschnittstelle zur CrossLab Infrastruktur übernimmt. Als Beispiel für ein steuerbares elektromechanisches Modell wird das 3-Achs-Portal verwendet. Neben den realen Systemen sollen auch die virtuellen Versionen in Betracht gezogen werden. Daraus folgen vier verschiedene Experiment-Konfigurationen, die für die Konzeption betrachtet werden:

\begin{enumerate}
    \item Realer Microcontroller und reales 3-Achs-Portal
    \item Realer Microcontroller und virtuelles 3-Achs-Portal
    \item Virtueller Microcontroller und reales 3-Achs-Portal
    \item Virtueller Microcontroller und virtuelles 3-Achs-Portal
\end{enumerate}

Weiterhin ist zu beachten, dass auch mehrere Steuereinheiten und Modelle in einem Experiment enthalten sein können.

\input{content/06_konzeption/01_datenspeicherung.tex}
\section{Kompilierung} \label{kompilierung}

Die Kompilierung des Quellcodes kann vom CM übernommen werden. Alternativ steht auch hier die Möglichkeit eines externen Servers zur Verfügung. Weiterhin kann auch die Möglichkeit der Kompilierung innerhalb des Browsers des Nutzers erforscht werden. Hierbei gibt es bereits eine Version des C-Compilers clang in WebAssembly. Diese ermöglicht jedoch in der aktuellen Form nur die Kompilierung von C/C++ Code zu WebAssembly. Im Falle des GOLDi Remotelab müsste allerdings eine Kompilierung des Quellcodes für die verwendeten Microcontroller erfolgen. Diese wird normalerweise mit Hilfe des Compilers avr-gcc durchgeführt. Allerdings bietet auch der clang Compiler ein avr Ziel an, welches jedoch einen experimentellen Status hat.
\section{Language Server} \label{language_server}

\begin{itemize}
    \item Language Server über Cloud-instanziierbares Gerät oder über clangd im Browser
\end{itemize}

Eine Möglichkeit zur Einbindung von Language Servern ist es, diese auf dem CM zu installieren und sich dann zu diesen zu verbinden. Allerdings ist das Language Server Protokoll nur auf einen einzigen Client pro Server ausgelegt. Daher müsste in einem Experiment mit mehreren Nutzern für jeden dieser Nutzer eine eigene Instanz des Language Server erstellt werden. Eine spezielle Alternative könnte die Verwendung des zu WebAssembly kompilerten Language Servers clangd darstellen. Dieser könnte im Browser des jeweiligen Nutzers instanziiert werden, wodurch keine zusätzliche Last auf dem CM entsteht. Allerdings muss hierbei die neu entstandene Last auf dem Rechner des Nutzers in Betracht gezogen werden. Eine weitere Möglichkeit ist die Bereitstellung weiterer Server zur Bereitstellung der Language Server.
\input{content/06_konzeption/04_debugging.tex}
\section{Kollaboratives Zusammenarbeiten} \label{kollaboration}

\begin{itemize}
    \item Ein Nutzer teilt sein Projekt mit den anderen Nutzern
    \item Alle Nutzer greifen auf ein geteiltes Projekt zu
    \item Kompilierung, Upload und Debugging sollen auch geteilt werden
\end{itemize}

Um den Nutzern des GOLDi Remotelab ein kollaboratives Zusammenarbeiten zu ermöglichen muss eine entsprechende Extension konzipiert werden. Dazu ist es zunächst wichtig, unterschiedliche Ansätze von Kollaboration zu vergleichen. Durch den Vergleich soll ermittelt werden, welche Ansätze sich für den Einsatz im GOLDi Remotelab eignen. Dabei werden die folgenden Ansätze betrachtet:

\begin{enumerate}
    \item Ein Nutzer teilt sein Projekt mit anderen Nutzern innerhalb eines Experiments
    \item Nutzer arbeiten zusammen an einem geteilten Projekt
\end{enumerate}

\subsection{Conflict-free Replicated Data Types}

Durch die Nutzung von Conflict-free Replicated Data Types kann es den Nutzern ermöglicht werden gemeinsam an Projekten zu arbeiten. Dazu

\subsection{Operational Transformation}

\subsection{Notizen}

Wenn ein Nutzer ein Experiment startet wird entweder ein neues Projekt angelegt (ggf. mit einem entsprechenden Template) oder es wird ein Projekt geladen, dass zu dem entsprechenden Experiment passt.