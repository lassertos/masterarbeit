\chapter{Konzeption} \label{konzeption}

Im Fokus dieser Arbeit liegt die Programmierung von Mikrocontrollern im Rahmen des GOLDi Remotelab. Bei den verwendeten Microcontrollern handelt es sich um ATmega2560. Damit diese mit der CrossLab Infrastruktur kommunizieren können sind sie über einen FPGA mit einem Raspberry Pi Compute Module 4 verbunden. Dabei übernimmt der FPGA die Kommunikation zwischen dem CM und dem Microcontroller, während das CM die Kommunikationsschnittstelle zur CrossLab Infrastruktur übernimmt. Als Beispiel für ein steuerbares elektromechanisches Modell wird das 3-Achs-Portal verwendet. Neben den realen Systemen sollen auch die virtuellen Versionen in Betracht gezogen werden. Daraus folgen vier verschiedene Experiment-Konfigurationen, die für die Konzeption betrachtet werden:

\begin{enumerate}
    \item Realer Microcontroller und reales 3-Achs-Portal
    \item Realer Microcontroller und virtuelles 3-Achs-Portal
    \item Virtueller Microcontroller und reales 3-Achs-Portal
    \item Virtueller Microcontroller und virtuelles 3-Achs-Portal
\end{enumerate}

Weiterhin ist zu beachten, dass auch mehrere Steuereinheiten und Modelle in einem Experiment enthalten sein können.

\section{Datenspeicherung} \label{datenspeicherung}

Um sicherzustelllen, dass die IDE komplett im Browser verwendet werden kann, müssen die Nutzer in der Lage sein ihre Projekte in dieser verwalten zu können. Da noch nicht alle Browser die File System API komplett implementieren muss die Projektverwaltung auf einem anderen Weg implementiert werden. Dazu kann die IndexedDB API genutzt werden. Diese ermöglicht die persistente Speicherung größerer Datenmengen direkt im Browser. Um die IndexedDB zur Projektverwaltung nutzen zu können muss eine entsprechende Extension für Visual Studio Code geschrieben werden. Dafür kann die FileSystemProvider API verwendet werden. Diese ermöglicht es eigene Dateisysteme in Visual Studio Code einzubinden. Weiterhin gibt es noch die FileSearchProvider API und die TextSearchProvider API. Die FileSearchProvider API ermöglicht es die Logik für die Suche nach Dateien in einem bestimmten Dateisystem zu implementieren. Die TextSearchProvider API ermöglicht es die Logik für die Durchsuchung des Textinhalts von Dateien zu implementieren. Sowohl die FileSearchProvider API als auch die TextSearchProvider API befinden sich aktuell noch in einem experimentellen Zustand. Das bedeutet, dass sie verwendet werden können, allerdings dürfen Extensions, die diese APIs nutzen nicht auf dem Marketplace veröffentlicht werden. Dies sollte allerdings für die Verwendung innerhalb des GOLDi Remotelab kein Problem darstellen.

Weiterhin besteht jedoch auch die Möglichkeit die Projekte der Nutzer auf einem Server zu speichern. Dies würde es den Nutzern ermöglichen zwischen ihren Endgeräten wechseln zu können, ohne ihren aktuellen Stand zu verlieren. Allerdings müssen dann entsprechende Server bereitgestellt werden. Ein anderer Ansatz könnte es den Nutzern erlauben ihre Projekt aus GitHub oder GitLab zu laden.
\section{Kompilierung} \label{kompilierung}

Die Kompilierung des Quellcodes kann vom CM übernommen werden. Alternativ steht auch hier die Möglichkeit eines externen Servers zur Verfügung. Weiterhin kann auch die Möglichkeit der Kompilierung innerhalb des Browsers des Nutzers erforscht werden. Hierbei gibt es bereits eine Version des C-Compilers clang in WebAssembly. Diese ermöglicht jedoch in der aktuellen Form nur die Kompilierung von C/C++ Code zu WebAssembly. Im Falle des GOLDi Remotelab müsste allerdings eine Kompilierung des Quellcodes für die verwendeten Microcontroller erfolgen. Diese wird normalerweise mit Hilfe des Compilers avr-gcc durchgeführt. Allerdings bietet auch der clang Compiler ein avr Ziel an, welches jedoch einen experimentellen Status hat.
\section{Language Server} \label{language_server}

\begin{itemize}
    \item Language Server über Cloud-instanziierbares Gerät oder über clangd im Browser
\end{itemize}

Eine Möglichkeit zur Einbindung von Language Servern ist es, diese auf dem CM zu installieren und sich dann zu diesen zu verbinden. Allerdings ist das Language Server Protokoll nur auf einen einzigen Client pro Server ausgelegt. Daher müsste in einem Experiment mit mehreren Nutzern für jeden dieser Nutzer eine eigene Instanz des Language Server erstellt werden. Eine spezielle Alternative könnte die Verwendung des zu WebAssembly kompilerten Language Servers clangd darstellen. Dieser könnte im Browser des jeweiligen Nutzers instanziiert werden, wodurch keine zusätzliche Last auf dem CM entsteht. Allerdings muss hierbei die neu entstandene Last auf dem Rechner des Nutzers in Betracht gezogen werden. Eine weitere Möglichkeit ist die Bereitstellung weiterer Server zur Bereitstellung der Language Server.
\section{Debugging} \label{debugging}

\begin{itemize}
    \item Debugging von realem Microcontroller über RPi (avr-gcc)
    \item Debugging von virtuellem Microcontroller über Cloud-instanziierbares Gerät oder über avr-gcc im Browser
\end{itemize}

Das Debuggen von Programmen auf den vorhandenen Microcontrollern gestaltet sich schwierig. Eine Möglichkeit ist die Nutzung der Bibliothek avr\_debug. Diese wird zusammen mit dem Programm kompiliert und auf den Microcontroller hochgeladen. Dort erstellt sie ein Interface für den Debugger gdb. Dieses Interface nutzt die Serielle Schnittstelle des Microcontrollers zur Kommunikation mit gdb. Das CM agiert in diesem Szenario als Schnittstelle zwischen gdb und unserer IDE. Ein Nachteil dieses Vorgehens ist der hohe Speicherverbrauch der Bibliothek, welcher die Anzahl möglicher Programme einschränkt. Allerdings ist ein Vorteil dieses Ansatzes, dass keine zusätzlichen Kosten durch die Anschaffung externe Debugger entstehen.

Ein weiteres Problem, was beim Debuggen eines laufenden Experimentes beachtet werden muss, ist die fortlaufende Ansteuerung von weiteren Geräten. Nehmen wir als Beispiel ein einfaches Experiment bestehend aus einem Microcontroller und einem 3-Achs-Portal. Wenn wir das Program des Microcontrollers unterbrechen, während dieser den Portalkran aktiv nach rechts bewegt, so wird diese Bewegung nicht unterbrochen. Um sicherzustellen, dass die Signale von Aktoren während eines Breakpoints nicht an andere Geräte weitergeleitet werden müssen die anderen Geräte entsprechend benachrichtigt werden.
\section{Kollaboratives Zusammenarbeiten} \label{kollaboration}

\begin{itemize}
    \item Ein Nutzer teilt sein Projekt mit den anderen Nutzern
    \item Alle Nutzer greifen auf ein geteiltes Projekt zu
    \item Kompilierung, Upload und Debugging sollen auch geteilt werden
\end{itemize}

Um den Nutzern des GOLDi Remotelab ein kollaboratives Zusammenarbeiten zu ermöglichen muss eine entsprechende Extension konzipiert werden. Dazu ist es zunächst wichtig, unterschiedliche Ansätze von Kollaboration zu vergleichen. Durch den Vergleich soll ermittelt werden, welche Ansätze sich für den Einsatz im GOLDi Remotelab eignen. Dabei werden die folgenden Ansätze betrachtet:

\begin{enumerate}
    \item Ein Nutzer teilt sein Projekt mit anderen Nutzern innerhalb eines Experiments
    \item Nutzer arbeiten zusammen an einem geteilten Projekt
\end{enumerate}

\subsection{Ansatz 1}

In Ansatz 1 besitzt nur ein Nutzer das Projekt. Das bedeutet, dass nur bei diesem Nutzer der aktuelle Stand des Projekts gespeichert wird. Bei allen anderen Nutzern werden lediglich die Änderungen während eines Experiments synchronisiert. Sobald das Experiment beendet wurde werden keine weiteren Änderungen am Projekt mehr synchronisiert.

\subsection{Ansatz 2}

In Ansatz 2 wird der aktuelle Zustand des Projekts auf einem Server gespeichert. Alle Nutzer mit den entsprechenden Berechtigungen können dementsprechend auf den aktuellen Stand des Projekts zugreifen und Änderungen vornehmen, solange sie eine aktive Verbindung zum Server haben. Sollten sie keine aktive Verbindung zum Server haben, sollten Ihnen keine Änderungen am Projekt erlaubt sein.