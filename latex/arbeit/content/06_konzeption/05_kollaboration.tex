\section{Kollaboratives Zusammenarbeiten} \label{kollaboration}

\begin{itemize}
    \item Ein Nutzer teilt sein Projekt mit den anderen Nutzern
    \item Alle Nutzer greifen auf ein geteiltes Projekt zu
    \item Kompilierung, Upload und Debugging sollen auch geteilt werden
\end{itemize}

Um den Nutzern des GOLDi Remotelab ein kollaboratives Zusammenarbeiten zu ermöglichen muss eine entsprechende Extension konzipiert werden. Dazu ist es zunächst wichtig, unterschiedliche Ansätze von Kollaboration zu vergleichen. Durch den Vergleich soll ermittelt werden, welche Ansätze sich für den Einsatz im GOLDi Remotelab eignen. Dabei werden die folgenden Ansätze betrachtet:

\begin{enumerate}
    \item Ein Nutzer teilt sein Projekt mit anderen Nutzern innerhalb eines Experiments
    \item Nutzer arbeiten zusammen an einem geteilten Projekt
\end{enumerate}

\subsection{Conflict-free Replicated Data Types}

Durch die Nutzung von Conflict-free Replicated Data Types kann es den Nutzern ermöglicht werden gemeinsam an Projekten zu arbeiten. Dazu

\subsection{Operational Transformation}

\subsection{Notizen}

Wenn ein Nutzer ein Experiment startet wird entweder ein neues Projekt angelegt (ggf. mit einem entsprechenden Template) oder es wird ein Projekt geladen, dass zu dem entsprechenden Experiment passt.