\section{Kollaboratives Zusammenarbeiten} \label{kollaboration}

\begin{itemize}
    \item Ein Nutzer teilt sein Projekt mit den anderen Nutzern
    \item Alle Nutzer greifen auf ein geteiltes Projekt zu
    \item Kompilierung, Upload und Debugging sollen auch geteilt werden
\end{itemize}

Um den Nutzern des GOLDi Remotelab ein kollaboratives Zusammenarbeiten zu ermöglichen muss eine entsprechende Extension konzipiert werden. Dazu ist es zunächst wichtig, unterschiedliche Ansätze von Kollaboration zu vergleichen. Durch den Vergleich soll ermittelt werden, welche Ansätze sich für den Einsatz im GOLDi Remotelab eignen. Dabei werden die folgenden Ansätze betrachtet:

\begin{enumerate}
    \item Ein Nutzer teilt sein Projekt mit anderen Nutzern innerhalb eines Experiments
    \item Nutzer arbeiten zusammen an einem geteilten Projekt
\end{enumerate}

\subsection{Ansatz 1}

In Ansatz 1 besitzt nur ein Nutzer das Projekt. Das bedeutet, dass nur bei diesem Nutzer der aktuelle Stand des Projekts gespeichert wird. Bei allen anderen Nutzern werden lediglich die Änderungen während eines Experiments synchronisiert. Sobald das Experiment beendet wurde werden keine weiteren Änderungen am Projekt mehr synchronisiert.

\subsection{Ansatz 2}

In Ansatz 2 wird der aktuelle Zustand des Projekts auf einem Server gespeichert. Alle Nutzer mit den entsprechenden Berechtigungen können dementsprechend auf den aktuellen Stand des Projekts zugreifen und Änderungen vornehmen, solange sie eine aktive Verbindung zum Server haben. Sollten sie keine aktive Verbindung zum Server haben, sollten Ihnen keine Änderungen am Projekt erlaubt sein.