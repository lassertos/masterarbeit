\chapter{Stand der Technik} \label{stand_der_technik}

\begin{itemize}
    \item Standalone IDEs / Code Editoren
          \begin{itemize}
              \item VSCode
              \item Brackets
              \item Phoenix
              \item IntelliJ IDEs
              \item JupyterLab
          \end{itemize}
    \item IDE Frameworks
          \begin{itemize}
              \item Theia
              \item OpenSumi
          \end{itemize}
    \item Remote Development Platforms
          \begin{itemize}
              \item Eclipse Che
              \item Coder
              \item Gitpod
              \item Replit
              \item Codeanywhere
              \item Stackblitz
              \item Github Codespaces
              \item Amazon Cloud9
              \item CodeOcean?
          \end{itemize}
    \item Educational
          \begin{itemize}
              \item Codeboard
              \item Coding Rooms
              \item CodeHS
              \item JDoodle
              \item KIRA
              \item Codio
              \item Online IDE
              \item Codegrade
          \end{itemize}
\end{itemize}

Im Bereich der IDEs gibt es bereits viele bestehende Softwarelösungen. Für die Zwecke dieser Arbeit ergeben sich drei Kategorien:

\begin{itemize}
    \item \textbf{Standalone IDEs / IDE-Frameworks:} \\ Diese Kategorie beinhaltet Softwarelösungen wie VSCode, Theia und OpenSumi.
    \item \textbf{Educational IDE Plattformen:} \\ Diese Kategorie beinhaltet Softwarelösungen wie CodeHS, Codegrade und KIRA.
    \item \textbf{Workspace Management:} \\ Diese Kategorie beinhaltet Softwarelösungen wie Eclipse Che, Coder und Gitpod.
\end{itemize}

Im folgenden werden die einzelnen Kategorien genauer betrachtet.

\section{Standalone IDEs / IDE-Frameworks}
In der Kategorie Standalone IDEs / IDE-Frameworks befinden sich unter anderem Visual Studio Code, Theia und OpenSumi.

\section{Educational IDE Plattormen}


\section{Workspace Management}