\chapter{Anforderungsanalyse} \label{anforderungsanalyse}

In diesem Kapitel sollen die Anforderung an die zu entwickelnde integrierte Entwicklungsumgebung dargelegt werden. Diese werden in Gesprächen mit Experten des GOLDi-Remotelab sowie Experten des CrossLab Projekts erhoben.

\begin{tabularx}{\textwidth}{|lX|}
    \hline
    \textbf{Anforderung Nr.:}     & 1                                                                     \\
    \textbf{Beschreibung:}        & Die IDE soll komplett im Browser nutzbar sein.                        \\
    \textbf{Begründung:}          & Studierende haben oftmals Probleme mit der Installation von Software. \\
    \textbf{Eignungskriterium:}   & ?                                                                     \\
    \textbf{Kundenzufriedenheit:} & 5                                                                     \\ \textbf{Kundenunzufriedenheit:} & 5 \\
    \hline
\end{tabularx}

\begin{tabularx}{\textwidth}{|lX|}
    \hline
    \textbf{Anforderung Nr.:}     & 2                                     \\
    \textbf{Beschreibung:}        & Die IDE soll erweiterbar sein.        \\
    \textbf{Begründung:}          & Durch die Erweiterbarkeit der IDE ... \\
    \textbf{Eignungskriterium:}   & ?                                     \\
    \textbf{Kundenzufriedenheit:} & 5                                     \\ \textbf{Kundenunzufriedenheit:} & 5 \\
    \hline
\end{tabularx}

\begin{tabularx}{\textwidth}{|lX|}
    \hline
    \textbf{Anforderung Nr.:}     & 3                                                              \\
    \textbf{Beschreibung:}        & Die IDE soll kollaboratives Editieren von Dateien ermöglichen. \\
    \textbf{Begründung:}          & Teamwork ist ein wichtiges Lernziel.                           \\
    \textbf{Eignungskriterium:}   & ?                                                              \\
    \textbf{Kundenzufriedenheit:} & 5                                                              \\ \textbf{Kundenunzufriedenheit:} & 5 \\
    \hline
\end{tabularx}

\begin{tabularx}{\textwidth}{|lX|}
    \hline
    \textbf{Anforderung Nr.:}     & 4                                                                              \\
    \textbf{Beschreibung:}        & Die IDE soll das Kompilieren von C Programmen für Microcontroller ermöglichen. \\
    \textbf{Begründung:}          & ?                                                                              \\
    \textbf{Eignungskriterium:}   & ?                                                                              \\
    \textbf{Kundenzufriedenheit:} & ?                                                                              \\ \textbf{Kundenunzufriedenheit:} & ? \\
    \hline
\end{tabularx}

\begin{tabularx}{\textwidth}{|lX|}
    \hline
    \textbf{Anforderung Nr.:}     & 5                                                                                       \\
    \textbf{Beschreibung:}        & Die IDE soll das Hochladen von kompilierten Programmen auf Microcontroller ermöglichen. \\
    \textbf{Begründung:}          & ?                                                                                       \\
    \textbf{Eignungskriterium:}   & ?                                                                                       \\
    \textbf{Kundenzufriedenheit:} & ?                                                                                       \\ \textbf{Kundenunzufriedenheit:} & ? \\
    \hline
\end{tabularx}

\begin{tabularx}{\textwidth}{|lX|}
    \hline
    \textbf{Anforderung Nr.:}     & 6                                                                          \\
    \textbf{Beschreibung:}        & Die IDE soll das Debuggen von Programmen auf Microcontrollern ermöglichen. \\
    \textbf{Begründung:}          & ?                                                                          \\
    \textbf{Eignungskriterium:}   & ?                                                                          \\
    \textbf{Kundenzufriedenheit:} & ?                                                                          \\ \textbf{Kundenunzufriedenheit:} & ? \\
    \hline
\end{tabularx}

\begin{tabularx}{\textwidth}{|lX|}
    \hline
    \textbf{Anforderung Nr.:}     & 7                                      \\
    \textbf{Beschreibung:}        & Die IDE soll CrossLab-kompatibel sein. \\
    \textbf{Begründung:}          & ?                                      \\
    \textbf{Eignungskriterium:}   & ?                                      \\
    \textbf{Kundenzufriedenheit:} & ?                                      \\ \textbf{Kundenunzufriedenheit:} & ? \\
    \hline
\end{tabularx}

\begin{tabularx}{\textwidth}{|lX|}
    \hline
    \textbf{Anforderung Nr.:}     & 8                                                             \\
    \textbf{Beschreibung:}        & Die IDE soll die Anbindung von Language Servern unterstützen. \\
    \textbf{Begründung:}          & ?                                                             \\
    \textbf{Eignungskriterium:}   & ?                                                             \\
    \textbf{Kundenzufriedenheit:} & ?                                                             \\ \textbf{Kundenunzufriedenheit:} & ? \\
    \hline
\end{tabularx}

\begin{tabularx}{\textwidth}{|lX|}
    \hline
    \textbf{Anforderung Nr.:}     & 9                                    \\
    \textbf{Beschreibung:}        & Die IDE soll kostenlos nutzbar sein. \\
    \textbf{Begründung:}          & ?                                    \\
    \textbf{Eignungskriterium:}   & ?                                    \\
    \textbf{Kundenzufriedenheit:} & ?                                    \\ \textbf{Kundenunzufriedenheit:} & ? \\
    \hline
\end{tabularx}

\begin{tabularx}{\textwidth}{|lX|}
    \hline
    \textbf{Anforderung Nr.:}     & 10                                                          \\
    \textbf{Beschreibung:}        & Die IDE soll auch außerhalb eines Experiments nutzbar sein. \\
    \textbf{Begründung:}          & ?                                                           \\
    \textbf{Eignungskriterium:}   & ?                                                           \\
    \textbf{Kundenzufriedenheit:} & ?                                                           \\ \textbf{Kundenunzufriedenheit:} & ? \\
    \hline
\end{tabularx}

\begin{tabularx}{\textwidth}{|lX|}
    \hline
    \textbf{Anforderung Nr.:}     & 11                                                                                                \\
    \textbf{Beschreibung:}        & Das erlernte Wissen im Umgang mit der IDE soll auch außerhalb des GOLDi Remotelab anwendbar sein. \\
    \textbf{Begründung:}          & ?                                                                                                 \\
    \textbf{Eignungskriterium:}   & ?                                                                                                 \\
    \textbf{Kundenzufriedenheit:} & ?                                                                                                 \\ \textbf{Kundenunzufriedenheit:} & ? \\
    \hline
\end{tabularx}

\begin{tabularx}{\textwidth}{|lX|}
    \hline
    \textbf{Anforderung Nr.:}     & 12                                                    \\
    \textbf{Beschreibung:}        & Debug-Sessions sollten zwischen Nutzern teilbar sein. \\
    \textbf{Begründung:}          & ?                                                     \\
    \textbf{Eignungskriterium:}   & ?                                                     \\
    \textbf{Kundenzufriedenheit:} & ?                                                     \\ \textbf{Kundenunzufriedenheit:} & ? \\
    \hline
\end{tabularx}

\begin{tabularx}{\textwidth}{|lX|}
    \hline
    \textbf{Anforderung Nr.:}     & 13                                                                \\
    \textbf{Beschreibung:}        & Die IDE sollte in Lehrplattformen einbindbar sein (z.B. über LTI) \\
    \textbf{Begründung:}          & ?                                                                 \\
    \textbf{Eignungskriterium:}   & ?                                                                 \\
    \textbf{Kundenzufriedenheit:} & ?                                                                 \\ \textbf{Kundenunzufriedenheit:} & ? \\
    \hline
\end{tabularx}

% \begin{tabularx}{\textwidth}{|lX|}
%     \hline
%     \textbf{Anforderung Nr.:}     & ? \\
%     \textbf{Beschreibung:}        & ? \\
%     \textbf{Begründung:}          & ? \\
%     \textbf{Eignungskriterium:}   & ? \\
%     \textbf{Kundenzufriedenheit:} & ? \\ \textbf{Kundenunzufriedenheit:} & ? \\
%     \hline
% \end{tabularx}