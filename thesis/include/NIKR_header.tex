% === Dokumentklasse ==========================================================
\documentclass[%
    a4paper,							% A4
    12pt,								% Schriftgröße 12
    twoside,							% zweiseitig
    fleqn								% Formeln linksbündig ausrichten
]{book}


% === Grundlegende Pakete =====================================================
\usepackage{etoolbox}					% portiert viele nützliche Sachen aus e-TeX (z.B. booleans, ifs)
\usepackage[							% erweiterte Angabe von Farben
    pdftex,								% Farbtreiber auswählen
    dvipsnames							% vordefinierte Farben laden
]{xcolor}								% erweiterte Angabe von Farben
\usepackage{xparse}						% high-level Interface für Dokumentbefehle wie \NewDocument(Command,Environment)

\usepackage[utf8]{inputenc}				% Kodierung der *.tex Dateien ist UTF-8 nicht ISO-8859-1
\usepackage{cmap}						% character map tables -> pdf Inhalt wird besser durchsuch- und kopierbar
\usepackage[T1]{fontenc}				% OT1 encoding für deutsche Sonderzeichen -> pdf Inhalt kann Sonderzeichen durchsucht werden
\usepackage{lmodern}					% Latin-Modern-Schriftart (Computer Modern in Verbindung mit OT1 führt zu Bitmap-Fonts auf Windows)
\usepackage[nopatch=eqnum]{microtype}					% mikrotypografische Einstellungen, die den gesetzten Text nochmals wesentlich verbessern (weniger badboxes)


% === Benutzerdefinierte Einstellungen ========================================
% === Art der Arbeit ==========================================================
% von den nachfolgenden Blöcken bitte den richtigen auswählen und die anderen auskommentieren/löschen

% --- Diplom ----------
%\newcommand{\settingsDegree}{{Diplom}}
%\newcommand{\settingsDegreeName}{{Diplominformatiker}}
%\newcommand{\settingsDegreeName}{{Diplomingenieur}}

% --- Bachelor ----------
% \newcommand{\settingsDegree}{{Bachelor}}
% \newcommand{\settingsDegreeName}{{Bachelor of Science}}

% --- Master ----------
\newcommand{\settingsDegree}{{Master}}
\newcommand{\settingsDegreeName}{{Master of Science}}

% === Name, Abgabedatum und Sprache der Arbeit ================================
\newcommand{\settingsName}{{Pierre Helbing}}
\newcommand{\settingsFinishDate}{{DD.MM.YYYY}}
\newcommand{\settingsLanguage}{german}   		% german / american

% === Weitere Einstellungen ===================================================
% --- Suchpfad (Unterverzeichnis) für eingebundene Grafiken ----------
\newcommand{\settingsGraphicsPath}{image/}

% --- Hinweiskapitel ----------
\newbool{settingsWithHints}
\setbool{settingsWithHints}{false}				% true / false

% --- Zeilennummern ----------
\newbool{settingsWithLineNumbers}
\setbool{settingsWithLineNumbers}{false}			% true / false

% --- Todos ----------
\newbool{settingsWithTodos}
\setbool{settingsWithTodos}{true}				% true / false

% --- Anzahl an Nummerierungsebenen im Text und Inhaltsverzeichnis ----------
% 1: \section
% 2: \section + \subsection
% Achtung: 3 oder 4 nur nach Absprache mit Betreuer !
% 3: \section + \subsection + \subsubsection
% 4: \section + \subsection + \subsubsection + \paragraph
\setcounter{secnumdepth}{2}
\setcounter{tocdepth}{2}

% --- Anforderungen ----------

\usepackage{tabularx}
\usepackage{multicol}
\newenvironment{myreq}[1]{%
    \setlist[description]{font=\normalfont\color{darkgray}}%
    \begin{tcolorbox}[colframe=black,colback=white, sharp corners, boxrule=1pt]%
        \bfseries\color{blue}%
        \begin{description}#1}%
            {\end{description}\end{tcolorbox}}

\newcommand{\threeinline}[3]{\begin{multicols}{3}#1 #2 #3\end{multicols}}
\newcommand{\twoinline}[2]{\begin{multicols}{2}#1 #2\end{multicols}}

\newcommand{\reqno}{\item[Requirement \#:]}
\newcommand{\reqtype}{\item[Requirement Type:]}
\newcommand{\reqevent}{\item[Event/BUC/PUC \#:]}
\newcommand{\reqdesc}{\item[Description:]}
\newcommand{\reqrat}{\item[Rationale:]}
\newcommand{\reqorig}{\item[Originator:]}
\newcommand{\reqfit}{\item[Fit Criterion:]}
\newcommand{\reqsatis}{\item[Customer Satisfaction:]}
\newcommand{\reqdissat}{\item[Customer Dissatisfaction:]}
\newcommand{\reqdep}{\item[Dependencies:]}
\newcommand{\reqconf}{\item[Conflicts:]}
\newcommand{\reqmater}{\item[Materials:]}
\newcommand{\reqhist}{\item[History:]}
\usepackage{multicol}
\usepackage{multirow}
\usepackage[dvipsnames]{xcolor}
\usepackage{enumitem}
\usepackage[most]{tcolorbox}
\usepackage{parskip}
\usepackage{pgf-umlsd}

% based on \newthread macro from pgf-umlsd.sty
% add fourth argument (third mandatory) that defines the xshift of a thread.
% that argument gets passed to \newinst, which has an optional
% argument that defines the shift
\newcommand{\newthreadShift}[4][gray!30]{
\newinst[#4]{#2}{#3}
\stepcounter{threadnum}
\node[below of=inst\theinstnum,node distance=0.8cm] (thread\thethreadnum) {};
\tikzstyle{threadcolor\thethreadnum}=[fill=#1]
\tikzstyle{instcolor#2}=[fill=#1]
}

\newcommand{\quoted}[1]{\glqq#1\grqq}

\newenvironment{note}
{
    \begin{tcolorbox}[boxrule=0pt,frame hidden,sharp corners,enhanced,borderline west={2pt}{0pt}{pink}, colback=black!3,breakable]
        \textbf{Notiz:} \\
        }
        {
    \end{tcolorbox}
}

% === Übersetzungen ===========================================================
% Definitionen je nach \mylanguage (siehe NIKR_settings.tex)
\ifdefstring{\settingsLanguage}{german}{%
    \newcommand{\acroname}{Abkürzungsverzeichnis}	% Name für Abkürzungsverzeichnis
    \newcommand{\todoname}{Todo Liste}	% Name für Todo-Liste
    \newcommand{\pagename}{Seite}		% Name für Seite
}%
{%
    \newcommand{\acroname}{Acronyms}	% Name für Abkürzungsverzeichnis
    \newcommand{\todoname}{Todo list}	% Name für Todo-Liste
    \newcommand{\pagename}{page}		% Name für Seite
}%


% === Wichtige Pakete und Einstellungen =======================================
\usepackage{calc}						% ermöglicht Arithmetik in den Argumenten von Befehlen

% Einstellungen je nach \mylanguage (siehe NIKR_settings.tex)
\ifdefstring{\settingsLanguage}{german}{%
    \usepackage[ngerman]{babel}			% Spracheinstellungen (für deutsch z.B. Contents -> Inhaltsverzeichnis, etc.)
    \usepackage{bibgerm}            	% Stylefile für deutsche Literaturstellenangabe
    \bibliographystyle{unsrt}		    % Stil für Literaturangaben festlegen
}%
{%
    \usepackage[american]{babel}		% Spracheinstellungen (für deutsch z.B. Contents -> Inhaltsverzeichnis, etc.)
    \bibliographystyle{unsrt}			% Stil für Literaturangaben festlegen
    \frenchspacing						% einfaches Leerzeiches nach Satzende (für deutsch bereits Standard)
}%
\usepackage[%
    noadjust        					% noadjust verhindert automatische Leerzeichen um die Referenz, was am Zeilenanfang zu Problemen führt
]{cite}     							% erlaubt Zeilenumbruch innerhalb von Zitierungen

\usepackage{graphicx}					% erweiterte Argumente in \in­clude­graph­ics
\graphicspath{{\settingsGraphicsPath}}	% Standard-Pfad für Bilder siehe NIKR_settings.tex

\usepackage[bf, margin=0.75cm]{caption}				% erlaubt erweiterte Formatierungen in \caption (siehe unten)
\usepackage{subcaption}					% mehrere Abbildungen nebeneinander

\usepackage{amsmath}					% ermöglicht \DeclareMathOperator
\usepackage{amssymb}					% mathmatische Symbole und Sonderzeichen
\usepackage{nicefrac}					% für \nicefrac
\usepackage{nccmath}					% für \mfrac

\usepackage{icomma}						% intelligentes Komma (macht Verwendung von {,} überflüssig)
\usepackage{siunitx}					% für einheitliche Angabe von Einheiten

\usepackage{fancyhdr}					% Kopf- und Fußzeilen (siehe unten)

\usepackage{hhline}						% erweitere Rahmengestaltung in Tabellen

\usepackage[%							% Einbettung von Links im Dokument und erlaubt die Nutzugn von \url
    hyperfootnotes=false,				% keine Fußnoten als Link im Dokument (geht nicht mit footmisc)
    pagebackref=true					% Backrefs im Literaturverzeichnis
]{hyperref}           					% Einbettung von Links im Dokument und erlaubt die Nutzugn von \url
\renewcommand*{\backref}[1]{\textit{(\pagename:~#1)}}   % Format für backrefs

\usepackage[%							% erlaubt erweiterte Formatierungen von Fußnoten (siehe unten)
    multiple,							% mehrere mit Komma abtrennen
    hang								% linksbündig, \footnotemargin entscheidet über Einrückung
]{footmisc}								% erlaubt erweiterte Formatierungen von Fußnoten (siehe unten)
\patchcmd{\footref}{\ref}{\ref*}{}{}	% Hyperlink in \footref entfernen

\usepackage[nohyperlinks]{acronym}		% Abkürzungsverzeichnis

\usepackage{setspace}					% für \setstretch (ändern Zeilenabstand, aber nicht floating Umgebungen)

\usepackage[%							% Todos
    colorinlistoftodos,					% farbige Markierungen in Todo-Liste
    prependcaption,						% caption=val
    textsize=tiny,						% Schriftgröße
    linecolor=red,						% Standard-Linienfarbe für \todo
    backgroundcolor=red!25,				% Standard-Hintergrundfarbe für \todo
    bordercolor=red,					% Standard-Rahmenfarbe für \todo
    textwidth=2cm,						% Standard-Breite für \todo
]{todonotes}							% Todos
\ifbool{settingsWithTodos}{%
    \setlength{\marginparwidth}{2cm}	% sonst werden Notes am Rand nicht richtig angezeigt
    \NewDocumentCommand{\todoaddref}{O{} m}{%
        \todo[linecolor=blue,backgroundcolor=blue!25,bordercolor=blue,#1]{#2}%
    }
    \NewDocumentCommand{\todouncertain}{O{} m}{%
        \todo[linecolor=green,backgroundcolor=green!25,bordercolor=green,#1]{#2}%
    }
    \NewDocumentCommand{\todooptional}{O{} m}{%
        \todo[linecolor=cyan,backgroundcolor=cyan!25,bordercolor=cyan,#1]{#2}%
    }
    \pretocmd{\mainmatter}{\listoftodos[\todoname]{\markboth{\MakeUppercase{\todoname}}{\MakeUppercase{\todoname}}}}{}{}
}{
    \presetkeys{todonotes}{disable}{}	% disable \todo
    \NewDocumentCommand{\todoaddref}{O{} m}{}
    \NewDocumentCommand{\todouncertain}{O{} m}{}
    \NewDocumentCommand{\todooptional}{O{} m}{}
}

\usepackage[switch*,pagewise]{lineno}	% Zeilennummern
\ifbool{settingsWithLineNumbers}{%
    \renewcommand\linenumberfont{\textbf\sffamily\color{black!50}\footnotesize}
    \apptocmd{\mainmatter}{\linenumbers}{}{}
    \pretocmd{\backmatter}{\nolinenumbers}{}{}
}{}

\usepackage{placeins}					% FloatBarriers

\usepackage{enumitem}					% erweiterte Formatierung von \enumerate, \itemize und \description

\usepackage[linewidth=0.5pt]{mdframed}	% für Boxen in Hinweisen

\usepackage[ddmmyyyy]{datetime}			% Datumsangabe
\renewcommand{\dateseparator}{.}		% Punkt als Trennzeichen in Datumsangabe


% === Längen und Abstände =====================================================
% horizontales Layout
\setlength{\oddsidemargin}{0.2in}
\setlength{\evensidemargin}{0.0in}
\setlength{\textwidth}{\paperwidth - 2.2in}

% vertikales Layout
%\setlength{\topskip}{0.0cm}
\setlength{\headheight}{15.1pt}
%\setlength{\headsep}{0.0cm}
\setlength{\topmargin}{0.0cm}
\setlength{\footskip}{0.6in}
\setlength{\textheight}{\paperheight - 2.0in}
\addtolength{\textheight}{-1.0\headheight}
\addtolength{\textheight}{-1.0\headsep}
\addtolength{\textheight}{-1.0\footskip}

% Zeilenabstand
\setstretch{1.3}
\AtBeginEnvironment{tabular}{\setstretch{1.3}}

% Einrückung von Formeln
\setlength{\mathindent}{1.0cm}

% Absätze
\setlength{\parindent}{0.0cm}

% Fußnoten
\renewcommand{\footnotelayout}{\setstretch{1.2}}
\setlength\footnotemargin{10pt}

% Listen (noitemsep, nosep, ...)
\setlist{noitemsep}

\setlength{\topsep}{0.3cm}


% === Bild- und Tabellenunterschrift ==========================================
\renewcommand{\captionfont}{\small \setstretch{1.3}}
\newcommand{\NIcaption}[2]{\caption[#1]{#1\protect\\ \emph{#2}}}
% \setcaptionmargin{0.75cm}


% === Abkürzungsverzeichnis ===================================================
% Verwendung vor jedem Kapitel zurücksetzen
\pretocmd{\chapter}{\acresetall}{}{}


% === Seitenstil ==============================================================
% Pagestyle plain überschreiben
\pagestyle{fancy}
% Kapitel- und Abschnittangaben ohne Punkt
\renewcommand{\sectionmark}[1]{\markright{\uppercase{\thesection~~#1}}}
\renewcommand{\chaptermark}[1]{\markboth{\uppercase{\chaptername\ \thechapter~~#1}}{}}
\fancypagestyle{plain}{%
    \fancyhead[ER]{\itshape\leftmark}%
    \fancyhead[OL]{\itshape\rightmark}%
    \fancyhead[EL,OR]{\thepage}%
    \fancyfoot[EL,OL]{}%
    \fancyfoot[EC,OC]{}%
    \renewcommand{\headrulewidth}{0.4pt}%
    \renewcommand{\footrulewidth}{0.4pt}%
}


% === Verweise ================================================================
% Klammern in Formel-Referenzen entfernen
\makeatletter
\renewcommand\tagform@[1]{\maketag@@@{\ignorespaces#1\unskip\@@italiccorr}}
\makeatother


% === Angabe von Einheiten ====================================================
\ifdefstring{\settingsLanguage}{german}{%
    \sisetup{locale=DE}		% deutsche Lokalisierung (konvertiert 1.00 automatisch zu 1,00)
}%
{%
    \sisetup{locale=US}		% englische Lokalisierung (konvertiert 1,00 automatisch zu 1.00)
}%


% === Mathematische Definitionen ==============================================
% Darstellung von Vektoren und Matrizen
\renewcommand{\vec}[1]{\underline{\mathbf{\MakeLowercase{#1}}}}		% Vektoren
\newcommand{\veci}[1]{\underline{\MakeLowercase{#1}}}				% Vektoren als Indizes
\newcommand{\mat}[1]{\underline{\mathbf{\MakeUppercase{#1}}}}		% Matrizen
\newcommand{\mati}[1]{\underline{\MakeUppercase{#1}}}				% Matrizen als Indizes
% zusätzliche mathematische Operatoren (damit sie nicht als Formelzeichen geschrieben werden)
\DeclareMathOperator{\step}{step}									% Stufenfunktion
\DeclareMathOperator{\sign}{sign}									% Vorzeichen


% === Manuelle Definition von Worttrennungen ==================================
\hyphenation{
    Convolutional
    Neural
    Net-work
    Net-works
    Ko-ef-fi-zi-ent
    Ko-ef-fi-zi-ent-en
    Drop-out
    pixel-genaue
    Patch
    Patch-größen
    ein-ge-reicht
}


% === Pseudocode-Darstellung ==================================================
% Import nicht oben, weil \parindent zuvor gesetzt werden muss!
% siehe: https://ctan.org/pkg/algorithm2e?lang=de
\ifdefstring{\settingsLanguage}{german}{%
    \usepackage[%						% Pseudocode
        linesnumbered,					% mit Zeilennummern
        noend,							% Ende von Befehlen, wie etwa While, unterdrücken
        ruled,							% Layout
        german,							% deutsche Bezeichnung und deutsches Verzeichnis
        %onelanguage,					% Keywords übersetzen
        algochapter						% Nummerierung analog zu Abbildungen
    ]{algorithm2e}						% Pseudocode
}{%
    \usepackage[%						% Pseudocode
        linesnumbered,					% mit Zeilennummern
        noend,							% Ende von Befehlen, wie etwa While, unterdrücken
        ruled,							% Layout
        algochapter						% Nummerierung analog zu Abbildungen
    ]{algorithm2e}						% Pseudocode
}%

\newenvironment{NIalgorithm}{%
    % Algorithmus um 1.5em einrücken, damit Zeilennummern nicht im Rand sind
    \setlength{\algomargin}{1.5em}%
    % Padding oben und unten für Caption
    \setlength{\interspacetitleruled}{\smallskipamount}%
    % Padding oben und unten für Algorithmus
    \SetAlgoInsideSkip{smallskip}%
    % Kommentarstyle ändern
    \newcommand\NIcommentstyle[1]{\ttfamily\textcolor{black!60}{##1}}
    \SetCommentSty{NIcommentstyle}%
    % zweiten Teil in NIcaption unterdrücken (falls NIcaption genutzt wird)
    \renewcommand{\NIcaption}[2]{\caption[##1]{##1}}%
    \begin{algorithm}%
        % Zeilenabstand minimal vergrößern
        \setstretch{1.1}%
        % kleine Schrift
        \small%
        % Semikolon unterdrücken
        \DontPrintSemicolon%
        % korrekt ausgerichtete mehrzeilige Input- bzw. Outputdefinitionen mittels \Input und \Output
        \SetKwInOut{Input}{Input}%
        \SetKwInOut{Output}{Output}%
        % Abschnitt für In- und Outputs zurückrücken
        \pretocmd{\Input}{\Indentp{-1.5em}}{}{}%
        \apptocmd{\Output}{\Indentp{1.5em}}{}{}%
        }{%
    \end{algorithm}%
}