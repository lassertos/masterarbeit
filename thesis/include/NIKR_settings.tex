% === Art der Arbeit ==========================================================
% von den nachfolgenden Blöcken bitte den richtigen auswählen und die anderen auskommentieren/löschen

% --- Diplom ----------
%\newcommand{\settingsDegree}{{Diplom}}
%\newcommand{\settingsDegreeName}{{Diplominformatiker}}
%\newcommand{\settingsDegreeName}{{Diplomingenieur}}

% --- Bachelor ----------
% \newcommand{\settingsDegree}{{Bachelor}}
% \newcommand{\settingsDegreeName}{{Bachelor of Science}}

% --- Master ----------
\newcommand{\settingsDegree}{{Master}}
\newcommand{\settingsDegreeName}{{Master of Science}}

% === Name, Abgabedatum und Sprache der Arbeit ================================
\newcommand{\settingsName}{{Pierre Helbing}}
\newcommand{\settingsFinishDate}{{03.02.2025}}
\newcommand{\settingsLanguage}{german}   		% german / american

% === Weitere Einstellungen ===================================================
% --- Suchpfad (Unterverzeichnis) für eingebundene Grafiken ----------
\newcommand{\settingsGraphicsPath}{image/}

% --- Hinweiskapitel ----------
\newbool{settingsWithHints}
\setbool{settingsWithHints}{false}				% true / false

% --- Zeilennummern ----------
\newbool{settingsWithLineNumbers}
\setbool{settingsWithLineNumbers}{false}			% true / false

% --- Todos ----------
\newbool{settingsWithTodos}
\setbool{settingsWithTodos}{true}				% true / false

% --- Anzahl an Nummerierungsebenen im Text und Inhaltsverzeichnis ----------
% 1: \section
% 2: \section + \subsection
% Achtung: 3 oder 4 nur nach Absprache mit Betreuer !
% 3: \section + \subsection + \subsubsection
% 4: \section + \subsection + \subsubsection + \paragraph
\setcounter{secnumdepth}{2}
\setcounter{tocdepth}{2}

% --- Anforderungen ----------

\usepackage{tabularx}
\usepackage{multicol}
\newenvironment{myreq}[1]{%
    \setlist[description]{font=\normalfont\color{darkgray}}%
    \begin{tcolorbox}[colframe=black,colback=white, sharp corners, boxrule=1pt]%
        \bfseries\color{blue}%
        \begin{description}#1}%
            {\end{description}\end{tcolorbox}}

\newcommand{\threeinline}[3]{\begin{multicols}{3}#1 #2 #3\end{multicols}}
\newcommand{\twoinline}[2]{\begin{multicols}{2}#1 #2\end{multicols}}

\newcommand{\reqno}{\item[Requirement \#:]}
\newcommand{\reqtype}{\item[Requirement Type:]}
\newcommand{\reqevent}{\item[Event/BUC/PUC \#:]}
\newcommand{\reqdesc}{\item[Description:]}
\newcommand{\reqrat}{\item[Rationale:]}
\newcommand{\reqorig}{\item[Originator:]}
\newcommand{\reqfit}{\item[Fit Criterion:]}
\newcommand{\reqsatis}{\item[Customer Satisfaction:]}
\newcommand{\reqdissat}{\item[Customer Dissatisfaction:]}
\newcommand{\reqdep}{\item[Dependencies:]}
\newcommand{\reqconf}{\item[Conflicts:]}
\newcommand{\reqmater}{\item[Materials:]}
\newcommand{\reqhist}{\item[History:]}