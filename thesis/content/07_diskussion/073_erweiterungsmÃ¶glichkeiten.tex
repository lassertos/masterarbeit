\section{Erweiterungsmöglichkeiten}\label{section:diskussion:erweiterungsmöglichkeiten}
In diesem Abschnitt werden Erweiterungsmöglichkeiten für die entwickelte IDE vorgestellt. Dabei wird zunächst die Kollaboration über einen zentralen Server betrachtet. Danach wird die Sammlung und Analyse von Nutzerdaten erläutert. Daraufhin werden mögliche Anpassungen für die Kollaboration in Lehrszenarien beschrieben. Schließlich wird die Verwendung von WebAssembly \cite{noauthor_webassembly_nodate} zur Bereitstellung weiterer Funktionen innerhalb des Browsers der Nutzer betrachtet.

\paragraph{Zentrale Kollaboration}
In der prototypischen Implementierung wurde nur eine verteilte Kollaboration über Yjs betrachtet. Allerdings wäre auch die Kollaboration über einen zentralen Server denkbar. Diese könnte es Nutzern ermöglichen auch über Experimente hinweg miteinander zu kollaborieren. Angenommen Studierende müssen in Gruppen mehrere Aufgaben lösen. In der aktuellen Version der IDE können Nutzer nur dann auf die Projekte ihrer Teammitglieder zugreifen, wenn diese innerhalb eines Experiments mit ihnen geteilt werden. Ansonsten können Nutzer nur auf ihre eigenen Projekte zugreifen. Wenn die Kollaboration über einen zentralen Server erfolgt, kann dieser die Projekte der Nutzer speichern. Dadurch können alle Teammitglieder auf die geteilten Projekte zugreifen, selbst wenn der ursprüngliche Ersteller des Projekts nicht an dem aktuellen Experiment teilnimmt.

\paragraph{Sammlung und Analyse von Nutzerdaten}
Die prototypische Implementierung besitzt keine Funktionen für die Sammlung von Nutzerdaten. Diese könnten ggf. hinzugefügt werden, um eine spätere Analyse dieser zu ermöglichen. Dadurch könnten z.B. Lehrende einen besseren Einblick in die Arbeitsweise und Probleme der Lernenden erhalten. Somit könnten die Aufgabenstellungen entsprechend angepasst werden, um den Lernerfolg zu maximieren. Mögliche Daten, die hierfür erhoben werden könnten, sind z.B. die Anzahl der ausgeführten Kompilationen und Programmierungen oder die beim Programmieren bzw. Debuggen verbrachte Zeit.

\paragraph{Kollaboration in Lehrszenarien}
In Lehrszenarien gibt es neben der bereits erwähnten Kollaboration zwischen den Lernenden auch die Möglichkeit der Kollaboration zwischen Lernenden und Lehrenden. Angenommen die Lernenden haben ein Problem während der Bearbeitung einer Aufgabe. Dabei vermuten sie ein Problem auf der Seite der verwendeten Hardware. Ein Lehrender könnte in diesem Fall dem laufenden Experiment beitreten und die Fehlerursache suchen. Dafür könnte er u.a. vorübergehend die Programmierung der Steuereinheiten für die Lernenden deaktivieren und dann eine Musterlösung für die Programmierung verwenden. Sollte der Fehler weiterhin bestehen, kann der Lehrende z.B. über das Debuggen des Programms oder entsprechende Testfälle herausfinden, wo das Problem liegt und dieses dann beheben. Außerdem wäre auch ein Pair-Programming Modus denkbar, bei dem zwei Nutzer zusammen an einem Programm arbeiten, wobei ein Nutzer nur Lesezugriff besitzt, während der andere Nutzer den kompletten Zugriff auf die Dateien erhält. Dabei könnte auch ein Rollenwechsel erlaubt werden.

\paragraph{WebAssembly}
WebAssembly \cite{noauthor_webassembly_nodate} ist ein Binärformat, das von modernen Browsern unterstützt wird. Es kann als Ziel für die Kompilierung verschiedenster Programmiersprachen verwendet werden. Emscripten \cite{noauthor_emscripten_nodate} ist eine Toolchain, welche die Kompilierung von u.a. C und C++ Programmen nach WebAssembly ermöglicht. Dafür nutzt es LLVM \cite{noauthor_llvm_nodate}. Pyodide \cite{noauthor_pyodide_nodate} ist eine Python Distribution, die komplett im Browser ausgeführt werden kann und mithilfe von Emscripten entwickelt wurde. Weitere interessante Anwendungen von WebAssembly sind u.a. Ports des C/C++ Compilers Clang \cite{noauthor_clang_nodate}\cite{smith_binjiwasm-clang_2024} sowie des dazugehörigen Language Servers Clangd \cite{noauthor_clangd_nodate}\cite{yu_guyutongxueclangd--browser_2024} und v86 \cite{fabian_copyv86_2025}. Letzeres emuliert eine x86-kompatible CPU und kann verwendet werden, um z.B. virtuelle Maschinen innerhalb des Browsers eines Nutzers zu starten. Dies wird von Wokwi \cite{noauthor_wokwi_nodate}, einer Platform zur Simulation von Microcontrollern, verwendet, um den Debugger GDB innerhalb des Browsers der Nutzer bereitzustellen \cite{noauthor_running_2021}. Somit könnte in späteren Arbeiten untersucht werden, ob manche Funktionen der IDE mithilfe von WebAssembly im Browser des Nutzers ausgeführt werden können. Dadurch könnten die benötigten Serverressourcen verringert werden.