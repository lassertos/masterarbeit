\section{Offene Aufgaben}\label{section:diskussion:offene-aufgaben}

In diesem Abschnitt sollen offene Aufgaben vorgestellt werden. Dabei wird zunächst betrachtet, wie die Komplexität der konzipierten und implementierten Lösungen verringert werden kann. Anschließend werden Sicherheitsaspekte besprochen, die für eine finale Version der IDE beachtet werden sollten. Danach wird auf die Nutzung mehrerer Steuereinheiten innerhalb eines Experiments eingegangen. Schließlich wird die noch durchzuführende Evaluation der konzipierten und implementierten Lösungen erläutert.

\paragraph{Komplexität der Lösungen}
Die entwickelten Lösungen besitzen eine relativ hohe Komplexität in der Anwendung bzw. Implementierung. So müssen zunächst entsprechende Laborgeräte für die jeweiligen Compiler, Debugger, Language Server, etc. entwickelt werden. Diese müssen dann in einem Experiment korrekt konfiguriert werden. Außerdem muss die Experimentkonfiguration dann noch getestet werden um sicherzustellen, dass sie korrekt funktioniert. Als Beispiel soll die Anwendung in der Lehre betrachtet werden. Ein Lehrender hat ggf. nicht die Zeit sich in das komplette CrossLab System einzuarbeiten und neue Laborgeräte zu implementieren. Dementsprechend müssten grundlegende Laborgeräte implementiert und in Kategorien zusammengefasst werden um die Konfiguration von Experimenten für Lehrende zu vereinfachen. Dabei sollten entsprechende Werkzeuge bzw. Bibliotheken konzipiert und implementiert werden, welche bei der Implementierung von Laborgeräten und bei der Zusammenstellung von Experimenten zusätzlich helfen können. Ein Beispiel hierfür könnte ein Experimentkonfigurator sein, der die verschiedenen Laborgeräte samt ihrer angebotenen Services visualisieren kann und deren Verbindung zu einem Experiment ermöglicht.

\paragraph{Sicherheitsaspekte}
Während der prototypischen Implementierung war die Betrachtung von Sicherheitsaspekten keine Priorität. Dementsprechend sollten diese vor dem Einsatz der IDE sowie der entwickelten Laborgeräte überprüft werden um sicherzustellen, dass das System nicht ausgenutzt werden kann. Beispiele für zu betrachtende Sicherheitsprobleme sind die in \cite{wu_ceclipse_2011} erwähnten \quoted{Wrong file operations}, \quoted{Banned operation calling} und \quoted{Excessive resource consumption}.

\paragraph{Mehrere Steuereinheiten}
Während der prototypischen Implementierung wurden nur Experimente mit einer Steuereinheit betrachtet. Für Experimente mit mehreren Steuereinheiten müssen entsprechende Anpassungen vorgenommen werden. Darunter auch die Anpassung der bereitgestellten Benutzerinterfaces.

\paragraph{Evaluation der Lösungen}
Die Implementierung muss evaluiert werden. Dadurch kann festgestellt werden, ob die entwickelten Lösungen den gewünschten Effekt haben. Darunter u.a. der geringere Ressourcenverbrauch und die einfachere Konfiguration von Experimenten, die eine IDE enthalten sollen. Dabei bietet sich ein Probelauf innerhalb eines Praktikumsversuchs an, bei dem die Studierenden eine entsprechende Aufgabe mit der IDE lösen müssen. Hierbei kann auch evaluiert werden, ob das Benutzerinterface für die Studierenden angemessen ist. Auch ein Vergleich mit z.B. WIDE könnte interessant sein, um zu sehen, wie die neuen Funktionen der IDE von den Studierenden wahrgenommen werden.