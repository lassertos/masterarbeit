\section{Alternative Lösungsansätze}\label{section:diskussion:alternative-lösungsansätze}

In diesem Abschnitt werden alternative Lösungsansätze vorgestellt. Dabei wird zunächst eine cloud-instanziierbare Version der IDE betrachtet. Danach wird ein serverseitiges integriertes Dateisystem erläutert. Schließlich wird noch die Möglichkeit betrachtet, einen anderen Code Editor statt \ac{VSCode} zu verwenden.

\paragraph{Cloud-instanziierbare IDE}
Eine Implementierung der IDE als cloud-instanzi-ierbares Laborgerät könnte einige Vereinfachungen ermöglichen. So könnten neben der IDE auch entsprechende Compiler, Debugger und Language Server auf dem gleichen System bereitgestellt werden. Dadurch wird deren Nutzung vereinfacht und kann ggf. ohne die entsprechenden CrossLab-Services erfolgen. Allerdings benötigt die Ausführung der IDE auf einem Server entsprechende Ressourcen, was die Skalierbarkeit dieser Lösung beeinträchtigen könnte. Zudem könnte dadurch die eigenständige Ausführung der IDE nicht immer gewährleistet werden, da Nutzer ggf. auf freie Serverresourcen warten müssen. Zudem erlaubt die Anbindung von Compilern, Debuggern und Language Servern über die entsprechenden CrossLab-Services auch die einfache Rekonfiguration eines Experiments. Somit ist es u.a. möglich, gewisse Funktionen nur in manchen Experimenten anzubieten.

\paragraph{Serverseitiges integriertes Dateisystem}
In der prototypischen Implementierung wurde ein projektbasiertes Dateisystem implementiert, was die Indexed Database API für die persistente Speicherung der Daten verwendet. Dabei werden die Daten nur auf dem jeweiligen Gerät und dem entsprechenden Browser des Nutzers gespeichert. Somit kann kein direkter Zugriff auf diese von einem anderen Gerät oder Browser erfolgen. Die serverseitige Speicherung von Projekten könnte den geräteunabhängigen Zugriff auf diese ermöglichen. Allerdings werden hierfür entsprechende Serverresourcen benötigt. Außerdem muss sichergestellt werden, dass die Speicherung jederzeit möglich ist, damit die eigenständige Ausführbarkeit der IDE nicht gefährdet wird.

\paragraph{Alternative Code Editoren} Man könnte für die Implementierung der IDE auch einen anderen Code Editor als \ac{VSCode} verwenden. Dabei kann man u.a. auf IDE-Frameworks wie Eclipse Theia und OpenSumi zurückgreifen. Diese erlauben tiefergreifende Anpassungen, als durch die alleinige Nutzung der VSCode Extension API möglich ist. Allerdings entsteht durch die Nutzung der entsprechenden tiefergreifenden Schnittstellen eine Bindung an das jeweilige Framework. Dadurch wird ein späterer Wechsel auf ein anderes Framework bzw. einen anderen Code Editor erschwert. Eine weitere Möglichkeit ist die Einbindung reiner Code Editoren wie z.B. Ace, dem Monaco Editor oder eine Eigenimplementierung in ein eigenes Benutzerinterface. Dadurch hat man komplette Kontrolle über das Benutzerinterface und kann es auf die entsprechende Zielgruppe anpassen. Weiterhin können ggf. Probleme, wie z.B. das Neuladen der IDE beim Wechseln von Ordnern, umgangen werden, da man die komplette Kontrolle über die Implementierung besitzt. Allerdings ist mit dieser Lösung auch ein entsprechend höherer Implementierungsaufwand verbunden und ein späterer Wechsel des Code Editors ggf. nicht einfach möglich.