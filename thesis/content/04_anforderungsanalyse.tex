\chapter{Anforderungsanalyse}\label{section:anforderungsanalyse}

In diesem Kapitel werden die Anforderung an die zu entwickelnde IDE dargelegt. Diese wurden in Gesprächen mit Experten des GOLDi-Remotelab sowie Experten des CrossLab Projekts erhoben. Um einen Kontext für die Anforderungen herzustellen, wird in \autoref{section:anforderungsanalyse:beispielszenario} ein entsprechendes Beispielszenario eines Praktikumsversuchs angeführt. Daraufhin werden in \autoref{section:anforderungsanalyse:anforderungen} die erhobenen Anforderungen beschrieben.

\section{Beispielszenario: Praktikumsversuch}\label{section:anforderungsanalyse:beispielszenario}

In diesem Abschnitt wird ein Beispielszenario für den Einsatz der zu entwickelnden integrierten Entwicklungsumgebung beschrieben werden. Bei dem Beispielszenario handelt es sich um einen Praktikumsversuch, bei dem die Studierenden in Zweiergruppen mehrere verschiedene Aufgaben lösen müssen. Diese Aufgaben nutzen unterschiedliche Experimentkonfigurationen.

\paragraph{Vorbereitung}
Für den Praktikumsversuch müssen ggf. neue CrossLab-kompatible Geräte implementiert und angebunden werden. Dafür können entweder bereits vorhandene Services genutzt oder neue Services entwickelt werden. Sollten neue Services entwickelt werden ist es ggf. notwendig bereits vorhandene Geräte mit diesen zu erweitern. Weiterhin werden Testfälle ausgearbeitet, welche die automatische Überprüfung von abgegebenen Lösungen ermöglichen.

\paragraph{Durchführung}
Nachdem der Praktikumsversuch fertiggestellt wurde kann er den Studierenden vorgestellt werden. Dabei zeigt ein Dozent den Studierenden die Aufgaben und wie diese zu bearbeiten sind. Danach können die Studierenden mit der Bearbeitung der Aufgaben beginnen. Dafür starten sie die entsprechenden Experimente über ein ihnen bereitgestelltes Interface, z.B. über ein vorhandenes \ac{LMS}. Die Studierenden sollen die gestellten Aufgaben gemeinsam bearbeiten. Sollten während der Bearbeitung der Aufgaben Probleme auftreten können sich die Studierenden an einen entsprechenden Betreuer wenden. Die Studierenden können ihre erarbeitete Lösung jederzeit mit den hinterlegten Testfällen überprüfen. Sobald alle Testfälle erfolgreich beendet werden ist die entsprechende Aufgabe erfüllt und die Ergebnisse werden entsprechend gespeichert.

\paragraph{Nachbereitung}
Während der Ausführung der Experimente können Daten erfasst werden, die es Lehrenden ermöglicht nachzuvollziehen bei welchen Aufgaben die Studierenden Probleme hatten bzw. wofür sie die meiste Zeit aufwenden mussten. Dazu kann z.B. die Code-Historie der Studierenden gespeichert werden sowie die in einem Experiment verbrachte Zeit. Durch die Analyse der entsprechenden Daten können Lehrende Änderungen am Praktikumsversuch vornehmen um diesen zu verbessern.

\section{Anforderungen}\label{section:anforderungsanalyse:anforderungen}

\newarray\Requirements
\expandarrayelementfalse
\newcounter{requirement}
\newenvironment{requirement}[1]
{
\def\reqlabel{requirement:#1}
\refstepcounter{requirement}
\label{\reqlabel}
\Requirements(\value{requirement})={\autoref{requirement:#1} & #1}
\tabularx{\textwidth}{p{2.5cm} X}
\toprule
\multicolumn{1}{p{3.25cm}}{\textbf{Anforderung \arabic{requirement}}} & \multicolumn{1}{c}{\textbf{#1}} \\
\midrule
}{
\bottomrule
\endtabularx
}
\newcommand{\reqdescription}{Beschreibung &}
\newcommand{\reqrationale}{Begründung &}
\def\requirementautorefname{Anforderung}
\def\replspaces#1#2{\expandafter\replspacesA\expandafter#2#1 \end}
\def\replspacesA#1#2 #3{#2\ifx\end#3\else#1\afterfi{\replspacesA#1#3}\fi}
\def\afterfi#1#2\fi{\fi#1}
\def\replace#1#2#3{%
    \def\tmp##1#2{##1#3\tmp}%
    \tmp#1\stopreplace#2\stopreplace}
\def\stopreplace#1\stopreplace{}

Im Folgenden werden die Anforderungen an die zu entwickelnde IDE beschrieben. Dabei wird für jede der Anforderungen eine entsprechende Beschreibung sowie Begründung angegeben. Eine Übersicht aller Anforderungen ist in \autoref{table:anforderungen} gegeben.

\begin{requirement}{CrossLab-Kompatibilität}
    \reqdescription Die zu entwickelnde IDE soll CrossLab-Services anbieten und konsumieren können. \\
    \reqrationale Durch die Verwendung von CrossLab-Services für die verschiedenen Funktionen der IDE kann diese in einem Experiment mit anderen Laborgeräten verbunden werden. Dadurch können Funktionen der IDE auf weitere Laborgeräte ausgelagert werden, wodurch die Erweiterbarkeit der IDE verbessert werden kann. \\
\end{requirement}

\begin{requirement}{Erweiterbarkeit}
    \reqdescription Die zu entwickelnde IDE soll Schnittstellen zum Hinzufügen von CrossLab-Services und Benutzerinterfaces besitzen. \\
    \reqrationale Um die Weiterentwicklung der IDE zu vereinfachen sollen entsprechende Schnittstellen zur Verfügung stehen. Dabei sollte mindestens das Hinzufügen neuer CrossLab-Services und Benutzerinterfaces möglich sein. \\
\end{requirement}

\begin{requirement}{Kostenlos nutzbar}
    \reqdescription Die zu entwickelnde IDE soll kostenlos nutzbar sein. Daher sollten auch die Kosten für die Implementierung und den Betrieb möglichst gering sein. \\
    \reqrationale Um die IDE in einer Vielzahl von verschiedenen Szenarien einsetzen zu können ist es vom Vorteil keine assoziierten Kosten für die Nutzung dieser zu haben. Somit kann sie z.B. auch im GOLDi-Remotelab und anderen CrossLab-kompatiblen online Laboren eingesetzt werden. \\
\end{requirement}

\begin{requirement}{Komplett im Browser nutzbar}
    \reqdescription Die zu entwickelnde IDE soll komplett im Browser nutzbar sein. \\
    \reqrationale Durch die Nutzbarkeit der kompletten IDE direkt im Browser des Nutzers wird die Verwendung dieser vereinfacht, da Nutzer keine weitere Software installieren müssen. \\
\end{requirement}

\begin{requirement}{Nur CrossLab-Nutzerkonto nötig}
    \reqdescription Die zu entwickelnde IDE soll nur ein CrossLab-Nutzerkonto zur Verwendung benötigen. \\
    \reqrationale Durch die Notwendigkeit eines zweiten Nutzerkontos könnte die IDE für gewisse Nutzergruppen uninteressant werden. Somit soll nur ein CrossLab-Nutzerkonto für die Nutzung der IDE vorausgesetzt werden. \\
\end{requirement}

\begin{requirement}{Standalone nutzbar}
    \reqdescription Die zu entwickelnde IDE soll standalone nutzbar sein. Das bedeutet, dass sie als einziges Laborgerät in einem Experiment verwendet werden kann. Dabei kann es zu Einschränkungen der angebotenen Funktionen kommen. Die Editierung von Quellcode soll in allen Fällen gewährleistet werden. \\
    \reqrationale Da Experimente in der CrossLab-Architektur meist aus mehreren verbundenen Laborgeräten bestehen kann es dazu kommen, dass manche dieser ggf. nicht immer verfügbar sind, da sie aktuell von anderen Nutzern verwendet werden. Wenn die IDE standalone nutzbar ist können Nutzer dennoch an ihren Programmen weiterarbeiten. \\
\end{requirement}

\begin{requirement}{Kollaboration}
    \reqdescription Die zu entwickelnde IDE soll Echtzeit-Kollaboration durch die Synchronisation von geteilten Daten und den Austausch von Zustandsinformationen ermöglichen. \\
    \reqrationale Durch die Ermöglichung der Synchronisation von Daten und dem Austausch von Zustandsinformationen zwischen Nutzern innerhalb eines Experiments kann die Zusammenarbeit dieser gefördert werden. \\
\end{requirement}

\begin{requirement}{Kollaboration: CrossLab-Kompatibilität}
    \reqdescription Um die Echtzeit-Kollaboration von Laborgeräten innerhalb eines Experiments zu ermöglichen, sollen entsprechende CrossLab-Services entwickelt werden. Diese sollen die Synchronisation von geteilten Daten sowie den Austausch von Zustandsinformationen erlauben. Die zu entwickelnde IDE soll diese CrossLab-Services für die Echtzeit-Kollaboration mit anderen Laborgeräten nutzen. \\
    \reqrationale Die Bereitstellung von CrossLab-Services für Echtzeit-Kollaboration vereinfacht die Entwicklung neuer kollaborativer Laborgeräte. \\
\end{requirement}

\begin{requirement}{Dateisystem}
    \reqdescription Die zu entwickelnde IDE soll ein integriertes Dateisystem mit einem entsprechenden Benutzerinterface besitzen, dass die Erstellung, Bearbeitung, Verschiebung, Löschung und persistente Speicherung von Dateien und Ordnern ermöglicht. \\
    \reqrationale Ein in der IDE integriertes Dateisystem vereinfacht die Nutzung der IDE, da kein externes Dateisystem benötigt wird. \\
\end{requirement}

\begin{requirement}{Dateisystem: CrossLab-Kompatibilität}
    \reqdescription Für die Bereitstellung und Nutzung von Dateisystemen sollen entsprechende CrossLab-Services konzipiert werden. Die zu entwickelnde IDE soll diese CrossLab-Services für die Anbindung weiterer Dateisysteme verwenden. \\
    \reqrationale Die Möglichkeit weitere Dateisysteme über CrossLab-Services hinzuzufügen kann z.B. den Zugriff auf lokale Dateien des Nutzers oder eine geräteunabhängige Speicherung von Dateien auf einem Server ermöglichen. \\
\end{requirement}

\begin{requirement}{Dateisystem: Kollaboration}
    \reqdescription Die zu entwickelnde IDE soll das Teilen von Ordnern und den darin enthaltenen Dateien zwischen Nutzern innerhalb eines Experiments ermöglichen. Änderungen innerhalb geteilter Ordner sollen zwischen allen teilnehmenden Nutzern synchronisiert werden. Geteilte Ordner können nur von ihrem Besitzer gelöscht, verschoben oder umbenannt werden. Das Teilen von Ordnern soll auch beendet werden können. \\
    \reqrationale Durch das Teilen von Ordnern und den enthaltenen Dateien können Nutzer gemeinsam an diesen arbeiten. Dadurch können z.B. Gruppenarbeiten im Rahmen eines Praktikumsversuch effizienter durchgeführt werden, während die Lernenden gleichzeitig ihre Teamfähigkeit verbessern können. \\
\end{requirement}

\begin{requirement}{Kompilierung}
    \reqdescription Die zu entwickelnde IDE soll die Kompilierung von Quellcode unterstützen. Dabei sollen entsprechende Bedienelemente für die Kompilierung bereitgestellt werden.  \\
    \reqrationale Die Kompilierung des Quellcodes ist in vielen Programmiersprachen ein wichtiger Schritt um das Programm auf dem Zielsystem ausführen zu können. Dementsprechend sollte die IDE bei Vorhandensein eines Compilers entsprechende Bedienelemente zur Kompilierung des Quellcodes bereitstellen. \\
\end{requirement}

\begin{requirement}{Kompilierung: CrossLab-Kompatibilität}
    \reqdescription Für die Bereitstellung und Nutzung von Compilern sollen entsprechende CrossLab-Services konzipiert werden. Die zu entwickelnde IDE soll für die Anbindung von Compilern diese CrossLab-Services verwenden. \\
    \reqrationale Die Möglichkeit weitere Compiler über CrossLab-Services an die IDE anschließen zu können erlaubt die Unterstützung vieler verschiedener Steuereinheiten innerhalb eines Experiments. \\
\end{requirement}

\begin{requirement}{Debuggen}
    \reqdescription Die zu entwickelnde IDE soll das Debuggen von Programmen der Nutzer ermöglichen. Dabei sollen entsprechende Bedienelemente für das Debuggen bereitgestellt werden. Beim Start des Debuggens soll die neueste Version des aktuellen Programms auf die entsprechende Steuereinheit geladen werden. \\
    \reqrationale Nutzer können durch das Debuggen ihrer Programme schneller Fehler in diesen finden und beheben. Weiterhin erlaubt das Debuggen eines Programms einen besseren Einblick in dessen Laufzeitverhalten. \\
\end{requirement}

\begin{requirement}{Debuggen: CrossLab-Kompatibilität}
    \reqdescription Für die Bereitstellung und Nutzung von Debuggern sollen entsprechende CrossLab-Services konzipiert werden. Die zu entwickelnde IDE soll für die Anbindung von Debuggern diese CrossLab-Services verwenden. Weiterhin sollen auch CrossLab-Services für die Kommunikation zwischen einem Debugger und einer zu debuggenden Steuereinheit konzipiert werden. \\
    \reqrationale Die Möglichkeit weitere Debugger über entsprechende CrossLab-Services mit der IDE und Steuereinheiten verbinden zu können erlaubt eine größere Anzahl an verschiedenen Experimenten. \\
\end{requirement}

\begin{requirement}{Debuggen: Kollaboration}
    \reqdescription Die zu entwickelnde IDE soll es Nutzern ermöglichen gleichzeitig an einer Debug-Sitzung teilzunehmen, falls dies mit dem verwendeten Debugger möglich ist. Dabei müssen beide Nutzer Zugriff auf die gleichen Dateien besitzen. Weiterhin sollen die Breakpoints aller an der Debug-Sitzung teilnehmenden Nutzer synchronisiert werden. Pro Steuereinheit soll nur eine Debug-Sitzung gestartet werden können. Nur der Ersteller der Debug-Sitzung kann diese beenden. \\
    \reqrationale Durch das kollaborative Debuggen können Nutzer gemeinsam einen Einblick in das Laufzeitverhalten des Programs erlangen. Dadurch kann auch die Suche nach Fehlern sowie deren Behebung effizienter erfolgen. \\
\end{requirement}

\begin{requirement}{Language Server}
    \reqdescription Die zu entwickelnde IDE soll die Anbindung von Language Servern unterstützen. \\
    \reqrationale Durch die Anbindung von Language Servern können Editorfunktionen wie Code-Vervollständigung, Code-Navigation und Refactoring ermöglicht werden. Diese können die Benutzererfahrung verbessern. \\
\end{requirement}

\begin{requirement}{Language Server: CrossLab-Kompatibilität}
    \reqdescription Für die Bereitstellung und Nutzung von Language Servern sollen entsprechende CrossLab-Services konzipiert werden. Die zu entwickelnde IDE soll für die Anbindung von Language Servern diese CrossLab-Services verwenden. \\
    \reqrationale Durch die Anbindbarkeit von Language Servern über CrossLab-Services können diese von anderen Laborgeräten bereitgestellt und in Experimenten von der IDE genutzt werden. \\
\end{requirement}

\begin{requirement}{Hochladen von Programmen auf Steuereinheiten}
    \reqdescription Die zu entwickelnde IDE soll das Hochladen von Programmen auf Steuereinheiten unterstützen und entsprechende Bedienelemente dafür besitzen. Die Kompilierung und das Hochladen von Programmen kann in einen Schritt zusammengefasst werden um den Ablauf effizienter zu gestalten. \\
    \reqrationale Nutzer können innerhalb eines Experiments Programme für Steuereinheiten schreiben. Zur Ausführung müssen die Programme auf die entsprechende Steuereinheit hochgeladen werden.  \\
\end{requirement}

\begin{requirement}{Testen}
    \reqdescription Die zu entwickelnde IDE soll es ermöglichen Testfälle für ein Experiment zu konfigurieren und die Ausführung dieser während des Experiments unterstützen. Die Testfälle sollen über ein entsprechendes Benutzerinterface eingesehen und ausgeführt werden können. Testfälle sollen die Interaktionen zwischen verschiedenen Laborgeräten innerhalb eines Experiments überprüfen können. \\
    \reqrationale Die Möglichkeit Testfälle für ein Experiment zu konfigurieren erlaubt es z.B. Lehrenden die Ziele ihrer Lehrveranstaltung im Vorhinein festzulegen. Während des Experiments können die Lernenden dann ihre erstellte Lösung überprüfen. \\
\end{requirement}

\begin{requirement}{Testen: CrossLab-Kompatibilität}
    \reqdescription Für die Konfiguration und Ausführung von Testfällen sollen entsprechende CrossLab-Services konzipiert werden. Dabei müssen Laborgeräte die Möglichkeit besitzen, Funktionen bereitzustellen, die in den Testfällen verwendet werden können. Die zu entwickelnde IDE soll für die Konfiguration und Ausführung von Testfällen die konzipierten CrossLab-Services verwenden. \\
    \reqrationale Um die Konfiguration und Ausführung von Testfällen zu ermöglichen benötigt die zu entwickelnde IDE ggf. Zugriff auf andere Laborgeräte. Dies kann seitens der Laborgeräte durch die Bereitstellung von Funktionen zur Nutzung in Testfällen ermöglicht werden. Durch die Konzeption entsprechender CrossLab-Service kann die Bereitstellung der Funktionen sowie die Konfiguration und Ausführung der Testfälle innerhalb eines Experiments unterstützt werden. \\
\end{requirement}

\begin{table}[t]
    \centering
    \begin{tabular}{l l}
        \toprule
        \Requirements(1)  \\
        \Requirements(2)  \\
        \Requirements(3)  \\
        \Requirements(4)  \\
        \Requirements(5)  \\
        \Requirements(6)  \\
        \Requirements(7)  \\
        \Requirements(8)  \\
        \Requirements(9)  \\
        \Requirements(10) \\
        \Requirements(11) \\
        \Requirements(12) \\
        \Requirements(13) \\
        \Requirements(14) \\
        \Requirements(15) \\
        \Requirements(16) \\
        \Requirements(17) \\
        \Requirements(18) \\
        \Requirements(19) \\
        \Requirements(20) \\
        \Requirements(21) \\
        \bottomrule
    \end{tabular}
    \caption{Übersicht der Anforderungen}
    \label{table:anforderungen}
\end{table}
