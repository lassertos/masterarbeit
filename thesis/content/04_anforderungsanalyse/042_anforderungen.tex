\section{Anforderungen}\label{section:anforderungsanalyse:anforderungen}

\newarray\Requirements
\expandarrayelementfalse
\newcounter{requirement}
\newenvironment{requirement}[1]
{
\def\reqlabel{requirement:#1}
\refstepcounter{requirement}
\label{\reqlabel}
\Requirements(\value{requirement})={\autoref{requirement:#1} & #1}
\tabularx{\textwidth}{p{4cm} X}
\toprule
\multicolumn{1}{p{3.25cm}}{\textbf{Anforderung \arabic{requirement}}} & \multicolumn{1}{c}{\textbf{#1}} \\
\midrule
}{
\bottomrule
\endtabularx
}
\newcommand{\reqdescription}{Beschreibung &}
\newcommand{\reqrationale}{Begründung &}
\def\requirementautorefname{Anforderung}
\def\replspaces#1#2{\expandafter\replspacesA\expandafter#2#1 \end}
\def\replspacesA#1#2 #3{#2\ifx\end#3\else#1\afterfi{\replspacesA#1#3}\fi}
\def\afterfi#1#2\fi{\fi#1}
\def\replace#1#2#3{%
    \def\tmp##1#2{##1#3\tmp}%
    \tmp#1\stopreplace#2\stopreplace}
\def\stopreplace#1\stopreplace{}

Im Folgenden werden die Anforderungen an die zu entwickelnde IDE beschrieben. Dabei wird für jede der Anforderungen eine entsprechende Beschreibung sowie Begründung angegeben. Eine Übersicht aller Anforderungen ist in \autoref{table:anforderungen} gegeben.

\begin{requirement}{Ausführung im Browser}
    \reqdescription Die zu entwickelnde IDE soll über den Browser des Nutzers ausgeführt werden können. \\
    \reqrationale Durch die Nutzbarkeit der IDE im Browser des Nutzers wird die Verwendung dieser vereinfacht, da Nutzer keine weitere Software installieren müssen. \\
\end{requirement}

\begin{requirement}{Erweiterbarkeit}
    \reqdescription Die zu entwickelnde IDE soll Schnittstellen zum Hinzufügen von CrossLab-Services und Benutzerinterfaces besitzen. \\
    \reqrationale Um die Weiterentwicklung der IDE zu vereinfachen sollen entsprechende Schnittstellen zur Verfügung stehen. Dabei sollte mindestens das Hinzufügen neuer CrossLab-Services und Benutzerinterfaces möglich sein. \\
\end{requirement}

\begin{requirement}{Keine Kosten für Nutzer}
    \reqdescription Die zu entwickelnde IDE soll keine Kosten für den Nutzer erzeugen. Daher soll für die Implementierung nur quelloffene und frei nutzbare Software verwendet werden.  \\
    \reqrationale Um die IDE in einer Vielzahl von verschiedenen Szenarien einsetzen zu können ist es vom Vorteil keine assoziierten Kosten für die Nutzung dieser zu haben. Somit kann sie z.B. auch im GOLDi-Remotelab und anderen CrossLab-kompatiblen online Laboren eingesetzt werden. \\
\end{requirement}

\begin{requirement}{Eigenständig nutzbar}
    \reqdescription Die zu entwickelnde IDE soll als einziges Laborgerät in einem Experiment verwendet werden können. Dabei können die angebotenen Funktionen der IDE eingeschränkt sein. Das Editieren von Quellcode soll stets ermöglicht werden. \\
    \reqrationale Da Experimente in der CrossLab-Architektur meist aus mehreren verbundenen Laborgeräten bestehen kann es dazu kommen, dass manche dieser ggf. nicht immer verfügbar sind, da sie aktuell von anderen Nutzern verwendet werden. Wenn die IDE eigenständig nutzbar ist können Nutzer dennoch an ihren Programmen weiterarbeiten. \\
\end{requirement}

\begin{requirement}{Nur CrossLab-Nutzerkonto nötig}
    \reqdescription Die zu entwickelnde IDE soll nur das bereits zur Ausführung von CrossLab Experimenten benötigte Nutzerkonto voraussetzen. \\
    \reqrationale Durch die Notwendigkeit eines zweiten Nutzerkontos könnte die IDE für gewisse Nutzergruppen uninteressant werden. Somit soll nur ein CrossLab-Nutzerkonto für die Nutzung der IDE vorausgesetzt werden. \\
\end{requirement}

\begin{requirement}{Integriertes Dateisystem}
    \reqdescription Die zu entwickelnde IDE soll ein integriertes Dateisystem besitzen. \\
    Unteranforderung (a) & Das integrierte Dateisystem soll die Erstellung, Bearbeitung, Verschiebung, Löschung und persistente Speicherung von Dateien und Ordnern unterstützen. \\
    Unteranforderung (b) & Das integrierte Dateisystem der IDE soll bei der eigenständigen Ausführung unterstützt werden. (sh. \autoref{requirement:Eigenständig nutzbar}) \\
    \reqrationale Ein in der IDE integriertes Dateisystem vereinfacht die Nutzung der IDE, da kein externes Dateisystem benötigt wird. Durch die Unterstützung des integrierten Dateisystems bei der eigenständigen Ausführung der IDE haben Nutzer zudem weiterhin Zugriff auf ihre Dateien. \\
\end{requirement}

\begin{requirement}{Weitere Dateisysteme}
    \reqdescription Die zu entwickelnde IDE soll die Anbindung weiterer Dateisysteme unterstützen. \\
    Unteranforderung (a) & Es soll ein CrossLab-Services für die Bereitstellung und Nutzung von Dateisystemen entwickelt werden. \\
    Unteranforderung (b) & Die zu entwickelnde IDE soll den aus Unteranforderung (a) resultierenden CrossLab-Service zur Anbindung weiterer Dateisysteme nutzen. \\
\end{requirement}

\begin{requirement}{Kollaboration}
    \reqdescription Die zu entwickelnde IDE soll Echtzeit-Kollaboration durch die Synchronisation von geteilten Daten und den Austausch von Zustandsinformationen ermöglichen. \\
    Unteranforderung (a) & Es soll ein Crosslab-Service für die Synchronisation von geteilten Daten und den Austausch von Zustandsinformationen entwickelt werden. \\
    Unteranforderung (b) & Die zu entwickelnde IDE soll den aus Unteranforderung (a) resultierenden CrossLab-Service zur Bereitstellung der Echtzeit-Kollaboration verwenden. \\
    \reqrationale Durch die Ermöglichung der Synchronisation von Daten und dem Austausch von Zustandsinformationen zwischen Nutzern innerhalb eines Experiments kann die Zusammenarbeit dieser gefördert werden. \\
\end{requirement}

\vfill

\begin{requirement}{Teilen von Ordnern}
    \reqdescription Die zu entwickelnde IDE soll das Teilen von Ordnern und den darin enthaltenen Dateien zwischen Nutzern innerhalb eines Experiments ermöglichen. \\
    Unteranforderung (a) & Änderungen innerhalb geteilter Ordner sollen zwischen allen teilnehmenden Nutzern synchronisiert werden. \\
    Unteranforderung (b) & Geteilte Ordner können nur von ihrem Besitzer gelöscht, verschoben oder umbenannt werden. \\
    Unteranforderung (c) & Das Teilen von Ordnern soll von dem Besitzer beendet werden können. \\
    Unteranforderung (d) & Das Teilen von Ordnern soll über die nach \autoref{requirement:Kollaboration}(b) vorhandenen CrossLab-Services der IDE erfolgen. \\
    \reqrationale Durch das Teilen von Ordnern und den enthaltenen Dateien können Nutzer gemeinsam an diesen arbeiten. Dadurch können z.B. Gruppenarbeiten im Rahmen eines Praktikumsversuch effizienter durchgeführt werden, während die Lernenden gleichzeitig ihre Teamfähigkeit verbessern können. \\
\end{requirement}

\newpage

\mbox{}\vfill

\begin{requirement}{Kompilierung}
    \reqdescription Die zu entwickelnde IDE soll die Kompilierung von Quellcode unterstützen. \\
    Unteranforderung (a) & Es soll ein CrossLab-Service für die Bereitstellung und Nutzung von Compilern entwickelt werden. \\
    Unteranforderung (b) & Die zu entwickelnde IDE soll den aus Unteranforderung (a) resultierenden CrossLab-Service für die Nutzung von Compilern verwenden. \\
    \reqrationale Die Kompilierung des Quellcodes ist in vielen Programmiersprachen ein wichtiger Schritt um das Programm auf dem Zielsystem ausführen zu können. \\
\end{requirement}

\vfill

\begin{requirement}{Programmierung von Steuereinheiten}
    \reqdescription Die zu entwickelnde IDE soll die Programmierung Steuereinheiten unterstützen. \\
    Unteranforderung (a) & Es soll ein CrossLab-Service für Programmierung von Steuereinheiten entwickelt werden. \\
    Unteranforderung (b) & Die zu entwickelnde IDE soll den aus Unteranforderung (a) resultierenden CrossLab-Service für die Programmierung von Steuereinheiten verwenden. \\
    \reqrationale Nutzer können innerhalb eines Experiments Programme für Steuereinheiten schreiben. Zur Ausführung müssen die Programme auf die entsprechende Steuereinheit hochgeladen und für deren Programmierung verwendet werden. \\
\end{requirement}

\vfill\mbox{}

\newpage

\begin{requirement}{Debuggen}
    \reqdescription Die zu entwickelnde IDE soll das Debuggen von Programmen der Nutzer ermöglichen. \\
    Unteranforderung (a) & Es soll ein CrossLab-Service für die Bereitstellung und Nutzung von Debuggern entwickelt werden. \\
    Unteranforderung (b) & Es soll ein CrossLab-Service für die Kommunikation zwischen Debuggern und Steuereinheit entwickelt werden. \\
    Unteranforderung (c) & Die zu entwickelnde IDE soll den aus Unteranforderung (a) resultierenden CrossLab-Service für die Nutzung von Debuggern verwenden. \\
    \reqrationale Nutzer können durch das Debuggen ihrer Programme schneller Fehler finden und beheben. Zudem können sie das Laufzeitverhalten ihrer Programme besser analysieren. \\
\end{requirement}

\vfill

\begin{requirement}{Teilen von Debug-Sitzungen}
    \reqdescription Die zu entwickelnde IDE soll es Nutzern ermöglichen gleichzeitig an einer Debug-Sitzung teilzunehmen, falls dies mit dem verwendeten Debugger möglich ist. \\
    Unteranforderung (a) & Nutzer können einer Debug-Sitzung nur beitreten wenn sie Zugriff auf die Dateien des entsprechenden Programms besitzen. (sh. \autoref{requirement:Teilen von Ordnern}) \\
    Unteranforderung (b) & Die Breakpoints aller an einer Debug-Sitzung teilnehmenden Nutzer sollen synchronisiert werden. \\
    Unteranforderung (c) & Pro Steuereinheit soll nur eine Debug-Sitzung gestartet werden können. \\
    Unteranforderung (d) & Nur der Ersteller einer Debug-Sitzung kann diese beenden. \\
    Unteranforderung (e) & Das Teilen von Debug-Sitzungen soll über den nach \autoref{requirement:Kollaboration}(b) vorhandenen CrossLab-Service der IDE erfolgen. \\
    \reqrationale Durch das kollaborative Debuggen können Nutzer gemeinsam einen Einblick in das Laufzeitverhalten des Programs erlangen. Dadurch kann auch die Suche nach Fehlern sowie deren Behebung effizienter erfolgen. \\
\end{requirement}

\newpage

\begin{requirement}{Testen}
    \reqdescription Es soll ein CrossLab-Service für die Erstellung und Ausführung von Testfällen innerhalb eines Experiments entwickelt werden. \\
    Unteranforderung (a) & Der Producer des CrossLab-Service soll die Bereitstellung von Funktionen zur Verwendung in Testfällen ermöglichen. \\
    Unteranforderung (b) & Der Consumer des CrossLab-Service soll die Ausführung der bereitgestellten Funktionen ermöglichen. \\
    Unteranforderung (c) & Die Erstellung von Testfällen soll während der Konfiguration eines Experiments erfolgen können. \\
    Unteranforderung (d) & Die zu entwickelnde IDE soll den aus Unteranforderung (a) resultierenden CrossLab-Service zur Ausführung von Testfällen innerhalb eines Experiments verwenden. \\
    \reqrationale Die Möglichkeit Testfälle für ein Experiment zu konfigurieren erlaubt es z.B. Lehrenden die Ziele ihrer Lehrveranstaltung im Vorhinein festzulegen. Während des Experiments können die Lernenden dann ihre erstellte Lösung überprüfen. \\
\end{requirement}

\vfill

\begin{requirement}{Language Server}
    \reqdescription Die zu entwickelnde IDE soll die Anbindung von Language Servern unterstützen. \\
    Unteranforderung (a) & Es soll ein CrossLab-Service für die Bereitstellung und Nutzung von Language Servern entwickelt werden. \\
    Unteranforderung (b) & Die zu entwickelnde IDE soll den aus Unteranforderung (a) resultierenden CrossLab-Service für die Nutzung von Language Servern verwenden. \\
    \reqrationale Durch die Anbindung von Language Servern können Editorfunktionen wie Code-Vervollständigung, Code-Navigation und Refactoring ermöglicht werden. Diese können die Benutzererfahrung verbessern. \\
\end{requirement}

\newpage

\begin{table}[t]
    \centering
    \begin{tabular}{l l}
        \toprule
        \Requirements(1)  \\
        \Requirements(2)  \\
        \Requirements(3)  \\
        \Requirements(4)  \\
        \Requirements(5)  \\
        \Requirements(6)  \\
        \Requirements(7)  \\
        \Requirements(8)  \\
        \Requirements(9)  \\
        \Requirements(10) \\
        \Requirements(11) \\
        \Requirements(12) \\
        \Requirements(13) \\
        \Requirements(14) \\
        \Requirements(15) \\
        \bottomrule
    \end{tabular}
    \caption{Übersicht der Anforderungen}
    \label{table:anforderungen}
\end{table}