\section{Beispielszenario: Praktikumsversuch}\label{section:anforderungsanalyse:beispielszenario}

In diesem Abschnitt wird ein Beispielszenario für den Einsatz der zu entwickelnden integrierten Entwicklungsumgebung beschrieben werden. Bei dem Beispielszenario handelt es sich um einen Praktikumsversuch, bei dem die Studierenden in Zweiergruppen mehrere verschiedene Aufgaben lösen müssen. Diese Aufgaben nutzen unterschiedliche Experimentkonfigurationen.

\paragraph{Vorbereitung}
Für den Praktikumsversuch müssen ggf. neue CrossLab-kompatible Geräte implementiert und angebunden werden. Dafür können entweder bereits vorhandene Services genutzt oder neue Services entwickelt werden. Sollten neue Services entwickelt werden ist es ggf. notwendig bereits vorhandene Geräte mit diesen zu erweitern. Weiterhin werden Testfälle ausgearbeitet, welche die automatische Überprüfung von abgegebenen Lösungen ermöglichen.

\paragraph{Durchführung}
Nachdem der Praktikumsversuch fertiggestellt wurde kann er den Studierenden vorgestellt werden. Dabei zeigt ein Dozent den Studierenden die Aufgaben und wie diese zu bearbeiten sind. Danach können die Studierenden mit der Bearbeitung der Aufgaben beginnen. Dafür starten sie die entsprechenden Experimente über ein ihnen bereitgestelltes Interface, z.B. über ein vorhandenes \ac{LMS}. Die Studierenden sollen die gestellten Aufgaben gemeinsam bearbeiten. Sollten während der Bearbeitung der Aufgaben Probleme auftreten können sich die Studierenden an einen entsprechenden Betreuer wenden. Die Studierenden können ihre erarbeitete Lösung jederzeit mit den hinterlegten Testfällen überprüfen. Sobald alle Testfälle erfolgreich beendet werden ist die entsprechende Aufgabe erfüllt und die Ergebnisse werden entsprechend gespeichert.

\paragraph{Nachbereitung}
Während der Ausführung der Experimente können Daten erfasst werden, die es Lehrenden ermöglicht nachzuvollziehen bei welchen Aufgaben die Studierenden Probleme hatten bzw. wofür sie die meiste Zeit aufwenden mussten. Dazu kann z.B. die Code-Historie der Studierenden gespeichert werden sowie die in einem Experiment verbrachte Zeit. Durch die Analyse der entsprechenden Daten können Lehrende Änderungen am Praktikumsversuch vornehmen um diesen zu verbessern.
