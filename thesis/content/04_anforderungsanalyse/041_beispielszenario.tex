\section{Beispielszenario: Praktikumsversuch}\label{section:anforderungsanalyse:beispielszenario}

In diesem Abschnitt wird ein Beispielszenario für den Einsatz der zu entwickelnden IDE gegeben. Bei diesem Beispielszenario handelt es sich um einem Praktikumsversuch, bei dem die Studierenden in Zweiergruppen einen Steuerungsalgorithmus für einen 3-Achsen-Portalkran schreiben müssen. Dafür erhalten sie einen Microcontroller, der über ein entsprechendes CrossLab-Interface in Experimenten verwendet werden kann. Die Experimente können über das für den Praktikumsversuch verwendete \ac{LMS} gestartet werden.

Die Studierenden können die IDE verwenden um ihr Programm zu schreiben. Dabei können sie zusammen an dem gleichen Programm arbeiten. Zusätzlich zum Editieren des Quellcodes werden ihnen Funktionen wie z.B. Code-Vervollständigung, Code-Navigation und Refactoring zur Verfügung gestellt. Weiterhin wird es ihnen ermöglicht Dateien und Ordner zur Strukturierung ihres Programms zu erstellen. Außerdem können sie ihr Programm kompilieren lassen. Dabei erhalten sie die Ausgaben des Compilers. Falls die Kompilierung erfolgreich ist kann das Programm auf den Microcontroller geladen werden. Zur Überprüfung ihrer Lösung können die Studierenden vorkonfigurierte Testfälle ausführen. Sollten diese alle erfolgreich sein, so haben sie die Aufgabe erfolgreich gelöst. Sollte das Programm ein unerwartetes Verhalten aufweisen können die Studierenden eine Debug-Sitzung starten. Dabei können sie das Laufzeitverhalten des Programms besser nachvollziehen. Dadurch kann die Ursache für das unerwartete Verhalten effizienter identifiziert werden.

% In diesem Abschnitt wird ein Beispielszenario für den Einsatz der zu entwickelnden integrierten Entwicklungsumgebung beschrieben werden. Bei dem Beispielszenario handelt es sich um einen Praktikumsversuch, bei dem die Studierenden in Zweiergruppen mehrere verschiedene Aufgaben lösen müssen. Diese Aufgaben nutzen unterschiedliche Experimentkonfigurationen.

% \paragraph{Vorbereitung}
% Für den Praktikumsversuch müssen ggf. neue CrossLab-kompatible Geräte implementiert und angebunden werden. Dafür können entweder bereits vorhandene Services genutzt oder neue Services entwickelt werden. Sollten neue Services entwickelt werden ist es ggf. notwendig bereits vorhandene Geräte mit diesen zu erweitern. Weiterhin werden Testfälle ausgearbeitet, welche die automatische Überprüfung von abgegebenen Lösungen ermöglichen.

% \paragraph{Durchführung}
% Nachdem der Praktikumsversuch fertiggestellt wurde kann er den Studierenden vorgestellt werden. Dabei zeigt ein Dozent den Studierenden die Aufgaben und wie diese zu bearbeiten sind. Danach können die Studierenden mit der Bearbeitung der Aufgaben beginnen. Dafür starten sie die entsprechenden Experimente über ein ihnen bereitgestelltes Interface, z.B. über ein vorhandenes \ac{LMS}. Die Studierenden sollen die gestellten Aufgaben gemeinsam bearbeiten. Sollten während der Bearbeitung der Aufgaben Probleme auftreten können sich die Studierenden an einen entsprechenden Betreuer wenden. Die Studierenden können ihre erarbeitete Lösung jederzeit mit den hinterlegten Testfällen überprüfen. Sobald alle Testfälle erfolgreich beendet werden ist die entsprechende Aufgabe erfüllt und die Ergebnisse werden entsprechend gespeichert.

% \paragraph{Nachbereitung}
% Während der Ausführung der Experimente können Daten erfasst werden, die es Lehrenden ermöglicht nachzuvollziehen bei welchen Aufgaben die Studierenden Probleme hatten bzw. wofür sie die meiste Zeit aufwenden mussten. Dazu kann z.B. die Code-Historie der Studierenden gespeichert werden sowie die in einem Experiment verbrachte Zeit. Durch die Analyse der entsprechenden Daten können Lehrende Änderungen am Praktikumsversuch vornehmen um diesen zu verbessern.
