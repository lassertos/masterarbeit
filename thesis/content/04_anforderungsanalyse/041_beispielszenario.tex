\section{Beispielszenario: Praktikumsversuch}\label{section:anforderungsanalyse:beispielszenario}

In diesem Abschnitt wird ein Beispielszenario für den Einsatz der zu entwickelnden IDE gegeben. Bei diesem Beispielszenario handelt es sich um einem Praktikumsversuch, bei dem die Studierenden in Zweiergruppen einen Steuerungsalgorithmus für einen 3-Achsen-Portalkran schreiben müssen. Dafür erhalten sie einen Microcontroller, der über ein entsprechendes CrossLab-Interface in Experimenten verwendet werden kann. Die Experimente können über das für den Praktikumsversuch verwendete \ac{LMS} gestartet werden.

Die Studierenden können die IDE verwenden um ihr Programm zu schreiben. Dabei können sie zusammen an dem gleichen Programm arbeiten. Zusätzlich zum Editieren des Quellcodes werden ihnen Funktionen wie z.B. Code-Vervollständigung, Code-Navigation und Refactoring zur Verfügung gestellt. Weiterhin wird es ihnen ermöglicht Dateien und Ordner zur Strukturierung ihres Programms zu erstellen. Außerdem können sie ihr Programm kompilieren lassen. Dabei erhalten sie die Ausgaben des Compilers. Falls die Kompilierung erfolgreich ist kann das Programm auf den Microcontroller geladen werden. Zur Überprüfung ihrer Lösung können die Studierenden vorkonfigurierte Testfälle ausführen. Sollten diese alle erfolgreich sein, so haben sie die Aufgabe erfolgreich gelöst. Sollte das Programm ein unerwartetes Verhalten aufweisen können die Studierenden eine Debug-Sitzung starten. Dabei können sie das Laufzeitverhalten des Programms besser nachvollziehen. Dadurch kann die Ursache für das unerwartete Verhalten effizienter identifiziert werden.

Die verschiedenen Funktionen der IDE, wie z.B. die Kompilierung und das Debuggen von Programmen der Studierenden, können über entsprechende CrossLab-Services von anderen Laborgeräten bereitgestellt werden. Soll z.B. der Microcontroller durch einen FPGA ersetzt werden können die entsprechenden Laborgeräte ausgetauscht werden. Dadurch sind keine grundlegenden Änderungen an der IDE notwendig. Somit kann auch die IDE selbst ausgetauscht werden, falls z.B. das Benutzerinterface an den Kenntnisstand der Studierenden angepasst werden soll. Die IDE soll auch ohne weitere Laborgeräte in einem Experiment als Code Editor verwendet werden können, damit Nutzer jederzeit an ihren Programmen arbeiten können ohne auf die Verfügbarkeit aller Laborgeräte warten zu müssen.
