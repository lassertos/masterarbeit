\chapter{Konzeption}\label{section:konzeption}

In diesem Kapitel werden die Konzepte für die verschiedenen Funktionen und CrossLab-Services der IDE vorgestellt. Dabei wird in \autoref{section:konzeption:experimentkonfiguration} zunächst festgelegt welche Experimentkonfigurationen in den darauf folgenden Abschnitten betrachtet werden. Daraufhin wird in \autoref{section:konzeption:crosslab-kompatibilität} dargelegt wie die CrossLab-Kompatibilität der IDE erreicht werden kann. Danach wird in \autoref{section:konzeption:kollaboration} der Ansatz für die Bereitstellung von Kollaborationsmöglichkeiten beschrieben. In \autoref{section:konzeption:dateisystem} wird das Konzept des Dateisystems erläutert. Darauf folgend wird in \autoref{section:konzeption:kompilierung} die Funktionsweise der Kompilierung dargelegt. Anschließend wird in \autoref{section:konzeption:debugging} das Konzept zur Ermöglichung des Debuggens beschrieben. Weiterhin wird in \autoref{section:konzeption:testen} die Vorgehensweise für die Konfiguration und Ausführung von Tests erläutert. Schließlich wird in \autoref{section:konzeption:language-server} die Einbindung von Language Servern dargelegt.

\section{Experimentkonfiguration}\label{section:konzeption:experimentkonfiguration}

Die IDE soll in verschiedenen Experimentkonfigurationen nutzbar sein. In \autoref{requirement:Eigenständig nutzbar} wird verlangt, dass Experimente, welche nur die IDE als Laborgerät enthalten, ausführbar sein sollen. Weiterhin soll die IDE in Verbindung mit Steuereinheiten, wie z.B. Microcontrollern und FPGAs, genutzt werden können. Diese können wiederum mit steuerbaren Laborgeräten verbunden sein. Viele der Funktionen, die in den folgenden Abschnitten besprochen werden sind nur in speziellen Fällen in einer eigenständigen Variante der IDE nutzbar. Weiterhin ist die Nutzung von Steuereinheiten ohne verbundene steuerbare Laborgeräte oftmals nicht zielführend. Daher wird für die folgenden Abschnitte von einem Experiment mit der IDE sowie mindestens einer Steuereinheit und mindestens einem steuerbaren Laborgerät ausgegangen. Falls neue Laborgeräte oder CrossLab-Services eingeführt werden, wird deren mögliche Einbindung in die betrachtete Experimentkonfiguration beschrieben.
\input{content/05_konzeption/052_crosslab_kompatibilität.tex}
\section{Kollaboration}\label{section:konzeption:kollaboration}

\usetikzlibrary{arrows.meta}

\begin{note}
    \textbf{Notizen:}
    \begin{itemize}
        \item Erwähnung von \autoref{requirement:Kollaboration} und \autoref{requirement:Kollaboration: CrossLab-Kompatibilität}
        \item Beschreibung der grundlegenden Konzepte
        \item Klassendiagramm kollaborative Datentypen
        \item Klassendiagramm Collaboration Services \& dazugehörige Klassen
        \item Beschreibung der verschiedenen Interaktionen zwischen den Services
              \begin{itemize}
                  \item Beginn der Synchronisation
                  \item Synchronisation des geteilten JSON-Objekts
                  \item Austausch von Zustandsinformationen
                  \item high-level Sequenzdiagramme + low-level Beschreibung
              \end{itemize}
        \item Beschreibung der Einbindung in die betrachtete Experimentkonfiguration
    \end{itemize}
\end{note}

Nach \autoref{requirement:Kollaboration} soll die IDE die Echtzeit-Kollaboration mit anderen Laborgeräten innerhalb eines Experiments unterstützen. Dies soll laut \autoref{requirement:Kollaboration: CrossLab-Kompatibilität} über entsprechende CrossLab-Services erfolgen. Es gibt viele verschiedene Methoden zur Synchronisierung von Daten zwischen mehreren Teilnehmern. Beispiele derartiger Methoden sind \emph{\ac{OT}} \cite{sun_operational_1998}, \emph{Differential Synchronization} \cite{fraser_differential_2009} und \emph{\acp{CRDT}} \cite{shapiro_conflict-free_2011}. Aufgrund der Tatsache, dass jede dieser Methoden eigene Vor- und Nachteile besitzt, sollte die entwickelte Lösung möglichst unabhängig von dem zugrunde liegenden Synchronisationsalgorithmus sein. Daher werden zunächst die grundlegenden Konzepte beschrieben, bevor die konzipierten CrossLab-Services vorgestellt werden.

\begin{figure}[tbp]
    \centering
    \resizebox{\textwidth}{!}{\begin{tikzpicture}
            \begin{interface}[text width=4cm]{CollaborationType}{0,0}
                \operation{+ toJSON()}
                \operation{+ onUpdate()}
            \end{interface}
            \begin{class}[text width=5cm]{CollaborationObject}{-6,2.25}
                \operation{+ setProperty()}
                \operation{+ getProperty()}
                \operation{+ deleteProperty()}
            \end{class}
            \begin{class}[text width=5cm]{CollaborationNull}{-6,-0.75}
            \end{class}
            \begin{class}[text width=5cm]{CollaborationArray}{-6,-2.25}
                \operation{+ push()}
                \operation{+ get()}
                \operation{+ delete()}
            \end{class}
            \begin{class}[text width=5cm]{CollaborationNumber}{6,2.25}
                \operation{+ set()}
            \end{class}
            \begin{class}[text width=5cm]{CollaborationString}{6,0}
                \operation{+ set()}
                \operation{+ insert()}
                \operation{+ delete()}
            \end{class}
            \begin{class}[text width=5cm]{CollaborationBoolean}{6,-3.35}
                \operation{+ set()}
            \end{class}
            \draw[umlcd style dashed line, -{Triangle[length=2.5mm,open]}] (CollaborationObject.east) -- ([yshift=5mm] CollaborationType.west);
            \draw[umlcd style dashed line, -{Triangle[length=2.5mm,open]}] (CollaborationNull.east) -- (CollaborationType.west);
            \draw[umlcd style dashed line, -{Triangle[length=2.5mm,open]}] (CollaborationArray.east) -- ([yshift=-5mm] CollaborationType.west);
            \draw[umlcd style dashed line, -{Triangle[length=2.5mm,open]}] (CollaborationNumber.west) -- ([yshift=5mm] CollaborationType.east);
            \draw[umlcd style dashed line, -{Triangle[length=2.5mm,open]}] (CollaborationString.west) -- (CollaborationType.east);
            \draw[umlcd style dashed line, -{Triangle[length=2.5mm,open]}] (CollaborationBoolean.west) -- ([yshift=-5mm] CollaborationType.east);
        \end{tikzpicture}}
    \caption{Klassendiagramm kollaborative Datentypen}
    \label{figure:klassendiagramm-kollaborative-datentypen}
\end{figure}

\begin{figure}[tbp]
    \centering
    \begin{tikzpicture}
        \begin{class}[text width=7cm]{CollaborationServiceProducer}{-4,0}
        \end{class}
        \begin{class}[text width=7cm]{CollaborationServiceConsumer}{4,0}
            \operation{+ getAwareness()}
            \operation{+ joinRoom()}
            \operation{+ executeTransaction()}
            \operation{+ valueToCollaborationType()}
            \operation{+ getProperty()}
            \operation{+ onUpdate()}
        \end{class}
        \begin{class}[text width=7cm]{Room}{0,-5}
            \attribute{+ awareness: Awareness}
            \operation{+ addParticipant()}
            \operation{+ removeParticipant()}
            \operation{+ valueToCollaborationType()}
            \operation{+ executeTransaction()}
            \operation{+ startSynchronization()}
            \operation{+ getProperty()}
            \operation{+ onUpdate()}
        \end{class}
        \begin{interface}[text width=7cm]{Awareness}{-4,-14}
            \operation{+ getLocalState()}
            \operation{+ setLocalState()}
            \operation{+ setLocalStateField()}
            \operation{+ getStates()}
            \operation{+ onChange()}
            \operation{+ onUpdate()}
        \end{interface}
        \begin{class}[text width=7cm]{AwarenessProvider}{-4,-11}
            \implement{Awareness}
            \operation{+ applyUpdate()}
            \operation{+ encodeStates()}
        \end{class}
        \begin{class}[text width=7cm]{CollaborationProvider}{4,-11}
            \operation{+ handleCollaborationMessage()}
            \operation{+ startSynchronization()}
            \operation{+ executeTransaction()}
            \operation{+ valueToColloraborationType()}
            \operation{+ getProperty()}
            \operation{+ onCollaborationMessage()}
            \operation{+ onUpdate()}
        \end{class}
        \draw[stroke] ([xshift=-10mm]CollaborationServiceProducer.south) -- ([xshift=-10mm] CollaborationServiceProducer.south |- , |- Room.west) -- (Room.west) node [above, xshift=-6mm] () {rooms} node [below, xshift=-4mm] () {0..*};
        \draw[stroke] ([xshift=10mm] CollaborationServiceConsumer.south) -- ([xshift=10mm] CollaborationServiceConsumer.south |- , |- Room.east) -- (Room.east) node [above, xshift=6mm] () {rooms} node [below, xshift=4mm] () {0..*};
        \draw[stroke] ([xshift=-5mm] Room.south) -- (-0.5,-10.5) -- (-4,-10.5) -- (AwarenessProvider.north) node [left, yshift=2mm] () {awarenessProvider} node [right, yshift=2mm] () {1};
        \draw[stroke] ([xshift=5mm] Room.south) -- (0.5,-10.5) -- (3.5,-10.5) -- ([xshift=-5mm] CollaborationProvider.north) node [left, yshift=2mm] () {1} node [right, yshift=2mm] () {collaborationProvider};
    \end{tikzpicture}
    \caption{Klassendiagramm Collaboration Services}
    \label{figure:klassendiagramm-collaboration-services}
\end{figure}

Im Folgenden wird für die Erklärung der Konzepte von mehreren an der Kollaboration teilnehmenden Clients und einem zentralen Synchronisationsserver ausgegangen. Die Clients besitzen die Möglichkeit sogenannte \textit{Räume} zu erstellen bzw. ihnen beizutreten. Jeder Raum verwaltet ein geteiltes JSON-Objekt. Dieses nutzt Strings als Schlüssel mit entsprechenden kollaborativen Datentypen als Werten. Diese kollaborativen Datentypen sind in \autoref{figure:klassendiagramm-kollaborative-datentypen} dargestellt. Dabei gibt jeweils einen kollaborativen Datentypen für die JSON-kompatiblen Datentypen Objekt, Array, Number, String, Boolean und Null. Jeder kollaborative Datentyp bietet neben speziellen Funktionen zur Bearbeitung dessen auch eine Funktion zur Umwandlung in den zugrunde liegenden Datentyp sowie eine Funktion zur Registrierung von Event Handlern für Update-Events an. Änderungen an dem geteilten JSON-Objekt werden über den zentralen Server an die anderen Teilnehmer weitergeleitet. Zusätzlich können die Clients Zustandsinformationen an den Server melden, welcher diese ebenso an die anderen Clients weiterleitet. Die Zustandsinformationen eines Clients können nur von diesem selbst verändert werden. Allerdings können Clients die Zustandsinformationen eines anderen Teilnehmers löschen, falls für eine gewisse Zeit keine Aktualisierung dieser vorgenommen wurde. Dadurch können z.B. Verbindungsabbrüche von Clients behandelt werden. Basierend auf diesen Konzepten werden nachfolgend der \textit{Collaboration Service Producer} und der \textit{Collaboration Service Consumer} anhand beispielhafter Interaktionen vorgestellt. Dabei werden auch die Verwendung der in \autoref{figure:klassendiagramm-collaboration-services} dargestellten Klassen und Funktionen an passender Stelle erläutert.

\begin{figure}[tbp]
    \centering
    \resizebox{\textwidth}{!}{\begin{sequencediagram}
            \newthread{consumer1}{Consumer 1}
            \newthreadShift{producer}{Producer}{3cm}
            \newthreadShift{consumer2}{Consumer 2}{3cm}

            \begin{call}{consumer1}{trete Räumen bei}{consumer1}{}
            \end{call}
            \prelevel\prelevel
            \begin{call}{consumer2}{trete Räumen bei}{consumer2}{}
            \end{call}

            \begin{call}{consumer1}{Initialisierung}{producer}{}
                \begin{call}{producer}{registriere Consumer}{producer}{}
                \end{call}
            \end{call}
            \begin{call}{consumer1}{starte Synchronisation}{producer}{}
            \end{call}

            \begin{call}{consumer2}{Initialisierung}{producer}{}
                \begin{call}{producer}{registriere Consumer}{producer}{}
                \end{call}
            \end{call}
            \begin{call}{consumer2}{starte Synchronisation}{producer}{}
            \end{call}
        \end{sequencediagram}}
    \caption{Initialisierung Synchronisation}
    \label{figure:initialisierung-synchronisation}
\end{figure}

Zunächst wird die Initialisierung der Synchronisation zwischen dem Collaboration Service Producer und dem Collaboration Service Consumer betrachtet. Dafür ist in \autoref{figure:initialisierung-synchronisation} ein möglicher Ablauf dargestellt. Zunächst treten die Collaboration Service Consumer lokal den Räumen bei, für welche sie Daten bereitstellen bzw. welche in der Konfiguration der Verbindung mit dem jeweiligen Producer, während der Erstellung des Experiments, angegeben wurden. Dafür kann die Funktion \texttt{joinRoom()} des Collaboration Service Consumer verwendet werden. Der Beitritt in Räume kann auch nach dem Beginn der Synchronisation geschehen. Nachdem der Collaboration Service Consumer den Räumen beigetreten ist sendet dieser eine Initialisierungsnachricht an den Collaboration Service Producer, welche unter anderem einen einzigartigen Kennzeichner beinhaltet. Dieser Kennzeichner kann entweder entsprechend generiert werden oder über die Konfiguration der entsprechenden Laborgeräte während der Erstellung des Experiments angegeben werden. Die Informationen der Initialisierungsnachricht werden von dem Collaboration Service Producer verwendet um die Nutzer den entsprechenden Räumen zuzuordnen. Sobald der Collaboration Service Producer eine Antwort an den Collaboration Service Consumer gesendet hat startet dieser die Synchronisation. Diese erfolgt über die zugrunde liegende Synchronisationsmethode. Aufgrund dessen können nur Collaboration Service Consumer und Collaboration Service Producer verbunden werden wenn sie die gleiche Synchronisationsmethode verwenden.

\begin{figure}[tbp]
    \centering
    \resizebox{\textwidth}{!}{\begin{sequencediagram}
            \newthread{consumer1}{Consumer 1}
            \newthreadShift{producer}{Producer}{3cm}
            \newthreadShift{consumer2}{Consumer 2}{3cm}

            \begin{call}{consumer1}{ändere JSON-Objekt}{consumer1}{}
            \end{call}

            \begin{call}{consumer1}{sende Änderung}{producer}{}
            \end{call}

            \begin{call}{producer}{wende Änderung an}{producer}{}
            \end{call}

            \begin{call}{producer}{sende Änderung}{consumer2}{}
            \end{call}

            \begin{call}{consumer2}{wende Änderung an}{consumer2}{}
            \end{call}
        \end{sequencediagram}}

    \caption{Synchronisation des geteilten JSON-Objekts}
    \label{figure:synchronisation-des-geteilten-json-objekts}
\end{figure}

\begin{figure}[tbp]
    \centering
    \resizebox{\textwidth}{!}{\begin{sequencediagram}
            \newthread{consumer1}{Consumer 1}
            \newthreadShift{producer}{Producer}{3cm}
            \newthreadShift{consumer2}{Consumer 2}{3cm}

            \begin{call}{consumer1}{ändere Zustand}{consumer1}{}
            \end{call}

            \begin{call}{consumer1}{sende Zustände}{producer}{}
            \end{call}

            \begin{call}{producer}{aktualisiere Zustände}{producer}{}
            \end{call}

            \begin{call}{producer}{sende Zustände}{consumer2}{}
            \end{call}

            \begin{call}{consumer2}{aktualisiere Zustände}{consumer2}{}
            \end{call}
        \end{sequencediagram}}

    \caption{Austausch der Zustandsinformationen}
    \label{figure:austausch-der-zustandsinformationen}
\end{figure}

% In \autoref{figure:klassendiagramm-collaboration-services} ist ein Klassendiagramm für die Collaboration Services dargestellt. Jeder \texttt{Room} besitzt einen \texttt{CollaborationProvider} sowie einen \texttt{AwarenessProvider}. Der \texttt{CollaborationProvider} ist für die Verwaltung des geteilten JSON-Objekts verantwortlich. Dabei ist dessen Implementierung abhängig von der verwendeten Synchronisationsmethode und kann für Producer und Consumer unterschiedlich sein. Die Behandlung von Nachrichten für die zugrunde liegende Synchronisationsmethode geschieht über die Funktionen \texttt{handleCollaborationMessage()} für eingehende Nachrichten, während \texttt{onCollaborationMessage()} für die Registrierung von Handlern für ausgehende Nachrichten verwendet werden kann. Die Funktion \texttt{startSynchronization()} wird innerhalb der gleichnamigen Funktion des entsprechenden \texttt{Room} genutzt um die Synchronisation zu starten. Die Funktionen \texttt{getProperty()}, \texttt{executeTransaction()} und \texttt{valueToCollaborationType()} können alle, unter Angabe des entsprechenden Raums, über den Collaboration Service Consumer aufgerufen werden. Die Funktion \texttt{getProperty()} ermöglicht den Zugriff auf die Eigenschaften des geteilten JSON-Objekts. Dabei wird ein entsprechender kollaborativer Datentyp zurückgegeben. Es werden nur Eigenschaften des geteilten Objekts synchronisiert, die mindestens einmal über die Operation \texttt{getProperty()} abgefragt wurden. Die Funktion \texttt{executeTransaction()} ermöglicht die Aggregation mehrerer Änderungen an dem geteilten JSON-Objekt innerhalb einer Transaktion. Die Änderungen werden dann in einem einzelnen Update zusammengefasst. Weiterhin dient die Funktion \texttt{valueToCollaborationType()} zur Umwandlung eines unterstützen JSON-Datentyps in einen entsprechenden kollaborativen Datentypen. Das Interface \texttt{Awareness} definiert Funktionen für die Verwaltung der geteilten Zustandsinformationen. Die Klasse \texttt{AwarenessProvider} realisiert dieses Interface und bietet zusätzlich noch die Funktionen \texttt{applyUpdate()} und \texttt{encodeStates()} an. Dabei wird die Funktion \texttt{applyUpdate()} für die Anwendung von eingehenden Aktualisierungen verwendet, während \texttt{encodeStates()} für die Erstellung ausgehender Aktualisierungen verwendet wird. Die über \texttt{onUpdate()} registrierten Event Handler für Updates werden bei jeglicher Aktualisierung der Zustandsinformationen aufgerufen, während über \texttt{onChange()} registrierte Event Handler nur bei einer tatsächlichen Veränderung der Zustandsinformationen aufgerufen werden. Die Klasse \texttt{AwarenessProvider} ist unabhängig von der zugrunde liegenden Synchronisationsmethode. Die Zustandsinformationen für einen Raum können über die Funktion \texttt{getAwareness()} des Collaboration Service Consumer erhalten werden. Für den Collaboration Service Producer werden keine weiteren Funktionen definiert.

% In \autoref{figure:klassendiagramm-collaboration-services} ist ein Klassendiagramm für die Collaboration Services dargestellt. Bei Experimenten mit einem zentralen Synchronisationspunkt (z.B. bei der Verwendung von \ac{OT}) bietet dieser einen Producer an während die restlichen Geräte, die an der Kollaboration teilnehmen, einen Consumer nutzen. Dahingegen nutzen bei Experimenten ohne einen zentralen Synchronisationspunkt (z.B. bei der Verwendung von \acp{CRDT}) alle Geräte, die an der Kollaboration teilnehmen, einen Prosumer. In der Experimentbeschreibung sollten bei den Konfigurationen der Verbindungen von Kollaborationsdiensten stets die Synchronisationsmethode sowie die sogenannten \emph{Räume} angegeben werden, die in der Verbindung genutzt werden sollen. Räume besitzen einen eindeutigen Namen und ein JSON-Objekt, das zwischen allen Teilnehmern innerhalb des Raums synchronisiert wird. Das synchronisierte JSON-Objekt kann z.B. Ordner oder Dateien abbilden, die von den Teilnehmern geteilt werden. Jeder Raum besitzt einen sogenannten \emph{Provider}. Dieser nutzt die in der Konfiguration der Verbindung angegebene Synchronisationsmethode um den Inhalt des Raums zu synchronisieren. Weiterhin bietet der Provider eine Schnittstelle um Statusinformationen auszutauschen. Diese Informationen sind teilnehmerspezifisch, d.h. sie können nur von dem jeweiligen Teilnehmer aktualisiert werden. Ein Beispiel für derartige Statusinformationen ist z.B. die aktuelle Position eines Teilnehmers innerhalb einer Datei.

% Die verwendeten Räume sowie deren zugrunde liegende Synchronisationsmethode werden bereits in der Experimentbeschreibung in der Konfiguration der Verbindung zwischen dem Collaboration Service Producer und dem Collaboration Service Consumer angegeben. Dementsprechend können diese auf der Seite des Collaboration Service Consumer erstellt werden, bevor die Verbindung hergestellt wird. 

% \begin{figure}[htbp]
%     \centering
%     \begin{sequencediagram}
%         \newthread{consumer}{Consumer}
%         \newthreadShift{producer}{Producer}{4cm}

%         \begin{call}{consumer}{erstelle Räume}{consumer}{}
%         \end{call}

%         \begin{call}{consumer}{sende ID}{producer}{}
%             \begin{call}{producer}{registriere Consumer}{producer}{}
%             \end{call}
%         \end{call}

%         \begin{call}{consumer}{starte Synchronisation}{producer}{}
%         \end{call}
%     \end{sequencediagram}
%     \caption{Initialisierung Kollaboration}\label{abbildung:initialisierung-kollaboration}
% \end{figure}

% Die Kommunikation zwischen den Kollaborationsteilnehmern erfolgt über ein entsprechendes Nachrichtenprotokoll. In \autoref{abbildung:initialisierung-kollaboration} ist der Verbindungsaufbau zwischen einem Consumer und einem Producer dargestellt. Zunächst erstellen beide die in der Verbindungskonfiguration festgelegten Räume. Dabei verknüpft der Consumer den Raum direkt mit der Verbindung. Der Producer hingegen wartet auf die Initialisierungsnachricht des Consumer, welche dessen ID beinhaltet. Die ID kann dann genutzt werden um den Consumer dem entsprechenden Räumen zuzuweisen. Sobald der Producer das erfolgreiche Ende der Initialisierung an den Consumer meldet beginnt dieser mit der Synchronisation. Da die verschiedenen Synchronisationsmethoden ggf. unterschiedliche Nachrichtenformate besitzen wird eine allgemeine Nachricht definiert, die dann die spezifischen Informationen für die zugrunde liegende Methode beinhalten. Daraus folgt auch, dass es nicht möglich ist einen Consumer mit einem Producer zu verbinden, der eine andere Synchronisationsmethode verwendet. Weiterhin ist darauf zu achten, dass ggf. mehrere Provider für eine Synchronisationsmethode benötigt werden. Dies ist z.B. der Fall bei Methoden mit einem zentralen Synchronisationspunkt.

% Während die Behandlung von Aktualisierungen der Räume durch das Protokoll der zugrunde liegenden Synchronisationsmethode erfolgt, wird für die Behandlung von Statusaktualisierungen der Teilnehmer ein allgemeines Protokoll eingeführt. Dabei ist der Status eines Teilnehmers immer als ein JSON-Objekt darstellbar, wobei ein Wert von \texttt{null} angibt, dass der Teilnehmer nicht mehr erreichbar ist. Zu Beginn der Synchronisation schicken Consumer ihren aktuellen Status an den Producer. Dieser speichert den aktuellen Status und sendet ihn an die restlichen Consumer. Wenn sich der Status eines Consumer kann er entweder den kompletten Status an den Producer senden oder nur die vorgenommenen Änderungen. Der Producer aktualisiert seine gespeicherten Statusinformationen für den Consumer und leitet die Änderungen an die restlichen Consumer weiter. Diese aktualisieren ebenfalls ihre lokalen Statusinformationen und können dann auf die vorgenommenen Änderungen reagieren. Sollte ein Consumer nicht innerhalb eines vordefinierten Zeitraums seinen Status aktualisieren wird dieser auf \texttt{null} gesetzt und die Änderung an die restlichen Consumer weitergeleitet.
\section{Dateisystem}\label{section:konzeption:dateisystem}

\begin{note}
    \textbf{Notizen:}
    \begin{itemize}
        \item Erwähnung von \autoref{requirement:Dateisystem}, \autoref{requirement:Dateisystem: CrossLab-Kompatibilität} und \autoref{requirement:Dateisystem: Kollaboration}
        \item Ggf. Erwähnung von \autoref{requirement:Standalone nutzbar}
        \item Vergleich verschiedener Konzepte (client- vs. server-side)
        \item Beschreibung der CrossLab-Services + Klassendiagramm
        \item Beschreibung der Kollaboration
        \item Beschreibung der Einbindung in die betrachtete Experimentkonfiguration
    \end{itemize}
\end{note}

Nach \autoref{requirement:Dateisystem} soll die IDE ein integriertes Dateisystem besitzen. Für dieses werden im Folgenden zwei verschiedene Lösungsansätze in beschrieben, ein client-seitiger und eine server-seitiger.

Für den client-seitigen Lösungsansatz bietet sich eine Speicherung der Dateien des Nutzers innerhalb des Browsers an. Für die persistente Speicherung von Daten innerhalb des Browsers kann die Indexed Database API \cite{noauthor_indexed-database-api_nodate} genutzt werden. Diese wird von allen aktuellen Browsern unterstützt und erlaubt die langfristige Speicherung von größeren Datenmengen. Der Vorteil des client-seitigen Ansatzes ist die Tatsache, das kein weiterer Speicherplatz für die Nutzer bereitgestellt werden muss, da die Daten auf dem Rechner des Nutzers gespeichert werden. Allerdings sind die Daten sowohl an die Domain der IDE, das Gerät des Nutzers als auch an den spezifischen Browser gebunden und müssen durch entsprechendes Exportieren und Importieren übertragen werden.

Der server-seitige Lösungsansatz basiert darauf jedem Nutzer einen entsprechenden Bereich zuzuteilen, in welchem seine Dateien gespeichert werden. Dies kann entweder über das Dateisystem des Servers oder über eine Datenbank geschehen. Der Vorteil dieser Art der Datenspeicherung liegt darin, dass sie geräteunabhängig ist. Allerdings werden für die Speicherung der Nutzerdaten entsprechender Speicherplatz auf dem Server benötigt wodurch höhere Kosten und ein höherer Verwaltungsaufwand bestehen. Zudem muss sichergestellt werden, dass das System nicht ausgenutzt werden kann.

\autoref{requirement:Dateisystem: CrossLab-Kompatibilität} verlangt die Entwicklung von CrossLab-Services für die Bereitstellung und Nutzung von Dateisystemen. Dementsprechend werden im Folgenden der \textit{Filesystem Service Producer} und der \textit{Filesystem Service Consumer} beschrieben. Die grundlegende Kommunikation zwischen den beiden Services geschieht über den Austausch von Nachrichten. Diese bestehen aus einem Typen und dem dazugehörigen Inhalt. Durch die Definition der Nachrichten kann eine Validierung dieser innerhalb der Services geschehen. In \autoref{figure:klassendiagramm-dateisystem-services} ist ein Klassendiagramm für die beiden Services dargestellt. Aus diesem können die bereitgestellten Operationen eines Dateisystems abgelesen werden. So müssen Dateisysteme die Erstellung von Ordnern und Dateien, das Lesen, Verschieben und Löschen dieser sowie das Schreiben von Dateien unterstützen. Dabei besitzt der Consumer Funktionen um die einzelnen Operationen auszuführen, während der Producer die Möglichkeit bietet auf eingehende Anfragen zu reagieren und entsprechende Antworten an den Consumer zu senden.

\begin{figure}[tbp]
    \centering
    \begin{tikzpicture}
        \begin{class}[text width=6cm]{FilesystemServiceProducer}{0,0}
            \operation{+ onRequest()}
            \operation{+ send()}
        \end{class}
        \begin{class}[text width=6cm]{FilesystemServiceConsumer}{7,0}
            \operation{+ createDirectory()}
            \operation{+ delete()}
            \operation{+ move()}
            \operation{+ readDirectory()}
            \operation{+ readFile()}
            \operation{+ writeFile()}
        \end{class}
    \end{tikzpicture}
    \caption{Klassendiagramm Filesystem Services}
    \label{figure:klassendiagramm-dateisystem-services}
\end{figure}

Eine mögliche Erweiterung der Services besteht in der Unterstützung mehrerer Kommunikationspartner. Sollten bei der Erstellung eines Experiments mehrere Verbindungen hergestellt werden so wird für jeden Kommunikationspartner ein eindeutiger Kennzeichner erstellt. Dieser wird dann in den jeweiligen Funktionen mit angegeben um die Nachrichten an den korrekten Kommunikationspartner zu schicken. Zudem könnte auch die Überwachung von Ordnern und Dateien angeboten werden. Dafür würde der Consumer eine entsprechende Anfrage an den Producer senden, in welcher er die zu überwachenden Pfade angibt. Sollte dann eine Änderung in einem der überwachten Pfade auftreten wird eine entsprechende Nachricht vom Producer an den Consumer gesendet.

Nach \autoref{requirement:Dateisystem: Kollaboration} soll das Dateisystem das Teilen von Ordnern mit anderen Nutzern innerhalb eines Experiments unterstützen. Dafür können die in \autoref{section:konzeption:kollaboration} beschriebenen Kollaborationsmechanismen genutzt werden. Beim Erstellen eines Experiments öffnen Nutzer einen Raum zum Teilen ihrer Ordner. Die geteilte Datenstruktur hat dabei die Kennzeichner der verschiedenen Teilnehmer als Schlüssel mit den geteilten Ordnern als den dazugehörigen Wert. Am Anfang einer Sitzung hat ein Nutzer noch keine geteilten Ordner und setzt somit seinen eigenen Wert auf ein leeres Objekt. Wenn ein Nutzer einen Ordner teilt so wird dieser den anderen Nutzern angezeigt und sie können mit den Inhalt einsehen und bearbeiten. Die Implementierung der Synchronisation ist hierbei abhängig von dem verwendeten Code Editor. Weiterhin kann auch die aktuelle Position von Nutzern über deren Zustandsinformationen mit den anderen Nutzern geteilt werden. Die Position kann dann den anderen Nutzern innerhalb der entsprechenden Datei angezeigt werden.

% Das bisherige WIDE System nutzt ein projektbasiertes Dateisystem. In diesem muss der Name eines Projektes einzigartig sein. Weiterhin können nicht mehrere Projekte gleichzeitig geöffnet werden. Zudem werden Metadaten zu einem Projekt gespeichert. Diese umfassen das elektromechanische Modell, die Steuereinheit sowie die Programmiersprache, die bei der Erstellung des Projekts genutzt bzw. ausgewählt wurden. Dadurch ist es möglich dem Nutzer nur die Projekte anzuzeigen, die in dem aktuellen Experiment von Interesse sein könnten. In der neuen CrossLab Architektur ist es nun allerdings möglich mehrere Steuereinheiten und elektromechanische Modelle sowie weitere Laborgeräte zu einem Experiment zusammenzustellen. Deshalb sollte es dem Nutzer ermöglicht werden mehrere Projekte gleichzeitig öffnen und bearbeiten zu können. Ein weiteres Feature von WIDE ist die Bereitstellung von Beispielprojekten. Diese können von Nutzern verwendet werden um einen Einblick in die Programmierung einer gegebenen Steuereinheit zu bekommen. Um dieses Feature weiterhin unterstützen zu können sollte eine Konfigurationsmöglichkeit gegeben werden, welche die Bereitstellung derartiger Beispiele ermöglicht.
\section{Kompilierung und Programmierung}\label{section:konzeption:kompilierung-und-programmierung}

% \begin{note}
%     \textbf{Notizen:}
%     \begin{itemize}
%         \item Erwähnung von \autoref{requirement:Kompilierung} und \autoref{requirement:Programmierung von Steuereinheiten}
%         \item Beschreibung der CrossLab-Services + Klassendiagramme
%         \item Konzept für die Bereitstellung von Compilern als Laborgeräte
%         \item Beschreibung der Einbindung in die betrachtete Experimentkonfiguration
%         \item (Beschreibung möglicher Einstellungen?)
%     \end{itemize}
% \end{note}

Nach \autoref{requirement:Kompilierung} soll die Kompilierung der Programme von Nutzern innerhalb der IDE ermöglicht werden. Laut Unteranforderung (a) soll daher ein CrossLab-Service für die Bereitstellung und Nutzung von Compilern entwickelt werden. Diese sollen nach Unteranforderung (b) von der IDE für die Nutzung von Compilern verwendet werden. Weiterhin soll nach \autoref{requirement:Programmierung von Steuereinheiten} die Programmierung von Steuereinheiten innerhalb der IDE ermöglicht werden. Laut Unteranforderung (a) soll daher ein CrossLab-Service für die Programmierung von Steuereinheiten entwickelt werden. Diese sollen nach Unteranforderung (b) von der IDE für die Programmierung von Steuereinheiten verwendet werden. Dementsprechend werden im Folgenden der \textit{Compilation Service} sowie der \textit{Programming Service} vorgestellt.

In \autoref{figure:klassendiagramm-compilation-service} ist ein Klassendiagramm für den Compilation Service angegeben. Dabei besitzt der Compilation Service Consumer nur eine Funktion \texttt{compile()}, die es ermöglicht die Kompilierung eines Ordners anzufragen. Dabei können zusätzliche Optionen für den Compiler sowie das gewünschte Format der Ausgabe angegeben werden. Die möglichen Ausgabeformate können auf der Seite des Compilation Service Producer über entsprechende Schemata definiert und über die Funktion \texttt{addResultFormat()} hinzugefügt werden. Die registrierten Ausgabeformate können dann in der Servicebeschreibung des Compilation Service Producer hinterlegt werden. Während der Erstellung eines Experiments kann das gewünschte Ausgabeformat für die Verbindung eines Compilation Service Consumer und eines Compilation Service Producer in der Verbindungskonfiguration angegeben werden. Die Behandlung eingehender Kompilieranfragen kann durch Event Handler für \texttt{Compile}-Events des Compilation Service Producer erfolgen. Die Behandlung von Kompilieranfragen ist abhängig vom verwendeten Compiler. Bei einer erfolgreichen Kompilierung wird das Ergebnis in dem festgelegten Ausgabeformat an den Compilation Service Consumer gesendet zusammen mit den Meldungen des Compilers. Falls die Kompilierung fehlschlägt werden nur die Fehlermeldungen des Compilers versendet.

\begin{figure}[tbp]
    \centering
    \begin{tikzpicture}
        \begin{class}[text width=6.5cm]{CompilationServiceProducer}{0,0}
            \operation{+ addResultFormat()}
            \operation{+ onCompile()}
        \end{class}
        \begin{class}[text width=6.5cm]{CompilationServiceConsumer}{7.5,0}
            \operation{+ compile()}
        \end{class}
    \end{tikzpicture}
    \caption{Klassendiagramm Compilation Service}
    \label{figure:klassendiagramm-compilation-service}
\end{figure}

\begin{figure}[tbp]
    \centering
    \begin{tikzpicture}
        \begin{class}[text width=6.5cm]{ProgrammingServiceProducer}{0,0}
            \operation{+ onProgram()}
        \end{class}
        \begin{class}[text width=6.5cm]{ProgrammingServiceConsumer}{7.5,0}
            \operation{+ program()}
        \end{class}
    \end{tikzpicture}
    \caption{Klassendiagramm Programming Service}
    \label{figure:klassendiagramm-programming-service}
\end{figure}

In \autoref{figure:klassendiagramm-programming-service} ist ein Klassendiagramm für den Programming Service angegeben. Dabei besitzt der Programming Service Consumer nur die Funktion \texttt{program()}, die es ermöglicht eine Programmieranfrage zu starten. Diese muss entweder eine Datei, z.B. das Ergebnis einer Kompilierung, oder einen Ordner mit dem entsprechenden Programm enthalten. Dabei könnte der Programming Service Producer über seine Servicebeschreibung angeben, welche Formate unterstützt werden. Der Programming Service Producer löst bei einer eingehenden Programmieranfrage ein entsprechendes \texttt{Program}-Event aus. Dieses kann durch entsprechende Event Handler abgefangen und zur Programmierung der Steuereinheit verwendet werden.

Die betrachtete Experimentkonfiguration wird zunächst um ein weiteres Laborgerät erweitert. Dieses stellt einen Compiler über einen entsprechenden Compilation Service Producer bereit. Die IDE wird um einen Compilation Service Consumer sowie einen Programming Service Consumer erweitert. Die Steuereinheit wird um einen Programming Service Producer erweitert. Die IDE wird dann über die neu hinzugefügten CrossLab-Services mit dem Compiler und der Steuereinheit verbunden.
\section{Debugging}\label{section:konzeption:debugging}

% \begin{note}
%     \textbf{Notizen:}
%     \begin{itemize}
%         \item Erwähnung von \autoref{requirement:Debuggen}, \autoref{requirement:Debuggen: CrossLab-Kompatibilität} und \autoref{requirement:Debuggen: Kollaboration}
%         \item Beschreibung der CrossLab-Services + Klassendiagramm (Adapter)
%         \item Beschreibung der CrossLab-Services + Klassendiagramm (Target)
%         \item Konzept für die Ermöglichung der Kollaboration
%         \item Konzept für die Bereitstellung von Debuggern als Laborgeräte
%         \item Beschreibung der Einbindung in die betrachtete Experimentkonfiguration
%         \item (Beschreibung möglicher Einstellungen?)
%     \end{itemize}
% \end{note}

\begin{figure}[tbp]
    \centering
    \resizebox{\textwidth}{!}{
        \begin{tikzpicture}
            \begin{class}[text width=7.5cm]{DebuggingAdapterServiceProducer}{0,0}
                \operation{+ sendMessageDAP()}
                \operation{+ onStartSession()}
                \operation{+ onJoinSession()}
                \operation{+ onMessageDAP()}
            \end{class}
            \begin{class}[text width=7.5cm]{DebuggingAdapterServiceConsumer}{8,0}
                \operation{+ sendMessageDAP()}
                \operation{+ startSession()}
                \operation{+ joinSession()}
                \operation{+ onDapMessageDAP()}
            \end{class}
            \begin{class}[text width=7.5cm]{DebuggingTargetServiceProducer}{0,-3.5}
                \operation{+ sendDebuggingMessage()}
                \operation{+ onStartDebugging()}
                \operation{+ onEndDebugging()}
                \operation{+ onDebuggingMessage()}
            \end{class}
            \begin{class}[text width=7.5cm]{DebuggingTargetServiceConsumer}{8,-3.5}
                \operation{+ sendDebuggingMessage()}
                \operation{+ startDebugging()}
                \operation{+ endDebugging()}
                \operation{+ onDebuggingMessage()}
            \end{class}
        \end{tikzpicture}
    }
    \caption{Klassendiagramm Debugging Services}
    \label{figure:klassendiagramm-debugging-services}
\end{figure}

Nach \autoref{requirement:Debuggen} soll das Debuggen der Programme von Nutzern innerhalb der IDE ermöglicht werden. Dafür soll laut Unteranforderung (a) ein CrossLab-Service für die Bereitstellung und Nutzung von Debuggern sowie für die Kommunikation zwischen Debuggern und Steuereinheiten entwickelt werden. Weiterhin verlangt Unteranforderung (b) die Entwicklung eines CrossLab-Service für die Kommunikation zwischen Debuggern und Steuereinheiten. Der aus Unteranforderung (a) resultierende CrossLab-Service soll nach Unteranforderung (b) von der IDE für die Nutzung von Debuggern verwendet werden. Dementsprechend werden im Folgenden der \textit{Debugging Adapter Service} für die Bereitstellung und Nutzung von Debuggern sowie der \textit{Debugging Target Service} für die Kommunikation zwischen Debuggern und Steuereinheiten vorgestellt. \autoref{figure:klassendiagramm-debugging-services} zeigt ein Klassendiagramm für die beiden Debugging Services.

Der Debugging Adapter Service baut auf dem \ac{DAP} \cite{noauthor_debug-adapter-protocol_nodate} von Microsoft auf. Dieses spezifiert Nachrichten die zwischen einer IDE und einem sogenannten \textit{Debug Adapter} ausgetauscht werden. Ein Debug Adapter bildet die Schnittstelle zwischen einem Debugger und einer IDE. Das Protokoll erlaubt somit die Anbindung bestehender Debugger über die Implementierung eines entsprechenden Debug Adapters. Weiterhin wird der Implementierungsaufwand für die Einbindung von Debuggern in IDEs verringert, da in der Theorie nur eine Schnittstelle implementiert werden muss, anstatt jeden Debugger an eine eigens entwickelte Schnittstelle anpassen zu müssen. Somit sollte durch die Verwendung des \ac{DAP} die Einbindung von Debuggern vereinfacht werden.

\begin{figure}[tbp]
    \centering
    \begin{sequencediagram}
        \newthread{ide}{IDE}
        \newthreadShift{debugger}{Debugger}{3cm}
        \newthreadShift{steuereinheit}{Steuereinheit}{3cm}

        \begin{call}{ide}{starte Debug-Sitzung}{debugger}{}
            \begin{call}{debugger}{speichere Programm}{debugger}{}
            \end{call}
            \begin{call}{debugger}{kompiliere Programm}{debugger}{}
            \end{call}
            \begin{call}{debugger}{starte Debuggen}{steuereinheit}{}
                \begin{call}{steuereinheit}{lade Programm}{steuereinheit}{}
                \end{call}
            \end{call}
        \end{call}

        \begin{call}{ide}{DAP Nachrichten}{debugger}{DAP Nachrichten}
        \end{call}

        \prelevel\prelevel

        \begin{call}{debugger}{Debugger Nachrichten}{steuereinheit}{Debugger Nachrichten}
        \end{call}
    \end{sequencediagram}
    \caption{Start einer Debug-Sitzung}
    \label{figure:start-einer-debug-sitzung}
\end{figure}

\paragraph{Start einer Debug-Sitzung} \autoref{figure:start-einer-debug-sitzung} zeigt ein Sequenzdiagramm für den Start einer Debug-Sitzung. Die Funktion \texttt{startSession()} des Debugging Adapter Service Consumer kann zum Starten einer Debug-Sitzung werden. Dabei werden der Ordner, welcher das zu debuggende Programm enthält, und dessen URL bzw. Pfad sowie Konfigurationsoptionen für den Debugger an den Debugging Adapter Service Producer gesendet. Dieser löst dann ein \texttt{StartSession}-Event mit den übergebenen Daten aus, welches durch einen entsprechenden Event Handler abgefangen werden kann. Dadurch kann die Implementierung an den jeweiligen Debugger angepasst werden. Im Allgemeinen sollte zunächst der übersendete Ordner auf dem Dateisystem des Debugging Adapter Service Producer gespeichert werden. Weiterhin kann ggf. eine Kompilierung des Programms mit speziellen Optionen für das Debuggen vorgenommen werden, wobei der in \autoref{section:konzeption:kompilierung-und-programmierung} vorgestellte CrossLab-Service verwendet werden kann. Außerdem sollte der Debugger mit den Konfigurationsoptionen gestartet werden und es sollte ein eindeutiger Kennzeichner für die Debug-Sitzung generiert werden. Weiterhin sollte die zu debuggende Steuereinheit über den Start der Debug-Sitzung informiert werden. Dafür kann die Funktion \texttt{startDebugging()} des Debugging Target Service Consumer verwendet werden. Dabei wird das Programm an den Debugging Target Service Producer übergeben, dieses kann wie bereits für den Programming Service in \autoref{section:konzeption:kompilierung-und-programmierung} beschrieben eine Datei oder ein Ordner sein. Das Programm kann in Event Handlern von \texttt{StartDebugging}-Events z.B. zur Programmierung der Steuereinheit verwendet werden. Zusätzlich können ggf. noch weitere Vorbereitungen getroffen werden bevor eine Antwort an den Debugging Target Service Consumer gesendet wird. Nachdem die Sitzung erfolgreich gestartet wurde, wird eine entsprechende Antwort an den Debugging Adapter Service Consumer gesendet. Diese enthält den Kennzeicher der Debug-Sitzung sowie weitere Konfigurationsoptionen, die beim Start des \ac{DAP} übergeben werden sollen. Bei diesen Konfigurationsoptionen handelt es sich um Informationen die nur dem Debugging Adapter Service Producer bekannt sind, wie z.B. der Pfad zu dem kompilierten Programm auf dessen Dateisystem. Sobald der Debugging Adapter Service Consumer die Antwort erhalten hat kann das \ac{DAP} mit den Konfigurationsoptionen und dem Kennzeichner der Debug-Sitzung gestartet werden. Der Austausch der \ac{DAP} Nachrichten erfolgt dabei über die Funktionen \texttt{sendMessageDAP()}. Eingehende \ac{DAP} Nachrichten können über Event Handler für \texttt{MessageDAP}-Events erhalten und an den Debug Adapter übergeben werden. Eine ähnliche Vorgehensweise wird für den Austausch von Debug-Nachrichten zwischen dem Debugging Target Service Producer und dem Debugging Target Service Consumer angewendet. Hierbei werden die Funktion \texttt{sendDebuggingMessage()} und \texttt{DebuggingMessage}-Events verwendet. Die in eingehenden und augehenden DAP Nachrichten enthaltenen URLs müssen für den Debug Adapter bzw. die IDE angepasst werden, da diese auf unterschiedlichen Dateisystemen arbeiten.

\begin{figure}[tbp]
    \centering
    \resizebox{\textwidth}{!}{\begin{sequencediagram}
            \newthread{ide}{IDE}
            \newthreadShift{debugger}{Debugger}{4cm}
            \newthreadShift{steuereinheit}{Steuereinheit}{3cm}

            \begin{call}{ide}{DAP Terminate Anfrage}{debugger}{}
                \begin{call}{debugger}{lösche Sitzungsdaten}{debugger}{}
                \end{call}
                \begin{call}{debugger}{stoppe Debugger}{debugger}{}
                \end{call}
                \begin{call}{debugger}{beende Debuggen}{steuereinheit}{}
                    \begin{call}{steuereinheit}{lade Programm neu}{steuereinheit}{}
                    \end{call}
                \end{call}
            \end{call}
        \end{sequencediagram}}
    \caption{Ende einer Debug-Sitzung}
    \label{figure:ende-einer-debug-sitzung}
\end{figure}

\paragraph{Ende einer Debug-Sitzung} \autoref{figure:ende-einer-debug-sitzung} zeigt ein Sequenzdiagramm für das Ende einer Debug-Sitzung. Dieses wird durch eine entsprechende Nachricht des \ac{DAP} wie z.B. \texttt{Terminate} ausgelöst. Sobald der Debug Adapter diese Nachricht erhält werden die Sitzungsdaten gelöscht und der Debugger gestoppt. Weiterhin wird die Funktion \texttt{endDebugging()} des Debugging Target Service Consumer verwendet um den Debugging Target Service Producer über das Ende des Debug-Sitzung zu informieren. In einem Event Handler des dadurch ausgelösten \texttt{EndDebugging}-Events könnte z.B. das aktuelle Programm neugeladen werden und weitere vorgenommene Änderungen für das Debugging rückgängig gemacht werden. Danach wird eine entsprechende Antwort an den Debugging Target Service Consumer gesendet. Sobald diese erhalten wurde wird noch eine finale Antwort auf die ursprüngliche \ac{DAP} Anfrage gesendet.

\paragraph{Kollaboration} Nach \autoref{requirement:Teilen von Debug-Sitzungen} soll es Nutzern innerhalb eines Experiments ermöglicht werden laufenden Debug-Sitzungen beizutreten. Dafür soll laut Unteranforderung (e) der bereits vorhandene Collaboration Service der IDE verwendet werden. Über diesen könnte z.B. beim Start einer Debug-Sitzung der Kennzeichner der Sitzung und der verwendete Ordner in den Zustandsinformationen des Nutzers hinterlegt werden. Dadurch können andere Nutzer über die gestartete Sitzung informiert werden. Falls sie Zugriff auf den angegebenen Ordner haben können sie der Debug-Sitzung mithilfe der Funktion \texttt{joinSession()} des Debugging Adapter Service Consumer beitreten. Dabei wird der Kennzeichner der beizutretenden Debug-Sitzung angegeben. Die Antwort enthält einen neuen Kennzeichner für die Debug-Sitzung des beitretenden Nutzers sowie Konfigurationsoptionen für die Ausführung des \ac{DAP}. Um eine kollaborative Debug-Sitzung zu ermöglichen ist zudem eine spezielle Behandlung von einigen Nachrichten des \ac{DAP} nötig. Zum Beispiel darf nur eine \texttt{Initialize} Anfrage an einen Debug Adapter gestellt werden. Dementsprechend muss die Antwort auf diese gespeichert werden um sie später bei einer erneuten \texttt{Initialize} Anfrage an den beitretenden Nutzer senden zu können. Dabei wird die Anfrage beitretender Nutzer nicht an den Debug Adapter übergeben. Weiterhin sind z.B. die Nachrichten für Breakpoints, Stacktraces, das Starten und Stoppen des Programms sowie das Beenden der Debug-Sitzung zu betrachten. Die detaillierte Betrachtung aller dieser Nachrichten ist nicht das Ziel dieser Arbeit. In \autoref{section:prototypische-implementierung:debugging} wird die Behandlung der Nachrichten erläutert, welche für die prototypische Implementierung relevant sind.

Die betrachtete Experimentkonfiguration wird zunächst um ein Laborgerät erweitert. Dieses stellt einen Debugger über einen entsprechenden Debugging Adapter Service Producer bereit. Weiterhin besitzt dieses noch einen Debugging Target Service Consumer zur Kommunikation mit der zu debuggenden Steuereinheit. Die IDE wird um einen Debugging Adapter Service Consumer erweitert. Die Steuereinheit wird um einen Debugging Target Service Producer erweitert. Das neue Laborgerät wird über die entsprechenden CrossLab-Services mit der IDE und der Steuereinheit verbunden.
\section{Testen}\label{section:konzeption:testen}

% \begin{note}
%     \textbf{Notizen:}
%     \begin{itemize}
%         \item Erwähnung von \autoref{requirement:Testen}
%         \item Beschreibung der CrossLab-Services + Klassendiagramm
%         \item Beschreibung der Einbindung in die betrachtete Experimentkonfiguration
%     \end{itemize}
% \end{note}

Nach \autoref{requirement:Testen} soll ein CrossLab-Service für die Erstellung und Ausführung von Testfällen innerhalb eines Experiments entwickelt werden. Dabei soll nach Unteranforderung (a) der Producer in der Lage sein, Funktionen zur Verwendung in Testfällen bereitzustellen. Unteranforderung (b) verlangt, dass diese Funktionen vom Consumer ausgeführt werden können. Die Erstellung von Testfällen soll laut Unteranforderung (c) während der Konfiguration eines Experiments erfolgen können. Außerdem soll der entwickelte CrossLab-Service nach Unteranforderung (d) von der IDE zur Ausführung von Testfällen innerhalb eines Experiment verwendet werden. Dementsprechend wird im Folgenden der \textit{Testing Service} vorgestellt.

\autoref{figure:klassendiagramm-testing-service} zeigt ein Klassendiagramm für den Testing Service. Der Testing Service Producer ermöglicht es Laborgeräten Funktionen für die Erstellung von Testfällen bereitzustellen. Diese können über die Funktion \texttt{registerFunction()} registriert werden. Dabei werden mindestens der Name der Funktion und deren Implementierung benötigt. Zusätzlich könnte man die Angabe von Schemata für die Argumente und den Rückgabewert der Funktion verlangen. Diese ermöglichen die Validierung der Eingaben und Ausgaben der Funktion. Der Testing Service Consumer erlaubt das Hinzufügen von Testfällen über die Funktion \texttt{addTest()}. Tests bestehen dabei aus einem Namen, einer Liste an Funktionen und ggf. eine Liste von weiteren Tests. Funktionen werden durch ihren Namen, den Kennzeichner des bereitstellenden Testing Service Producer und ihre Argumente beschrieben. Zusätzlich kann ein erwarteter Rückgabewert angegeben werden. Dieser wird während dem Testen mit dem tatsächlichen Rückgabewert verglichen. Sollten die Werte dabei unterschiedlich sein, schlägt der Testfall fehl. Die weiteren Testfälle werden nach den Funktionen in der angegebenen Reihenfolge ausgeführt. Der Testing Service Consumer kann das Testen über die Funktion \texttt{startTesting()} beginnen. Dadurch wird ein entsprechendes \texttt{StartTesting}-Event bei den Test Service Producern ausgelöst. Über entsprechende Event Handler können Vorbereitungen für die Ausführung der Testfälle getroffen werden, bevor eine Antwort an den Testing Service Consumer gesendet wird. Sobald alle Testing Service Producer eine Antwort gesendet haben, kann der Testing Service Consumer mithilfe der Funktion \texttt{runTest()} Testfälle ausführen. Dafür werden für jeden Testfall zunächst dessen Funktionen der Reihe nach aufgerufen und danach die weiteren enthaltenen Testfälle in der angegebenen Reihenfolge ausgeführt. Nachdem alle ausgewählten Testfälle ausgeführt wurden, kann die Operation \texttt{endTesting()} des Testing Service Consumer genutzt werden, um das Testen zu beenden. Hierbei wird wieder ein entsprechendes \texttt{EndTesting}-Event von den Testing Service Producern ausgelöst, welches über Event Handler genutzt werden kann, um den Normalzustand wiederherzustellen.

\begin{figure}[tbp]
    \centering
    \begin{tikzpicture}
        \begin{class}[text width=6cm]{TestingServiceProducer}{0,0}
            \operation{+ registerFunction()}
            \operation{+ onStartTesting()}
            \operation{+ onEndTesting()}
            \operation{+ onFunctionCall()}
        \end{class}
        \begin{class}[text width=6cm]{TestingServiceConsumer}{7,0}
            \operation{+ addTest()}
            \operation{+ runTest()}
            \operation{+ startTesting()}
            \operation{+ endTesting()}
        \end{class}
    \end{tikzpicture}
    \caption{Klassendiagramm Testing Service}
    \label{figure:klassendiagramm-testing-service}
\end{figure}

Die Einbindung des Testing Service in die betrachtete Experimentkonfiguration kann z.B. über das Hinzufügen eines Testing Service Consumer bei der IDE und eines Testing Service Producer bei der Steuereinheit erfolgen. Für ein konkreteres Beispiel wird ein Microcontroller als Steuereinheit angenommen. Dieser könnte das Setzen und Auslesen der Werte seiner Pins als Funktionen für Testfälle anbieten. Diese können für die Erstellung von Testfällen genutzt werden, die während dem laufenden Experiment von der IDE ausgeführt werden können.
\section{Language Server}\label{section:konzeption:language-server}

% \begin{note}
%     \textbf{Notizen:}
%     \begin{itemize}
%         \item Erwähnung von \autoref{requirement:Language Server} und \autoref{requirement:Language Server: CrossLab-Kompatibilität}
%         \item Beschreibung der CrossLab-Services + Klassendiagramm
%         \item Konzept für die Bereitstellung von Debuggern als Laborgeräte
%         \item Beschreibung der Einbindung in die betrachtete Experimentkonfiguration
%         \item (Beschreibung möglicher Einstellungen?)
%     \end{itemize}
% \end{note}

\begin{figure}[tbp]
    \centering
    \resizebox{\textwidth}{!}{
        \begin{tikzpicture}
            \begin{class}[text width=7.5cm]{LanguageServerServiceProducer}{0,0}
                \operation{+ sendMessageLSP()}
                \operation{+ onReadFile()}
                \operation{+ onInitialize()}
                \operation{+ onMessageLSP()}
                \operation{+ onFilesystemEvent()}
            \end{class}
            \begin{class}[text width=7.5cm]{LanguageServerServiceConsumer}{8,0}
                \operation{+ initialize()}
                \operation{+ readFile()}
                \operation{+ sendMessageLSP()}
                \operation{+ sendFilesystemEvent()}
                \operation{+ onMessageLSP()}
            \end{class}
        \end{tikzpicture}
    }
    \caption{Klassendiagramm Language Server Service}
    \label{figure:klassendiagramm-language-server-service}
\end{figure}

Nach \autoref{requirement:Language Server} soll die Anbindung von Language Servern an die IDE ermöglicht werden. Dafür soll laut Unteranforderung (a) ein CrossLab-Service für die Bereitstellung und Nutzung von Language Servern entwickelt werden. Dieser soll nach Unteranforderung (b) von der IDE für die Nutzung von Language Servern verwendet werden. Bevor die konzipierten CrossLab-Services vorgestellt werden, wird zunächst ein kurzer Überblick über das \textit{\ac{LSP}} \cite{noauthor_language-server-protocol_nodate} sowie die Herausforderungen für die Anbindung an die IDE gegeben.

% sollte wahrscheinlich eher in Grundlagen erklärt werden da es bereits in den Anforderungen erwähnt wird
Das \acl{LSP} ist ein von Microsoft \cite{noauthor_microsoft_nodate} entwickeltes Protokoll zur Kommunikation zwischen einem \textit{Language Client} und einem \textit{Language Server}. Language Server ermöglichen Editorfunktionen, wie z.B. Code-Vervollständigung, Code-Navigation und Refactoring für ausgewählte Programmiersprachen. Language Clients sind meist als Teil eines Code Editors implementiert und sind mit allen Language Servern kompatibel. Das Protokoll ist für die lokale Kommunikation zwischen einem Language Client und einem Language Server entworfen, d.h. es wird angenommen, dass beide auf demselben System ausgeführt werden. Dies hat zur Folge, dass für den verteilten Anwendungsfall entsprechende Vorkehrungen getroffen werden müssen um die Funktionalität zu gewährleisten. So muss es dem Language Server ermöglicht werden auf die Dateien des Remote-Systems zugreifen zu können und umgekehrt. Es gibt Language Server, die den Quellcode lokal verarbeiten bzw. sogar kompilieren und basierend auf diesen Ergebnissen Antworten an den Language Client zurücksenden. Dazu werden Dateien auf dem Server verwendet, auf welche der Language Client ggf. Zugriff benötigt. Unter Betrachtung dieser Herausforderungen wird im Folgenden der \textit{Language Server Service} vorgestellt.

\autoref{figure:klassendiagramm-language-server-service} zeigt ein Klassendiagramm für den Language Server Service. Die Funktion \texttt{initialize()} des Language Server Service Consumer kann für die Initialisierung eines Language Servers verwendet werden. Dabei wird der aktuelle Ordner des Nutzers und dessen Pfad bzw. URI sowie ggf. Konfigurationsoptionen für die Initialisierung des Language Server übergeben. Dadurch wird ein enstprechendes \texttt{Initialize}-Event von dem verbundenen Language Server Service Producer ausgelöst. Über entsprechende Event Handler kann dieses abgefangen und für die Initialisierung des Language Server verwendet werden. Dabei kann u.a. der übergebene Ordner in dem lokalen Dateisystem hinterlegt werden und der Language Server mit den angegebenen Konfigurationsoptionen gestartet werden. Die übergebene URI des Ordners kann zur Umschreibung von URIs innerhalb der Nachrichten des \ac{LSP} verwendet werden. Diese unterscheiden sich ggf. zwischen den Systemen des Language Server Service Producer und des Language Server Service Consumer. Nachdem der Language Server initialisiert wurde wird eine Antwort an den Language Server Service Consumer gesendet. Dateisystem-Events innerhalb des verwendeten Ordners auf der Seite des Language Server Service Consumer müssen an den Language Server Service Producer mithilfe der Funktion \texttt{sendFilesystemEvent()} gesendet werden. Diese umfassen die Erstellung, Änderung und Löschung von Dateien und Ordnern. Die Behandlung dieser Dateisystem-Events kann über entsprechende Event Handler erfolgen. Sobald die Initialisierung erfolgt ist kann das \ac{LSP} mit den Konfigurationsoptionen gestartet werden. Der Austausch der \ac{LSP} Nachrichten erfolgt dabei über die Funktion \texttt{sendMessageLSP()}. Zur Behandlung dieser Nachrichten können Event Handler für die entsprechenden \texttt{MessageLSP}-Events registriert werden. Da der Nutzer ggf. Zugriff auf die lokalen Dateien des Language Server benötigt, z.B. für die Betrachtung von Bibliotheksdateien, ermöglicht der Language Server Service Consumer das Lesen von Dateien über die Funktion \texttt{readFile()}. Das dadurch ausgelöste \texttt{ReadFile}-Event kann auf der Seite des Language Server Service Producer durch einen entsprechenden Event Handler behandelt werden.

Die betrachtete Experimentkonfiguration wird zunächst um ein weiteres Laborgerät erweitert. Dieses stellt einen Language Server über einen entsprechenden Language Server Service Producer bereit. Die IDE wird um einen Language Server Service Consumer erweitert. Die beiden Laborgeräte werden dann über den Language Server Service miteinander verbunden, wodurch die IDE die Funktionen des bereitgestellten Language Server verwenden kann.