\chapter{Konzeption}\label{section:konzeption}

In diesem Kapitel werden die Konzepte für die verschiedenen Funktionen und CrossLab-Services der IDE vorgestellt. Dafür wird in \autoref{section:konzeption:crosslab-kompatibilität} zunächst dargelegt wie die CrossLab-Kompatibilität erreicht werden kann. Danach wird in \autoref{section:konzeption:kollaboration} der Ansatz für die Bereitstellung von Kollaborationsmöglichkeiten beschrieben. In \autoref{section:konzeption:dateisystem} wird das Konzept für das Dateisystem erläutert. Darauf folgend wird in \autoref{section:konzeption:kompilierung} die Funktionsweise der Kompilierung dargelegt. Danach wird in \autoref{section:konzeption:debugging} das Konzept zur Ermöglichung von Debugging beschrieben. In \autoref{section:konzeption:testen} wird die Vorgehensweise für die Hinterlegung und Ausführung von Tests erläutert. Weiterhin wird in \autoref{section:konzeption:language-server} die Einbindung von Language Servern dargelegt. Schließlich wird in \autoref{section:konzeption:simulation} ein Konzept für die Simulation von Steuereinheiten beschrieben.

\input{content/05_konzeption/051_crosslab_kompatibilität.tex}
\section{Kollaboration}\label{section:konzeption:kollaboration}

Es gibt viele verschiedene Methoden zur Synchronisierung von Daten zwischen mehreren Teilnehmern. Beispiele derartiger Methoden sind \emph{\ac{OT}} \cite{sun_operational_1998}, \emph{Differential Synchronization} \cite{fraser_differential_2009} und \emph{\ac{CRDTs}} \cite{shapiro_conflict-free_2011}. Aufgrund der Tatsache, dass jede dieser Methoden ihre Vor- und Nachteile besitzt, sollte die entwickelte Lösung unabhängig von dem zugrunde liegenden Synchronisationsalgorithmus sein. Der Kollaborationsdienst ist als Consumer, Producer und Prosumer nutzbar. Bei Experimenten mit einem zentralen Synchronisationspunkt (z.B. bei der Verwendung von \ac{OT}) bietet dieser einen Producer an während die restlichen Geräte, die an der Kollaboration teilnehmen, einen Consumer nutzen. Dahingegen nutzen bei Experimenten ohne einen zentralen Synchronisationspunkt (z.B. bei der Verwendung von \ac{CRDTs}) alle Geräte, die an der Kollaboration teilnehmen, einen Prosumer. In der Experimentbeschreibung sollten bei den Konfigurationen der Verbindungen von Kollaborationsdiensten stets die Synchronisationsmethode sowie die sogenannten \emph{Räume} angegeben werden, die in der Verbindung genutzt werden sollen. Räume besitzen einen eindeutigen Namen und ein JSON-Objekt, das zwischen allen Teilnehmern innerhalb des Raums synchronisiert wird. Das synchronisierte JSON-Objekt kann z.B. Ordner oder Dateien abbilden, die von den Teilnehmern geteilt werden. Jeder Raum besitzt einen sogenannten \emph{Provider}. Dieser nutzt die in der Konfiguration der Verbindung angegebene Synchronisationsmethode um den Inhalt des Raums zu synchronisieren. Weiterhin bietet der Provider eine Schnittstelle um Statusinformationen auszutauschen. Diese Informationen sind teilnehmerspezifisch, d.h. sie können nur von dem jeweiligen Teilnehmer aktualisiert werden. Ein Beispiel für derartige Statusinformationen ist z.B. die aktuelle Position eines Teilnehmers innerhalb einer Datei.

\begin{figure}[htbp]
    \centering
    \begin{sequencediagram}
        \newthread{consumer}{Consumer}
        \newthreadShift{producer}{Producer}{4cm}

        \begin{call}{consumer}{erstelle Räume}{consumer}{}
        \end{call}

        \prelevel\prelevel

        \begin{call}{producer}{erstelle Räume}{producer}{}
        \end{call}

        \postlevel

        \begin{call}{consumer}{sende ID}{producer}{}
            \begin{call}{producer}{registriere Consumer}{producer}{return}
            \end{call}
        \end{call}

        \postlevel

        \begin{call}{consumer}{starte Synchronisation}{producer}{}
        \end{call}
    \end{sequencediagram}
    \caption{Initialisierung Kollaboration}\label{abbildung:initialisierung-kollaboration}
\end{figure}

Die Kommunikation zwischen den Kollaborationsteilnehmern erfolgt über ein entsprechendes Nachrichtenprotokoll. In \autoref{abbildung:initialisierung-kollaboration} ist der Verbindungsaufbau zwischen einem Consumer und einem Producer dargestellt. Zunächst erstellen beide die in der Verbindungskonfiguration festgelegten Räume. Dabei verknüpft der Consumer den Raum direkt mit der Verbindung. Der Producer hingegen wartet auf die Initialisierungsnachricht des Consumer, welche dessen ID beinhaltet. Die ID kann dann genutzt werden um den Consumer dem entsprechenden Räumen zuzuweisen. Sobald der Producer das erfolgreiche Ende der Initialisierung an den Consumer meldet beginnt dieser mit der Synchronisation. Da die verschiedenen Synchronisationsmethoden ggf. unterschiedliche Nachrichtenformate besitzen wird eine allgemeine Nachricht definiert, die dann die spezifischen Informationen für die zugrundeliegende Methode beinhalten. Daraus folgt auch, dass es nicht möglich ist einen Consumer mit einem Producer zu verbinden, der eine andere Synchronisationsmethode verwendet. Weiterhin ist darauf zu achten, dass ggf. mehrere Provider für eine Synchronisationsmethode benötigt werden. Dies ist z.B. der Fall bei Methoden mit einem zentralen Synchronisationspunkt.

Während die Behandlung von Aktualisierungen der Räume durch das Protokoll der zugrundeliegenden Synchronisationsmethode erfolgt, wird für die Behandlung von Statusaktualisierungen der Teilnehmer ein allgemeines Protokoll eingeführt. Dabei ist der Status eines Teilnehmers immer als ein JSON-Objekt darstellbar, wobei ein Wert von \texttt{null} angibt, dass der Teilnehmer nicht mehr erreichbar ist. Zu Beginn der Synchronisation schicken Consumer ihren aktuellen Status an den Producer. Dieser speichert den aktuellen Status und sendet ihn an die restlichen Consumer. Wenn sich der Status eines Consumer kann er entweder den kompletten Status an den Producer senden oder nur die vorgenommenen Änderungen. Der Producer aktualisiert seine gespeicherten Statusinformationen für den Consumer und leitet die Änderungen an die restlichen Consumer weiter. Diese aktualisieren ebenfalls ihre lokalen Statusinformationen und können dann auf die vorgenommenen Änderungen reagieren. Sollte ein Consumer nicht innerhalb eines vordefinierten Zeitraums seinen Status aktualisieren wird dieser auf \texttt{null} gesetzt und die Änderung an die restlichen Consumer weitergeleitet.
\section{Dateisystem}\label{section:konzeption:dateisystem}

Nach \autoref{requirement:Dateisystem} soll die IDE ein integriertes Dateisystem besitzen. Für dieses werden im Folgenden zwei verschiedene Lösungsansätze in beschrieben, ein client-seitiger und eine server-seitiger.

Für den client-seitigen Lösungsansatz bietet sich eine Speicherung der Dateien des Nutzers innerhalb des Browsers an. Für die persistente Speicherung von Daten innerhalb des Browsers kann die Indexed Database API \cite{noauthor_indexed-database-api_nodate} genutzt werden. Diese wird von allen aktuellen Browsern unterstützt und erlaubt die langfristige Speicherung von größeren Datenmengen. Der Vorteil des client-seitigen Ansatzes ist die Tatsache, das kein weiterer Speicherplatz für die Nutzer bereitgestellt werden muss, da die Daten auf dem Rechner des Nutzers gespeichert werden. Allerdings sind die Daten sowohl an die Domain der IDE, das Gerät des Nutzers als auch an den spezifischen Browser gebunden und müssen durch entsprechendes Exportieren und Importieren übertragen werden.

Der server-seitige Lösungsansatz basiert darauf jedem Nutzer einen entsprechenden Bereich zuzuteilen, in welchem seine Dateien gespeichert werden. Dies kann entweder über das Dateisystem des Servers oder über eine Datenbank geschehen. Der Vorteil dieser Art der Datenspeicherung liegt darin, dass sie geräteunabhängig ist. Allerdings werden für die Speicherung der Nutzerdaten entsprechender Speicherplatz auf dem Server benötigt wodurch höhere Kosten und ein höherer Verwaltungsaufwand bestehen. Zudem muss sichergestellt werden, dass das System nicht ausgenutzt werden kann.

\autoref{requirement:Dateisystem: CrossLab-Kompatibilität} verlangt die Entwicklung von CrossLab-Services für die Bereitstellung und Nutzung von Dateisystemen. Dementsprechend werden im Folgenden der \textit{Dateisystem Service Producer} und der \textit{Dateisystem Service Consumer} beschrieben. Die grundlegende Kommunikation zwischen den beiden Services geschieht über den Austausch von Nachrichten. Diese bestehen aus einem Typen und dem dazugehörigen Inhalt. Durch die Definition der Nachrichten kann eine Validierung dieser innerhalb der Services geschehen. In \autoref{figure:klassendiagramm-dateisystem-services} ist ein Klassendiagramm für die beiden Services dargestellt. Aus diesem können die bereitgestellten Operationen eines Dateisystems abgelesen werden. So müssen Dateisysteme die Erstellung von Ordnern und Dateien, das Lesen, Verschieben und Löschen dieser sowie das Schreiben von Dateien unterstützen. Dabei besitzt der Consumer Funktionen um die einzelnen Operationen auszuführen, während der Producer die Möglichkeit bietet auf eingehende Anfragen zu reagieren und entsprechende Antworten an den Consumer zu senden.

\begin{figure}[tbp]
    \centering
    \begin{tikzpicture}
        \begin{class}[text width=8.25cm]{DateisystemServiceConsumer}{-4,0}
            \operation{+ createDirectory(path, content)}
            \operation{+ delete(path)}
            \operation{+ move(path, newPath)}
            \operation{+ readDirectory(path)}
            \operation{+ readFile(path)}
            \operation{+ writeFile(path, content)}
        \end{class}
        \begin{class}[text width=6cm]{DateisystemServiceProducer}{4,0}
            \operation{+ onRequest(request)}
            \operation{+ send(message)}
        \end{class}
    \end{tikzpicture}
    \caption{Klassendiagramm Dateisystem Services}
    \label{figure:klassendiagramm-dateisystem-services}
\end{figure}

Eine mögliche Erweiterung der Services besteht in der Unterstützung mehrerer Kommunikationspartner. Sollten bei der Erstellung eines Experiments mehrere Verbindungen hergestellt werden so wird für jeden Kommunikationspartner ein eindeutiger Kennzeichner erstellt. Dieser wird dann in den jeweiligen Funktionen mit angegeben um die Nachrichten an den korrekten Kommunikationspartner zu schicken. Zudem könnte auch die Überwachung von Ordnern und Dateien angeboten werden. Dafür würde der Consumer eine entsprechende Anfrage an den Producer senden, in welcher er die zu überwachenden Pfade angibt. Sollte dann eine Änderung in einem der überwachten Pfade auftreten wird eine entsprechende Nachricht vom Producer an den Consumer gesendet.

Nach \autoref{requirement:Dateisystem: Kollaboration} soll das Dateisystem das Teilen von Ordnern mit anderen Nutzern innerhalb eines Experiments unterstützen. Dafür können die in \autoref{section:konzeption:kollaboration} beschriebenen Kollaborationsmechanismen genutzt werden. Beim Erstellen eines Experiments öffnen Nutzer einen Raum zum Teilen ihrer Ordner. Die geteilte Datenstruktur hat dabei die Kennzeichner der verschiedenen Teilnehmer als Schlüssel mit den geteilten Ordnern als den dazugehörigen Wert. Am Anfang einer Sitzung hat ein Nutzer noch keine geteilten Ordner und setzt somit seinen eigenen Wert auf ein leeres Objekt. Wenn ein Nutzer einen Ordner teilt so wird dieser den anderen Nutzern angezeigt und sie können mit den Inhalt einsehen und bearbeiten. Die Implementierung der Synchronisation ist hierbei abhängig von dem verwendeten Code Editor. Weiterhin kann auch die aktuelle Position von Nutzern über deren Zustandsinformationen mit den anderen Nutzern geteilt werden. Die Position kann dann den anderen Nutzern innerhalb der entsprechenden Datei angezeigt werden.

% Das bisherige WIDE System nutzt ein projektbasiertes Dateisystem. In diesem muss der Name eines Projektes einzigartig sein. Weiterhin können nicht mehrere Projekte gleichzeitig geöffnet werden. Zudem werden Metadaten zu einem Projekt gespeichert. Diese umfassen das elektromechanische Modell, die Steuereinheit sowie die Programmiersprache, die bei der Erstellung des Projekts genutzt bzw. ausgewählt wurden. Dadurch ist es möglich dem Nutzer nur die Projekte anzuzeigen, die in dem aktuellen Experiment von Interesse sein könnten. In der neuen CrossLab Architektur ist es nun allerdings möglich mehrere Steuereinheiten und elektromechanische Modelle sowie weitere Laborgeräte zu einem Experiment zusammenzustellen. Deshalb sollte es dem Nutzer ermöglicht werden mehrere Projekte gleichzeitig öffnen und bearbeiten zu können. Ein weiteres Feature von WIDE ist die Bereitstellung von Beispielprojekten. Diese können von Nutzern verwendet werden um einen Einblick in die Programmierung einer gegebenen Steuereinheit zu bekommen. Um dieses Feature weiterhin unterstützen zu können sollte eine Konfigurationsmöglichkeit gegeben werden, welche die Bereitstellung derartiger Beispiele ermöglicht.
\section{Kompilierung}\label{section:konzeption:kompilierung}

% Die Kompilierung des Quellcodes kann vom CM übernommen werden. Alternativ steht auch hier die Möglichkeit eines externen Servers zur Verfügung. Weiterhin kann auch die Möglichkeit der Kompilierung innerhalb des Browsers des Nutzers erforscht werden. Hierbei gibt es bereits eine Version des C-Compilers clang in WebAssembly. Diese ermöglicht jedoch in der aktuellen Form nur die Kompilierung von C/C++ Code zu WebAssembly. Im Falle des GOLDi Remotelab müsste allerdings eine Kompilierung des Quellcodes für die verwendeten Microcontroller erfolgen. Diese wird normalerweise mit Hilfe des Compilers avr-gcc durchgeführt. Allerdings bietet auch der clang Compiler ein avr Ziel an, welches jedoch einen experimentellen Status hat.

Für die Kompilierung des Quellcodes können bereits vorhandene Compiler verwendet werden. Allerdings ist es notwendig diese über eine entsprechende Schnittstelle in den Code Editor zu integrieren. Dadurch soll es dem Nutzer ermöglicht werden seinen Quellcode zu kompilieren. Hierbei ist zu beachten, dass falls mehrere Steuereinheiten in einem Experiment vorhanden sind der Nutzer wählen kann für welche das jeweilige Programm kompiliert werden soll. Weiterhin können die Compiler über verschiedene Konfigurationsmöglichkeiten an die Bedürfnisse des Nutzers bzw. des Experiments angepasst werden.

\subsection{CrossLab Compilation Service}

Der CrossLab Compilation Service besitzt zwei Rollen: Einen Konsumenten und einen Produzenten. Der Konsument kann Kompilieranfragen an einen Produzenten stellen, welcher diese entgegennimmt und ein entsprechendes Ergebnis an den Konsumenten zurücksendet. Dabei ist zu beachten, dass ein Konsument mit mehreren Produzenten verbunden sein kann. Die Konfiguration der erwarteten Ergebnisformate sowie der Optionen des Compilers werden über die Servicekonfiguration auf der Ebene einer Verbindung zwischen Konsument und Produzent vorgenommen. Es ist den Produzenten freigestellt, welche Konfigurationsmöglichkeiten und welche Ergebnisformate für Konsumenten angeboten werden.
\section{Debugging} \label{konzeption:debugging}

% Notiz: Debugging Server benötigt Zugriff auf zur Kompilierung verwendete Dateien (sollte wahrscheinlich zusammen mit Compiler auf einem System laufen)

% \begin{itemize}
%     \item Debugging von realem Microcontroller über RPi (avr-gdb)
%     \item Debugging von virtuellem Microcontroller über Cloud-instanziierbares Gerät oder über avr-gcc im Browser
% \end{itemize}

% Das Debuggen von Programmen auf den vorhandenen Microcontrollern gestaltet sich schwierig. Eine Möglichkeit ist die Nutzung der Bibliothek avr\_debug. Diese wird zusammen mit dem Programm kompiliert und auf den Microcontroller hochgeladen. Dort erstellt sie ein Interface für den Debugger gdb. Dieses Interface nutzt die Serielle Schnittstelle des Microcontrollers zur Kommunikation mit gdb. Das CM agiert in diesem Szenario als Schnittstelle zwischen gdb und unserer IDE. Ein Nachteil dieses Vorgehens ist der hohe Speicherverbrauch der Bibliothek, welcher die Anzahl möglicher Programme einschränkt. Allerdings ist ein Vorteil dieses Ansatzes, dass keine zusätzlichen Kosten durch die Anschaffung externe Debugger entstehen.

% Ein weiteres Problem, was beim Debuggen eines laufenden Experimentes beachtet werden muss, ist die fortlaufende Ansteuerung von weiteren Geräten. Nehmen wir als Beispiel ein einfaches Experiment bestehend aus einem Microcontroller und einem 3-Achs-Portal. Wenn wir das Program des Microcontrollers unterbrechen, während dieser den Portalkran aktiv nach rechts bewegt, so wird diese Bewegung nicht unterbrochen. Um sicherzustellen, dass die Signale von Aktoren während eines Breakpoints nicht an andere Geräte weitergeleitet werden müssen die anderen Geräte entsprechend benachrichtigt werden.
\section{Testen}\label{section:konzeption:testen}
\section{Language Server}\label{section:konzeption:language-server}

% \begin{itemize}
%     \item Language Server über Cloud-instanziierbares Gerät oder über clangd im Browser
% \end{itemize}

% Eine Möglichkeit zur Einbindung von Language Servern ist es, diese auf dem CM zu installieren und sich dann zu diesen zu verbinden. Allerdings ist das Language Server Protokoll nur auf einen einzigen Client pro Server ausgelegt. Daher müsste in einem Experiment mit mehreren Nutzern für jeden dieser Nutzer eine eigene Instanz des Language Server erstellt werden. Eine spezielle Alternative könnte die Verwendung des zu WebAssembly kompilerten Language Servers clangd darstellen. Dieser könnte im Browser des jeweiligen Nutzers instanziiert werden, wodurch keine zusätzliche Last auf dem CM entsteht. Allerdings muss hierbei die neu entstandene Last auf dem Rechner des Nutzers in Betracht gezogen werden. Eine weitere Möglichkeit ist die Bereitstellung weiterer Server zur Bereitstellung der Language Server.

Das \textit{\ac{LSP}} \todoaddref[]{Language Server Protocol} ist ein von Microsoft entwickeltes Protokoll zur Kommunikation zwischen einem \textit{Language Client} und einem \textit{Language Server}. Language Server sind meistens für eine spezielle Programmiersprache entwickelt und stellen Dienste wie z.B. Autovervollständigung für diese an. Language Clients sind meist als Teil eines Code Editors implementiert und sind unabhängig von speziellen Programmiersprachen. Das Protokoll ist für die lokale Kommunikation zwischen einem Language Client und einem Language Server entworfen, d.h. es wird angenommen, dass beide auf demselben System ausgeführt werden. Dies hat zur Folge, dass für den verteilten Anwendungsfall entsprechende Vorkehrungen getroffen werden müssen um die Funktionalität zu gewährleisten. So muss es dem Language Server ermöglicht werden auf die Dateien des Remote-Systems zugreifen zu können und umgekehrt. Es gibt Language Server, die den Quellcode lokal verarbeiten bzw. sogar kompilieren und basierend auf diesen Ergebnissen Antworten an den Language Client zurücksenden. Dazu werden Dateien auf dem Server verwendet, auf welche der Language Client gegebenenfalls Zugriff benötigt.
\section{Simulation}\label{konzeption:simulation}