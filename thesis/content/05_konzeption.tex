\chapter{Konzeption}\label{section:konzeption}

In diesem Kapitel werden die Konzepte für die verschiedenen Funktionen und CrossLab-Services der IDE vorgestellt. Dabei wird in \autoref{section:konzeption:experimentkonfiguration} zunächst festgelegt welche Experimentkonfigurationen in den darauf folgenden Abschnitten betrachtet werden. Daraufhin wird in \autoref{section:konzeption:crosslab-kompatibilität} dargelegt wie die CrossLab-Kompatibilität der IDE erreicht werden kann. Danach wird in \autoref{section:konzeption:kollaboration} der Ansatz für die Bereitstellung von Kollaborationsmöglichkeiten beschrieben. In \autoref{section:konzeption:dateisystem} wird das Konzept des Dateisystems erläutert. Darauf folgend wird in \autoref{section:konzeption:kompilierung} die Funktionsweise der Kompilierung dargelegt. Anschließend wird in \autoref{section:konzeption:debugging} das Konzept zur Ermöglichung des Debuggens beschrieben. Weiterhin wird in \autoref{section:konzeption:testen} die Vorgehensweise für die Konfiguration und Ausführung von Tests erläutert. Schließlich wird in \autoref{section:konzeption:language-server} die Einbindung von Language Servern dargelegt.

\section{Experimentkonfiguration}\label{section:konzeption:experimentkonfiguration}

Die IDE soll in verschiedenen Experimentkonfigurationen nutzbar sein. In \autoref{requirement:Eigenständig nutzbar} wird verlangt, dass Experimente, welche nur die IDE als Laborgerät enthalten, ausführbar sein sollen. Weiterhin soll die IDE in Verbindung mit Steuereinheiten, wie z.B. Microcontrollern und FPGAs, genutzt werden können. Diese können wiederum mit steuerbaren Laborgeräten verbunden sein. Viele der Funktionen, die in den folgenden Abschnitten besprochen werden sind nur in speziellen Fällen in einer eigenständigen Variante der IDE nutzbar. Weiterhin ist die Nutzung von Steuereinheiten ohne verbundene steuerbare Laborgeräte oftmals nicht zielführend. Daher wird für die folgenden Abschnitte von einem Experiment mit der IDE sowie mindestens einer Steuereinheit und mindestens einem steuerbaren Laborgerät ausgegangen. Falls neue Laborgeräte oder CrossLab-Services eingeführt werden, wird deren mögliche Einbindung in die betrachtete Experimentkonfiguration beschrieben.
\section{CrossLab-Kompatibilität}\label{section:konzeption:crosslab-kompatibilität}

% \begin{note}
%     \textbf{Notizen:}
%     \begin{itemize}
%         \item Erwähnung \autoref{requirement:Erweiterbarkeit}
%         \item Vergleich der möglichen Anbindungen der IDE als Laborgerät
%         \item Grundlegender Kommunikationsmechanismus der entwickelten Services
%     \end{itemize}
% \end{note}

Für die verschiedenen Funktionen der zu entwickelnden IDE sollen entsprechende CrossLab-Services entwickelt und von der IDE verwendet werden. \autoref{requirement:Erweiterbarkeit} verlangt zudem die Erweiterbarkeit der IDE um zusätzliche CrossLab-Services. Um dies zu erreichen gibt es verschiedene Möglichkeiten. Angenommen die IDE unterstützt das Hinzufügen von Erweiterungen. So könnte eine zentrale Komponente genutzt werden, um alle vorhandenen CrossLab-Services zu verwalten. Diese kann von der IDE selbst oder von einer entsprechenden Erweiterung bereitgestellt werden. Diese zentrale Komponente könnte entweder selbst in der Lage sein CrossLab-Services, die von anderen Erweiterungen bereitgestellt werden, zum Laborgerät hinzuzufügen oder sie könnte eine entsprechende Schnittstelle bereitstellen, die es anderen Erweiterungen ermöglicht das Laborgerät um zusätzliche Services zu erweitern. In der ersten Variante könnte die zentrale Komponente einschränken welche CrossLab-Services zu dem Laborgerät hinzugefügt werden. So könnte z.B. für eine IDE mit verschiedenen Erweiterungen in der Experimentkonfiguration eine Liste von Erweiterungen festgelegt werden, deren CrossLab-Services geladen werden sollen. Somit müssen ggf. nicht alle Erweiterungen für alle Experimente geladen werden. Der beschriebene Ablauf ist in \autoref{abbildung:initialisierung-laborgerät-ide} dargestellt.

\begin{figure}[tbp]
    \centering
    \begin{sequencediagram}
        \newthread{ide}{IDE}
        \newinst[4]{erweiterung}{Erweiterung}

        \begin{call}{ide}{lese Experimentkonfiguration}{ide}{}
        \end{call}

        \begin{sdblock}{alt}{[Erweiterung inaktiv]}
            \begin{call}{ide}{starte Erweiterung}{erweiterung}{}
            \end{call}
        \end{sdblock}

        \begin{call}{ide}{lade CrossLab-Services}{erweiterung}{}
        \end{call}
    \end{sequencediagram}
    \caption{Initialisierung Laborgerät IDE}\label{abbildung:initialisierung-laborgerät-ide}
\end{figure}

Weiterhin besteht die Frage welche Art eines Laborgeräts für die Einbindung der IDE in die CrossLab-Architektur am besten geeignet ist. Dabei ist zu beachten, dass die IDE von mehreren Nutzern gleichzeitig und auch eigenständig in Experimenten verwendet werden soll (sh. \autoref{requirement:Eigenständig nutzbar} und \autoref{requirement:Kollaboration}). Daher kommt nur die Einbindung als cloud- oder edge-instanziierbares Gerät in Frage. Die Instanzen von cloud-instanziierbaren Laborgeräten werden auf Servern ausgeführt und benötigen dementsprechende Ressourcen. Aufgrund dieser Tatsache kann es ggf. dazu kommen, dass Nutzer warten müssen bis die entsprechenden Serverkapazitäten vorhanden sind. Dies könnte die Benutzererfahrung verschlechtern. Eine Einbindung der IDE als edge-instanziierbares Laborgerät kann dieses Problem umgehen, da die Instanzen auf der Seite des Nutzers ausgeführt werden. Allerdings muss dabei beachtet werden, dass für eine Implementierung der IDE als edge-instanziierbares Gerät die grundlegenden Funktionen dieser komplett im Browser des Nutzers ausgeführt werden müssen. Zu den grundlegenden Funktionen gehören dabei ein Dateisystem für die Bearbeitung und persistente Speicherung von Dateien und Ordnern sowie der Code Editor zum Editieren von Dateien. Zusätzliche Funktionen, wie z.B. Kompilierung und Debuggen, müssen in den meisten Fällen auf Servern ausgeführt werden und benötigen somit weiterhin entsprechende Ressourcen. Eine Implementierung der IDE als cloud-instanziierbares Gerät kann allerdings die Anbindung und Nutzung von Compilern, Debuggern und Language Servern stark vereinfachen, da diese auf demselben System laufen können. Die in den folgenden Abschnitten beschriebenen Konzepte sollen in beiden Varianten angewandt werden können.

\section{Kollaboration}\label{section:konzeption:kollaboration}

\usetikzlibrary{arrows.meta}

% \begin{note}
%     \textbf{Notizen:}
%     \begin{itemize}
%         \item Erwähnung von \autoref{requirement:Kollaboration}
%         \item Beschreibung der grundlegenden Konzepte
%         \item Klassendiagramm kollaborative Datentypen
%         \item Klassendiagramm Collaboration Service \& dazugehörige Klassen
%         \item Beschreibung der verschiedenen Interaktionen zwischen Producer und Consumer
%               \begin{itemize}
%                   \item Beginn der Synchronisation
%                   \item Synchronisation des geteilten JSON-Objekts
%                   \item Austausch von Zustandsinformationen
%                   \item high-level Sequenzdiagramme + low-level Beschreibung
%               \end{itemize}
%         \item Beschreibung der Einbindung in die betrachtete Experimentkonfiguration
%         \item Nutzung von JSON-Schema für Beschreibung des geteilten JSON-Objekts
%     \end{itemize}
% \end{note}

Nach \autoref{requirement:Kollaboration} soll die IDE die Echtzeit-Kollaboration mit anderen Laborgeräten innerhalb eines Experiments unterstützen. Dafür soll laut Unteranforderung (a) ein entsprechender CrossLab-Service entwickelt werden. Dieser soll nach Unteranforderung (b) für die Ermöglichung der Echtzeit-Kollaboration von der IDE verwendet werden.

Es gibt viele verschiedene Methoden zur Synchronisierung von Daten zwischen mehreren Teilnehmern. Beispiele derartiger Methoden sind \emph{\ac{OT}} \cite{sun_operational_1998}, \emph{Differential Synchronization} \cite{fraser_differential_2009} und \emph{\acp{CRDT}} \cite{shapiro_conflict-free_2011}. Aufgrund der Tatsache, dass jede dieser Methoden eigene Vor- und Nachteile besitzt, sollte die entwickelte Lösung möglichst unabhängig von dem zugrunde liegenden Synchronisationsalgorithmus sein, damit für den jeweiligen Anwendungsfall der passende Algorithmus verwendet werden kann\todo{Feedback einholen}. Daher werden zunächst die grundlegenden Konzepte beschrieben, bevor der entwickelte CrossLab-Service vorgestellt wird.

\begin{figure}[tbp]
    \centering
    \resizebox{\textwidth}{!}{\begin{tikzpicture}
            \begin{interface}[text width=4cm]{CollaborationType}{0,0}
                \operation{+ toJSON()}
                \operation{+ onUpdate()}
            \end{interface}
            \begin{class}[text width=5cm]{CollaborationObject}{-6,2.25}
                \operation{+ setProperty()}
                \operation{+ getProperty()}
                \operation{+ deleteProperty()}
            \end{class}
            \begin{class}[text width=5cm]{CollaborationNull}{-6,-0.75}
            \end{class}
            \begin{class}[text width=5cm]{CollaborationArray}{-6,-2.25}
                \operation{+ push()}
                \operation{+ get()}
                \operation{+ delete()}
            \end{class}
            \begin{class}[text width=5cm]{CollaborationNumber}{6,2.25}
                \operation{+ set()}
            \end{class}
            \begin{class}[text width=5cm]{CollaborationString}{6,0}
                \operation{+ set()}
                \operation{+ insert()}
                \operation{+ delete()}
            \end{class}
            \begin{class}[text width=5cm]{CollaborationBoolean}{6,-3.35}
                \operation{+ set()}
            \end{class}
            \draw[umlcd style dashed line, -{Triangle[length=2.5mm,open]}] (CollaborationObject.east) -- ([yshift=5mm] CollaborationType.west);
            \draw[umlcd style dashed line, -{Triangle[length=2.5mm,open]}] (CollaborationNull.east) -- (CollaborationType.west);
            \draw[umlcd style dashed line, -{Triangle[length=2.5mm,open]}] (CollaborationArray.east) -- ([yshift=-5mm] CollaborationType.west);
            \draw[umlcd style dashed line, -{Triangle[length=2.5mm,open]}] (CollaborationNumber.west) -- ([yshift=5mm] CollaborationType.east);
            \draw[umlcd style dashed line, -{Triangle[length=2.5mm,open]}] (CollaborationString.west) -- (CollaborationType.east);
            \draw[umlcd style dashed line, -{Triangle[length=2.5mm,open]}] (CollaborationBoolean.west) -- ([yshift=-5mm] CollaborationType.east);
        \end{tikzpicture}}
    \caption{Klassendiagramm kollaborative Datentypen}
    \label{figure:klassendiagramm-kollaborative-datentypen}
\end{figure}

\begin{figure}[tbp]
    \centering
    \begin{tikzpicture}
        \begin{class}[text width=7cm]{CollaborationServiceProducer}{-4,0}
        \end{class}
        \begin{class}[text width=7cm]{CollaborationServiceConsumer}{4,0}
            \operation{+ getAwareness()}
            \operation{+ joinRoom()}
            \operation{+ executeTransaction()}
            \operation{+ valueToCollaborationType()}
            \operation{+ getProperty()}
            \operation{+ onUpdate()}
        \end{class}
        \begin{class}[text width=7cm]{Room}{0,-5}
            \attribute{+ awareness: Awareness}
            \operation{+ addParticipant()}
            \operation{+ removeParticipant()}
            \operation{+ valueToCollaborationType()}
            \operation{+ executeTransaction()}
            \operation{+ startSynchronization()}
            \operation{+ getProperty()}
            \operation{+ onUpdate()}
        \end{class}
        \begin{interface}[text width=7cm]{Awareness}{-4,-14}
            \operation{+ getLocalState()}
            \operation{+ setLocalState()}
            \operation{+ setLocalStateField()}
            \operation{+ getStates()}
            \operation{+ onChange()}
            \operation{+ onUpdate()}
        \end{interface}
        \begin{class}[text width=7cm]{AwarenessProvider}{-4,-11}
            \implement{Awareness}
            \operation{+ applyUpdate()}
            \operation{+ encodeStates()}
        \end{class}
        \begin{class}[text width=7cm]{CollaborationProvider}{4,-11}
            \operation{+ handleCollaborationMessage()}
            \operation{+ startSynchronization()}
            \operation{+ executeTransaction()}
            \operation{+ valueToColloraborationType()}
            \operation{+ getProperty()}
            \operation{+ onCollaborationMessage()}
            \operation{+ onUpdate()}
        \end{class}
        \draw[stroke] ([xshift=-10mm]CollaborationServiceProducer.south) -- ([xshift=-10mm] CollaborationServiceProducer.south |- , |- Room.west) -- (Room.west) node [above, xshift=-6mm] () {rooms} node [below, xshift=-4mm] () {0..*};
        \draw[stroke] ([xshift=10mm] CollaborationServiceConsumer.south) -- ([xshift=10mm] CollaborationServiceConsumer.south |- , |- Room.east) -- (Room.east) node [above, xshift=6mm] () {rooms} node [below, xshift=4mm] () {0..*};
        \draw[stroke] ([xshift=-5mm] Room.south) -- (-0.5,-10.5) -- (-4,-10.5) -- (AwarenessProvider.north) node [left, yshift=2mm] () {awarenessProvider} node [right, yshift=2mm] () {1};
        \draw[stroke] ([xshift=5mm] Room.south) -- (0.5,-10.5) -- (3.5,-10.5) -- ([xshift=-5mm] CollaborationProvider.north) node [left, yshift=2mm] () {1} node [right, yshift=2mm] () {collaborationProvider};
    \end{tikzpicture}
    \caption{Klassendiagramm Collaboration Service}
    \label{figure:klassendiagramm-collaboration-service}
\end{figure}

Im Folgenden wird für die Erklärung der Konzepte von mehreren an der Kollaboration teilnehmenden Clients und einem zentralen Synchronisationsserver ausgegangen. Die Clients besitzen die Möglichkeit, sogenannte \textit{Räume} zu erstellen bzw. ihnen beizutreten. Jeder Raum verwaltet ein geteiltes JSON-Objekt. Dieses nutzt Strings als Schlüssel mit entsprechenden kollaborativen Datentypen als Werten. Diese kollaborativen Datentypen sind in \autoref{figure:klassendiagramm-kollaborative-datentypen} dargestellt. \todo{ggf. Position anpassen}Dabei gibt es jeweils einen kollaborativen Datentypen für die JSON-kompatiblen Datentypen Object, Array, Number, String, Boolean und Null. Jeder kollaborative Datentyp bietet neben speziellen Funktionen zur Bearbeitung dessen auch eine Funktion zur Umwandlung in den zugrunde liegenden Datentyp sowie eine Funktion zur Registrierung von Event Handlern für \texttt{Update}-Events an. Änderungen an dem geteilten JSON-Objekt werden über den zentralen Server an die anderen Teilnehmer weitergeleitet. Zusätzlich können die Clients Zustandsinformationen an den Server melden, welcher diese ebenso an die anderen Clients weiterleitet. Die Zustandsinformationen der Clients sind ebenfalls JSON-Objekte, allerdings werden für diese die normalen JSON-Datentypen verwendet. Die Zustandsinformationen eines Clients können nur von diesem selbst verändert werden. Clients können jedoch die Zustandsinformationen eines anderen Teilnehmers löschen, falls für eine gewisse Zeit keine Aktualisierung dieser vorgenommen wurde\todo{Erklärungssatz}. Dadurch können z.B. Verbindungsabbrüche von Clients behandelt werden. Basierend auf diesen Konzepten wird nachfolgend der \textit{Collaboration Service} anhand bestimmter Szenarien vorgestellt. Dabei werden auch die in \autoref{figure:klassendiagramm-collaboration-service} dargestellten Klassen und Funktionen an passender Stelle erläutert.

\begin{figure}[tbp]
    \centering
    \resizebox{\textwidth}{!}{\begin{sequencediagram}
            \newthread{consumer1}{Consumer 1}
            \newthreadShift{producer}{Producer}{3cm}
            \newthreadShift{consumer2}{Consumer 2}{3cm}

            \begin{call}{consumer1}{trete Räumen bei}{consumer1}{}
            \end{call}
            \prelevel\prelevel
            \begin{call}{consumer2}{trete Räumen bei}{consumer2}{}
            \end{call}

            \begin{call}{consumer1}{Initialisierung}{producer}{}
                \begin{call}{producer}{registriere Consumer}{producer}{}
                \end{call}
            \end{call}
            \begin{call}{consumer1}{starte Synchronisation}{producer}{}
            \end{call}

            \begin{call}{consumer2}{Initialisierung}{producer}{}
                \begin{call}{producer}{registriere Consumer}{producer}{}
                \end{call}
            \end{call}
            \begin{call}{consumer2}{starte Synchronisation}{producer}{}
            \end{call}
        \end{sequencediagram}}
    \caption{Beispiel Initialisierung Synchronisation}
    \label{figure:initialisierung-synchronisation}
\end{figure}

Zunächst wird die Initialisierung der Synchronisation zwischen dem Collaboration Service Producer und dem Collaboration Service Consumer betrachtet. Dafür ist in \autoref{figure:initialisierung-synchronisation} ein möglicher Ablauf dargestellt. Zunächst tritt der Collaboration Service Consumer den Räumen bei, für welche er Daten bereitstellt bzw. welche in der Konfiguration der Verbindung mit dem jeweiligen Producer angegeben wurden. Dafür kann die Funktion \texttt{joinRoom()} des Collaboration Service Consumer verwendet werden. Der Beitritt in Räume kann auch nach dem Beginn der Synchronisation geschehen. Nachdem der Collaboration Service Consumer den Räumen beigetreten ist, sendet dieser eine Initialisierungsnachricht an den Collaboration Service Producer, welche u.a. einen einzigartigen Kennzeichner für den Collaboration Service Consumer beinhaltet. Dieser Kennzeichner kann entweder generiert werden oder über die Konfiguration der entsprechenden Laborgeräte während der Konfiguration des Experiments angegeben werden. Die Informationen der Initialisierungsnachricht werden von dem Collaboration Service Producer verwendet, um die Nutzer den entsprechenden Räumen zuzuordnen. Sobald der Collaboration Service Producer eine Antwort an den Collaboration Service Consumer gesendet hat, startet dieser die Synchronisation der einzelnen Räume über die Funktion \texttt{startSynchronization()}. Die Synchronisation erfolgt über die zugrunde liegende Synchronisationsmethode. Aufgrund dessen können Collaboration Service Consumer und Collaboration Service Producer nur verbunden werden, wenn sie die gleiche Synchronisationsmethode unterstützen.

\begin{figure}[tbp]
    \centering
    \resizebox{\textwidth}{!}{\begin{sequencediagram}
            \newthread{consumer1}{Consumer 1}
            \newthreadShift{producer}{Producer}{3cm}
            \newthreadShift{consumer2}{Consumer 2}{3cm}

            \begin{call}{consumer1}{ändere JSON-Objekt}{consumer1}{}
            \end{call}

            \begin{call}{consumer1}{sende Änderung}{producer}{}
            \end{call}

            \begin{call}{producer}{wende Änderung an}{producer}{}
            \end{call}

            \begin{call}{producer}{sende Änderung}{consumer2}{}
            \end{call}

            \begin{call}{consumer2}{wende Änderung an}{consumer2}{}
            \end{call}
        \end{sequencediagram}}

    \caption{Beispiel Synchronisation des geteilten JSON-Objekts}
    \label{figure:synchronisation-des-geteilten-json-objekts}
\end{figure}

Über die Funktion \texttt{getProperty()} des Collaboration Service Consumer kann der Zugriff auf die Eigenschaften des geteilten JSON-Objekts für einen spezifischen Raum erfolgen. Dabei wird zunächst die gleichnamige Funktion der Klasse \texttt{Room} aufgerufen, welche diesen Aufruf an den assoziierten \texttt{CollaborationProvider} weiterleitet. Der Rückgabewert der Funktion ist einer der zuvor erwähnten kollaborativen Datentypen. Es werden nur die Eigenschaften synchronisiert, die mindestens einmal über die Funktion \texttt{getProperty()} abgefragt wurden. Der erhaltene kollaborative Datentyp kann dann über dessen bereitgestellte Funktionen verändert werden, wobei kollaborative Objekte und Arrays nur kollaborative Datentypen als Eigenschaften bzw. Items unterstützen. Um einen unterstützten JSON-Datentypen in einen entsprechenden kollaborativen Datentypen umzuwandeln, kann die Funktion \texttt{valueToCollaborationType()} verwendet werden. Die Funktion \texttt{executeTransaction()} des Collaboration Service Consumers kann verwendet werden um mehrere Änderungen vorzunehmen, die dann in einem Update gesammelt werden. Updates und andere Nachrichten der zugrunde liegenden Synchronisationsmethode werden vom \texttt{CollaborationProvider} über ein entsprechendes \texttt{CollaborationMessage}-Event bekanntgegeben. Diese werden von dem dazugehörigen Raum behandelt. Die Nachrichten werden dabei an alle Teilnehmer des Raums gesendet. Eingehende Nachrichten der zugrunde liegenden Synchronisationsmethode können über die Funktion \texttt{handleCollaborationMessage()} des \texttt{CollaborationProvider} behandelt werden. Der weitere Verlauf der Synchronisation ist stark von der verwendeten Methode abhängig. Wenn alle Teilnehmer über Peer-to-Peer Verbindungen direkt miteinander verbunden sind, können die Updates eines Teilnehmers direkt an alle anderen gesendet werden. Dadurch ist keine Weiterleitung danach mehr nötig. Im Falle der Kommunikation über einen zentralen Server muss dieser die Änderung einzelner Teilnehmer an die anderen Teilnehmer weiterleiten. Wenn ein externes Update zur Änderung des geteilten JSON-Objekts führt, wird ein \texttt{Update}-Event vom \texttt{CollaborationProvider} ausgelöst. Event Handler für diese können über den Collaboration Service Consumer registriert werden.

\begin{figure}[tbp]
    \centering
    \resizebox{\textwidth}{!}{\begin{sequencediagram}
            \newthread{consumer1}{Consumer 1}
            \newthreadShift{producer}{Producer}{3cm}
            \newthreadShift{consumer2}{Consumer 2}{3cm}

            \begin{call}{consumer1}{ändere Zustand}{consumer1}{}
            \end{call}

            \begin{call}{consumer1}{sende Zustände}{producer}{}
            \end{call}

            \begin{call}{producer}{aktualisiere Zustände}{producer}{}
            \end{call}

            \begin{call}{producer}{sende Zustände}{consumer2}{}
            \end{call}

            \begin{call}{consumer2}{aktualisiere Zustände}{consumer2}{}
            \end{call}
        \end{sequencediagram}}

    \caption{Beispiel Austausch der Zustandsinformationen}
    \label{figure:austausch-der-zustandsinformationen}
\end{figure}

Die Zustandsinformationen für einen Raum können über die Funktion \texttt{getAwareness()} des Collaboration Service Consumers erhalten werden. Der lokale Zustand kann dann über die Funktionen \texttt{setLocalState()} und \texttt{setLocalStateField()} bearbeitet werden. Änderungen der Zustandsinformationen werden über \texttt{Update}- und \texttt{Change}-Events bekanntgegeben, wobei \texttt{Change}-Events nur ausgelöst werden, wenn sich die Zustandsinformationen tatsächlich verändert haben. Die Events werden vom Raum abgefangen und für die Weiterleitung der neuen Zustandsinformationen verwendet. Dafür ist in \autoref{figure:austausch-der-zustandsinformationen} ein entsprechendes Sequenzdiagramm dargestellt. Es werden immer alle vorhandenen Zustandsinformationen verschickt, wofür u.a. die Funktion \texttt{encodeStates()} des \texttt{AwarenessProvider} verwendet wird. Sobald die neuen Zustandsinformationen bei den entsprechenden Collaboration Service Producern ankommen, aktualisieren diese zunächst ihre bekannten Zustandsinformationen mithilfe der Funktion \texttt{applyUpdate()} des \texttt{AwarenessProvider}. Sollte dabei eine Änderung vorgenommen werden, wird eine entsprechende Nachricht an alle verbundenen Collaboration Service Consumer mit Ausnahme des ursprünglichen Senders des Updates verschickt. Die Behandlung von externen Updates ist identisch für Collaboration Service Consumer.

Für die Einbindung des Collaboration Service in die betrachtete Experimentkonfiguration gibt grundsätzlich zwei Möglichkeiten. Diese hängen von der verwendeten Synchronisationsmethode ab, wobei in beiden Fällen mindestens eine weitere Instanz der IDE hinzugefügt wird. Wenn ein zentraler Synchronisationsserver benötigt wird, muss dieser als ein entsprechendes Laborgerät in das Experiment eingebunden werden. Dabei bietet es einen Collaboration Service Producer an, während die IDEs einen entsprechenden Collaboration Service Consumer anbieten. Die zweite Variante geht von einer verteilten Synchronisationsmethode aus. Dabei bieten die IDEs sowohl einen Collaboration Service Producer als auch einen Collaboration Service Consumer an. Dadurch können diese direkt miteinander verbunden werden.
\section{Dateisystem}\label{section:konzeption:dateisystem}

% \begin{note}
%     \textbf{Notizen:}
%     \begin{itemize}
%         \item Erwähnung von \autoref{requirement:Integriertes Dateisystem}, \autoref{requirement:Weitere Dateisysteme} und \autoref{requirement:Teilen von Ordnern}
%         \item Ggf. Erwähnung von \autoref{requirement:Eigenständig nutzbar}
%         \item Vergleich verschiedener Konzepte (client- vs. server-side)
%         \item Beschreibung CrossLab-Service + Klassendiagramm
%         \item Beschreibung der Kollaboration
%         \item Beschreibung der Einbindung in die betrachtete Experimentkonfiguration
%     \end{itemize}
% \end{note}

Nach \autoref{requirement:Integriertes Dateisystem} soll die IDE ein integriertes Dateisystem besitzen. Für dieses werden im Folgenden zunächst ein client-seitiger sowie ein server-seitiger Ansatz beschrieben.

Für den client-seitigen Lösungsansatz bietet sich eine Speicherung der Dateien des Nutzers innerhalb des Browsers an. Hierbei kann für die persistente Speicherung von Daten die Indexed Database API \cite{noauthor_indexed-database-api_nodate} genutzt werden. Diese wird von allen modernen Browsern unterstützt und erlaubt die langfristige Speicherung von größeren Datenmengen. Der Vorteil des client-seitigen Ansatzes ist die Tatsache, das kein weiterer Speicherplatz für die Nutzer bereitgestellt werden muss, da die Daten auf deren Geräten gespeichert werden. Allerdings sind die Daten sowohl an die Domain der IDE, das Gerät des Nutzers als auch an den spezifischen Browser gebunden und müssen ggf. durch entsprechendes Exportieren und Importieren übertragen werden.

Der server-seitige Lösungsansatz basiert darauf jedem Nutzer einen entsprechenden Speicherbereich zuzuteilen, in welchem seine Dateien gespeichert werden. Dies kann entweder über das Dateisystem des Servers oder über eine Datenbank geschehen. Der Vorteil dieser Art der Datenspeicherung liegt darin, dass sie unabhängig vom verwendeten Gerät und Browser ist. Allerdings werden für die Speicherung der Nutzerdaten entsprechender Speicherplatz auf dem Server benötigt wodurch höhere Kosten und ein höherer Verwaltungsaufwand entstehen können. Zudem muss sichergestellt werden, dass das System nicht ausgenutzt werden kann.

Nach \autoref{requirement:Weitere Dateisysteme} soll die IDE die Anbindung weiterer Dateisysteme unterstützen. Laut Unteranforderung (a) soll daher ein entsprechender CrossLab-Service für die Bereitstellung und Nutzung von Dateisystemen entwickelt werden. Diese sollen dann nach Unteranforderung (b) von der IDE für die Anbindung weiterer Dateisysteme verwendet werden. Dementsprechend wird im Folgenden der \textit{Filesystem Service} beschrieben. In \autoref{figure:klassendiagramm-dateisystem-service} ist ein Klassendiagramm für den Filesystem Service dargestellt. Aus diesem können die bereitgestellten Operationen eines Dateisystems abgelesen werden. So müssen Dateisysteme die Erstellung von Ordnern und Dateien, das Lesen, Verschieben und Löschen dieser sowie das Schreiben von Dateien unterstützen. Weiterhin soll die Registrierung sogenannter \textit{Watcher} ermöglicht werden. Diese können von einem Filesystem Service Consumer für bestimmte Pfade registriert werden um über Änderungen innerhalb dieser informiert zu werden. Der Filesystem Service Consumer besitzt Funktionen um die einzelnen Operationen auszuführen, während der Filesystem Service Producer die Möglichkeit bietet auf eingehende Anfragen zu reagieren und entsprechende Antworten an den Filesystem Service Consumer zu senden.

\begin{figure}[tbp]
    \centering
    \begin{tikzpicture}
        \begin{class}[text width=6cm]{FilesystemServiceProducer}{0,0}
            \operation{+ onCreateDirectory()}
            \operation{+ onDelete()}
            \operation{+ onMove()}
            \operation{+ onReadDirectory()}
            \operation{+ onReadFile()}
            \operation{+ onWriteFile()}
            \operation{+ onRegisterWatcher()}
            \operation{+ onUnregisterWatcher()}
        \end{class}
        \begin{class}[text width=6cm]{FilesystemServiceConsumer}{7,0}
            \operation{+ createDirectory()}
            \operation{+ delete()}
            \operation{+ move()}
            \operation{+ readDirectory()}
            \operation{+ readFile()}
            \operation{+ writeFile()}
            \operation{+ registerWatcher()}
            \operation{+ unregisterWatcher()}
            \operation{+ onWatcherEvent()}
        \end{class}
    \end{tikzpicture}
    \caption{Klassendiagramm Filesystem Service}
    \label{figure:klassendiagramm-dateisystem-service}
\end{figure}

Nach \autoref{requirement:Teilen von Ordnern} soll das Dateisystem das Teilen von Ordnern mit anderen Nutzern innerhalb eines Experiments unterstützen. Dafür soll nach Unteranforderung (d) der bereits vorhandene Collaboration Service der IDE verwendet werden. Beim Erstellen eines Experiments öffnen Nutzer einen Raum zum Teilen ihrer Ordner. Das geteilte JSON-Objekt hat dabei die Kennzeichner der verschiedenen Teilnehmer als Schlüssel mit den geteilten Ordnern des jeweiligen Teilnehmers als den dazugehörigen Wert. Wenn ein Nutzer einen Ordner teilt so soll dieser anderen Nutzern angezeigt werden. Diese können dann den Inhalt einsehen und bearbeiten. Die Implementierung der Synchronisation ist hierbei abhängig von dem verwendeten Code Editor. Weiterhin kann auch die aktuelle Position von Nutzern über deren Zustandsinformationen mit den anderen Nutzern geteilt werden. Die Position kann dann den anderen Nutzern innerhalb der entsprechenden Datei angezeigt werden.

Für die Einbindung des Filesystem Service in die betrachtete Experimentkonfiguration kann ein Filesystem Service Consumer zur IDE hinzugefügt werden. Für die Bereitstellung von Dateisystemen können Laborgeräte einen entsprechenden Filesystem Service Producer anbieten. Ein Beispiel für ein derartiges Laborgerät wäre ein Storage-Server, der es Nutzern ermöglichen könnte ihre Dateien, unabhängig von ihrem verwendeten Browser und Gerät, speichern und abrufen zu können. Weiterhin kann die IDE auch einen Filesystem Service Producer zur Bereitstellung ihres integrierten Dateisystems anbieten.
\section{Kompilierung und Programmierung}\label{section:konzeption:kompilierung-und-programmierung}

% \begin{note}
%     \textbf{Notizen:}
%     \begin{itemize}
%         \item Erwähnung von \autoref{requirement:Kompilierung} und \autoref{requirement:Programmierung von Steuereinheiten}
%         \item Beschreibung der CrossLab-Services + Klassendiagramme
%         \item Konzept für die Bereitstellung von Compilern als Laborgeräte
%         \item Beschreibung der Einbindung in die betrachtete Experimentkonfiguration
%         \item (Beschreibung möglicher Einstellungen?)
%     \end{itemize}
% \end{note}

Nach \autoref{requirement:Kompilierung} soll die Kompilierung der Programme von Nutzern innerhalb der IDE ermöglicht werden. Laut Unteranforderung (a) soll daher ein CrossLab-Service für die Bereitstellung und Nutzung von Compilern entwickelt werden. Diese sollen nach Unteranforderung (b) von der IDE für die Nutzung von Compilern verwendet werden. Weiterhin soll nach \autoref{requirement:Programmierung von Steuereinheiten} die Programmierung von Steuereinheiten innerhalb der IDE ermöglicht werden. Laut Unteranforderung (a) soll daher ein CrossLab-Service für die Programmierung von Steuereinheiten entwickelt werden. Dieser soll nach Unteranforderung (b) von der IDE für die Programmierung von Steuereinheiten verwendet werden. Dementsprechend werden im Folgenden der \textit{Compilation Service} sowie der \textit{Programming Service} vorgestellt.

In \autoref{figure:klassendiagramm-compilation-service} ist ein Klassendiagramm für den Compilation Service angegeben. Dabei besitzt der Compilation Service Consumer nur eine Funktion \texttt{compile()}, die es ermöglicht, die Kompilierung eines Ordners anzufragen. Dabei können zusätzliche Optionen für den Compiler sowie das gewünschte Format der Ausgabe angegeben werden. Die möglichen Ausgabeformate können auf der Seite des Compilation Service Producer über entsprechende Schemata definiert und über die Funktion \texttt{addResultFormat()} hinzugefügt werden. Die registrierten Ausgabeformate können dann in der Servicebeschreibung des Compilation Service Producer hinterlegt werden. Während der Erstellung eines Experiments kann das gewünschte Ausgabeformat für die Verbindung eines Compilation Service Consumer und eines Compilation Service Producer in der Verbindungskonfiguration angegeben werden. Die Behandlung eingehender Kompilieranfragen kann durch Event Handler für \texttt{Compile}-Events des Compilation Service Producer erfolgen. Die Behandlung von Kompilieranfragen ist abhängig vom verwendeten Compiler. Bei einer erfolgreichen Kompilierung wird das Ergebnis in dem festgelegten Ausgabeformat zusammen mit den Meldungen des Compilers an den Compilation Service Consumer gesendet. Falls die Kompilierung fehlschlägt, werden nur die Fehlermeldungen des Compilers versendet.

\begin{figure}[tbp]
    \centering
    \begin{tikzpicture}
        \begin{class}[text width=6.5cm]{CompilationServiceProducer}{0,0}
            \operation{+ addResultFormat()}
            \operation{+ onCompile()}
        \end{class}
        \begin{class}[text width=6.5cm]{CompilationServiceConsumer}{7.5,0}
            \operation{+ compile()}
        \end{class}
    \end{tikzpicture}
    \caption{Klassendiagramm Compilation Service}
    \label{figure:klassendiagramm-compilation-service}
\end{figure}

\begin{figure}[tbp]
    \centering
    \begin{tikzpicture}
        \begin{class}[text width=6.5cm]{ProgrammingServiceProducer}{0,0}
            \operation{+ onProgram()}
        \end{class}
        \begin{class}[text width=6.5cm]{ProgrammingServiceConsumer}{7.5,0}
            \operation{+ program()}
        \end{class}
    \end{tikzpicture}
    \caption{Klassendiagramm Programming Service}
    \label{figure:klassendiagramm-programming-service}
\end{figure}

In \autoref{figure:klassendiagramm-programming-service} ist ein Klassendiagramm für den Programming Service angegeben. Dabei besitzt der Programming Service Consumer nur die Funktion \texttt{program()}, die es ermöglicht, eine Programmieranfrage zu starten. Diese muss entweder eine Datei, z.B. das Ergebnis einer Kompilierung, oder einen Ordner mit dem entsprechenden Programm enthalten. Dabei könnte der Programming Service Producer über seine Servicebeschreibung angeben, welche Formate unterstützt werden. Der Programming Service Producer löst bei einer eingehenden Programmieranfrage ein entsprechendes \texttt{Program}-Event aus. Dieses kann durch entsprechende Event Handler abgefangen und zur Programmierung der Steuereinheit verwendet werden.

Die betrachtete Experimentkonfiguration wird zunächst um ein weiteres Laborgerät erweitert. Dieses stellt einen Compiler über einen entsprechenden Compilation Service Producer bereit. Die IDE wird um einen Compilation Service Consumer sowie einen Programming Service Consumer erweitert. Die Steuereinheit wird um einen Programming Service Producer erweitert. Die IDE wird dann über die neu hinzugefügten CrossLab-Services mit dem Compiler und der Steuereinheit verbunden.
\section{Debugging}\label{section:konzeption:debugging}

\begin{note}
    \textbf{Notizen:}
    \begin{itemize}
        \item Erwähnung von \autoref{requirement:Debuggen}, \autoref{requirement:Debuggen: CrossLab-Kompatibilität} und \autoref{requirement:Debuggen: Kollaboration}
        \item Beschreibung der CrossLab-Services + Klassendiagramm (Adapter)
        \item Beschreibung der CrossLab-Services + Klassendiagramm (Target)
        \item Konzept für die Ermöglichung der Kollaboration
        \item Konzept für die Bereitstellung von Debuggern als Laborgeräte
        \item Beschreibung der Einbindung in die betrachtete Experimentkonfiguration
        \item (Beschreibung möglicher Einstellungen?)
    \end{itemize}
\end{note}

\begin{figure}[tbp]
    \centering
    \resizebox{\textwidth}{!}{
        \begin{tikzpicture}
            \begin{class}[text width=7.5cm]{DebuggingAdapterServiceProducer}{0,0}
                \operation{+ sendMessageDAP()}
                \operation{+ onStartSession()}
                \operation{+ onJoinSession()}
                \operation{+ onMessageDAP()}
            \end{class}
            \begin{class}[text width=7.5cm]{DebuggingAdapterServiceConsumer}{8,0}
                \operation{+ sendMessageDAP()}
                \operation{+ startSession()}
                \operation{+ joinSession()}
                \operation{+ onDapMessageDAP()}
            \end{class}
            \begin{class}[text width=7.5cm]{DebuggingTargetServiceProducer}{0,-3.5}
                \operation{+ sendDebuggingMessage()}
                \operation{+ onStartDebugging()}
                \operation{+ onEndDebugging()}
                \operation{+ onDebuggingMessage()}
            \end{class}
            \begin{class}[text width=7.5cm]{DebuggingTargetServiceConsumer}{8,-3.5}
                \operation{+ sendDebuggingMessage()}
                \operation{+ startDebugging()}
                \operation{+ endDebugging()}
                \operation{+ onDebuggingMessage()}
            \end{class}
        \end{tikzpicture}
    }
    \caption{Klassendiagramm Debugging Services}
    \label{figure:klassendiagramm-debugging-services}
\end{figure}

Nach \autoref{requirement:Debuggen} soll das Debuggen der Programme von Nutzern innerhalb der IDE ermöglicht werden. Dafür sollen laut \autoref{requirement:Debuggen: CrossLab-Kompatibilität} CrossLab-Services für die Bereitstellung und Nutzung von Debuggern sowie für die Kommunikation zwischen Debuggern und Steuereinheiten konzipiert werden. Dementsprechend werden im Folgenden der \textit{Debugging Adapter Service Producer} und der \textit{Debugging Adapter Service Consumer} für die Bereitstellung und Nutzung von Debuggern sowie der \textit{Debugging Target Service Producer} und der \textit{Debugging Target Service Consumer} für die Kommunikation zwischen Debuggern und Steuereinheiten vorgestellt. \autoref{figure:klassendiagramm-debugging-services} zeigt ein Klassendiagramm für die Debugging Services.

Der Debugging Adapter Service baut auf dem \ac{DAP} \cite{noauthor_debug-adapter-protocol_nodate} von Microsoft auf. Dieses spezifiert Nachrichten die zwischen einer IDE und einem sogenannten \textit{Debug Adapter} ausgetauscht werden. Ein Debug Adapter bildet die Schnittstelle zwischen einem Debugger und einer IDE. Das Protokoll erlaubt somit die Anbindung bestehender Debugger über die Implementierung eines entsprechenden Debug Adapters. Weiterhin wird der Implementierungsaufwand für die Einbindung von Debuggern in IDEs verringert, da in der Theorie nur eine Schnittstelle implementiert werden muss, anstatt jeden Debugger an eine eigens entwickelte Schnittstelle anpassen zu müssen. Somit sollte durch die Verwendung des \ac{DAP} die Einbindung der meisten Debugger stark vereinfacht werden.

\begin{figure}[tbp]
    \centering
    \begin{sequencediagram}
        \newthread{ide}{IDE}
        \newthreadShift{debugger}{Debugger}{3cm}
        \newthreadShift{steuereinheit}{Steuereinheit}{3cm}

        \begin{call}{ide}{starte Debug-Sitzung}{debugger}{}
            \begin{call}{debugger}{speichere Programm}{debugger}{}
            \end{call}
            \begin{call}{debugger}{kompiliere Programm}{debugger}{}
            \end{call}
            \begin{call}{debugger}{starte Debuggen}{steuereinheit}{}
                \begin{call}{steuereinheit}{lade Programm}{steuereinheit}{}
                \end{call}
            \end{call}
        \end{call}

        \begin{call}{ide}{DAP Nachrichten}{debugger}{DAP Nachrichten}
        \end{call}

        \prelevel\prelevel

        \begin{call}{debugger}{Debugger Nachrichten}{steuereinheit}{Debugger Nachrichten}
        \end{call}
    \end{sequencediagram}
    \caption{Start einer Debug-Sitzung}
    \label{figure:start-einer-debug-sitzung}
\end{figure}

\paragraph{Start einer Debug-Sitzung} \autoref{figure:start-einer-debug-sitzung} zeigt ein Sequenzdiagramm für den Start einer Debug-Sitzung. Die Funktion \texttt{startSession()} des Debugging Adapter Service Consumer kann zum Starten einer Debug-Sitzung werden. Dabei werden der Ordner, welcher das zu debuggende Programm enthält, sowie Konfigurationsoptionen für den Debugger an den Debugging Adapter Service Producer gesendet. Dieser löst dann ein \texttt{StartSession}-Event mit den übergebenen Daten aus, welches durch einen entsprechenden Event Handler abgefangen werden kann. Dadurch kann die Implementierung an den jeweiligen Debugger angepasst werden. Im Allgemeinen sollte zunächst der übersendete Ordner auf dem Dateisystem des Debugging Adapter Service Producer gespeichert werden. Weiterhin kann ggf. eine Kompilierung des Programms mit speziellen Optionen für das Debuggen vorgenommen werden, wobei die in \autoref{section:konzeption:kompilierung} vorgestellten CrossLab-Services verwendet werden können. Außerdem sollte der Debugger mit den Konfigurationsoptionen gestartet werden und es sollte ein eindeutiger Kennzeichner für die Debug-Sitzung generiert werden. Weiterhin sollte die zu debuggende Steuereinheit über den Start der Debug-Sitzung informiert werden. Dafür kann die Funktion \texttt{startDebugging()} des Debugging Target Service Consumer verwendet werden. Dabei wird das Programm an den Debugging Target Service Producer übergeben. Dieses kann in Event Handlern von \texttt{StartDebugging}-Events z.B. zur Programmierung der Steuereinheit verwendet werden. Zusätzlich können ggf. noch weitere Vorbereitungen getroffen werden bevor eine Antwort an den Debugging Target Service Consumer gesendet wird. Nachdem die Sitzung erfolgreich gestartet wurde, wird eine entsprechende Antwort an den Debugging Adapter Service Consumer gesendet. Diese enthält den Kennzeicher der Debug-Sitzung sowie weitere Konfigurationsoptionen, die beim Start des \ac{DAP} übergeben werden sollen. Bei diesen Konfigurationsoptionen handelt es sich um Informationen die nur dem Debugging Adapter Service Producer bekannt sind, wie z.B. der Pfad zu dem kompilierten Programm auf dessen Dateisystem. Sobald der Debugging Adapter Service Consumer die Antwort erhalten hat kann das \ac{DAP} mit den Konfigurationsoptionen und dem Kennzeichner der Debug-Sitzung gestartet werden. Der Austausch der \ac{DAP} Nachrichten erfolgt dabei über die Funktionen \texttt{sendMessageDAP()}. Eingehende \ac{DAP} Nachrichten können über Event Handler für \texttt{MessageDAP}-Events erhalten und an den Debugger übergeben werden. Eine ähnliche Vorgehensweise wird für den Austausch von Debug-Nachrichten zwischen dem Debugging Target Service Producer und dem Debugging Target Service Consumer angewendet. Hierbei werden die Funktion \texttt{sendDebuggingMessage()} und \texttt{DebuggingMessage}-Events verwendet.

\begin{figure}[tbp]
    \centering
    \resizebox{\textwidth}{!}{\begin{sequencediagram}
            \newthread{ide}{IDE}
            \newthreadShift{debugger}{Debugger}{4cm}
            \newthreadShift{steuereinheit}{Steuereinheit}{3cm}

            \begin{call}{ide}{DAP Terminate Anfrage}{debugger}{}
                \begin{call}{debugger}{lösche Sitzungsdaten}{debugger}{}
                \end{call}
                \begin{call}{debugger}{stoppe Debugger}{debugger}{}
                \end{call}
                \begin{call}{debugger}{beende Debuggen}{steuereinheit}{}
                    \begin{call}{steuereinheit}{lade Programm neu}{steuereinheit}{}
                    \end{call}
                \end{call}
            \end{call}
        \end{sequencediagram}}
    \caption{Ende einer Debug-Sitzung}
    \label{figure:ende-einer-debug-sitzung}
\end{figure}

\paragraph{Ende einer Debug-Sitzung} \autoref{figure:ende-einer-debug-sitzung} zeigt ein Sequenzdiagramm für das Ende einer Debug-Sitzung. Dieses wird durch eine entsprechende Nachricht des \ac{DAP} wie z.B. \texttt{Terminate} ausgelöst. Sobald der Debugger diese Nachricht erhält werden die Sitzungsdaten gelöscht und der Debugger gestoppt. Weiterhin wird die Funktion \texttt{endDebugging()} des Debugging Target Service Consumer verwendet um den Debugging Target Service Producer über das Ende des Debug-Sitzung zu informieren. In einem Event Handler des dadurch ausgelösten \texttt{EndDebugging}-Events könnte z.B. das aktuelle Programm neugeladen werden und weitere vorgenommene Änderungen für das Debugging rückgängig gemacht werden. Danach wird eine entsprechende Antwort an den Debugging Target Service Consumer gesendet. Sobald diese erhalten wurde wird noch eine finale Antwort auf die ursprüngliche \ac{DAP} Anfrage gesendet.

\paragraph{Kollaboration} Nach \autoref{requirement:Debuggen: Kollaboration} soll es Nutzern innerhalb eines Experiments ermöglicht werden laufenden Debug-Sitzungen beizutreten. Dafür können die in \autoref{section:konzeption:kollaboration} vorgestellten CrossLab-Services verwendet werden. Dabei könnte z.B. beim Start einer Debug-Sitzung der Kennzeichner der Sitzung und der verwendete Ordner in den Zustandsinformationen des Nutzers hinterlegt werden. Dadurch können andere Nutzer über die gestartete Sitzung informiert werden. Falls sie Zugriff auf den angegebenen Ordner haben können sie der Debug-Sitzung mithilfe der Funktion \texttt{joinSession()} des Debugging Adapter Service Consumer beitreten. Dabei wird der Kennzeichner der beizutretenden Debug-Sitzung angegeben. Die Antwort enthält einen neuen Kennzeichner für die Debug-Sitzung des beitretenden Nutzers sowie Konfigurationsoptionen für die Ausführung des \ac{DAP}. Um eine kollaborative Debug-Sitzung zu ermöglichen ist zudem eine spezielle Behandlung von einigen Nachrichten des \ac{DAP} nötig. Zum Beispiel darf nur eine \texttt{Initialize} Anfrage an einen Debug Adapter gestellt werden. Dementsprechend muss die Antwort auf diese gespeichert werden um sie später bei einer erneuten \texttt{Initialize} Anfrage an den beitretenden Nutzer senden zu können. Dabei wird die Anfrage beitretender Nutzer nicht an den Debug Adapter übergeben. Weiterhin sind z.B. die Nachrichten für Breakpoints, Stacktraces, das Starten und Stoppen des Programms sowie das Beenden der Debug-Sitzung zu betrachten. Die detaillierte Betrachtung aller dieser Nachrichten ist nicht das Ziel dieser Arbeit. In \autoref{section:prototypische-implementierung:debugging} wird die Behandlung der für die prototypische Implementierung relevanten Nachrichten erläutert.
\section{Testen}\label{section:konzeption:testen}

% \begin{note}
%     \textbf{Notizen:}
%     \begin{itemize}
%         \item Erwähnung von \autoref{requirement:Testen}
%         \item Beschreibung der CrossLab-Services + Klassendiagramm
%         \item Beschreibung der Einbindung in die betrachtete Experimentkonfiguration
%     \end{itemize}
% \end{note}

Nach \autoref{requirement:Testen} soll ein CrossLab-Service für die Erstellung und Ausführung von Testfällen innerhalb eines Experiments entwickelt werden. Dabei soll nach Unteranforderung (a) der Producer in der Lage sein, Funktionen zur Verwendung in Testfällen bereitzustellen. Unteranforderung (b) verlangt, dass diese Funktionen vom Consumer ausgeführt werden können. Die Erstellung von Testfällen soll laut Unteranforderung (c) während der Konfiguration eines Experiments erfolgen können. Außerdem soll der entwickelte CrossLab-Service nach Unteranforderung (d) von der IDE zur Ausführung von Testfällen innerhalb eines Experiment verwendet werden. Dementsprechend wird im Folgenden der \textit{Testing Service} vorgestellt.

\autoref{figure:klassendiagramm-testing-service} zeigt ein Klassendiagramm für den Testing Service. Der Testing Service Producer ermöglicht es Laborgeräten Funktionen für die Erstellung von Testfällen bereitzustellen. Diese müssen über die Funktion \texttt{registerFunction()} registriert werden. Dabei werden mindestens der Name der Funktion und deren Implementierung benötigt. Zusätzlich könnte man die Angabe von Schemata für die Argumente und den Rückgabewert der Funktion verlangen. Diese ermöglichen die Validierung der Eingaben und Ausgaben der Funktion. Der Testing Service Consumer erlaubt das Hinzufügen von Testfällen über die Funktion \texttt{addTest()}. Tests bestehen dabei aus einem Namen, einer Liste an Funktionen und ggf. eine Liste von weiteren Tests. Funktionen werden durch ihren Namen, den Kennzeichner des bereitstellenden Testing Service Producer und ihre Argumente beschrieben. Zusätzlich kann ein erwarteter Rückgabewert angegeben werden. Dieser wird während dem Testen mit dem tatsächlichen Rückgabewert verglichen. Sollten die Werte dabei unterschiedlich sein, schlägt der Testfall fehl. Angenommen, das Experiment beinhaltet einen Microcontroller, der das Setzen und Auslesen seiner Pins über entsprechende Funktionen ermöglicht. Durch die Angabe von erwarteten Rückgabewerten kann beim Auslesen der Pins sichergestellt werden, dass der zu diesem Zeitpunkt erwartete Wert vorliegt. Die weiteren Testfälle werden nach den Funktionen in der angegebenen Reihenfolge ausgeführt. Der Testing Service Consumer kann das Testen über die Funktion \texttt{startTesting()} beginnen. Dadurch wird ein entsprechendes \texttt{StartTesting}-Event bei den Test Service Producern ausgelöst. Über entsprechende Event Handler können Vorbereitungen für die Ausführung der Testfälle getroffen werden, bevor eine Antwort an den Testing Service Consumer gesendet wird. Sobald alle Testing Service Producer eine Antwort gesendet haben, kann der Testing Service Consumer mithilfe der Funktion \texttt{runTest()} Testfälle ausführen. Dafür werden für jeden Testfall zunächst dessen Funktionen der Reihe nach aufgerufen und danach die weiteren enthaltenen Testfälle in der angegebenen Reihenfolge ausgeführt. Nachdem alle ausgewählten Testfälle ausgeführt wurden, kann die Operation \texttt{endTesting()} des Testing Service Consumer genutzt werden, um das Testen zu beenden. Hierbei wird wieder ein entsprechendes \texttt{EndTesting}-Event von den Testing Service Producern ausgelöst, welches über Event Handler genutzt werden kann, um den Normalzustand wiederherzustellen.

\begin{figure}[tbp]
    \centering
    \begin{tikzpicture}
        \begin{class}[text width=6cm]{TestingServiceProducer}{0,0}
            \operation{+ registerFunction()}
            \operation{+ onStartTesting()}
            \operation{+ onEndTesting()}
            \operation{+ onFunctionCall()}
        \end{class}
        \begin{class}[text width=6cm]{TestingServiceConsumer}{7,0}
            \operation{+ addTest()}
            \operation{+ runTest()}
            \operation{+ startTesting()}
            \operation{+ endTesting()}
        \end{class}
    \end{tikzpicture}
    \caption{Klassendiagramm Testing Service}
    \label{figure:klassendiagramm-testing-service}
\end{figure}

Die Einbindung des Testing Service in die betrachtete Experimentkonfiguration kann z.B. über das Hinzufügen eines Testing Service Consumer bei der IDE und eines Testing Service Producer bei der Steuereinheit erfolgen. Für ein konkreteres Beispiel wird ein Microcontroller als Steuereinheit angenommen. Dieser könnte das Setzen und Auslesen der Werte seiner Pins als Funktionen für Testfälle anbieten. Diese können für die Erstellung von Testfällen genutzt werden, die während dem laufenden Experiment von der IDE ausgeführt werden können.

\begin{figure}[htbp]
    \centering
    \includegraphics[width=\textwidth]{diagrams/experimentkonfigurationen/Experimentkonfiguration-05.drawio.pdf}
    \caption{Experimentkonfiguration}
    \label{figure:experimentkonfiguration:testen}
\end{figure}
\section{Language Server}\label{section:konzeption:language-server}

\begin{note}
    \textbf{Notizen:}
    \begin{itemize}
        \item Erwähnung von \autoref{requirement:Language Server} und \autoref{requirement:Language Server: CrossLab-Kompatibilität}
        \item Beschreibung der CrossLab-Services + Klassendiagramm
        \item Konzept für die Bereitstellung von Debuggern als Laborgeräte
        \item Beschreibung der Einbindung in die betrachtete Experimentkonfiguration
        \item (Beschreibung möglicher Einstellungen?)
    \end{itemize}
\end{note}

\begin{figure}[tbp]
    \centering
    \resizebox{\textwidth}{!}{
        \begin{tikzpicture}
            \begin{class}[text width=7.5cm]{LanguageServerServiceProducer}{0,0}
                \operation{+ sendMessageLSP()}
                \operation{+ onInitialize()}
                \operation{+ onMessageLSP()}
                \operation{+ onFilesystemEvent()}
            \end{class}
            \begin{class}[text width=7.5cm]{LanguageServerServiceConsumer}{8,0}
                \operation{+ initialize()}
                \operation{+ sendMessageLSP()}
                \operation{+ sendFilesystemEvent()}
                \operation{+ onMessageLSP()}
            \end{class}
        \end{tikzpicture}
    }
    \caption{Klassendiagramm Language Server Services}
    \label{figure:klassendiagramm-language-server-services}
\end{figure}

Nach \autoref{requirement:Language Server} soll die Anbindung von Language Servern an die IDE ermöglicht werden. Dafür soll laut \autoref{requirement:Language Server: CrossLab-Kompatibilität} die Bereitstellung und Nutzung von Language Servern über entsprechende CrossLab-Services ermöglicht werden. Bevor die konzipierten CrossLab-Services vorgestellt werden, wird zunächst ein kurzer Überblick über das \textit{\ac{LSP}} \cite{noauthor_language-server-protocol_nodate} sowie die Herausforderungen für die Anbinung an die IDE gegeben.

% sollte wahrscheinlich eher in Grundlagen erklärt werden da es bereits in den Anforderungen erwähnt wird
Das \acl{LSP} ist ein von Microsoft entwickeltes Protokoll zur Kommunikation zwischen einem \textit{Language Client} und einem \textit{Language Server}. Language Server ermöglichen Editorfunktionen, wie z.B. Code-Vervollständigung, Code-Navigation und Refactoring für ausgewählte Programmiersprachen. Language Clients sind meist als Teil eines Code Editors implementiert und sind nicht auf spezifische Programmiersprachen beschränkt. Das Protokoll ist für die lokale Kommunikation zwischen einem Language Client und einem Language Server entworfen, d.h. es wird angenommen, dass beide auf demselben System ausgeführt werden. Dies hat zur Folge, dass für den verteilten Anwendungsfall entsprechende Vorkehrungen getroffen werden müssen um die Funktionalität zu gewährleisten. So muss es dem Language Server ermöglicht werden auf die Dateien des Remote-Systems zugreifen zu können und umgekehrt. Es gibt Language Server, die den Quellcode lokal verarbeiten bzw. sogar kompilieren und basierend auf diesen Ergebnissen Antworten an den Language Client zurücksenden. Dazu werden Dateien auf dem Server verwendet, auf welche der Language Client ggf. Zugriff benötigt. Unter Betrachtung dieser Herausforderungen werden im Folgenden der \textit{Language Server Service Producer} und der \textit{Language Server Service Consumer} vorgestellt.

\autoref{figure:klassendiagramm-language-server-services} zeigt ein Klassendiagramm für die Language Server Services. Die Funktion \texttt{initialize()} des Language Server Service Consumer kann für die Initialisierung eines Language Servers verwendet werden. Dabei wird der aktuelle Ordner des Nutzers sowie dessen URL und ggf. Konfigurationsoptionen für die Initialisierung des Language Server übergeben. Dadurch wird ein enstprechendes \texttt{Initialize}-Event von dem verbundenen Language Server Service Producer ausgelöst. Über entsprechende Event Handler kann dieses abgefangen und für die Initialisierung des Language Server verwendet werden. Dabei kann u.a. der übergebene Ordner in dem lokalen Dateisystem hinterlegt werden und der Language Server mit den angegebenen Konfigurationsoptionen gestartet werden. Die übergebene URL des Ordners kann zur Umschreibung von URLs innerhalb der Nachrichten des \ac{LSP} verwendet werden. Diese unterscheiden sich ggf. zwischen den Systemen des Language Server Service Producer und des Language Server Service Consumer. Nachdem der Language Server initialisiert wurde wird eine Antwort an den Language Server Service Consumer gesendet. Diese enthält einen Booleschen Wert, welcher angibt ob Dateisystem-Events für den verwendeten Ordner und dessen Inhalt benötigt werden, sowie ggf. weitere Konfigurationsoptionen für den Start des \ac{LSP}. Sollten Dateisystem-Events benötigt werden müssen diese entsprechend auf der Seite des Language Server Service Consumer für Dateien und Ordner innerhalb des verwendeten Ordners erstellt und an den Language Server Service Producer mithilfe der Funktion \texttt{sendFilesystemEvent()} gesendet werden. Diese umfassen die Erstellung und Löschung von Dateien und Ordnern sowie Änderungen von Dateien. Die Behandlung dieser Dateisystem-Events kann über entsprechende Event Handler erfolgen. Sobald die Initialisierung erfolgt ist kann das \ac{LSP} mit den Konfigurationsoptionen gestartet werden. Der Austausch der \ac{LSP} Nachrichten erfolgt dabei über die Funktion \texttt{sendMessageLSP()}. Zur Behandlung dieser Nachrichten können Event Handler für die entsprechenden \texttt{MessageLSP}-Events registriert werden.