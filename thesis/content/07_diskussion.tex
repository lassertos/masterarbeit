\chapter{Diskussion}\label{section:diskussion}

% \begin{note}
%     \textbf{Notizen:}
%     \begin{itemize}
%         \item Wurden die Anforderungen erfüllt?
%         \item Betrachtung alternativer Lösungsansätze
%               \begin{itemize}
%                   \item Cloud-instanziierbare IDE
%                   \item Andere Implementierung eines Dateisystems
%                   \item Kollaboration über einen zentralen Synchronisationspunkt
%               \end{itemize}
%         \item Betrachtung offener Probleme und Aufgaben
%               \begin{itemize}
%                   \item Hohe Komplexität der Lösungen
%                   \item Analyse der benötigten Serverresourcen
%                   \item Kollaboration über mehrere Experimente hinweg
%                   \item Evaluation der Lösungen
%                   \item Experimente mit mehreren Steuereinheiten
%               \end{itemize}
%     \end{itemize}
% \end{note}

In diesem Kapitel wird in \autoref{section:diskussion:erfüllung-der-anforderungen} zunächst die Erfüllung der in \autoref{section:anforderungsanalyse:anforderungen} aufgestellten Anforderungen im Hinblick auf die prototypische Implementierung betrachtet. Danach werden in \autoref{section:diskussion:alternative-lösungsansätze} mögliche alternative Lösungsansätze aufgezeigt. Daraufhin werden in \autoref{section:diskussion:erweiterungsmöglichkeiten} verschiedene Erweiterungsmöglichkeiten für die entwickelte IDE dargelegt. Schließlich werden in \autoref{section:diskussion:offene-aufgaben} die noch bestehenden offenen Aufgaben erläutert.

\input{content/07_diskussion/071_erfüllung_der_anfoderungen.tex}
\section{Alternative Lösungsansätze}\label{section:diskussion:alternative-lösungsansätze}

In diesem Abschnitt werden alternative Lösungsansätze vorgestellt. Dabei wird zunächst eine cloud-instanziierbare Version der IDE betrachtet. Danach wird ein serverseitiges integriertes Dateisystem erläutert. Schließlich wird noch die Möglichkeit betrachtet, einen anderen Code Editor statt \ac{VSCode} zu verwenden.

\paragraph{Cloud-instanziierbare IDE}
Eine Implementierung der IDE als cloud-instanzi-ierbares Laborgerät könnte einige Vereinfachungen ermöglichen. So könnten neben der IDE auch entsprechende Compiler, Debugger und Language Server auf dem gleichen System bereitgestellt werden. Dadurch wird deren Nutzung vereinfacht und kann ggf. ohne die entsprechenden CrossLab-Services erfolgen. Allerdings benötigt die Ausführung der IDE auf einem Server entsprechende Ressourcen, was die Skalierbarkeit dieser Lösung beeinträchtigen könnte. Zudem könnte dadurch die eigenständige Ausführung der IDE nicht immer gewährleistet werden, da Nutzer ggf. auf freie Serverresourcen warten müssen. Zudem erlaubt die Anbindung von Compilern, Debuggern und Language Servern über die entsprechenden CrossLab-Services auch die einfache Rekonfiguration eines Experiments. Somit ist es u.a. möglich, gewisse Funktionen nur in manchen Experimenten anzubieten.

\paragraph{Serverseitiges integriertes Dateisystem}
In der prototypischen Implementierung wurde ein projektbasiertes Dateisystem implementiert, was die Indexed Database API für die persistente Speicherung der Daten verwendet. Dabei werden die Daten nur auf dem jeweiligen Gerät und dem entsprechenden Browser des Nutzers gespeichert. Somit kann kein direkter Zugriff auf diese von einem anderen Gerät oder Browser erfolgen. Die serverseitige Speicherung von Projekten könnte den geräteunabhängigen Zugriff auf diese ermöglichen. Allerdings werden hierfür entsprechende Serverresourcen benötigt. Außerdem muss sichergestellt werden, dass die Speicherung jederzeit möglich ist, damit die eigenständige Ausführbarkeit der IDE nicht gefährdet wird.

\paragraph{Alternative Code Editoren} Man könnte für die Implementierung der IDE auch einen anderen Code Editor als \ac{VSCode} verwenden. Dabei kann man u.a. auf IDE-Frameworks wie Eclipse Theia und OpenSumi zurückgreifen. Diese erlauben tiefergreifende Anpassungen, als durch die alleinige Nutzung der VSCode Extension API möglich ist. Allerdings entsteht durch die Nutzung der entsprechenden tiefergreifenden Schnittstellen eine Bindung an das jeweilige Framework. Dadurch wird ein späterer Wechsel auf ein anderes Framework bzw. einen anderen Code Editor erschwert. Eine weitere Möglichkeit ist die Einbindung reiner Code Editoren wie z.B. Ace, dem Monaco Editor oder eine Eigenimplementierung in ein eigenes Benutzerinterface. Dadurch hat man komplette Kontrolle über das Benutzerinterface und kann es auf die entsprechende Zielgruppe anpassen. Weiterhin können ggf. Probleme, wie z.B. das Neuladen der IDE beim Wechseln von Ordnern, umgangen werden, da man die komplette Kontrolle über die Implementierung besitzt. Allerdings ist mit dieser Lösung auch ein entsprechend höherer Implementierungsaufwand verbunden und ein späterer Wechsel des Code Editors ggf. nicht einfach möglich.
\input{content/07_diskussion/073_erweiterungsmöglichkeiten.tex}
\section{Offene Aufgaben}\label{section:diskussion:offene-aufgaben}

In diesem Abschnitt sollen offene Aufgaben vorgestellt werden. Dabei wird zunächst betrachtet, wie die Komplexität der konzipierten und implementierten Lösungen verringert werden kann. Anschließend werden Sicherheitsaspekte besprochen, die für eine finale Version der IDE beachtet werden sollten. Danach wird auf die Nutzung mehrerer Steuereinheiten innerhalb eines Experiments eingegangen. Schließlich wird die noch durchzuführende Evaluation der konzipierten und implementierten Lösungen erläutert.

\paragraph{Komplexität der Lösungen}
Die entwickelten Lösungen besitzen eine relativ hohe Komplexität in der Anwendung bzw. Implementierung. So müssen zunächst entsprechende Laborgeräte für die jeweiligen Compiler, Debugger, Language Server, etc. entwickelt werden. Diese müssen dann in einem Experiment korrekt konfiguriert werden. Außerdem muss die Experimentkonfiguration dann noch getestet werden, um sicherzustellen, dass sie korrekt funktioniert. Als Beispiel soll die Anwendung in der Lehre betrachtet werden. Ein Lehrender hat ggf. nicht die Zeit, sich in das komplette CrossLab System einzuarbeiten und neue Laborgeräte zu implementieren. Dementsprechend müssten grundlegende Laborgeräte implementiert und in Kategorien zusammengefasst werden, um die Konfiguration von Experimenten für Lehrende zu vereinfachen. Dabei sollten entsprechende Werkzeuge bzw. Bibliotheken konzipiert und implementiert werden, welche bei der Implementierung von Laborgeräten und bei der Zusammenstellung von Experimenten zusätzlich helfen können. Ein Beispiel hierfür könnte ein Experimentkonfigurator sein, der die verschiedenen Laborgeräte samt ihrer angebotenen Services visualisieren kann und deren Verbindung zu einem Experiment ermöglicht.

\paragraph{Sicherheitsaspekte}
Während der prototypischen Implementierung war die Betrachtung von Sicherheitsaspekten keine Priorität. Dementsprechend sollten diese vor dem Einsatz der IDE sowie der entwickelten Laborgeräte überprüft werden, um sicherzustellen, dass das System nicht ausgenutzt werden kann. Beispiele für zu betrachtende Sicherheitsprobleme sind die in \cite{wu_ceclipse_2011} erwähnten \quoted{Wrong file operations}, \quoted{Banned operation calling} und \quoted{Excessive resource consumption}.

\paragraph{Mehrere Steuereinheiten}
Während der prototypischen Implementierung wurden nur Experimente mit einer Steuereinheit betrachtet. Für Experimente mit mehreren Steuereinheiten müssen entsprechende Anpassungen vorgenommen werden. Darunter auch die Anpassung der bereitgestellten Benutzerinterfaces.

\paragraph{Evaluation der Lösungen}
Die Implementierung muss evaluiert werden. Dadurch kann festgestellt werden, ob die entwickelten Lösungen den gewünschten Effekt haben. Darunter u.a. der geringere Ressourcenverbrauch und die einfachere Konfiguration von Experimenten, die eine IDE enthalten sollen. Dabei bietet sich ein Probelauf innerhalb eines Praktikumsversuchs an, bei dem die Studierenden eine entsprechende Aufgabe mit der IDE lösen müssen. Hierbei kann auch evaluiert werden, ob das Benutzerinterface für die Studierenden angemessen ist. Auch ein Vergleich mit WIDE könnte interessant sein, um zu sehen, wie die neuen Funktionen der IDE von den Studierenden wahrgenommen werden.