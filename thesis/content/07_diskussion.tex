\chapter{Diskussion}\label{section:diskussion}

% \begin{note}
%     \textbf{Notizen:}
%     \begin{itemize}
%         \item Wurden die Anforderungen erfüllt?
%         \item Betrachtung alternativer Lösungsansätze
%               \begin{itemize}
%                   \item Cloud-instanziierbare IDE
%                   \item Andere Implementierung eines Dateisystems
%                   \item Kollaboration über einen zentralen Synchronisationspunkt
%               \end{itemize}
%         \item Betrachtung offener Probleme und Aufgaben
%               \begin{itemize}
%                   \item Hohe Komplexität der Lösungen
%                   \item Analyse der benötigten Serverresourcen
%                   \item Kollaboration über mehrere Experimente hinweg
%                   \item Evaluation der Lösungen
%                   \item Experimente mit mehreren Steuereinheiten
%               \end{itemize}
%     \end{itemize}
% \end{note}

In diesem Kapitel wird in \autoref{section:diskussion:erfüllung-der-anforderungen} zunächst die Erfüllung der in \autoref{section:anforderungsanalyse:anforderungen} aufgestellten Anforderungen im Hinblick auf die prototypische Implementierung betrachtet. Danach werden in \autoref{section:diskussion:alternative-lösungsansätze} mögliche alternative Lösungsansätze aufgezeigt. Daraufhin werden in \autoref{section:diskussion:erweiterungsmöglichkeiten} verschiedene Erweiterungsmöglichkeiten für die entwickelte IDE dargelegt. Schließlich werden in \autoref{section:diskussion:offene-aufgaben} die noch bestehenden offenen Aufgaben erläutert.

\section{Erfüllung der Anforderungen}\label{section:diskussion:erfüllung-der-anforderungen}

In diesem Abschnitt wird die Erfüllung der in \autoref{section:anforderungsanalyse:anforderungen} gestellten Anforderungen betrachtet. Dafür werden im Folgenden die einzelnen Anforderungen betrachtet. Dabei wird der Grad der Erfüllung sowie eine kurze Erläuterung zu diesem angegeben.

\begin{tabularx}{\textwidth}{X}
    \toprule
    \autoref{requirement:Ausführung im Browser} \hfill Ausführung im Browser \hfill Erfüllt \\
    \\
    Die IDE kann im Browser des Nutzers ausgeführt werden.                                  \\
    \bottomrule                                                                             \\
\end{tabularx}
\begin{tabularx}{\textwidth}{X}
    \toprule
    \autoref{requirement:Erweiterbarkeit} \hfill Erweiterbarkeit \hfill Erfüllt \\
    \\
    Die IDE kann mithilfe der VSCode Extension API erweitert werden.
    \\
    \bottomrule
\end{tabularx}
\vfill
\begin{tabularx}{\textwidth}{X}
    \toprule
    \autoref{requirement:Keine Kosten für Nutzer} \hfill Keine Kosten für Nutzer \hfill Erfüllt
    \\
    \\
    Die IDE besitzt neben den benötigten Serverresourcen keine weiteren Kosten. Es wurde quelloffene und frei nutzbare Software für die Implementierung verwendet.
    \\
    \bottomrule
\end{tabularx}
\vfill
\begin{tabularx}{\textwidth}{X}
    \toprule
    \autoref{requirement:Eigenständig nutzbar} \hfill Eigenständig nutzbar \hfill Erfüllt
    \\
    \\
    Die IDE kann als einziges Laborgerät in einem Experiment verwendet werden. Dabei werden die Editierung von Quellcode und das integrierte Dateisystem unterstützt.
    \\
    \bottomrule
\end{tabularx}
\vfill
\begin{tabularx}{\textwidth}{X}
    \toprule
    \autoref{requirement:Nur CrossLab-Nutzerkonto nötig} \hfill Nur CrossLab-Nutzerkonto nötig \hfill Erfüllt
    \\
    \\
    Die IDE benötigt zur Verwendung nur das zur Ausführung von Experimenten benötigte CrossLab-Nutzerkonto.
    \\
    \bottomrule
\end{tabularx}
\vfill
\begin{tabularx}{\textwidth}{X}
    \toprule
    \autoref{requirement:Integriertes Dateisystem} \hfill Integriertes Dateisystem \hfill Erfüllt
    \\
    \\
    Die IDE besitzt ein integriertes Dateisystem, welches die Erstellung, Bearbeitung, Verschiebung, Löschung und persistente Speicherung von Dateien und Ordnern unterstützt. Weiterhin kann das integrierte Dateisystem auch in der eigenständigen Ausführung verwendet werden.
    \\
    \bottomrule
\end{tabularx}
\vfill
\begin{tabularx}{\textwidth}{X}
    \toprule
    \autoref{requirement:Weitere Dateisysteme} \hfill Weitere Dateisysteme \hfill Erfüllt
    \\
    \\
    Es wurde der Filesystem Service für die Bereitstellung und Nutzung von Dateisystemen entwickelt. Dieser wird von der IDE für die Bereitstellung des integrierten Dateisystems sowie für die Anbindung weiterer Dateisysteme verwendet.
    \\
    \bottomrule
\end{tabularx}
\vfill
\begin{tabularx}{\textwidth}{X}
    \toprule
    \autoref{requirement:Kollaboration} \hfill Kollaboration \hfill Erfüllt
    \\
    \\
    Es wurde der Collaboration Service für die Synchronisation von geteilten Daten und den Austausch von Zustandsinformationen entwickelt. Dieser wird von der IDE zur Bereitstellung der Echtzeit-Kollaboration verwendet.
    \\
    \bottomrule
\end{tabularx}
\vfill
\begin{tabularx}{\textwidth}{X}
    \toprule
    \autoref{requirement:Teilen von Ordnern} \hfill Teilen von Ordnern \hfill Teilweise erfüllt
    \\
    \\
    Nutzer können ihre Projekte innerhalb eines Experiments mit anderen teilnehmenden Nutzern teilen. Änderungen innerhalb der geteilten Projekte werden zwischen den Nutzern synchronisiert. Geteilte Projekte können nur von ihrem Besitzer gelöscht, verschoben oder umbenannt werden. Zudem kann das Teilen von Projekten auch von dem Besitzer beendet werden. Das Teilen der Projekte erfolgt über den bereits vorhandenen Collaboration Service der IDE.
    \\
    \bottomrule
\end{tabularx}
\vfill
\begin{tabularx}{\textwidth}{X}
    \toprule
    \autoref{requirement:Kompilierung} \hfill Kompilierung \hfill Erfüllt
    \\
    \\
    Es wurde der Compilation Service für die Bereitstellung und Nutzung von Compilern entwickelt. Dieser wird von der IDE für die Anbindung und Nutzung von Compilern verwendet.
    \\
    \bottomrule
\end{tabularx}
\vfill
\begin{tabularx}{\textwidth}{X}
    \toprule
    \autoref{requirement:Programmierung von Steuereinheiten} \hfill Programmierung von Steuereinheiten  \hfill Erfüllt                                                    \\
    \\
    Es wurde der Programming Service für die Programmierung von Steuereinheiten entwickelt. Dieser wird von der IDE für die Programmierung von Steuereinheiten verwendet. \\
    \bottomrule
\end{tabularx}
\vfill
\begin{tabularx}{\textwidth}{X}
    \toprule
    \autoref{requirement:Debuggen} \hfill Debuggen \hfill Erfüllt
    \\
    \\
    Es wurde der Debugging Adapter Service für die Bereitstellung und Nutzung von Debuggern entwickelt. Dieser wird von der IDE für die Anbindung und Nutzung von Debuggern verwendet. Außerdem wurde der Debugging Target Service für die Kommunikation zwischen Debuggern und Steuereinheiten entwickelt.
    \\
    \bottomrule
\end{tabularx}
\vfill
\begin{tabularx}{\textwidth}{X}
    \toprule
    \autoref{requirement:Teilen von Debug-Sitzungen} \hfill Teilen von Debug-Sitzungen \hfill Teilweise erfüllt
    \\
    \\
    Nutzer können gemeinsam an einer Debug-Sitzung teilnehmen, wenn sie Zugriff auf das zu debuggende Programm besitzen. Die Breakpoints der Nutzer werden zwischen diesen synchronisiert. Nur der Ersteller einer Debug-Sitzung kann diese beenden. Für das Teilen von Debug-Sitzungen wird der bereits vorhandene Collaboration Service der IDE verwendet. Während der prototypischen Implementierung wurden nur Experimente mit einer Steuereinheit betrachtet. Dementsprechend wurde nur eine aktive Debug-Sitzung innerhalb eines Experiments erlaubt.
    \\
    \bottomrule \\
\end{tabularx}
\begin{tabularx}{\textwidth}{X}
    \toprule
    \autoref{requirement:Testen} \hfill Testen \hfill Erfüllt
    \\
    \\
    Es wurde der Testing Service für die Erstellung und Ausführung von Testfällen innerhalb eines Experiments entwickelt. Der Testing Service Producer ermöglicht die Bereitstellung von Funktionen zur Verwendung in Testfällen. Der Testing Service Consumer ermöglicht die Ausführung von Testfällen. Die Erstellung von Testfällen kann während der Konfiguration eines Experiments erfolgen. Die IDE verwendet den Testing Service für die Ausführung von Testfällen innerhalb eines Experiments.
    \\
    \bottomrule \\
\end{tabularx}
\begin{tabularx}{\textwidth}{X}
    \toprule
    \autoref{requirement:Language Server} \hfill Language Server \hfill Erfüllt
    \\
    \\
    Es wurde der Language Server Service für die Bereitstellung und Nutzung von Language Servern entwickelt. Dieser wird von der IDE für die Anbindung und Nutzung von Language Servern verwendet.
    \\
    \bottomrule
\end{tabularx}

% \begin{itemize}
%     \item \autoref{requirement:Ausführung im Browser} (Ausführung im Browser): \hfill Erfüllt \\ Die IDE kann im Browser des Nutzers ausgeführt werden.
%     \item \autoref{requirement:Erweiterbarkeit} (Erweiterbarkeit): Erfüllt \\ Die IDE kann mithilfe der VSCode Extension API erweitert werden.
%     \item \autoref{requirement:Keine Kosten für Nutzer} (Keine Kosten für Nutzer): Erfüllt \\ Die IDE besitzt neben den benötigten Serverresourcen keine weiteren assoziierten Kosten. Es wurde quelloffene und frei nutzbare Software für die Implementierung verwendet.
%     \item \autoref{requirement:Eigenständig nutzbar} (Eigenständig nutzbar): Erfüllt \\ Die IDE kann als einziges Laborgerät in einem Experiment verwendet werden. Dabei wird die Editierung von Quellcode und das integrierte Dateisystem unterstützt.
%     \item \autoref{requirement:Nur CrossLab-Nutzerkonto nötig} (Nur CrossLab-Nutzerkonto nötig): Erfüllt \\ Die IDE benötigt zur Verwendung nur das zur Ausführung von Experimenten benötigte CrossLab-Nutzerkonto.
%     \item \autoref{requirement:Integriertes Dateisystem} (Integriertes Dateisystem): Erfüllt \\ Die IDE besitzt ein integriertes Dateisystem, welches die Erstellung, Bearbeitung, Verschiebung, Löschun und persistente Speicherung von Dateien und Ordnern unterstützt. Weiterhin kann das integrierte Dateisystem auch in der eigenständigen Ausführung verwendet werden.
%     \item \autoref{requirement:Weitere Dateisysteme} (Weitere Dateisysteme): Erfüllt \\ Es wurde ein CrossLab-Service für die Bereitstellung und Nutzung von Dateisystemen entwickelt (Filesystem Service). Dieser wird von der IDE für die Bereitstellung des integrierten Dateisystems sowie für die Anbindung weiterer Dateisysteme verwendet.
%     \item \autoref{requirement:Kollaboration} (Kollaboration): Erfüllt \\ Es wurde ein CrossLab-Service für die Synchronisation von geteilten Daten und den Austausch von Zustandsinformationen entwickelt (Collaboration Service). Dieser wird von der IDE zur Bereitstellung der Echtzeit-Kollaboration verwendet.
%     \item \autoref{requirement:Teilen von Ordnern} (Teilen von Ordnern): Teilweise erfüllt \\ Nutzer können ihre Projekte innerhalb eines Experiments mit anderen teilnehmenden Nutzern teilen. Änderungen innerhalb der geteilten Projekte werden zwischen den Nutzern synchronisiert. Geteilte Projekte können nur von ihrem Besitzer gelöscht, verschoben oder umbenannt werden. Zudem kann das Teilen von Projekten auch von dem Besitzer beendet werden. Das Teilen der Projekte erfolgt über den bereits vorhandenen Collaboration Service der IDE.
%     \item \autoref{requirement:Kompilierung} (Kompilierung): Erfüllt \\ Es wurde ein CrossLab-Service für die Bereitstellung und Nutzung von Compilern entwickelt (Compilation Service). Dieser wird von der IDE für die Anbindung und Nutzung von Compilern verwendet.
%     \item \autoref{requirement:Programmierung von Steuereinheiten} (Programmierung von Steuereinheiten): Erfüllt \\ Es wurde ein CrossLab-Service für die Programmierung von Steuereinheiten entwickelt (Programming Service). Dieser wird von der IDE für die Programmierung von Steuereinheiten verwendet.
%     \item \autoref{requirement:Debuggen} (Debuggen): Erfüllt \\ Es wurde ein CrossLab-Service für die Bereitstellung und Nutzung von Debuggern entwickelt (Debugging Adapter Service). Dieser wird von der IDE für die Anbindung und Nutzung von Debuggern verwendet. Außerdem wurde ein CrossLab-Service für die Kommunikation zwischen Debuggern und Steuereinheiten entwickelt (Debugging Target Service).
%     \item \autoref{requirement:Teilen von Debug-Sitzungen} (Teilen von Debug-Sitzungen): Teilweise erfüllt \\ Nutzer können gemeinsam an einer Debug-Sitzung teilnehmen, wenn sie Zugriff auf das zu debuggende Programm besitzen. Die Breakpoints der Nutzer werden zwischen diesen synchronisiert. Nur der Ersteller einer Debug-Sitzung kann diese beenden. Für das Teilen von Debug-Sitzungen wird der bereits vorhandene Collaboration Service der IDE verwendet. Während der prototypischen Implementierung wurden nur Experimente mit einer Steuereinheit betrachtet, dementsprechend wurde nur eine aktive Debug-Sitzung innerhalb eines Experiments erlaubt.
%     \item \autoref{requirement:Testen} (Testen): Erfüllt \\ Es wurde ein CrossLab-Service für die Erstellung und Ausführung von Testfällen innerhalb eines Experiments entwickelt (Testing Service). Der Testing Service Producer ermöglicht die Bereitstellung von Funktionen zur Verwendung in Testfällen. Der Testing Service Consumer ermöglicht die Ausführung von Testfällen. Die Erstellung von Testfällen kann während der Konfiguration eines Experiments erfolgen. Die IDE verwendet den Testing Service für die Ausführung von Testfällen innerhalb eines Experiments.
%     \item \autoref{requirement:Language Server} (Language Server): Erfüllt \\ Es wurde ein CrossLab-Service für die Bereitstellung und Nutzung von Language Servern entwickelt (Language Server Service). Dieser wird von der IDE für die Anbindung und Nutzung von Language Servern verwendet.
% \end{itemize}
\section{Alternative Lösungsansätze}\label{section:diskussion:alternative-lösungsansätze}

In diesem Abschnitt werden alternative Lösungsansätze vorgestellt. Dabei wird zunächst eine cloud-instanziierbare Version der IDE betrachtet. Danach wird ein serverseitiges integriertes Dateisystem erläutert. Schließlich wird noch die Möglichkeit betrachtet einen anderen Code Editor statt \ac{VSCode} zu verwenden.

\paragraph{Cloud-instanziierbare IDE}
Eine Implementierung der IDE als cloud-instanzi-ierbares Laborgerät könnte einige Vereinfachungen ermöglichen. So könnten neben der IDE auch entsprechende Compiler, Debugger und Language Server auf dem gleichen System bereitgestellt werden. Dadurch wird deren Nutzung vereinfacht und kann ggf. ohne die entsprechenden CrossLab-Services erfolgen. Allerdings benötigt die Ausführung der IDE auf einem Server entsprechende Ressourcen, was die Skalierbarkeit dieser Lösung beeinträchtigen könnte. Zudem könnte dadurch die eigenständige Ausführung der IDE nicht immer gewährleistet werden, da Nutzer ggf. auf freie Serverresourcen warten müssen. Zudem erlaubt die Anbindung von Compilern, Debuggern und Language Servern über die entsprechenden CrossLab-Services auch die einfache Rekonfiguration eines Experiments. Somit ist es u.a. möglich gewisse Funktionen nur in manchen Experimenten anzubieten.

\paragraph{Serverseitiges integriertes Dateisystem}
In der prototypischen Implementierung wurde ein projektbasiertes Dateisystem implementiert, was die Indexed Database API für die persistente Speicherung der Daten verwendet. Dabei werden die Daten nur auf dem jeweiligen Gerät und dem entsprechenden Browser des Nutzers gespeichert. Somit kann kein direkter Zugriff auf diese von einem anderen Gerät oder Browser erfolgen. Die serverseitige Speicherung von Projekten könnte den geräteunabhängigen Zugriff auf diese ermöglichen. Allerdings werden hierfür entsprechende Serverresourcen benötigt. Außerdem muss sichergestellt werden, dass die Speicherung jederzeit möglich ist, damit die eigenständige Ausführbarkeit der IDE nicht gefährdet wird.

\paragraph{Alternative Code Editoren} Man könnte für die Implementierung der IDE auch einen anderen Code Editor als \ac{VSCode} verwenden. Dabei kann man u.a. auf IDE-Frameworks wie Eclipse Theia und OpenSumi zurückgreifen. Diese erlauben tiefergreifende Anpassungen als durch die alleinige Nutzung der VSCode Extension API möglich ist. Allerdings entsteht durch die Nutzung der entsprechenden tiefergreifenden Schnittstellen eine Bindung an das jeweilige Framework. Dadurch wird ein späterer Wechsel auf ein anderes Framework bzw. einen anderen Code Editor erschwert. Eine weitere Möglichkeit ist die Einbindung reiner Code Editoren, wie z.B. Ace, der Monaco Editor oder eine Eigenimplementierung, in ein eigenes Benutzerinterface. Dadurch hat man komplette Kontrolle über das Benutzerinterface und kann es auf die entsprechende Zielgruppe anpassen. Weiterhin können ggf. Probleme, wie z.B. das Neuladen der IDE beim Wechseln von Ordnern, umgangen werden, da man die komplette Kontrolle über die Implementierung besitzt. Allerdings ist mit dieser Lösung auch ein entsprechend höherer Implementierungsaufwand verbunden und ein späterer Wechsel des Code Editors ist ggf. nicht einfach möglich.

\todo{Idee: Edge-instanziierbare Laborgeräte für Compiler, etc.}
\section{Erweiterungsmöglichkeiten}\label{section:diskussion:erweiterungsmöglichkeiten}
In diesem Abschnitt werden Erweiterungsmöglichkeiten für die entwickelte IDE vorgestellt. Dabei wird zunächst die Kollaboration über einen zentralen Server betrachtet. Danach wird die Sammlung und Analyse von Nutzerdaten erläutert. Daraufhin werden mögliche Anpassungen für die Kollaboration in Lehrszenarien beschrieben. Schließlich wird die Verwendung von WebAssembly \cite{noauthor_webassembly_nodate} zur Bereitstellung weiterer Funktionen innerhalb des Browsers der Nutzer betrachtet.

\paragraph{Zentrale Kollaboration}
In der prototypischen Implementierung wurde nur eine verteilte Kollaboration über Yjs betrachtet. Allerdings wäre auch die Kollaboration über einen zentralen Server denkbar. Diese könnte es Nutzern ermöglichen auch über Experimente hinweg miteinander zu kollaborieren. Angenommen Studierende müssen in Gruppen mehrere Aufgaben lösen. In der aktuellen Version der IDE können Nutzer nur dann auf die Projekte ihrer Teammitglieder zugreifen, wenn diese innerhalb eines Experiments mit ihnen geteilt werden. Ansonsten können Nutzer nur auf ihre eigenen Projekte zugreifen. Wenn die Kollaboration über einen zentralen Server erfolgt, kann dieser die Projekte der Nutzer speichern. Dadurch können alle Teammitglieder auf die geteilten Projekte zugreifen. Auch wenn der ursprüngliche Ersteller des Projekts nicht an dem aktuellen Experiment teilnimmt.

\paragraph{Sammlung und Analyse von Nutzerdaten}
Die prototypische Implementierung besitzt keine Funktionen für die Sammlung von Nutzerdaten. Diese könnten ggf. hinzugefügt werden um eine spätere Analyse dieser zu ermöglichen. Dadurch könnten z.B. Lehrende einen besseren Einblick in die Arbeitsweise und Probleme der Lernenden erhalten. Somit könnten die Aufgabenstellungen entsprechend angepasst werden um den Lernerfolg zu maximieren. Mögliche Daten die hierfür erhoben werden könnten sind z.B. die Anzahl der ausgeführten Kompilationen und Programmierungen oder die beim Programmieren bzw. Debuggen verbrachte Zeit.

\paragraph{Kollaboration in Lehrszenarien}
In Lehrszenarien gibt es neben der bereits erwähnten Kollaboration zwischen den Lernenden auch die Möglichkeit der Kollaboration zwischen Lernenden und Lehrenden. Angenommen die Lernenden haben ein Problem während der Bearbeitung einer Aufgabe. Dabei vermuten sie ein Problem auf der Seite der verwendeten Hardware. Ein Lehrender könnte in diesem Fall dem laufenden Experiment beitreten und die Fehlerursache suchen. Dafür könnte er u.a. vorübergehend die Programmierung der Steuereinheiten für die Lernenden deaktivieren und dann eine Musterlösung für die Programmierung verwenden. Sollte der Fehler weiterhin bestehen kann der Lehrende z.B. über das Debuggen des Programms oder entsprechende Testfälle herausfinden wo das Problem liegt und dieses dann beheben. Außerdem wäre auch ein Pair-Programming Modus denkbar, bei dem zwei Nutzer zusammen an einem Programm arbeiten, wobei ein Nutzer nur Lesezugriff besitzt, während der andere Nutzer den kompletten Zugriff auf die Dateien erhält. Dabei könnte auch ein Wechsel der Rollen erlaubt werden.

\paragraph{WebAssembly}
WebAssembly \cite{noauthor_webassembly_nodate} ist ein Binärformat, dass von modernen Browsern unterstützt wird. Es kann als Ziel für die Kompilierung verschiedenster Programmiersprachen verwendet werden. Emscripten \cite{noauthor_emscripten_nodate} ist eine Toolchain, welche die Kompilierung von u.a. C und C++ Programmen nach WebAssembly ermöglicht. Dafür nutzt es LLVM \cite{noauthor_llvm_nodate}. Pyodide \cite{noauthor_pyodide_nodate} ist eine Python Distribution, die komplett im Browser ausgeführt werden kann, und wurde mithilfe von Emscripten entwickelt. Weitere interessante Anwendungen von WebAssembly sind u.a. Ports des C/C++ Compilers Clang \cite{noauthor_clang_nodate}\cite{smith_binjiwasm-clang_2024} sowie des dazugehörigen Language Servers Clangd \cite{noauthor_clangd_nodate}\cite{yu_guyutongxueclangd--browser_2024} und v86 \cite{fabian_copyv86_2025}. Letzeres emuliert eine x86-kompatible CPU und kann verwendet werden um z.B. virtuelle Maschinen innerhalb des Browsers eines Nutzers zu starten. Dies wird von Wokwi \cite{noauthor_wokwi_nodate}, einer Platform zur Simulation von Microcontrollern, verwendet um den Debugger GDB innerhalb des Browsers der Nutzer bereitzustellen \cite{noauthor_running_2021}. Somit könnte in späteren Arbeiten untersucht werden, ob manche Funktionen der IDE mithilfe von WebAssembly im Browser des Nutzers ausgeführt werden können. Dadurch könnten die benötigten Serverressourcen verringert werden.
\section{Offene Aufgaben}\label{section:diskussion:offene-aufgaben}

In diesem Abschnitt sollen offene Aufgaben vorgestellt werden. Dabei wird zunächst betrachtet, wie die Komplexität der konzipierten und implementierten Lösungen verringert werden kann. Anschließend werden Sicherheitsaspekte besprochen, die für eine finale Version der IDE beachtet werden sollten. Danach wird auf die Nutzung mehrerer Steuereinheiten innerhalb eines Experiments eingegangen. Schließlich wird die noch durchzuführende Evaluation der konzipierten und implementierten Lösungen erläutert.

\paragraph{Komplexität der Lösungen}
Die entwickelten Lösungen besitzen eine relativ hohe Komplexität in der Anwendung bzw. Implementierung. So müssen zunächst entsprechende Laborgeräte für die jeweiligen Compiler, Debugger, Language Server, etc. entwickelt werden. Diese müssen dann in einem Experiment korrekt konfiguriert werden. Außerdem muss die Experimentkonfiguration dann noch getestet werden um sicherzustellen, dass sie korrekt funktioniert. Als Beispiel soll die Anwendung in der Lehre betrachtet werden. Ein Lehrender hat ggf. nicht die Zeit sich in das komplette CrossLab System einzuarbeiten und neue Laborgeräte zu implementieren. Dementsprechend müssten grundlegende Laborgeräte implementiert und in Kategorien zusammengefasst werden um die Konfiguration von Experimenten für Lehrende zu vereinfachen. Dabei sollten entsprechende Werkzeuge bzw. Bibliotheken konzipiert und implementiert werden, welche bei der Implementierung von Laborgeräten und bei der Zusammenstellung von Experimenten zusätzlich helfen können. Ein Beispiel hierfür könnte ein Experimentkonfigurator sein, der die verschiedenen Laborgeräte samt ihrer angebotenen Services visualisieren kann und deren Verbindung zu einem Experiment ermöglicht.

\paragraph{Sicherheitsaspekte}
Während der prototypischen Implementierung war die Betrachtung von Sicherheitsaspekten keine Priorität. Dementsprechend sollten diese vor dem Einsatz der IDE sowie der entwickelten Laborgeräte überprüft werden um sicherzustellen, dass das System nicht ausgenutzt werden kann. Beispiele für zu betrachtende Sicherheitsprobleme sind die in \cite{wu_ceclipse_2011} erwähnten \quoted{Wrong file operations}, \quoted{Banned operation calling} und \quoted{Excessive resource consumption}.

\paragraph{Mehrere Steuereinheiten}
Während der prototypischen Implementierung wurden nur Experimente mit einer Steuereinheit betrachtet. Für Experimente mit mehreren Steuereinheiten müssen entsprechende Anpassungen vorgenommen werden. Darunter auch die Anpassung der bereitgestellten Benutzerinterfaces.

\paragraph{Evaluation der Lösungen}
Die Implementierung muss evaluiert werden. Dadurch kann festgestellt werden, ob die entwickelten Lösungen den gewünschten Effekt haben. Darunter u.a. der geringere Ressourcenverbrauch und die einfachere Konfiguration von Experimenten, die eine IDE enthalten sollen. Dabei bietet sich ein Probelauf innerhalb eines Praktikumsversuchs an, bei dem die Studierenden eine entsprechende Aufgabe mit der IDE lösen müssen. Hierbei kann auch evaluiert werden, ob das Benutzerinterface für die Studierenden angemessen ist. Auch ein Vergleich mit z.B. WIDE könnte interessant sein, um zu sehen, wie die neuen Funktionen der IDE von den Studierenden wahrgenommen werden.