\section*{Abstract}
Die CrossLab-Architektur bietet neue Möglichkeiten für die Bereitstellung von Experimenten in online Laboren. Experimente innerhalb der CrossLab-Architektur bestehen aus mehreren über Services miteinander verbundenen Laborgeräten. Dadurch wird die Austauschbarkeit und Wiederverwendbarkeit der einzelnen Laborgeräte erhöht. Integrierte Entwicklungsumgebungen (IDEs) sind ein zentrales Werkzeug zur Bearbeitung von Programmieraufgaben. Daher werden sie auch in online Laboren für z.B. die Programmierung von Microcontrollern verwendet. Dementsprechend soll eine CrossLab-kompatible IDE entwickelt werden. Dazu wird zunächst der Stand der Technik basierend auf einer systematischen Literaturrecherche dargelegt. Daraufhin werden die Anforderungen an die zu entwickelnde IDE ermittelt. Auf deren Grundlage werden Konzepte für die verschiedenen Funktionen der IDE erstellt. Diese umfassen, unter Beachtung der CrossLab-Kompatibilität, die Bereitstellung von Dateisystemen, Compilern, Debuggern und Language Servern sowie die Programmierung von Steuereinheiten, die Ausführung von Testfällen und die Echtzeit-Kollaboration zwischen Nutzern innerhalb eines Experiments. Es wird eine prototypische Implementierung mit Visual Studio Code und der Visual Studio Code Extension API vorgenommen. Die Ergebnisse der Arbeit werden diskutiert, wobei die Erfüllung der Anforderungen, alternative Lösungsansätze, Erweiterungsmöglichkeiten und offene Aufgaben betrachtet werden. Für weitere Arbeiten können die Sicherheitsaspekte der IDE, Möglichkeiten zur Reduktion der Komplexität der konzipierten Lösungen sowie die Nutzung von WebAssembly erforscht werden.
\vspace*{\fill}
\section*{Abstract}
The CrossLab architecture offers new possibilities for the provision of experiments in online laboratories. Experiments within the CrossLab architecture consist of several laboratory devices connected to each other via services. This increases the interchangeability and reusability of the individual laboratory devices. Integrated development environments (IDEs) are a central tool for processing programming tasks. They are therefore also used in online laboratories for programming microcontrollers, for example. Accordingly, a CrossLab-compatible IDE is to be developed. To this end, the state of the art is first presented based on a systematic literature review. The requirements for the IDE to be developed are then determined. On this basis, concepts for the various functions of the IDE are created. Taking into account CrossLab compatibility, these include the provision of file systems, compilers, debuggers and language servers as well as the programming of control units, the execution of test cases and real-time collaboration between users within an experiment. A prototype implementation with Visual Studio Code and the Visual Studio Code Extension API is carried out.  The results of the work are discussed, considering the fulfillment of the requirements, alternative solutions, extension possibilities and open tasks. For further work, the security aspects of the IDE, possibilities for reducing the complexity of the designed solutions and the use of WebAssembly can be explored.