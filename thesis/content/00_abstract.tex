\section*{Abstract}
Die CrossLab-Architektur bietet neue Möglichkeiten für die Bereitstellung von Experimenten in online Laboren. Daher wird eine CrossLab-kompatible integrierte Entwicklungsumgebung (IDE) entwickelt. Dazu wird zunächst der Stand der Technik mithilfe einer systematischen Literaturrecherche ermittelt. Daraufhin werden die Anforderungen an die zu entwickelnde IDE ermittelt. Basierend auf diesen werden Konzepte für die verschiedenen Funktionen der IDE erstellt. Diese umfassen, unter Beachtung der CrossLab-Kompatibilität, die Bereitstellung von Dateisystemen, Compilern, Debuggern und Language Servern sowie die Programmierung von Steuereinheiten, die Ausführung von Testfällen und die Echtzeit-Kollaboration zwischen Nutzern innerhalb eines Experiments. Es wird eine prototypische Implementierung mit Visual Studio Code und der VSCode Extension API vorgenommen. Die Ergebnisse der Arbeit werden diskutiert, wobei die Erfüllung der Anforderungen, alternative Lösungsansätze, Erweiterungsmöglichkeiten und offene Aufgaben betrachtet werden. Für weitere Arbeiten können die Sicherheitsaspekte der IDE, Möglichkeiten zur Reduktion der Komplexität der konzipierten Lösungen sowie die Nutzung von WebAssembly erforscht werden.
\vspace*{\fill}
\section*{Abstract}
The CrossLab architecture offers new possibilities for the provision of experiments in online laboratories. A CrossLab-compatible integrated development environment (IDE) is therefore being developed. To this end, the state of the art is first determined with the help of a systematic literature search. The requirements for the IDE to be developed are then determined. Based on these, concepts for the various functions of the IDE are created. Taking into account CrossLab compatibility, these include the provision of file systems, compilers, debuggers and language servers as well as the programming of control units, the execution of test cases and real-time collaboration between users within an experiment. A prototypical implementation with Visual Studio Code and the VSCode Extension API is carried out. The results of the work are discussed, considering the fulfillment of the requirements, alternative solutions, extension possibilities and open tasks. For further work, the security aspects of the IDE, possibilities for reducing the complexity of the designed solutions and the use of WebAssembly can be explored.