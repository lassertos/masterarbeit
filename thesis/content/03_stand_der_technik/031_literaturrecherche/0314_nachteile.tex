\subsection{Nachteile}\label{section:stand-der-technik:literaturrecherche:nachteile}

Neben den zuvor beschriebenen Vorteilen besitzen online IDEs auch einige Nachteile, welche im Folgenden beschrieben werden.

\paragraph{Onlinezwang}
Ein Nachteil von online IDEs ist der oftmals damit verbundene Onlinezwang und die damit einhergehenden Probleme, die durch Latenzen, Instabilitäten und Ausfälle des Netzwerks ausgelöst werden. So können Latenzen zu einer schlechteren Nutzererfahrung führen indem z.B. während einer Echtzeit Kollaboration die Änderungen anderer Nutzer stark verzögert ankommen. Instabilitäten können dazu führen, dass Änderungen verloren gehen bzw. der Nutzer auf eine stabilere Verbindung warten muss. Netwerkausfälle bedeuten in den meisten Fällen, dass der Nutzer nicht mit der Programmierung fortfahren kann bis die Netzwerkverbindung wiederhergestellt wird. Außerdem kann es auch vorkommen, dass ein Fehler auf der Serverseite auftritt, wodurch Nutzer die online IDE nicht nutzen können. In diesem Fall müssen Nutzer darauf warten, bis das zugrundeliegende Problem behoben wird.

\paragraph{Abhängigkeit von Anbietern}
In den meisten Fällen besitzen Nutzer keine Kontrolle über die installierte Software. Sollte der Anbieter Updates durchführen kann dies dazu führen, dass zuvor funktionierende Programme des Nutzers nun versionsabhängige Fehler beinhalten. Weiterhin können die angebotenen IDEs auch komplett eingestellt werden, wodurch Nutzer keine Möglichkeit mehr besitzen diese zu nutzen. Zudem können Anbieter auch ihre Preise anpassen, wodurch die Kosten für die online IDE steigen können.

\paragraph{Implementierungs und Verwaltungsaufwand}
Der Implementierungs- und Verwaltungsaufwand von online IDEs kann sehr hoch sein \cite{malan_standardizing_2022} und somit einige der Vorteile von online IDEs aufwiegen. In gewissen Szenarien ist auch die Skalierbarkeit von online IDEs ein Problem, da diese oftmals mit erhöhten Kosten oder einem entsprechend höherem Verwaltungsaufwand verbunden ist. Diese Probleme sind entsprechend gravierender, wenn kein entsprechendes Personal für verfügbar ist und diese Aufgaben z.B. an Lehrende übertragen werden.

\paragraph{Sicherheit}
Anbieter von online IDEs müssen sicherstellen, dass Nutzer keine ungewünschen Aktionen ausführen können, die das System beeinträchtigen könnten. Derartige Aktionen sind z.B. \quoted{Wrong file operations}, \quoted{Banned operation calling}, \quoted{Excessive resource consumption} \cite{wu_ceclipse_2011} und \quoted{Arbitrary code execution} \cite{srinivasa_bad_2022}.

\paragraph{Erweiterbarkeit}
Zudem sind viele der während der Literaturrecherche gefundenen online IDEs sehr anwendungsspezifisch und somit in einigen Aspekten stark eingeschränkt, z.B. in den unterstützten Editoren, Programmiersprachen sowie den vorhandenen Einstellungs- und Erweiterungsmöglichkeiten. Weiterhin werden viele der gefundenen IDEs nicht mehr angeboten bzw. unterstützt und oftmals sind diese auch nicht quelloffen.

