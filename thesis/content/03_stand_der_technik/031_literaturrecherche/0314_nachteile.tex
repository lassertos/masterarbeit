\subsection{Nachteile}\label{section:stand-der-technik:literaturrecherche:nachteile}

% Nachteile:
% \begin{itemize}
%     \item + Onlinezwang
%     \item + Netzwerklatenz / Instabilität
%     \item + Keine Kontrolle über Updates
%     \item + Ggf. Bindung an online Provider wie Github Codespaces / AWS
%     \item + Skalierbarkeit (Ressourcenaufwand)
%     \item + IDEs aus Literatur oftmals nicht lange unterstützt
%     \item + Ggf. hoher Verwaltungsaufwand
%     \item + Oftmals sehr anwendungsspezifisch
%     \item (Zu einfaches / schwieriges Nutzerinterface)
%     \item + Neue Sicherheitsprobleme (\quoted{Wrong file operations}, \quoted{Banned operation calling}, \quoted{Excessive resource consumption}, Nutzung der IDE für Bots - \quoted{Arbitrary code execution})
%     \item (Abhängigkeitsmanagement / Bibliothekmanagement)
%     \item (Schwierigkeiten mit GUI Programmen)
%     \item Schwierige Aufsetzung der IDE für Lehrende
% \end{itemize}

Ein Nachteil von online IDEs ist der oftmals damit verbundene Onlinezwang und die damit einhergehenden Probleme, die durch Latenzen, Instabilitäten und Ausfälle des Netzwerks ausgelöst werden. So können Latenzen zu einer schlechteren Nutzererfahrung führen indem z.B. während einer Echtzeit Kollaboration die Änderungen anderer Nutzer stark verzögert ankommen. Instabilitäten können dazu führen, dass Änderungen verloren gehen bzw. der Nutzer auf eine stabilere Verbindung warten muss. Netwerkausfälle bedeuten in den meisten Fällen, dass der Nutzer nicht mit der Programmierung fortfahren kann bis die Netzwerkverbindung wiederhergestellt wird. Außerdem kann es auch vorkommen, dass ein Fehler auf der Serverseite auftritt, wodurch Nutzer die online IDE nicht nutzen können. In diesem Fall können Nutzer nur darauf warten, bis das Problem behoben wird. Zudem haben Nutzer keine Kontrolle über Updates, was dazu führen kann, dass neue Fehler in das System gelangen bzw. dass zuvor funktionierende Programme des Nutzers nun nicht mehr ausgeführt werden können. Nutzer sind an den entsprechenden Anbieter gebunden. \todoaddref[]{Cloud IDE Provider} In gewissen Szenarien ist auch die Skalierbarkeit von online IDEs ein Problem, da diese oftmals mit erhöhten Kosten verbunden ist. Außerdem kann eine online IDE in gewisser Hinsicht auch einen erhöhten Verwaltungsaufwand mit sich bringen, da je nach angebotenen Features, sichergestellt werden muss, dass Nutzer keine ungewünschen Aktionen ausführen können, wie z.B. \quoted{Wrong file operations}, \quoted{Banned operation calling}, \quoted{Excessive resource consumption} \cite{wu_ceclipse_2011} und \quoted{Arbitrary code execution} \cite{srinivasa_bad_2022}.  Zudem sind viele der während der Literaturrecherche gefundenen online IDEs sehr anwendungsspezifisch und somit in einigen Aspekten stark eingeschränkt, z.B. in den unterstützten Editoren, Programmiersprachen sowie den vorhandenen Erweiterungsmöglichkeiten. \todoaddref[]{Einschränkun-gen von vorhandenen online IDEs} Zudem werden viele der gefundenen IDEs nicht mehr angeboten bzw. unterstützt und oftmals ist auch kein Quellcode auffindbar.