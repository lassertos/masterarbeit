\subsection{Anforderungen}\label{section:stand-der-technik:literaturrecherche:anforderungen}

% Anforderungen:
% \begin{itemize}
%     \item Angemessenes Nutzerinterface
%     \item Syntax-Highlighting
%     \item Code-Vervollständigung
%     \item Kompilierung
%     \item Debugging
%     \item Kollaboration
%     \item (Erweiterbarkeit)
%     \item Sehr anwendungsspezifisch (z.B. Flashing von Microcontrollern, Rollenwechsel in Kollaboration)
%     \item Kommunikationsmöglichkeiten
%     \item Projektmanagement Features
%     \item Versionskontrolle
%     \item Programmiersprachen Support
%     \item Programmausührung
%     \item Testen
%     \item Input / Output
%     \item Guides für Nutzer
%     \item Möglichkeiten zur Datenerhebung
%     \item Interfaces für Lehrende
% \end{itemize}

Anforderungen an online IDEs sind meist sehr anwendungsspezifisch. Dementsprechend ist ein für die Zielgruppe angemessenes Nutzerinterface von hoher Bedeutung. Ein Student, der bereits mehrere Programmierkurse belegt hat und Erfahrungen mit IDEs sammeln konnte wird ein ihm vertrautes Interface mit bereits bekannten Funktionen wertschätzen, während das gleiche Interface einen Schüler, der gerade erst anfängt Programmieren zu lernen, überfordern würde. Ein für Schüler ausgelegtes Interface würde umgekehrt den Studenten frustrieren, da dieser ggf. bereits bekannte Funktionen nicht nutzen kann. Im Allgemeinen sollten aber die in einer IDE verwendeten Code Editoren Standardfunktionen wie z.B. Syntax Highlighting und Code-Vervollständigung für textbasierte Programmiersprachen unterstützen. Weiterhin sind oftmals auch die Kompilierung und Ausführung sowie das Debugging und Testen von Programmen wichtige Anforderungen an online IDEs. Zudem sollten den Nutzern auch Möglichkeiten zur Kollaboration angeboten werden. Dabei kann es den Nutzern unter anderem ermöglicht werden gleichzeitig an Projekten zu arbeiten, wobei gleichzeitig auch entsprechende Kommunikationsmöglichkeiten bereitgestellt werden können. Allerdings kann die Kollaboration auch asynchron über Versionskontrollsysteme wie z.B. Git \todoaddref[]{Git} und über Foren z.B. innerhalb eines verwendeten LMS ermöglicht werden. Eine weitere Anforderung ist die Möglichkeit zur Datenerhebung.