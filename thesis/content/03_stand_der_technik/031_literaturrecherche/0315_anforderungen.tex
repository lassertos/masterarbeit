\subsection{Anforderungen}\label{section:stand-der-technik:literaturrecherche:anforderungen}

Die Anforderungen an online IDEs sind sehr anwendungsspezifisch. Allerdings gibt es dennoch einige allgemeinere Anforderungen, die im Folgenden erläutert werden.

% Anforderungen:
% \begin{itemize}
%     \item Angemessenes Nutzerinterface
%     \item Syntax-Highlighting
%     \item Code-Vervollständigung
%     \item Kompilierung
%     \item Debugging
%     \item Kollaboration
%     \item (Erweiterbarkeit)
%     \item Sehr anwendungsspezifisch (z.B. Flashing von Microcontrollern, Rollenwechsel in Kollaboration)
%     \item Kommunikationsmöglichkeiten
%     \item Projektmanagement Features
%     \item Versionskontrolle
%     \item Programmiersprachen Support
%     \item Programmausührung
%     \item Testen
%     \item Input / Output
%     \item Guides für Nutzer
%     \item Möglichkeiten zur Datenerhebung
%     \item Interfaces für Lehrende
% \end{itemize}

\paragraph{Interface}
Das Nutzerinterface sollte an die Zielgruppe angepasst sein \cite{malan_standardizing_2022}. Ein professioneller Programmierer erwartet ein anderes Interface als ein Schüler, der zum ersten Mal programmiert. Dementsprechend könnte für erstere z.B. ein bereits vorhandenes IDE Interface genutzt werden, während für zweitere eine vereinfachtes Interface angeboten werden sollte, dass z.B. nur die nötigsten Funktionen anbietet um die unerfahrenen Nutzer nicht zu überfordern.

\paragraph{Editor Funktionen}
Online IDEs sollten Editor Funktionen, wie z.B. Syntax Highlighting, Code-Vervollständigung, Code-Navigation und Refactoring anbieten. Dabei müssen diese nicht für alle Programmiersprachen angeboten werden, aber es ist vom Vorteil wenn diese über entsprechende Erweiterungsmechanismen hinzugefügt werden können.

\paragraph{Projektmanagement}
Es sollte Nutzern ermöglicht werden ihre Projekte in der online IDE speichern zu können. Dabei gibt es die Möglichkeiten der clientseitigen Speicherung, bei denen die Daten z.B. im Browser des Nutzers hinterlegt werden, und der serverseitigen Speicherung. Letztere ermöglicht wie zuvor erwähnt den systemunabhängigen Zugriff auf die Projekte des Nutzers.

\paragraph{Kollaboration}
Eine häufig genannte Anforderung ist die Bereitstellung von Kollaborationsmöglichkeiten für die Nutzer der online IDE. Dabei gibt es synchrone und asynchrone Kollaborationsmöglichkeiten. Erstere beinhalten z.B. die Möglichkeit mit anderen Nutzern gleichzeitig an einer geteilten Datei arbeiten zu können. Zweitere beinhalten z.B. Versionskontrollsysteme, Chats und Foren.

\paragraph{Erweiterbarkeit}
Eine erweiterbare online IDE hat den Vorteil, dass sie ggf. für mehrere verschiedene Anwendungsfälle genutzt werden kann. Eine bereits genannte Erweiterungsmöglichkeit ist z.B. die Unterstützung neuer Programmiersprachen über die Bereitstellung der entsprechenden Editor Funktionen.

\paragraph{Datenerhebung}
Wie bereits in den Vorteilen erwähnt ermöglicht die Datenerhebung ein besseres Verständnis für das Nutzerverhalten. Zudem können die erhobenen Daten genutzt werden um die Nutzererfahrung zu verbessern. Daher ist die Möglichkeit der Datenerhebung eine oftmals genannte Anforderung.
