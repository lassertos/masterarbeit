\subsection{Vorteile}\label{section:stand-der-technik:literaturrecherche:vorteile}

% Vorteile:
% \begin{itemize}
%     \item + Keine Installation
%     \item + Keine Zeitrestriktion (im Gegensatz zu Computerlaboren)
%     \item + Vorbereitete Entwicklungsumgebung mit allen benötigten Tools und Bibliotheken bereits vorgegeben
%     \item + Gleiche Entwicklungsumgebung für alle
%     \item + Einfacheres Teilen von Projekten
%     \item + Vereinfachte Kollaboration
%     \item + Von überall nutzbar
%     \item + Gleiche Daten auf allen Geräten (bei Server-Speicherung)
%     \item + Ggf. kein leistungsstarker Rechner notwendig
%     \item + Skalierbarkeit
%     \item + Einbindbarkeit in LMS
%     \item + Offline Nutzbarkeit von browserbasierten IDEs
%     \item + Einfachere Datenerhebung
%     \item (Einfaches Nutzerinterface)
% \end{itemize}

Der am meisten genannte Vorteil von online IDEs ist die Tatsache, dass diese keine Installation benötigen. Somit wird es Nutzern ermöglicht direkt mit der Programmierung zu beginnen ohne zuvor die benötigte Software installieren zu müssen. Weiterhin erlaubt es den Lehrenden sich besser auf ihre Lehrveranstaltungen zu konzentrieren, da sie weniger Zeit mit damit verbringen müssen den Studierenden bei Problemen mit der Einrichtung verschiedener Toolchains zu helfen. Dabei ist zu beachten, dass die Systeme der Studierenden meist sehr unterschiedlich sind im Bezug auf installierte Betriebssysteme, Software und deren Hardware. Zudem erlauben online IDEs den Zugriff von überall mit einem entsprechenden Browser und einer Internetverbindung. Somit wird theoretisch auch der Zugriff mit mobilen Endgeräten wie Smartphones und Tablets ermöglicht, wobei die meisten online IDEs aktuell mit Laptops oder Desktops als Zielgeräten implementiert werden. Außnahmen bilden hier IDEs wie JavaScript Development Environment (JDE) \cite{uehara_javascript_2019} und TouchDevelop \cite{ball_beyond_2015}, welche spezielle Nutzerinterfaces für mobile Endgeräte besitzen. Zudem ist zu beachten, dass auch keine räumliche Bindung mehr für Nutzer besteht, so können Studenten auch außerhalb der angebotenen Zeiten für Computerlabore an ihren Projekten arbeiten. \todooptional[]{Corona?} Ein weiterer Vorteil ist, dass allen Nutzern eine einhaltliche Entwicklungsumgebung geboten werden kann. Somit kann man systemabhängigen bzw. versionsabhängigen Fehlern vermeiden. Weiterhin kann durch online IDEs die Kollaboration zwischen Nutzern erleichtert werden, indem diese z.B. nur einen Link teilen müssen um in Echtzeit mit anderen Nutzern an ihrem Projekt arbeiten zu können. Sollte zudem eine serverseitige Speicherung von Daten vorgenommen werden können Nutzer von all ihren Geräten immer auf den aktuellen Stand ihrer Projekte zugreifen ohne diesen manuell zwischen ihren Geräten teilen zu müssen. Durch die Auslagerung von rechenaufwändigen Prozessen können zudem auch Nutzer unterstützt werden, die keine leistungsstarken Endgeräte besitzen. Komplett im Browser implementierte online IDEs bieten zudem den Vorteil, dass sie ggf. nach dem ersten Laden der Website auch offline genutzt werden können, da die komplette IDE im Browser des Nutzers ausgeführt werden kann. \todo{Beispiele für Browser-IDEs} Browser- sowie cloudbasierte IDEs haben außerdem den Vorteil, dass sie leichter skaliert werden können als klassische Client-Server-Anwendungen. So skalieren browserbasierte IDEs dadurch gut, dass sie nach erstmaligen Laden keine weiteren Serverressourcen in Anspruch nehmen, wohingegen bei cloudbasierten IDEs dynamisch weitere Instanzen hinzugefügt werden können um hohe Lasten zu handhaben. Ein weiterer Vorteil ist die vereinfachte Einbindung von online IDEs in Lernmanagementsysteme (LMS) wie z.B. Moodle \todoaddref[]{Moodle}. Dies ermöglicht unter anderem einen einfacheren Zugang zu der IDE für Studenten sowie die automatische Benotung von ihren erstellten Programmen. \todoaddref[]{Vereinfachte Einbindbarkeit in LMS} Schließlich ist auch die vereinfachte Datenerhebung von Vorteil. Mithilfe der Nutzerdaten kann die IDE an die Nutzer angepasst werden, bzw. können z.B. Lehrende anhand der Nutzerdaten erkennen bei welchen Aufgaben Studenten die meisten Probleme haben bzw. womit sie die meiste Zeit verbringen. \todoaddref[]{Quellen für alle verschiedenen Vorteile}