\subsection{Vorteile}\label{section:stand-der-technik:literaturrecherche:vorteile}

Online IDEs bieten eine Vielzahl an Vorteilen gegenüber lokalen IDEs. Diese Vorteile werden im Folgenden erläutert.

\paragraph{Keine Installation}
Ein Vorteil von online IDEs ist die Tatsache, dass diese keine Installation benötigen \cite{srinivasa_bad_2022}\cite{tran_interactive_2013}\cite{yang_evaluations_2018}. Somit wird es Nutzern ermöglicht direkt mit der Programmierung zu beginnen ohne zuvor die benötigte Software auf ihrem System installieren zu müssen. Dadurch ist es auch möglich die IDE auf Systemen auszuführen, auf denen sie sonst nicht installierbar wäre, wie z.B. mobile Endgeräte \cite{jefferson_pyodideu_2024}\cite{ball_beyond_2015}\cite{uehara_javascript_2019}. Im Bezug auf die Lehre erlaubt es den Lehrenden und Lernenden sich besser auf die Inhalte der Lehrveranstaltung zu konzentrieren, da sie weniger Zeit damit verbringen müssen systemabhängige Probleme mit der Einrichtung der IDE zu lösen \cite{valez_student_2020}.

\paragraph{Zeit- und Ortsunabhängigkeit}
Nutzer können jederzeit und von überall auf online IDEs zugreifen, solange sie einen entsprechenden Browser sowie eine Internetverbindung haben. Weiterhin können einige komplett im Browser nutzbare IDEs nach dem erstmaligen Laden auch offline genutzt werden \cite{jefferson_pyodideu_2024}. Außerdem erlaubt eine serverseitige Speicherung von Daten einen systemunabhängigen Zugriff auf diese \cite{ball_beyond_2015}. Somit können Nutzer selbst wählen, wann und mit welchem Gerät sie die IDE nutzen möchten, ohne von Zugriffszeiten und spezieller Hardware abhängig zu sein.

\paragraph{Einheitliche Umgebung}
Nutzern kann, unabhängig von ihrem System, eine einhaltliche Entwicklungsumgebung angeboten werden \cite{molnar_evaluation_2023}\cite{tran_interactive_2013}. Somit können systemabhängige bzw. versionsabhängige Probleme vermieden werden, was z.B. in der Lehre dazu führt, dass Lehrende und Lernende weniger Zeit für die Behebung derartiger Probleme aufbringen müssen \cite{valez_student_2020}. Außerdem können Client-Server sowie cloudbasierte online IDEs die Benutzererfahrung für Nutzer mit weniger performanten Systemen verbessern, da ein Teil der Berechnungen von einem externen Server durchgeführt wird.

\paragraph{Einbindbarkeit in Lernmanagementsysteme}
Ein weiterer Vorteil von online IDEs ist die vereinfachte Einbindbarkeit von online IDEs in \ac{LMS}, wie z.B. Moodle \cite{noauthor_moodle_nodate}. Da online IDEs im Browser des Nutzers ausgeführt werden können diese direkt in LMS eingebunden werden. Weiterhin können durch die Implementierung entsprechender Schnittstellen, wie z.B. \ac{LTI} \cite{noauthor_lti_nodate}, auch weitere Funktionen ermöglicht werden, wie z.B. die automatische Bewertung der Lösungen von Lernenden. Diese Art der Integration kann die Benutzererfahrung der Lernenden verbessern.

\paragraph{Einfachere Datenerhebung}
Online IDEs können die Erhebung von Nutzerdaten vereinfachen \cite{efopoulos_wipe_2005}\cite{singh_pyguru_nodate}\cite{helminen_recording_2013}. So können u.a. Cursorbewegungen, Code-Änderungen und Zeitaufwand aufgenommen und später analysiert werden. Mithilfe der Nutzerdaten können z.B. Lehrende erkennen bei welchen Aufgaben Studenten die meisten Probleme haben bzw. womit sie die meiste Zeit verbringen.

\paragraph{Skalierbarkeit}
Browser- sowie cloudbasierte IDEs besitzen eine hohe Skalierbarkeit. Browserbasierte IDEs können nachdem sie im Browser des Nutzers geladen wurden ohne bzw. mit geringen zusätzlichen Serverressourcen genutzt werden \cite{ball_beyond_2015}\cite{jefferson_pyodideu_2024}. Cloudbasierte IDEs können sich an dynamische Lastverhältnisse anpassen \cite{noauthor_azure-cloud-services_nodate}\cite{noauthor_ec2-autoscaling_nodate}. So kann für jeden Nutzer beim Starten der IDE eine entsprechende Instanz gestartet werden, die beim Verlassen der IDE wieder gestoppt wird. Dabei ist der Cloud-Anbieter für die Bereitstellung der entsprechenden Serverressourcen verantwortlich. Klassische Client-Server Implementierungen können auch die Skalierbarkeit erhöhen, wenn man die zuvor genannten Vorteile betrachtet.