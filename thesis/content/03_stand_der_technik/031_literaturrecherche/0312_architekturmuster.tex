\subsection{Architekturmuster}\label{section:stand-der-technik:literaturrecherche:architekturmuster}

Die Publikationen beschreiben eine Vielzahl an verschiedenen webbasierten integrierten Entwicklungsumgebungen. Dabei kann eine Unterteilung in die folgenden drei Kategorien erfolgen:

\begin{itemize}
    \item \textbf{Client-Server-basierte Lösungen} \\
          Systeme dieser Art zeichnen sich dadurch aus, dass sie eine Client-Server-Archi-tektur verwenden. Hierbei werden Features, die nicht innerhalb eines Browsers ausgeführt werden können (z.B. Kompilierung) über einen entsprechenden Server bereitgestellt.
    \item \textbf{Browser-basierte Lösungen} \\
          Systeme dieser Art zeichnen sich dadurch aus, dass alle Features im Browser des Nutzers ausgeführt werden können, ohne die Hilfe eines separaten Servers.
    \item \textbf{Cloud-basierte Lösungen} \\
          Systeme dieser Art zeichnen sich dadurch aus, dass sie als Cloud-Service angeboten werden können. Meist erhalten Nutzer eine komplett eigene Umgebung samt Dateisystem, Compiler, Debugger und weiteren Tools. Diese Lösungen nutzen oftmals Containerisierung.
\end{itemize}