\subsection{Architekturmuster}\label{section:stand-der-technik:literaturrecherche:architekturmuster}

Die betrachteten Publikationen beschreiben eine Vielzahl an verschiedenen online IDEs. Dabei kann eine Unterteilung in die folgenden drei Kategorien erfolgen:

\begin{itemize}
      \item \textbf{Client-Server-basierte Lösungen} \\
            Systeme dieser Art zeichnen sich dadurch aus, dass sie eine Client-Server-Archi-tektur verwenden. Hierbei werden Features, die nicht innerhalb eines Browsers ausgeführt werden können (z.B. Kompilierung) über einen entsprechenden Server bereitgestellt. Beispiele für derartige online IDEs sind u.a. Adinda \cite{van_deursen_adinda_2010}, CEclipse \cite{wu_ceclipse_2011} und Collabode \cite{goldman_real-time_2011}.
      \item \textbf{Browser-basierte Lösungen} \\
            Systeme dieser Art zeichnen sich dadurch aus, dass alle Features im Browser des Nutzers ausgeführt werden können, ohne die Hilfe eines separaten Servers. Beispiele für derartige online IDEs sind u.a. TouchDevelop \cite{ball_beyond_2015} und auf PyodideU aufbauende IDE \cite{jefferson_pyodideu_2024}.
      \item \textbf{Cloud-basierte Lösungen} \\
            Systeme dieser Art zeichnen sich dadurch aus, dass sie als Cloud-Service angeboten werden können. Meist erhalten Nutzer eine komplett eigene Umgebung samt Dateisystem, Compiler, Debugger und weiteren Tools. Diese Lösungen nutzen oftmals Containerisierung. Beispiele für derartige online IDEs sind u.a. AWS Cloud9 \cite{noauthor_aws-cloud9_nodate} und GitHub Codespaces \cite{noauthor_github-codespaces_2024}.
\end{itemize}