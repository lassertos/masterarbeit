\subsection{Implementierungen}\label{section:stand-der-technik:literaturrecherche:implementierungen}

Es gibt eine Vielzahl an verschiedenen Implementierungen von online IDEs. Um einen Überblick über die unterschiedlichen Ansätze und die Änderungen über den Verlauf der Zeit zu bekommen, werden im Folgenden einige ausgewählte IDEs vorgestellt. Dabei werden die IDEs zeitlich sortiert vorgestellt.

\paragraph{Adinda}
Deursen et al. (2010) \cite{van_deursen_adinda_2010} beschreiben eine online IDE namens Adinda. Die grundlegende Idee von Adinda ist die Zerlegung der Funktionalität einer IDE in einen leichtgewichtigen Client sowie mehrere zusammenarbeitende (Web-)Services. Diese Services sollen dann bestimmte Aufgaben erfüllen, wie z.B. Kompilierung, Testen, kollaboratives Editieren und Datenerhebung. Es werden weiterhin verschiedenste Forschungsfragen aufgestellt, die für das vorgestellte System von Interesse sind. Die prototypische Implementierung von Adinda basiert auf WWWorkspace \cite{ryan_web_2007} und nutzt serverseitig die Eclipse IDE \cite{noauthor_eclipse_nodate}. Der Prototyp unterstützt das Erstellen von Nutzer-Workspaces, Java Projekten, Paketen und Klassendateien sowie Syntax Highlighting, Kompilierung und Code-Vervollständigung.

\paragraph{CEclipse}
Wu et al. (2011) \cite{wu_ceclipse_2011} stellen die online IDE Cloud Eclipse (CEclipse) vor. Die Ziele von CEclipse sind:
\begin{enumerate}
    \item die Bereitstellung von Funktionen der Eclipse IDE \cite{noauthor_eclipse_nodate}, wie zum Beispiel Code-Vervollständigung
    \item die Behandlung von den online IDE spezifischen Sicherheitsproblemen \quoted{\textit{Wrong file operations}}, \quoted{\textit{Banned operation calling}} und \quoted{\textit{Excessive resource consumption}}
    \item die Ausnutzung von Cloud Computing Möglichkeiten, um Entwickler besser zu unterstützen.
\end{enumerate}
Für $1.$ wurde ein entsprechendes Protokoll entwickelt, was es ermöglicht, die gewünschten Funktionen der Eclipse IDE aufzurufen und das Ergebnis im Browser darzustellen. Um die in $2.$ genannten Probleme handhaben zu können, wird ein \textit{Program Behavior Analysis Service} beschrieben. Durch die Einschränkung des Dateisystems auf einen speziellen Ordner kann das Problem \quoted{Wrong file operations} gelöst werden. Das Verbieten bzw. Erlauben von Methoden über eine Blacklist bzw. eine Whitelist kann zur Lösung des Problems \quoted{Banned operation calling} angewendet werden. Durch eine Zeitbegrenzung von laufenden Prozessen kann schließlich auch das Problem \quoted{Excessive resource consumption} behoben werden. Für $3.$ werden über den \textit{Program Behavior Mining Service} Daten über die Nutzung der IDE gesammelt werden. Diese Daten können dann z.B. dazu genutzt werden, dem Entwickler häufig verwendete Befehle mit höherer Priorität vorzuschlagen.

\paragraph{Collabode}
Goldman et al. (2011) \cite{goldman_real-time_2011} beschreiben Collabode, eine kollaborative online IDE für Java. Collabode ermöglicht es mehreren Nutzern gleichzeitig Änderungen an Dateien vorzunehmen. Die Änderungen werden in Echtzeit zwischen den Nutzern synchronisiert. Dabei wird ein spezieller Algorithmus verwendet. Dieser Algorithmus sorgt dafür, dass nur Änderungen eingepflegt werden, die keinen syntaktischen Fehler beinhalten oder erzeugen. Dadurch wird sichergestellt, dass Nutzer nur ihre eigenen Fehler sehen und das Programm unabhängig von den ggf. vorhandenen Fehlern ihrer Teammitglieder kompilieren können. Collabode nutzt EtherPad \cite{noauthor_etherpad_nodate} als Code Editor im Frontend und die Eclipse IDE \cite{noauthor_eclipse_nodate} für die Bereitstellung von Kompilierung, Syntax Highlighting, etc. im Backend.

\paragraph{CoRED}
Lautamäki et al. (2012) \cite{lautamaki_cored_2012} stellen den Collaborative Real-time Editor (CoRED) vor, einen online Code Editor für Java Programme. CoRED nutzt den ACE Editor \cite{noauthor_ace_nodate} im Frontend sowie das Java Development Kit (JDK) \cite{noauthor_jdk_nodate} zur Kompilierung im Backend. Fehlermeldungen während des Kompiliervorgangs werden an den Client zurückgesendet und dann im Frontend angezeigt. Zur Bereitstellung von Echtzeit Kollaboration wird der Algorithmus \textit{Differential Synchronization with shadows} \cite{fraser_differential_2009} von Neil Fraser eingesetzt. Weitere Features von CoRED sind das Sperren von Code Bereichen für andere Nutzer sowie die Möglichkeit, Kommentare im Code zu hinterlassen. Auf diese Kommentare können dann andere Nutzer antworten, wodurch eine weitere Interaktionsmöglichkeit besteht. Weiterhin bietet CoRED auch Code-Vervollständigung als Funktion an.

\paragraph{IDEOL}
Tran et al. (2013) \cite{tran_interactive_2013} stellen die online IDE IDEOL vor. IDEOL erlaubt es Nutzern, in Echtzeit miteinander zu kollaborieren. Dies beinhaltet das gleichzeitige Bearbeiten von Dateien mit Synchronisierung der Änderungen zwischen den Nutzern sowie ein Diskussionsforum mit einem Tagging-Mechanismus. Über diesen Mechanismus können Nutzer Codezeilen oder ganze Dateien in einer Diskussion taggen. Weiterhin können Nutzer auch Dateien an ihre Nachrichten anhängen. Zudem bietet IDEOL eine Übersicht über die Änderungen im Code. IDEOL bietet Support für die Entwicklung von C/C++ Programmen und erlaubt auch die Kompilierung, Ausführung sowie das Debuggen von diesen. Um die Echtzeit Kollaboration zu ermöglichen, unterscheidet IDEOL zwischen exklusiven und nicht-exklusiven Operationen. Zu den exklusiven Operationen zählen die Kompilierung, Ausführung und das Debuggen eines Programms. Dateien werden über einen Server synchronisiert und dort gespeichert. Die Behandlung von gleichzeitigen Operationen wird mithilfe eines Operational Transformation Algorithmus \cite{sun_operational_1998} vorgenommen. Exklusive Operationen werden auf der im Server persistent gespeicherten Version ausgeführt, sodass die im Arbeitsspeicher des Servers vorhandene Version weiterhin zur Bearbeitung verwendet werden kann. Nguyen et al. (2016) \cite{nguyen_enhancing_2016} nutzen IDEOL für die web-basierte kollaborative Umgebung EduCo. Das Ziel von EduCo ist die Bereitstellung von Funktionen für Lehrende und Lernende, besonders im Hinblick auf deren Interaktion und Kollaboration.

\paragraph{TouchDevelop}
Ball et al. (2015) \cite{ball_beyond_2015} beschreiben die online IDE TouchDevelop. Das Hauptfeature von TouchDevelop ist die Speicherung aller Programmänderungen, Versionen, Laufzeitinformationen, Bugs sowie Kommentare, Fragen und Feedback von Nutzern in einer zentralen Datenbank. Diese Daten können über entsprechende APIs abgefragt werden. Die Nutzeroberfläche von TouchDevelop unterscheidet sich stark von anderen textbasierten Editoren. So bekommt der Nutzer eine Auswahl an kontextabhängigen Optionen, z.B. if-Anweisungen, for-Schleifen oder verfügbare Variablen. Alle IDE Funktionen sind offline verfügbar, da sie komplett auf der Clientseite implementiert sind. Weiterhin nutzt TouchDevelop eine eigene Programmiersprache. Diese folgt dem imperativen Programmierparadigma, besitzt ein starkes Typsystem sowie eine Vielzahl an plattformübergreifenden APIs.

\paragraph{CodePilot}
Warner und Guo (2017) \cite{warner_codepilot_2017} stellen die online IDE CodePilot vor. Diese ermöglicht es Nutzern, Webapplikationen mithilfe von HTML, CSS und JavaScript zu entwickeln. Zudem können Nutzer mithilfe von Firepad \cite{noauthor_firepad_nodate} gleichzeitig an Projekten arbeiten. Weiterhin bietet CodePilot eine GitHub \cite{noauthor_github_nodate} Integration, wodurch Nutzer ihre Projekte aus GitHub importieren können. Über diese Integration werden auch bei jedem Commit die entsprechenden Änderungen an GitHub zurückgesendet. Weiterhin können Nutzer über einen Aktivitätsfeed die aktuellen Ereignisse nachverfolgen und mithilfe eines integrierten Text-Chats miteinander kommunizieren. Durch den Ace-Editor \cite{noauthor_ace_nodate} werden außerdem Syntax Highlighting und Code-Vervollständigung bereitgestellt. Zusätzlich bietet CodePilot die Möglichkeit, Issues zu erstellen. Diese werden dann mit GitHub synchronisiert und es wird ein Snapshot des aktuellen Projekts auf GitHub hinterlegt. Bei der Erstellung von Issues können auch Screenshots angefügt werden. Jede Issue enthält die Referenz auf den entsprechenden Snapshot des Projekts.

\paragraph{RIDE}
Über mehrere Publikationen hinweg wird die Entwicklung der Reflex IDE (RIDE) beschrieben. Zunächst wird von Bastrykina et al. (2021) \cite{bastrykina_developing_2021} ein entsprechender Kernel mit dem Xtext Framework \cite{noauthor_xtext_nodate} entwickelt. Dieser kann in der Eclipse IDE verwendet werden, um Funktionen wie Code-Vervollständigung und Code-Generation für die domainspezifische Sprache Reflex verfügbar zu machen. Zudem wird auch eine Integration mit dem \ac{LSP} \cite{noauthor_language-server-protocol_nodate} erreicht, wodurch diese Features auch für andere Code Editoren nutzbar sind. Darauf aufbauend wird von Gornev und Liakh (2021) \cite{gornev_ride_2021} die Konzipierung und Implementierung einer auf Theia basierten Web-Variante von RIDE vorgestellt. Gornev et al. (2022) \cite{gornev_towards_2022} beschreiben ein System, welches Docker verwendet, um die Web-Version von RIDE für mehrere simultane Nutzer bereitstellen zu können. Gornev und Bondarchuk (2023) \cite{gornev_towards_2023} stellen ein Framework vor, welches Echtzeit-Kollaboration in RIDE ermöglicht. Kuznetsov und Zyubin (2024) \cite{kuznetsov_development_2024} beschreiben darüber hinaus die Entwicklung eines Projektmanagement-Systems für RIDE.

\paragraph{PyodideU}
Jefferson et al. (2024) \cite{jefferson_pyodideu_2024} beschreiben eine IDE, die es Nutzern ermöglicht Python Code im Browser zu schreiben und auszuführen. Dabei wird das Programm des Nutzers lokal in dessen Browser durchgeführt. Dies wird durch den Einsatz von PyodideU, einer erweiterten Version der auf WebAssembly basierenden Python Distribution Pyodide \cite{noauthor_pyodide_nodate}, erreicht. Zusätzlich wird den Nutzern auch eine Grafikbibliothek samt eines Debuggers angeboten, der es ermöglicht, Zeile für Zeile und auch rückwärts durch das Programm zu gehen und die entsprechenden Änderungen an der Grafik zu sehen. Weiterhin wird durch PyodideU auch die synchrone Eingabe von Daten unterstützt, während Python im Main-Thread des Browsers läuft. Zudem wird auch ein Dateisystem bereitgestellt. Insgesamt wurde die IDE sowohl von Studenten als auch von Lehrenden als hilfreich wahrgenommen.

\paragraph{CS50}
Malan (2024) \cite{malan_containerizing_2024} beschreibt die verschiedenen Ansätze zur Bereitstellung einer integrierten Entwicklungsumgebung für die Teilnehmer des Einführungskurses in die Programmierung (CS50) an der Harvard University. Zunächst wurde seit 2007 ein On-Campus Cluster für die Studenten angeboten. Studenten konnten sich über SSH mit diesem verbinden und dort ihre Programme ausführen. Dieser Cluster wurde 2008 mithilfe von Amazon Web Services (AWS) \cite{noauthor_amazon_nodate} in die Cloud überführt. Diese cloud-basierte Lösung wurde 2011 durch Client-seitige virtuelle Maschinen ersetzt. In 2015 wurde eine auf Docker basierende Lösung erarbeitet, die zunächst die Cloud9 IDE \cite{noauthor_aws-cloud9_nodate} als Frontend nutzte. In 2021 wurde schließlich eine auf Github Codespaces aufbauende Lösung eingeführt, die \ac{VSCode} \cite{noauthor_vscode_nodate} als Code Editor verwendet.