\subsection{IDEs und Code Editoren}\label{section:stand-der-technik:weitere-entwicklungen:ides-und-code-editoren}

Dieser Unterabschnitt befasst sich mit IDEs und Code Editoren. Dabei werden im Folgenden Visual Studio Code \cite{noauthor_vscode_nodate}, JetBrains IDEs \cite{noauthor_jetbrains_nodate} und Stackblitz \cite{noauthor_stackblitz_nodate} vorgestellt.

\paragraph{Visual Studio Code}
\acf{VSCode} \cite{noauthor_vscode_nodate} ist ein von Microsoft \cite{noauthor_microsoft_nodate} entwickelter Code Editor. \ac{VSCode} baut auf dem ebenfalls von Microsoft entwickelten Monaco Editor \cite{noauthor_monaco_nodate} auf und bietet zusätzliche Features an, wie z.B. Workspace-Management, Nutzerinterfaces für Versionskontrolle und Debugging sowie die Möglichkeit Erweiterungen zu installieren. Diese Erweiterungen nutzen die VSCode Extension API \cite{noauthor_vscode-extension-api_nodate} und erlauben es \ac{VSCode} um verschiedenste Funktionen zu erweitern. \ac{VSCode} ist frei verfügbar und kann sowohl als Desktop-, Browser- oder Cloudanwendung genutzt werden. Somit kann \ac{VSCode} für unterschiedlichste Anwendungsfälle verwendet werden.

\paragraph{JetBrains IDEs}
JetBrains \cite{noauthor_jetbrains_nodate} bietet eine Vielzahl an spezialisierten IDEs an. Diese sind meist auf eine spezifische Programmiersprache ausgelegt und beinhalten bereits die zur Programmierung benötigten Softwarewerkzeuge, wie z.B. Compiler, Interpreter, Debugger, Testframeworks und mehr. Zusätzlich könnnen die IDEs mit entsprechenden Plugins \cite{noauthor_jetbrains-plugins_nodate} erweitert werden. Der Großteil der von JetBrains angebotenen IDEs ist kostenpflichtig und als Desktop- sowie als Cloudanwendung nutzbar.

\paragraph{Stackblitz}
Stackblitz \cite{noauthor_stackblitz_nodate} ist eine online IDE für Projekte, die auf der Programmiersprache JavaScript basieren. Dabei wird sowohl die direkte Entwicklung für Browser als auch die Entwicklung für die JavaScript Runtime Node.js \cite{noauthor_nodejs_nodate} unterstützt. Die komplette IDE wird im Browser des Nutzers ausgeführt und bietet Funktionen, wie z.B. Debugging, Versionskontrolle über Git und die Möglichkeit auch kurzzeitig offline weiterarbeiten zu können, bis die Verbindung wiederhergestellt wird. Node.js wird mithilfe von WebContainer \cite{noauthor_webcontainer_nodate} im Browser des Nutzers ausgeführt. Stackblitz besitzt eine kostenlose Variante für die private und nichtkommerzielle Nutzung sowie verschiedene kostenpflichtige Varianten.

% VSCode (https://code.visualstudio.com/)
% Brackets (https://brackets.io/)
% Phoenix Code (https://phcode.dev/)
% JetBrains IDEs (https://www.jetbrains.com/de-de/)
% JupyterLab (https://jupyter.org/)
% Eclipse (https://eclipseide.org/)
% Visual Studio (https://visualstudio.microsoft.com/de/vs/)
% Arduino IDE (https://docs.arduino.cc/software/ide/#ide-v2)
% Stackblitz (https://stackblitz.com/)
% Monaco Editor (https://microsoft.github.io/monaco-editor/)
% Ace (https://ace.c9.io/)