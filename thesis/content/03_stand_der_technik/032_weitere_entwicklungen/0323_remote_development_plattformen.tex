\subsection{Remote Development Plattformen}\label{section:stand-der-technik:weitere-entwicklungen:remote-development-plattformen}

Remote Development Plattformen ermöglichen es, komplette Entwicklungsumgebungen online zu nutzen. Diese werden auf entsprechenden Servern gehostet. Dabei werden meist Container verwendet, welche die komplette benötigte Software für die Entwicklungsumgebung des Nutzers beinhalten und gleichzeitig von anderen Nutzern isoliert sind. Als Beispiele für Remote Development Plattformen werden im Folgenden Eclipse Che \cite{noauthor_eclipse-che_nodate} und Github Codespaces \cite{noauthor_github-codespaces_2024} vorgestellt.

\paragraph{Eclipse Che}
Eclipse Che \cite{noauthor_eclipse-che_nodate} ist ein Projekt der Eclipse Foundation \cite{milinkovich_eclipse-foundation_nodate} und nutzt Kubernetes \cite{noauthor_kubernetes_nodate} zur Bereitstellung von online Entwicklungsumgebungen. Dafür werden Devfiles \cite{noauthor_devfile_nodate} verwendet, um die bereitgestellten IDEs für das jeweilige Projekt zu konfigurieren. Das Eclipse Che Dashboard kann diese Devfiles einlesen und die entsprechende IDE starten. Das Starten der benötigten Services wird von dem sogenannten DevWorkspace Operator übernommen, welcher die Kubernetes API entsprechend erweitert. Eclipse Che kann von Nutzern selbst gehostet werden, was die Verwendung von bereits vorhandener Hardware ermöglicht. Allerdings ist es auch möglich, Eclipse Che auf einer entsprechenden Cloud Plattform hosten zu lassen.

\paragraph{Github Codespaces}
Github Codespaces \cite{noauthor_github-codespaces_2024} ist ein von Microsoft \cite{noauthor_microsoft_nodate} angebotener Dienst zur Bereitstellung von online Entwicklungsumgebungen. Dabei können Nutzer für ihre Github Projekte über eine entsprechende Datei einen Devcontainer \cite{noauthor_devcontainer_nodate} definieren. Dadurch wird es Nutzern ermöglicht, das zur Entwicklung benötigte Betriebssystem, Programme und Einstellungen festzulegen. Diese Informationen werden dann genutzt, um dem Nutzer eine entsprechende Entwicklungsumgebung bereitzustellen. Im Gegensatz zu Eclipse Che kann Github Codespaces nicht selbst gehosted werden.
