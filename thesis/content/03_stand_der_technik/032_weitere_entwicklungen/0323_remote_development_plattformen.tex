\subsection{Remote Development Plattformen}\label{section:stand-der-technik:weitere-entwicklungen:remote-development-plattformen}

Remote Development Plattformen ermöglichen es Nutzern komplette Entwicklungsumgebungen online zu nutzen. Diese werden auf entsprechenden Servern gehostet. Dabei werden meist Container verwendet, welche die komplette benötigte Software für die Entwicklungsumgebung des Nutzers beinhalten und gleichzeitig von anderen Nutzern isoliert sind. Als Beispiele für Remote Development Plattformen werden im Folgenden Eclipse Che \cite{noauthor_eclipse-che_nodate} und Github Codespaces \cite{noauthor_github-codespaces_2024} vorgestellt.

\paragraph{Eclipse Che} \dots

\paragraph{Github Codespaces} \dots

% Eclipse Che (https://eclipse.dev/che/)
% Coder (https://coder.com/)
% Gitpod (https://www.gitpod.io/)
% Replit (https://replit.com/)
% Codeanywhere (https://codeanywhere.com/)
% Github Codespaces (https://github.com/features/codespaces)
% Amazon Cloud9 (discontinued) (https://aws.amazon.com/de/cloud9/)