\subsection{IDE Frameworks}\label{section:stand-der-technik:weitere-entwicklungen:ide-frameworks}

IDE Frameworks ermöglichen die Implementierung eigener IDEs, basierend auf einem entsprechenden Grundgerüst. Dabei werden im Folgenden Eclipse Theia \cite{noauthor_theia_nodate} und OpenSumi \cite{noauthor_opensumi_nodate} vorgestellt.

\paragraph{Eclipse Theia}
Eclipse Theia \cite{noauthor_theia_nodate} ist ein Projekt der Eclipse Foundation \cite{milinkovich_eclipse-foundation_nodate} bestehend aus der Theia Platform, einem Framework für die Entwicklung von IDEs, sowie der darauf aufbauenden Theia IDE. Theia nutzt den Monaco Editor \cite{noauthor_monaco_nodate} als Code Editor. Eine mithilfe der Theia Platform erstellten IDE besteht aus sogenannten Theia Erweiterungen. Diese sind Teil der Kompilierung und können tiefgreifende Änderungen an der IDE vornehmen. Weiterhin können auch Erweiterungen genutzt werden, die mithilfe der VSCode Extension API \cite{noauthor_vscode-extension-api_nodate} erstellt wurden. Diese werden zur Laufzeit hinzugefügt und haben somit nur einen begrenzten Zugriff auf die internen Schnittstellen der IDE. Daher bieten Theia Erweiterungen die Möglichkeit, tiefergreifende Änderungen vorzunehmen, als es mit VSCode Erweiterungen möglich ist. Mit der Theia Platform erstellte IDEs bestehen aus einem Backend sowie einem Frontend. Allerdings kann das Frontend mit entsprechendem Funktionsverlust auch ohne das Backend verwendet werden. Dadurch können die IDEs als Desktop-, Browser- und Cloudanwendungen implementiert werden.

\paragraph{OpenSumi}
OpenSumi \cite{noauthor_opensumi_nodate} ist ein von Alibaba \cite{noauthor_alibaba_nodate} entwickeltes Framework zur Entwicklung von IDEs. OpenSumi verwendet den Monaco Editor \cite{noauthor_monaco_nodate} als Code Editor. Ähnlich wie bei Theia werden auch bei OpenSumi zwei verschiedene Arten von Erweiterungen unterstützt: OpenSumi Erweiterungen und VSCode Erweiterungen. Dabei bieten OpenSumi Erweiterungen mehr Anpassungsmöglichkeiten als VSCode Erweiterungen, da diese auf weitere Schnittstellen zugreifen können. OpenSumi ermöglicht die Entwicklung von IDEs als Desktop-, Browser- und Cloudanwendungen.

% Theia
% OpenSumi