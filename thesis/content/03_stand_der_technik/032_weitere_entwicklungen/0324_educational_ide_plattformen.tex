\subsection{Educational IDE Plattformen}\label{section:stand-der-technik:weitere-entwicklungen:educational-ide-plattformen}

Educational IDE Plattformen zeichnen sich dadurch aus, dass sie neben der Entwicklungsumgebung auch viele weitere Features für die Anwendung in der Lehre bereitstellen. Beispiele für derartige Features sind die automatische Bewertung von abgegebenen Lösungen, Plagiaterkennung und die Möglichkeit der Einbindung in \acp{LMS}. Im folgenden werden CodeHS \cite{noauthor_codehs_nodate} und Codio \cite{noauthor_codio_nodate} vorgestellt.

\paragraph{CodeHS}
CodeHS \cite{noauthor_codehs_nodate} wurde für Schulen entwickelt und stellt eine Vielzahl an verschiedenen Werkzeugen bereit, darunter eine online IDE sowie ein eigenes \ac{LMS}. Die online IDE unterstützt neben allgemeineren Features wie Echtzeit-Kollaboration und Debugging auch lehrspezifische Features wie die automatische Bewertung der Programme und die Möglichkeit mit Mitschülern und Lehrern zu interagieren. Das von CodeHS angebotene \ac{LMS} erlaubt das Management von Klassen und den entsprechenden Lehrressourcen, wie z.B. Skripte und Aufgaben. Außerdem bietet es einen Überblick über den aktuellen Stand der Schüler, durch die Aufzeichnung von Punktzahlen, Code-Historien und der Zeit, die für einzelne Aufgaben benögigt wurde. Über die \ac{LTI} Schnittstelle ist teilweise auch eine Einbindung des CodeHS \ac{LMS} in bereits vorhandene \ac{LMS} möglich.

\paragraph{Codio}
Codio \cite{noauthor_codio_nodate} wurde für die Lehre entwickelt und bietet Nutzern eine online IDE samt \ac{LTI} Integration. Die online IDE bietet u.a. einen integrierten Debugger, die Möglichkeit grafische Anwendungen zu entwickeln sowie Echtzeit-Kollaboration. Dabei wird als zugrunde liegendes Betriebssystem Ubuntu verwendet, zu welchem auch ein entsprechender Zugriff ermöglicht werden kann. Neben der von Codio angebotenen online IDE können auch andere Editoren verwendet werden, wie z.B. \ac{VSCode}. Lehrende haben stets Zugriff auf den Code der Lernenden und können auch dessen Historie einsehen. Weiterhin bietet Codio auch die automatische Bewertung von Lösungen, Plagiatserkennung sowie die Möglichkeit der Integration in \ac{LMS} über die \ac{LTI} Schnittstelle.
