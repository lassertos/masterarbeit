\section{Kompilierung}\label{section:konzeption:kompilierung}

% Die Kompilierung des Quellcodes kann vom CM übernommen werden. Alternativ steht auch hier die Möglichkeit eines externen Servers zur Verfügung. Weiterhin kann auch die Möglichkeit der Kompilierung innerhalb des Browsers des Nutzers erforscht werden. Hierbei gibt es bereits eine Version des C-Compilers clang in WebAssembly. Diese ermöglicht jedoch in der aktuellen Form nur die Kompilierung von C/C++ Code zu WebAssembly. Im Falle des GOLDi Remotelab müsste allerdings eine Kompilierung des Quellcodes für die verwendeten Microcontroller erfolgen. Diese wird normalerweise mit Hilfe des Compilers avr-gcc durchgeführt. Allerdings bietet auch der clang Compiler ein avr Ziel an, welches jedoch einen experimentellen Status hat.

Für die Kompilierung des Quellcodes können bereits vorhandene Compiler verwendet werden. Allerdings ist es notwendig diese über eine entsprechende Schnittstelle in den Code Editor zu integrieren. Dadurch soll es dem Nutzer ermöglicht werden seinen Quellcode zu kompilieren. Hierbei ist zu beachten, dass falls mehrere Steuereinheiten in einem Experiment vorhanden sind der Nutzer wählen kann für welche das jeweilige Programm kompiliert werden soll. Weiterhin können die Compiler über verschiedene Konfigurationsmöglichkeiten an die Bedürfnisse des Nutzers bzw. des Experiments angepasst werden.

\subsection{CrossLab Compilation Service}

Der CrossLab Compilation Service besitzt zwei Rollen: Einen Konsumenten und einen Produzenten. Der Konsument kann Kompilieranfragen an einen Produzenten stellen, welcher diese entgegennimmt und ein entsprechendes Ergebnis an den Konsumenten zurücksendet. Dabei ist zu beachten, dass ein Konsument mit mehreren Produzenten verbunden sein kann. Die Konfiguration der erwarteten Ergebnisformate sowie der Optionen des Compilers werden über die Servicekonfiguration auf der Ebene einer Verbindung zwischen Konsument und Produzent vorgenommen. Es ist den Produzenten freigestellt, welche Konfigurationsmöglichkeiten und welche Ergebnisformate für Konsumenten angeboten werden.