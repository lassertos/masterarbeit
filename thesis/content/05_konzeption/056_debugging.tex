\section{Debugging}\label{section:konzeption:debugging}

% \begin{note}
%     \textbf{Notizen:}
%     \begin{itemize}
%         \item Erwähnung von \autoref{requirement:Debuggen}, \autoref{requirement:Debuggen: CrossLab-Kompatibilität} und \autoref{requirement:Debuggen: Kollaboration}
%         \item Beschreibung der CrossLab-Services + Klassendiagramm (Adapter)
%         \item Beschreibung der CrossLab-Services + Klassendiagramm (Target)
%         \item Konzept für die Ermöglichung der Kollaboration
%         \item Konzept für die Bereitstellung von Debuggern als Laborgeräte
%         \item Beschreibung der Einbindung in die betrachtete Experimentkonfiguration
%         \item (Beschreibung möglicher Einstellungen?)
%     \end{itemize}
% \end{note}

\begin{figure}[tbp]
    \centering
    \resizebox{\textwidth}{!}{
        \begin{tikzpicture}
            \begin{class}[text width=7.5cm]{DebuggingAdapterServiceProducer}{0,0}
                \operation{+ sendMessageDAP()}
                \operation{+ onStartSession()}
                \operation{+ onJoinSession()}
                \operation{+ onMessageDAP()}
            \end{class}
            \begin{class}[text width=7.5cm]{DebuggingAdapterServiceConsumer}{8,0}
                \operation{+ sendMessageDAP()}
                \operation{+ startSession()}
                \operation{+ joinSession()}
                \operation{+ onDapMessageDAP()}
            \end{class}
            \begin{class}[text width=7.5cm]{DebuggingTargetServiceProducer}{0,-3.5}
                \operation{+ sendDebuggingMessage()}
                \operation{+ onStartDebugging()}
                \operation{+ onEndDebugging()}
                \operation{+ onDebuggingMessage()}
            \end{class}
            \begin{class}[text width=7.5cm]{DebuggingTargetServiceConsumer}{8,-3.5}
                \operation{+ sendDebuggingMessage()}
                \operation{+ startDebugging()}
                \operation{+ endDebugging()}
                \operation{+ onDebuggingMessage()}
            \end{class}
        \end{tikzpicture}
    }
    \caption{Klassendiagramm Debugging Services}
    \label{figure:klassendiagramm-debugging-services}
\end{figure}

Nach \autoref{requirement:Debuggen} soll das Debuggen der Programme von Nutzern innerhalb der IDE ermöglicht werden. Dafür soll laut Unteranforderung (a) ein CrossLab-Service für die Bereitstellung und Nutzung von Debuggern sowie für die Kommunikation zwischen Debuggern und Steuereinheiten entwickelt werden. Weiterhin verlangt Unteranforderung (b) die Entwicklung eines CrossLab-Service für die Kommunikation zwischen Debuggern und Steuereinheiten. Der aus Unteranforderung (a) resultierende CrossLab-Service soll nach Unteranforderung (b) von der IDE für die Nutzung von Debuggern verwendet werden. Dementsprechend werden im Folgenden der \textit{Debugging Adapter Service} für die Bereitstellung und Nutzung von Debuggern sowie der \textit{Debugging Target Service} für die Kommunikation zwischen Debuggern und Steuereinheiten vorgestellt. \autoref{figure:klassendiagramm-debugging-services} zeigt ein Klassendiagramm für die beiden Debugging Services.

Der Debugging Adapter Service baut auf dem \textit{\ac{DAP}} \cite{noauthor_debug-adapter-protocol_nodate} von Microsoft auf. Dieses spezifiert Nachrichten die zwischen einer IDE und einem sogenannten \textit{Debug Adapter} ausgetauscht werden. Ein Debug Adapter bildet die Schnittstelle zwischen einem Debugger und einer IDE. Das Protokoll erlaubt somit die Anbindung bestehender Debugger über die Implementierung eines entsprechenden Debug Adapters. Weiterhin wird der Implementierungsaufwand für die Einbindung von Debuggern in IDEs verringert, da nur eine Schnittstelle implementiert werden muss, anstatt jeden Debugger an eine eigens entwickelte Schnittstelle anpassen zu müssen. Somit sollte durch die Verwendung des \ac{DAP} die Einbindung von Debuggern vereinfacht werden.

\begin{figure}[tbp]
    \centering
    \begin{sequencediagram}
        \newthread{ide}{IDE}
        \newthreadShift{debugger}{Debugger}{3cm}
        \newthreadShift{steuereinheit}{Steuereinheit}{3cm}

        \begin{call}{ide}{starte Debug-Sitzung}{debugger}{}
            \begin{call}{debugger}{speichere Programm}{debugger}{}
            \end{call}
            \begin{call}{debugger}{kompiliere Programm}{debugger}{}
            \end{call}
            \begin{call}{debugger}{starte Debuggen}{steuereinheit}{}
                \begin{call}{steuereinheit}{lade Programm}{steuereinheit}{}
                \end{call}
            \end{call}
        \end{call}

        \begin{call}{ide}{DAP Nachrichten}{debugger}{DAP Nachrichten}
        \end{call}

        \prelevel\prelevel

        \begin{call}{debugger}{Debugger Nachrichten}{steuereinheit}{Debugger Nachrichten}
        \end{call}
    \end{sequencediagram}
    \caption{Start einer Debug-Sitzung}
    \label{figure:start-einer-debug-sitzung}
\end{figure}

\paragraph{Start einer Debug-Sitzung} \autoref{figure:start-einer-debug-sitzung} zeigt ein Sequenzdiagramm für den Start einer Debug-Sitzung. Die Funktion \texttt{startSession()} des Debugging Adapter Service Consumer kann zum Starten einer Debug-Sitzung werden. Dabei werden der Ordner, welcher das zu debuggende Programm enthält, und dessen URI bzw. Pfad sowie Konfigurationsoptionen für den Debugger an den Debugging Adapter Service Producer gesendet. Dieser löst dann ein \texttt{StartSession}-Event mit den übergebenen Daten aus, welches durch einen entsprechenden Event Handler abgefangen werden kann. Dadurch kann die Implementierung an den jeweiligen Debugger angepasst werden. Im Allgemeinen sollte zunächst der übersendete Ordner auf dem Dateisystem des Debugging Adapter Service Producer gespeichert werden. Weiterhin kann ggf. eine Kompilierung des Programms mit speziellen Optionen für das Debuggen vorgenommen werden, wobei der in \autoref{section:konzeption:kompilierung-und-programmierung} vorgestellte CrossLab-Service verwendet werden kann. Außerdem sollte der Debugger mit den Konfigurationsoptionen gestartet werden und es sollte ein eindeutiger Kennzeichner für die Debug-Sitzung generiert werden. Weiterhin sollte die zu debuggende Steuereinheit über den Start der Debug-Sitzung informiert werden. Dafür kann die Funktion \texttt{startDebugging()} des Debugging Target Service Consumer verwendet werden. Dabei wird das Programm an den Debugging Target Service Producer übergeben, dieses kann wie bereits für den Programming Service in \autoref{section:konzeption:kompilierung-und-programmierung} beschrieben eine Datei oder ein Ordner sein. Das Programm kann in Event Handlern von \texttt{StartDebugging}-Events z.B. zur Programmierung der Steuereinheit verwendet werden. Zusätzlich können ggf. noch weitere Vorbereitungen getroffen werden bevor eine Antwort an den Debugging Target Service Consumer gesendet wird. Nachdem die Sitzung erfolgreich gestartet wurde, wird eine entsprechende Antwort an den Debugging Adapter Service Consumer gesendet. Diese enthält den Kennzeicher der Debug-Sitzung sowie weitere Konfigurationsoptionen, die beim Start des \ac{DAP} übergeben werden sollen. Bei diesen Konfigurationsoptionen handelt es sich um Informationen die nur dem Debugging Adapter Service Producer bekannt sind, wie z.B. der Pfad zu dem kompilierten Programm auf dessen Dateisystem. Sobald der Debugging Adapter Service Consumer die Antwort erhalten hat kann das \ac{DAP} mit den Konfigurationsoptionen und dem Kennzeichner der Debug-Sitzung gestartet werden. Der Austausch der \ac{DAP} Nachrichten erfolgt dabei über die Funktionen \texttt{sendMessageDAP()}. Eingehende \ac{DAP} Nachrichten können über Event Handler für \texttt{MessageDAP}-Events erhalten und an den Debug Adapter übergeben werden. Eine ähnliche Vorgehensweise wird für den Austausch von Debug-Nachrichten zwischen dem Debugging Target Service Producer und dem Debugging Target Service Consumer angewendet. Hierbei werden die Funktion \texttt{sendDebuggingMessage()} und \texttt{DebuggingMessage}-Events verwendet. Die in eingehenden und augehenden DAP Nachrichten enthaltenen URIs müssen für den Debug Adapter bzw. die IDE angepasst werden, da diese auf unterschiedlichen Dateisystemen arbeiten.

\begin{figure}[tbp]
    \centering
    \resizebox{\textwidth}{!}{\begin{sequencediagram}
            \newthread{ide}{IDE}
            \newthreadShift{debugger}{Debugger}{4cm}
            \newthreadShift{steuereinheit}{Steuereinheit}{3cm}

            \begin{call}{ide}{DAP Terminate Anfrage}{debugger}{}
                \begin{call}{debugger}{lösche Sitzungsdaten}{debugger}{}
                \end{call}
                \begin{call}{debugger}{stoppe Debugger}{debugger}{}
                \end{call}
                \begin{call}{debugger}{beende Debuggen}{steuereinheit}{}
                    \begin{call}{steuereinheit}{lade Programm neu}{steuereinheit}{}
                    \end{call}
                \end{call}
            \end{call}
        \end{sequencediagram}}
    \caption{Ende einer Debug-Sitzung}
    \label{figure:ende-einer-debug-sitzung}
\end{figure}

\paragraph{Ende einer Debug-Sitzung} \autoref{figure:ende-einer-debug-sitzung} zeigt ein Sequenzdiagramm für das Ende einer Debug-Sitzung. Dieses wird durch eine entsprechende Nachricht des \ac{DAP} wie z.B. \texttt{Terminate} ausgelöst. Sobald der Debug Adapter diese Nachricht erhält werden die Sitzungsdaten gelöscht und der Debugger gestoppt. Weiterhin wird die Funktion \texttt{endDebugging()} des Debugging Target Service Consumer verwendet um den Debugging Target Service Producer über das Ende des Debug-Sitzung zu informieren. In einem Event Handler des dadurch ausgelösten \texttt{EndDebugging}-Events könnte z.B. das aktuelle Programm neugeladen werden und weitere vorgenommene Änderungen für das Debugging rückgängig gemacht werden. Danach wird eine entsprechende Antwort an den Debugging Target Service Consumer gesendet. Sobald diese erhalten wurde wird noch eine finale Antwort auf die ursprüngliche \ac{DAP} Anfrage gesendet.

\paragraph{Kollaboration} Nach \autoref{requirement:Teilen von Debug-Sitzungen} soll es Nutzern innerhalb eines Experiments ermöglicht werden laufenden Debug-Sitzungen beizutreten. Dafür soll laut Unteranforderung (e) der bereits vorhandene Collaboration Service der IDE verwendet werden. Über diesen könnte z.B. beim Start einer Debug-Sitzung der Kennzeichner der Sitzung und der verwendete Ordner in den Zustandsinformationen des Nutzers hinterlegt werden. Dadurch können andere Nutzer über die gestartete Sitzung informiert werden. Falls sie Zugriff auf den angegebenen Ordner haben können sie der Debug-Sitzung mithilfe der Funktion \texttt{joinSession()} des Debugging Adapter Service Consumer beitreten. Dabei wird der Kennzeichner der beizutretenden Debug-Sitzung angegeben. Die Antwort enthält einen neuen Kennzeichner für die Debug-Sitzung des beitretenden Nutzers sowie Konfigurationsoptionen für die Ausführung des \ac{DAP}. Um eine kollaborative Debug-Sitzung zu ermöglichen ist zudem eine spezielle Behandlung von einigen Nachrichten des \ac{DAP} nötig. Zum Beispiel darf nur eine \texttt{Initialize} Anfrage an einen Debug Adapter gestellt werden. Dementsprechend muss die Antwort auf diese gespeichert werden um sie später bei einer erneuten \texttt{Initialize} Anfrage an den beitretenden Nutzer senden zu können. Dabei wird die Anfrage beitretender Nutzer nicht an den Debug Adapter übergeben. Weiterhin sind z.B. die Nachrichten für Breakpoints, Stacktraces, das Starten und Stoppen des Programms sowie das Beenden der Debug-Sitzung zu betrachten. Die detaillierte Betrachtung aller dieser Nachrichten ist nicht das Ziel dieser Arbeit. In \autoref{section:prototypische-implementierung:debugging} wird die Behandlung der Nachrichten erläutert, welche für die prototypische Implementierung relevant sind.

Die betrachtete Experimentkonfiguration wird zunächst um ein Laborgerät erweitert. Dieses stellt einen Debugger über einen entsprechenden Debugging Adapter Service Producer bereit. Weiterhin besitzt dieses noch einen Debugging Target Service Consumer zur Kommunikation mit der zu debuggenden Steuereinheit. Die IDE wird um einen Debugging Adapter Service Consumer erweitert. Die Steuereinheit wird um einen Debugging Target Service Producer erweitert. Das neue Laborgerät wird über die entsprechenden CrossLab-Services mit der IDE und der Steuereinheit verbunden.