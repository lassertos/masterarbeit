\section{Testen}\label{section:konzeption:testen}

\begin{note}
    \textbf{Notizen:}
    \begin{itemize}
        \item Erwähnung von \autoref{requirement:Testen} und \autoref{requirement:Testen: CrossLab-Kompatibilität}
        \item Beschreibung der CrossLab-Services + Klassendiagramm
        \item Beschreibung der Einbindung in die betrachtete Experimentkonfiguration
    \end{itemize}
\end{note}

Nach \autoref{requirement:Testen} soll die Konfiguration und Ausführung von Testfällen innerhalb der IDE ermöglicht werden. Dies soll laut \autoref{requirement:Testen: CrossLab-Kompatibilität} über die Bereitstellung entsprechender CrossLab-Services erfolgen. Deshalb werden im Folgenden der \textit{Testing Service Producer} und der \textit{Testing Service Consumer} vorgestellt.

% Um die Erstellung und Ausführung von Testfällen, welche die Interaktionen zwischen mehreren Laborgeräten innerhalb eines Experiments überprüfen sollen, zu unterstützen müssen Laborgeräte in der Lage sein Funktionen anzubieten, die während der Ausführung der Testfälle von anderen Laborgeräten aufgerufen werden können. Dafür können der \textit{Testing Service Producer} und der \textit{Testing Service Consumer} verwendet werden.

\autoref{figure:klassendiagramm-testing-services} zeigt ein Klassendiagramm für die Testing Services. Der Testing Service Producer ermöglicht es Laborgeräten Funktionen für die Erstellung von Testfällen bereitzustellen. Diese können über die Funktion \texttt{registerFunction()} registriert werden. Dabei werden mindestens der Name der Funktion und deren Implementierung benötigt. Zusätzlich könnte man die Angabe von Schemata für die Argumente und den Rückgabewert der Funktion verlangen. Diese ermöglichen die Validierung der Eingaben und Ausgaben der Funktion. Der Testing Service Consumer erlaubt das Hinzufügen von Testfällen über die Funktion \texttt{addTest()}. Tests bestehen dabei aus einem Namen, einer Liste an Funktionen und ggf. eine Liste von weiteren Tests. Funktionen werden durch ihren Namen, den Kennzeichner des bereitstellenden Testing Service Producer und ihre Argumente beschrieben. Zusätzlich kann ein erwarteter Rückgabewert angegeben werden. Dieser wird während dem Testen mit dem tatsächlichen Rückgabewert verglichen. Sollten die Werte dabei unterschiedlich sein schlägt der Testfall fehl. Die weiteren Testfälle werden nach den Funktionen in der angegebenen Reihenfolge ausgeführt. Der Testing Service Consumer kann das Testen über die Funktion \texttt{startTesting()} beginnen. Dadurch wird ein entsprechendes \texttt{StartTesting}-Event bei den Test Service Producern ausgelöst. Über entsprechende Event Handler können Vorbereitungen für die Ausführung der Testfälle getroffen werden, bevor eine Antwort an den Testing Service Consumer gesendet wird. Sobald alle Testing Service Producer eine Antwort gesendet haben kann der Testing Service Consumer mithilfe der Funktion \texttt{runTest()} Testfälle ausführen. Dafür werden für jeden Testfall zunächst dessen Funktionen der Reihe nach aufgerufen und danach die weiteren enthaltenen Testfälle in der angegebenen Reihenfolge ausgeführt. Nachdem alle Testfälle ausgeführt wurden kann die Operation \texttt{endTesting()} des Testing Service Consumer genutzu werden um das Testen zu beenden. Hierbei wird wieder ein entsprechendes \texttt{EndTesting}-Event von den Testing Service Producern ausgelöst, welches über Event Handler genutzt werden können um den Normalzustand wiederherzustellen.

% Im Folgenden wird die Einbindung der Testing Services in die betrachtete Experimentkonfiguration beschrieben. Alle Laborgeräte, die Funktionen für Testfälle bereitstellen sollen, müssen einen Testing Service Producer anbieten. Die IDE nutzt einen Testing Service Consumer um Testfälle ausführen zu können. Dieser wird mit allen für die Testfälle benötigten Laborgeräten verbunden. Die Konfiguration der Testfälle erfolgt während der Konfiguration des Experiments auf der Ebene der Rollen. Dementsprechend würde für die Rolle der IDE für die Eigenschaft \texttt{tests} ein Array von Testfällen angegeben werden.

\begin{figure}[tbp]
    \centering
    \begin{tikzpicture}
        \begin{class}[text width=6cm]{TestingServiceProducer}{0,0}
            \operation{+ registerFunction()}
            % \operation{+ executeFunction()}
            \operation{+ onStartTesting()}
            \operation{+ onEndTesting()}
            \operation{+ onFunctionCall()}
        \end{class}
        \begin{class}[text width=6cm]{TestingServiceConsumer}{7,0}
            \operation{+ addTest()}
            \operation{+ runTest()}
            \operation{+ startTesting()}
            \operation{+ endTesting()}
        \end{class}
    \end{tikzpicture}
    \caption{Klassendiagramm Testing Services}
    \label{figure:klassendiagramm-testing-services}
\end{figure}