\section{CrossLab-Kompatibilität}\label{section:konzeption:crosslab-kompatibilität}

% \begin{note}
%     \textbf{Notizen:}
%     \begin{itemize}
%         \item Erwähnung \autoref{requirement:Erweiterbarkeit}
%         \item Vergleich der möglichen Anbindungen der IDE als Laborgerät
%         \item Grundlegender Kommunikationsmechanismus der entwickelten Services
%     \end{itemize}
% \end{note}

Für die verschiedenen Funktionen der zu entwickelnden IDE sollen entsprechende CrossLab-Services entwickelt und von der IDE verwendet werden. \autoref{requirement:Erweiterbarkeit} verlangt zudem die Erweiterbarkeit der IDE um zusätzliche CrossLab-Services. Um dies zu erreichen, gibt es verschiedene Möglichkeiten. Angenommen die IDE unterstützt das Hinzufügen von Erweiterungen. So könnte eine zentrale Komponente genutzt werden, um alle vorhandenen CrossLab-Services zu verwalten. Diese kann von der IDE selbst oder von einer entsprechenden Erweiterung bereitgestellt werden. Diese zentrale Komponente könnte entweder selbst in der Lage sein CrossLab-Services, die von anderen Erweiterungen bereitgestellt werden, zum Laborgerät hinzuzufügen oder sie könnte eine entsprechende Schnittstelle bereitstellen, die es anderen Erweiterungen ermöglicht das Laborgerät um zusätzliche Services zu erweitern. In der ersten Variante könnte die zentrale Komponente einschränken, welche CrossLab-Services zu dem Laborgerät hinzugefügt werden. So könnte z.B. für eine IDE mit verschiedenen Erweiterungen in der Experimentkonfiguration eine Liste von Erweiterungen festgelegt werden, deren CrossLab-Services geladen werden sollen. Somit müssen ggf. nicht alle Erweiterungen für alle Experimente geladen werden. Der beschriebene Ablauf ist in \autoref{abbildung:initialisierung-laborgerät-ide} dargestellt.

\begin{figure}[tbp]
    \centering
    \begin{sequencediagram}
        \newthread{ide}{IDE}
        \newinst[4]{erweiterung}{Erweiterung}

        \begin{call}{ide}{lese Experimentkonfiguration}{ide}{}
        \end{call}

        \begin{sdblock}{alt}{[Erweiterung inaktiv]}
            \begin{call}{ide}{starte Erweiterung}{erweiterung}{}
            \end{call}
        \end{sdblock}

        \begin{call}{ide}{lade CrossLab-Services}{erweiterung}{}
        \end{call}
    \end{sequencediagram}
    \caption{Initialisierung Laborgerät IDE}\label{abbildung:initialisierung-laborgerät-ide}
\end{figure}

Weiterhin besteht die Frage, welche Art eines Laborgeräts für die Einbindung der IDE in die CrossLab-Architektur am besten geeignet ist. Dabei ist zu beachten, dass die IDE sowohl von mehreren Nutzern gleichzeitig als auch eigenständig in Experimenten verwendet werden soll (sh. \autoref{requirement:Eigenständig nutzbar} und \autoref{requirement:Kollaboration}). Aufgrund der Tatsache, dass konkrete Laborgeräte nur in einem Experiment gleichzeitig verwendet werden können, kommt nur die Einbindung als cloud- oder edge-instanziierbares Gerät in Frage\todo{Feedback einholen}. Die Instanzen von cloud-instanziierbaren Laborgeräten werden auf Servern ausgeführt und benötigen dementsprechende Ressourcen. Aufgrund dieser Tatsache kann es ggf. dazu kommen, dass Nutzer warten müssen, bis die entsprechenden Serverkapazitäten vorhanden sind. Dies könnte die Benutzererfahrung verschlechtern. Eine Einbindung der IDE als edge-instanziierbares Laborgerät kann dieses Problem umgehen, da die Instanzen auf der Seite des Nutzers ausgeführt werden. Allerdings muss dabei beachtet werden, dass für eine Implementierung der IDE als edge-instanziierbares Gerät die grundlegenden Funktionen dieser komplett im Browser des Nutzers ausgeführt werden müssen. Zu den grundlegenden Funktionen gehören dabei ein Dateisystem für die Bearbeitung und persistente Speicherung von Dateien und Ordnern sowie der Code Editor zum Editieren von Dateien. Zusätzliche Funktionen, wie z.B. Kompilierung und Debuggen, müssen in den meisten Fällen auf externen Servern ausgeführt werden und benötigen somit entsprechende Ressourcen. Eine Implementierung der IDE als cloud-instanziierbares Gerät kann allerdings die Anbindung und Nutzung von Compilern, Debuggern und Language Servern stark vereinfachen, da diese auf demselben System laufen können. Die in den folgenden Abschnitten beschriebenen Konzepte sollen in beiden Varianten angewandt werden können.
