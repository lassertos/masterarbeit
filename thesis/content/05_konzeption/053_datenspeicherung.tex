\section{Datenspeicherung} \label{konzeption:datenspeicherung}

Das bisherige WIDE System nutzt ein projektbasiertes Dateisystem. In diesem muss der Name eines Projektes einzigartig sein. Weiterhin können nicht mehrere Projekte gleichzeitig geöffnet werden. Zudem werden Metadaten zu einem Projekt gespeichert. Diese umfassen das elektromechanische Modell, die Steuereinheit sowie die Programmiersprache, die bei der Erstellung des Projekts genutzt bzw. ausgewählt wurden. Dadurch ist es möglich dem Nutzer nur die Projekte anzuzeigen, die in dem aktuellen Experiment von Interesse sein könnten. In der neuen CrossLab Architektur ist es nun allerdings möglich mehrere Steuereinheiten und elektromechanische Modelle sowie weitere Laborgeräte zu einem Experiment zusammenzustellen. Deshalb sollte es dem Nutzer ermöglicht werden mehrere Projekte gleichzeitig öffnen und bearbeiten zu können. Ein weiteres Feature von WIDE ist die Bereitstellung von Beispielprojekten. Diese können von Nutzern verwendet werden um einen Einblick in die Programmierung einer gegebenen Steuereinheit zu bekommen. Um dieses Feature weiterhin unterstützen zu können sollte eine Konfigurationsmöglichkeit gegeben werden, welche die Bereitstellung derartiger Beispiele ermöglicht.