\section{Kompilierung und Programmierung}\label{section:konzeption:kompilierung-und-programmierung}

\todo{Services für Programmierung beschreiben}

\begin{note}
    \textbf{Notizen:}
    \begin{itemize}
        \item Erwähnung von \autoref{requirement:Kompilierung} und \autoref{requirement:Kompilierung: CrossLab-Kompatibilität}
        \item Beschreibung der CrossLab-Services + Klassendiagramm
        \item Konzept für die Bereitstellung von Compilern als Laborgeräte
        \item Beschreibung der Einbindung in die betrachtete Experimentkonfiguration
        \item (Beschreibung möglicher Einstellungen?)
    \end{itemize}
\end{note}

Nach \autoref{requirement:Kompilierung} soll die Kompilierung der Programme von Nutzern innerhalb der IDE ermöglicht werden. Dafür soll laut \autoref{requirement:Kompilierung: CrossLab-Kompatibilität} die Bereitstellung und Nutzung von Compilern über entsprechende CrossLab-Services ermöglicht werden. Dementsprechend werden im Folgenden der \textit{Compilation Service Producer} und der \textit{Compilation Service Consumer} vorgestellt. In \autoref{figure:klassendiagramm-compilation-services} ist ein Klassendiagramm für die beiden CrossLab-Services angegeben. Dabei besitzt der Compilation Service Consumer nur eine Funktion \texttt{compile()}, die es Nutzern ermöglicht die Kompilierung eines Ordners anzufragen. Dabei können zusätzliche Optionen für den Compiler sowie das gewünschte Format der Ausgabe angegeben werden. Die möglichen Ausgabeformate können auf der Seite des Compilation Service Producer über entsprechende Schemas definiert und über die Funktion \texttt{addResultFormat()} hinzugefügt werden. Die registrierten Ausgabeformate können dann in der Servicebeschreibung des Compilation Service Producer hinterlegt werden. Während der Erstellung eines Experiments kann das gewünschte Ausgabeformat für die Verbindung eines Compilation Service Consumer und eines Compilation Service Producer in der Verbindungskonfiguration angegeben werden. Die Behandlung eingehender Kompilieranfragen kann durch Event Handler für \texttt{CompilationRequest}-Events des Compilation Service Producer erfolgen. Die Behandlung von Kompilieranfragen ist abhängig vom verwendeten Compiler. Bei einer erfolgreichen Kompilierung wird das Ergebnis in dem festgelegten Ausgabeformat an den Compilation Service Consumer gesendet zusammen mit den Meldungen des Compilers. Falls die Kompilierung fehlschlägt werden nur die Fehlermeldungen des Compilers versendet.

\begin{figure}[tbp]
    \centering
    \begin{tikzpicture}
        \begin{class}[text width=6cm]{CompilationServiceProducer}{0,0}
            \operation{+ addResultFormat()}
            \operation{+ onCompilationRequest()}
        \end{class}
        \begin{class}[text width=6cm]{CompilationServiceConsumer}{7,0}
            \operation{+ compile()}
        \end{class}
    \end{tikzpicture}
    \caption{Klassendiagramm Compilation Services}
    \label{figure:klassendiagramm-compilation-services}
\end{figure}