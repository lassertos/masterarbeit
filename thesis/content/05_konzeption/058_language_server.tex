\section{Language Server}\label{section:konzeption:language-server}

\begin{note}
    \textbf{Notizen:}
    \begin{itemize}
        \item Erwähnung von \autoref{requirement:Language Server} und \autoref{requirement:Language Server: CrossLab-Kompatibilität}
        \item Beschreibung der CrossLab-Services + Klassendiagramm
        \item Konzept für die Bereitstellung von Debuggern als Laborgeräte
        \item Beschreibung der Einbindung in die betrachtete Experimentkonfiguration
        \item (Beschreibung möglicher Einstellungen?)
    \end{itemize}
\end{note}

\begin{figure}[tbp]
    \centering
    \resizebox{\textwidth}{!}{
        \begin{tikzpicture}
            \begin{class}[text width=7.5cm]{LanguageServerServiceProducer}{0,0}
                \operation{+ send()}
                \operation{+ onInitialization()}
                \operation{+ onMessage()}
            \end{class}
            \begin{class}[text width=7.5cm]{LanguageServerServiceConsumer}{8,0}
                \operation{+ send()}
                \operation{+ initialize()}
                \operation{+ readFile()}
                \operation{+ writeFile()}
                \operation{+ createDirectory()}
                \operation{+ delete()}
                \operation{+ onMessage()}
            \end{class}
        \end{tikzpicture}
    }
    \caption{Klassendiagramm Language Server Services}
    \label{figure:klassendiagramm-language-server-services}
\end{figure}

Das \textit{\ac{LSP}} \cite{noauthor_language-server-protocol_nodate} ist ein von Microsoft entwickeltes Protokoll zur Kommunikation zwischen einem \textit{Language Client} und einem \textit{Language Server}. Language Server sind meistens für eine spezielle Programmiersprache entwickelt und stellen Dienste wie z.B. Autovervollständigung für diese an. Language Clients sind meist als Teil eines Code Editors implementiert und sind unabhängig von speziellen Programmiersprachen. Das Protokoll ist für die lokale Kommunikation zwischen einem Language Client und einem Language Server entworfen, d.h. es wird angenommen, dass beide auf demselben System ausgeführt werden. Dies hat zur Folge, dass für den verteilten Anwendungsfall entsprechende Vorkehrungen getroffen werden müssen um die Funktionalität zu gewährleisten. So muss es dem Language Server ermöglicht werden auf die Dateien des Remote-Systems zugreifen zu können und umgekehrt. Es gibt Language Server, die den Quellcode lokal verarbeiten bzw. sogar kompilieren und basierend auf diesen Ergebnissen Antworten an den Language Client zurücksenden. Dazu werden Dateien auf dem Server verwendet, auf welche der Language Client gegebenenfalls Zugriff benötigt.