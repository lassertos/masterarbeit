\section{Language Server}\label{section:konzeption:language-server}

\begin{note}
    \textbf{Notizen:}
    \begin{itemize}
        \item Erwähnung von \autoref{requirement:Language Server} und \autoref{requirement:Language Server: CrossLab-Kompatibilität}
        \item Beschreibung der CrossLab-Services + Klassendiagramm
        \item Konzept für die Bereitstellung von Debuggern als Laborgeräte
        \item Beschreibung der Einbindung in die betrachtete Experimentkonfiguration
        \item (Beschreibung möglicher Einstellungen?)
    \end{itemize}
\end{note}

\begin{figure}[tbp]
    \centering
    \resizebox{\textwidth}{!}{
        \begin{tikzpicture}
            \begin{class}[text width=7.5cm]{LanguageServerServiceProducer}{0,0}
                \operation{+ sendMessageLSP()}
                \operation{+ onInitialize()}
                \operation{+ onMessageLSP()}
                \operation{+ onFilesystemEvent()}
            \end{class}
            \begin{class}[text width=7.5cm]{LanguageServerServiceConsumer}{8,0}
                \operation{+ initialize()}
                \operation{+ sendMessageLSP()}
                \operation{+ sendFilesystemEvent()}
                \operation{+ onMessageLSP()}
            \end{class}
        \end{tikzpicture}
    }
    \caption{Klassendiagramm Language Server Services}
    \label{figure:klassendiagramm-language-server-services}
\end{figure}

Nach \autoref{requirement:Language Server} soll die Anbindung von Language Servern an die IDE ermöglicht werden. Dafür soll laut \autoref{requirement:Language Server: CrossLab-Kompatibilität} die Bereitstellung und Nutzung von Language Servern über entsprechende CrossLab-Services ermöglicht werden. Bevor die konzipierten CrossLab-Services vorgestellt werden, wird zunächst ein kurzer Überblick über das \textit{\ac{LSP}} \cite{noauthor_language-server-protocol_nodate} sowie die Herausforderungen für die Anbinung an die IDE gegeben.

% sollte wahrscheinlich eher in Grundlagen erklärt werden da es bereits in den Anforderungen erwähnt wird
Das \acl{LSP} ist ein von Microsoft entwickeltes Protokoll zur Kommunikation zwischen einem \textit{Language Client} und einem \textit{Language Server}. Language Server ermöglichen Editorfunktionen, wie z.B. Code-Vervollständigung, Code-Navigation und Refactoring für ausgewählte Programmiersprachen. Language Clients sind meist als Teil eines Code Editors implementiert und sind nicht auf spezifische Programmiersprachen beschränkt. Das Protokoll ist für die lokale Kommunikation zwischen einem Language Client und einem Language Server entworfen, d.h. es wird angenommen, dass beide auf demselben System ausgeführt werden. Dies hat zur Folge, dass für den verteilten Anwendungsfall entsprechende Vorkehrungen getroffen werden müssen um die Funktionalität zu gewährleisten. So muss es dem Language Server ermöglicht werden auf die Dateien des Remote-Systems zugreifen zu können und umgekehrt. Es gibt Language Server, die den Quellcode lokal verarbeiten bzw. sogar kompilieren und basierend auf diesen Ergebnissen Antworten an den Language Client zurücksenden. Dazu werden Dateien auf dem Server verwendet, auf welche der Language Client ggf. Zugriff benötigt. Unter Betrachtung dieser Herausforderungen werden im Folgenden der \textit{Language Server Service Producer} und der \textit{Language Server Service Consumer} vorgestellt.

\autoref{figure:klassendiagramm-language-server-services} zeigt ein Klassendiagramm für die Language Server Services. Die Funktion \texttt{initialize()} des Language Server Service Consumer kann für die Initialisierung eines Language Servers verwendet werden. Dabei wird der aktuelle Ordner des Nutzers sowie dessen URL und ggf. Konfigurationsoptionen für die Initialisierung des Language Server übergeben. Dadurch wird ein enstprechendes \texttt{Initialize}-Event von dem verbundenen Language Server Service Producer ausgelöst. Über entsprechende Event Handler kann dieses abgefangen und für die Initialisierung des Language Server verwendet werden. Dabei kann u.a. der übergebene Ordner in dem lokalen Dateisystem hinterlegt werden und der Language Server mit den angegebenen Konfigurationsoptionen gestartet werden. Die übergebene URL des Ordners kann zur Umschreibung von URLs innerhalb der Nachrichten des \ac{LSP} verwendet werden. Diese unterscheiden sich ggf. zwischen den Systemen des Language Server Service Producer und des Language Server Service Consumer. Nachdem der Language Server initialisiert wurde wird eine Antwort an den Language Server Service Consumer gesendet. Diese enthält einen Booleschen Wert, welcher angibt ob Dateisystem-Events für den verwendeten Ordner und dessen Inhalt benötigt werden, sowie ggf. weitere Konfigurationsoptionen für den Start des \ac{LSP}. Sollten Dateisystem-Events benötigt werden müssen diese entsprechend auf der Seite des Language Server Service Consumer für Dateien und Ordner innerhalb des verwendeten Ordners erstellt und an den Language Server Service Producer mithilfe der Funktion \texttt{sendFilesystemEvent()} gesendet werden. Diese umfassen die Erstellung und Löschung von Dateien und Ordnern sowie Änderungen von Dateien. Die Behandlung dieser Dateisystem-Events kann über entsprechende Event Handler erfolgen. Sobald die Initialisierung erfolgt ist kann das \ac{LSP} mit den Konfigurationsoptionen gestartet werden. Der Austausch der \ac{LSP} Nachrichten erfolgt dabei über die Funktion \texttt{sendMessageLSP()}. Zur Behandlung dieser Nachrichten können Event Handler für die entsprechenden \texttt{MessageLSP}-Events registriert werden.