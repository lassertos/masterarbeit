\section{Language Server} \label{konzeption-language_server}

% \begin{itemize}
%     \item Language Server über Cloud-instanziierbares Gerät oder über clangd im Browser
% \end{itemize}

% Eine Möglichkeit zur Einbindung von Language Servern ist es, diese auf dem CM zu installieren und sich dann zu diesen zu verbinden. Allerdings ist das Language Server Protokoll nur auf einen einzigen Client pro Server ausgelegt. Daher müsste in einem Experiment mit mehreren Nutzern für jeden dieser Nutzer eine eigene Instanz des Language Server erstellt werden. Eine spezielle Alternative könnte die Verwendung des zu WebAssembly kompilerten Language Servers clangd darstellen. Dieser könnte im Browser des jeweiligen Nutzers instanziiert werden, wodurch keine zusätzliche Last auf dem CM entsteht. Allerdings muss hierbei die neu entstandene Last auf dem Rechner des Nutzers in Betracht gezogen werden. Eine weitere Möglichkeit ist die Bereitstellung weiterer Server zur Bereitstellung der Language Server.

Das \textit{\ac{LSP}} \todoaddref[]{Language Server Protocol} ist ein von Microsoft entwickeltes Protokoll zur Kommunikation zwischen einem \textit{Language Client} und einem \textit{Language Server}. Language Server sind meistens für eine spezielle Programmiersprache entwickelt und stellen Dienste wie z.B. Autovervollständigung für diese an. Language Clients sind meist als Teil eines Code Editors implementiert und sind unabhängig von speziellen Programmiersprachen. Das Protokoll ist für die lokale Kommunikation zwischen einem Language Client und einem Language Server entworfen, d.h. es wird angenommen, dass beide auf demselben System ausgeführt werden. Dies hat zur Folge, dass für den verteilten Anwendungsfall entsprechende Vorkehrungen getroffen werden müssen um die Funktionalität zu gewährleisten. So muss es dem Language Server ermöglicht werden auf die Dateien des Remote-Systems zugreifen zu können und umgekehrt. Es gibt Language Server, die den Quellcode lokal verarbeiten bzw. sogar kompilieren und basierend auf diesen Ergebnissen Antworten an den Language Client zurücksenden. Dazu werden Dateien auf dem Server verwendet, auf welche der Language Client gegebenenfalls Zugriff benötigt.