\section{Debugging}\label{section:konzeption:debugging}

Um ein Programm debuggen zu können muss zunächst ein entsprechender Debugger an die IDE angebunden werden. Dabei ist zu beachten, dass die meisten Debugger nicht direkt im Browser ausgeführt werden können, weshalb sich cloud-instanziierbare Laborgeräte für die Anbindung eines Debuggers eignen. Dabei könnte für jeden Nutzer bzw. jedes Experiment eine entsprechende Instanz des Debuggers erstellt werden. Der Debugger muss dabei sowohl eine Verbindung zur IDE besitzen als auch eine weitere Verbindung zu der zu debuggenden Steuereinheit. Für beide dieser Verbindungen wird ein entsprechender Service benötigt. Für die Verbindung zwischen der IDE und dem Debugger wird der \textit{Debugging Adapter Service} genutzt und für die Verbindung zwischen dem Debugger und der Steuereinheit wird der \textit{Debugging Target Service} verwendet.

Der Debugging Adapter Service baut auf dem \ac{DAP} \cite{noauthor_debug-adapter-protocol_nodate} von Microsoft auf. Dieses spezifiert Nachrichten die zwischen einer IDE und einem sogenannten \textit{Debug Adapter} ausgetauscht werden. Ein Debug Adapter bildet die Schnittstelle zwischen einem Debugger und einer IDE. Das Protokoll erlaubt somit die Anbindung bestehender Debugger über die Implementierung eines entsprechenden Debug Adapters. Weiterhin wird der Implementierungsaufwand für die Einbindung von Debuggern in IDEs verringert, da in der Theorie nur eine Schnittstelle implementiert werden muss, anstatt jeden Debugger an eine eigens entwickelte Schnittstelle anpassen zu müssen. Somit sollte durch die Verwendung des \ac{DAP} die Einbindung der meisten Debugger stark vereinfacht werden. \dots

Der Debugging Target Service übernimmt die Kommunikation zwischen dem Debugger und einer Steuereinheit. Dabei wird beim Start des Debuggens zunächst das aktuelle Programm auf die Steuereinheit geladen. Beim Erhalt dieser ersten Nachricht können neben der Programmierung der Steuereinheit auch weitere Vorbereitungen vorgenommen werden. Ein Beispiel hierfür wäre ein Experiment bei dem die Steuereinheit im normalen Betrieb mit einem elektromechanischen Hardwaremodell verbunden ist, aber während dem Debuggen ein Interface auf der Nutzerseite als Eingabe für die Sensoren verwendet. Somit könnten Nutzer schneller in den zu debuggenden Zustand gelangen. Nachdem alle Vorbereitungen getroffen wurden wird eine Antwort an den Debugger gesendet. Dieser beginnt daraufhin mit dem Austausch der Debugger-spezifischen Nachrichten. Diese sind nicht standardisiert und daher wird kein spezifisches Format für diese Nachrichten angegeben. Die Debugger sollten das Debugging von entfernten Geräten unterstützen. Wenn das Debuggen beendet wird sendet der Debugger eine entsprechende Nachricht an die Steuereinheit. Diese kann somit wieder in ihren Normalbetrieb wechseln und sendet dann eine Antwort an den Debugger.

Um die Echtzeit-Kollaboration während des Debuggens zu ermöglichen \dots