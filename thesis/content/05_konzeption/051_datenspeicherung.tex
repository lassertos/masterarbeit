\section{Datenspeicherung} \label{konzeption-datenspeicherung}

% Um sicherzustelllen, dass die IDE komplett im Browser verwendet werden kann, müssen die Nutzer in der Lage sein ihre Projekte in dieser verwalten zu können. Da noch nicht alle Browser die File System API komplett implementieren muss die Projektverwaltung auf einem anderen Weg implementiert werden. Dazu kann die IndexedDB API genutzt werden. Diese ermöglicht die persistente Speicherung größerer Datenmengen direkt im Browser. Um die IndexedDB zur Projektverwaltung nutzen zu können muss eine entsprechende Extension für Visual Studio Code geschrieben werden. Dafür kann die FileSystemProvider API verwendet werden. Diese ermöglicht es eigene Dateisysteme in Visual Studio Code einzubinden. Weiterhin gibt es noch die FileSearchProvider API und die TextSearchProvider API. Die FileSearchProvider API ermöglicht es die Logik für die Suche nach Dateien in einem bestimmten Dateisystem zu implementieren. Die TextSearchProvider API ermöglicht es die Logik für die Durchsuchung des Textinhalts von Dateien zu implementieren. Sowohl die FileSearchProvider API als auch die TextSearchProvider API befinden sich aktuell noch in einem experimentellen Zustand. Das bedeutet, dass sie verwendet werden können, allerdings dürfen Extensions, die diese APIs nutzen nicht auf dem Marketplace veröffentlicht werden. Dies sollte allerdings für die Verwendung innerhalb des GOLDi Remotelab kein Problem darstellen.

% Weiterhin besteht jedoch auch die Möglichkeit die Projekte der Nutzer auf einem Server zu speichern. Dies würde es den Nutzern ermöglichen zwischen ihren Endgeräten wechseln zu können, ohne ihren aktuellen Stand zu verlieren. Allerdings müssen dann entsprechende Server bereitgestellt werden. Ein anderer Ansatz könnte es den Nutzern erlauben ihre Projekt aus GitHub oder GitLab zu laden.