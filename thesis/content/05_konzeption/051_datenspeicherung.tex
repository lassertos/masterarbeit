\section{Datenspeicherung} \label{konzeption-datenspeicherung}

% Um sicherzustelllen, dass die IDE komplett im Browser verwendet werden kann, müssen die Nutzer in der Lage sein ihre Projekte in dieser verwalten zu können. Da noch nicht alle Browser die File System API komplett implementieren muss die Projektverwaltung auf einem anderen Weg implementiert werden. Dazu kann die IndexedDB API genutzt werden. Diese ermöglicht die persistente Speicherung größerer Datenmengen direkt im Browser. Um die IndexedDB zur Projektverwaltung nutzen zu können muss eine entsprechende Extension für Visual Studio Code geschrieben werden. Dafür kann die FileSystemProvider API verwendet werden. Diese ermöglicht es eigene Dateisysteme in Visual Studio Code einzubinden. Weiterhin gibt es noch die FileSearchProvider API und die TextSearchProvider API. Die FileSearchProvider API ermöglicht es die Logik für die Suche nach Dateien in einem bestimmten Dateisystem zu implementieren. Die TextSearchProvider API ermöglicht es die Logik für die Durchsuchung des Textinhalts von Dateien zu implementieren. Sowohl die FileSearchProvider API als auch die TextSearchProvider API befinden sich aktuell noch in einem experimentellen Zustand. Das bedeutet, dass sie verwendet werden können, allerdings dürfen Extensions, die diese APIs nutzen nicht auf dem Marketplace veröffentlicht werden. Dies sollte allerdings für die Verwendung innerhalb des GOLDi Remotelab kein Problem darstellen.

% Weiterhin besteht jedoch auch die Möglichkeit die Projekte der Nutzer auf einem Server zu speichern. Dies würde es den Nutzern ermöglichen zwischen ihren Endgeräten wechseln zu können, ohne ihren aktuellen Stand zu verlieren. Allerdings müssen dann entsprechende Server bereitgestellt werden. Ein anderer Ansatz könnte es den Nutzern erlauben ihre Projekt aus GitHub oder GitLab zu laden.

Das bisherige WIDE System nutzt ein Projekt-basiertes Dateisystem. Dabei können Nutzer Projekte für verschiedene Kombinationen von elektromechanischen Modellen, Steuereinheiten und Programmiersprachen erstellen. Dabei muss für jeden dieser Parameter exakt ein Wert angegeben werden. Dies ist möglich, da Experimente im GOLDi Remotelab jeweils aus exakt einem elektromechanischen Modell und einer Steuereinheit bestehen. Diese Vorgehensweise bietet den Vorteil, dass beim Starten eines Experiments nur die Projekte des Nutzers angezeigt werden, welche für die ausgewählte Kombination aus Modell und Steuereinheit erstellt wurden. In der neuen CrossLab Architektur ist es jedoch möglich mehrere Modelle und Steuereinheiten sowie ggf. weitere Geräte zu einem Experiment zusammenzustellen. Dadurch ist es schwieriger, eine Vorsortierung für den Nutzer vorzunehmen. Diese könnte auf der Basis der verwendeten Steuereinheiten getroffen werden. Allerdings muss hierbei beachtet werden, dass ggf. unterschiedliche Geräte das gleiche Projekt nutzen könnten. Dies könnte über die Einteilung von Steuereinheiten in Gruppen von kompatiblen Geräten signalisiert werden. Dafür könnten entsprechende Metadaten an die Gruppen angefügt werden, die es dem Dateisystem Service ermöglichen eine mögliche Kompatibilität festzustellen. Allerdings sollte der Nutzer allgemein alle seine Projekte öffnen können, selbst falls diese nicht mit der aktuellen Steuereinheit kompatibel sind. Damit wird sichergestellt, dass keine Probleme durch eine fehlerhafte Filterung der Projekte des Nutzers entstehen. Statt einer Filterung der Projekte könnte eine Sortierung der Projekte nach ihrer Relevanz für das aktuelle Experiment vorgenommen werden. Dabei könnte auch betrachtet werden, ob das aktuelle Experiment mithilfe einer Vorlage erstellt wurde. Falls weitere Projekte für diese Vorlage erstellt wurden können diese dem Nutzer als relevanter markiert werden. 

Die Speicherung der Projekte des Nutzers kann einerseits direkt im Browser des Nutzers erfolgen, oder auf einem Server. Die Speicherung im Browser des Nutzers bietet den Vorteil, das keine weiteren serverseitigen Ressourcen benötigt werden. Bei einer serverseitigen Datenspeicherung ist es allerdings für den Nutzer möglich von allen seinen kompatiblen Endgeräten auf alle seine Projekte zuzugreifen.