\section{Testen}\label{section:konzeption:testen}

Um die Erstellung und Ausführung von Testfällen, welche die Interaktionen zwischen mehreren Laborgeräten innerhalb eines Experiments überprüfen sollen, zu unterstützen müssen Laborgeräte in der Lage sein Funktionen anzubieten, die während der Ausführung der Testfälle von anderen Laborgeräten aufgerufen werden können. Dafür können der \textit{Testing Service Producer} und der \textit{Testing Service Consumer} verwendet werden.

Der Testing Service Producer ermöglicht \dots

\begin{figure}[tbp]
    \centering
    \begin{tikzpicture}
        \begin{class}[text width=6cm]{TestingServiceProducer}{0,0}
            \operation{+ registerFunction()}
            \operation{+ executeFunction()}
            \operation{+ onStartTesting()}
            \operation{+ onEndTesting()}
            \operation{+ onFunctionCall()}
        \end{class}
        \begin{class}[text width=6cm]{TestingServiceConsumer}{-7,0}
            \operation{+ addTest()}
            \operation{+ runTest()}
            \operation{+ startTesting()}
            \operation{+ endTesting()}
        \end{class}
    \end{tikzpicture}
    \caption{Klassendiagramm Testing Services}
    \label{figure:klassendiagramm-testing-services}
\end{figure}