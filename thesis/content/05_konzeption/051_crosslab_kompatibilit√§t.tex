\section{CrossLab Kompatibilität}\label{konzeption:crosslab-kompatibilität}

Die Sicherstellung der CrossLab Kompatibilität für die verschiedenen Features der zu entwickelnden IDE ist eine der grundlegenden Anforderungen. Die CrossLab Architektur ermöglicht die Definition von sogenannten \emph{Services}. Services können als \emph{Consumer}, \emph{Producer} oder \emph{Prosumer} implementiert werden. Innerhalb eines Experiments können dann Consumer und Producer miteinander verbunden werden. Prosumer implementieren sowohl einen Consumer als auch einen Producer und können dementsprechend mit allen Varianten eine Verbindung aufbauen. Dementsprechend sollten für alle in den nachfolgenden Abschnitten beschriebenen Features der IDE entsprechende Services definiert werden.

Weiterhin ist zu beachten, dass jedes Gerät seine Services mit Hilfe eines sogenannten \texttt{DeviceHandler} verwaltet. Daher ist es empfehlenswert eine zentrale Komponente zur Verwaltung dieses zu nutzen. Über diese Komponente muss das Hinzufügen von neuen Services ermöglicht werden können. Hierbei können zwei unterschiedliche Ansätze genutzt werden. Der erste Ansatz setzt voraus, dass die zentrale Komponente in der Lage ist weitere CrossLab kompatible Komponenten zu finden und ihre Services zu dem \texttt{DeviceHandler} hinzuzufügen. Der zweite Ansatz geht davon aus, dass die zentrale Komponente eine Schnittstelle bereitstellt, die von anderen CrossLab kompatiblen Komponenten gefunden und zum Hinzufügen ihrer Services genutzt werden kann.