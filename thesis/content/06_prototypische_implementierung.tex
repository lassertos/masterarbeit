\chapter{Prototypische Implementierung}\label{section:prototypische-implementierung}

% TODO: schauen, was in Kapitel 6 verwendet werden kann
% Im Fokus dieser Arbeit liegt die Programmierung von Mikrocontrollern im Rahmen des GOLDi Remotelab. Bei den verwendeten Microcontrollern handelt es sich um ATmega2560. Damit diese mit der CrossLab Infrastruktur kommunizieren können sind sie über einen FPGA mit einem Raspberry Pi Compute Module 4 verbunden. Dabei übernimmt der FPGA die Kommunikation zwischen dem CM und dem Microcontroller, während das CM die Kommunikationsschnittstelle zur CrossLab Infrastruktur übernimmt. Als Beispiel für ein steuerbares elektromechanisches Modell wird das 3-Achs-Portal verwendet. Neben den realen Systemen sollen auch die virtuellen Versionen in Betracht gezogen werden. Daraus folgen vier verschiedene Experiment-Konfigurationen, die für die Konzeption betrachtet werden:

% \begin{enumerate}
%     \item Realer Microcontroller und reales 3-Achs-Portal
%     \item Realer Microcontroller und virtuelles 3-Achs-Portal
%     \item Virtueller Microcontroller und reales 3-Achs-Portal
%     \item Virtueller Microcontroller und virtuelles 3-Achs-Portal
% \end{enumerate}

% Weiterhin ist zu beachten, dass auch mehrere Steuereinheiten und Modelle in einem Experiment enthalten sein können.

\section{Code Editor}\label{section:prototypische-implementierung:code-editor}

% \begin{note}
%     \textbf{Notizen:}
%     \begin{itemize}
%         \item \autoref{requirement:Erweiterbarkeit}, \autoref{requirement:Kostenlos nutzbar}, \autoref{requirement:Komplett im Browser nutzbar} und \autoref{requirement:Nur CrossLab-Nutzerkonto nötig}
%         \item Betrachtung verschiedener Optionen: \\ Eigenimplementierung, Visual Studio Code, Eclipse Theia und OpenSumi
%         \item Begründung warum Implementierung über VSCode Extensions
%         \item Begründung warum VSCode als Code Editor
%         \item Ggf. nicht als eigenes Unterkapitel
%     \end{itemize}
% \end{note}

Es gibt viele verschiedene Code Editoren, die für die prototypsische Implementierung der IDE verwendet werden können. So kann ein bereits vorhandener Code Editor wie z.B. Ace oder der Monaco Editor genutzt werden um diese in ein eigenes Benutzerinterface einzubinden. Dies ermöglicht, mit Ausnahme einer Eigenimplementierung des Code Editors, die größtmögliche Kontrolle über die Implementierung der IDE. Allerdings wird dadurch auch der Implementierungsaufwand stark erhöht. Eine weitere Option ist die Nutzung von Code Editoren, die bereits etablierte Benutzerinterfaces und Erweiterungsmöglichkeiten besitzen. Beispiele hierfür sind \ac{VSCode} \cite{noauthor_vscode_nodate}, Eclipse Theia \cite{noauthor_theia_nodate} und OpenSumi \cite{noauthor_opensumi_nodate}. Hierbei bieten Theia und OpenSumi neben der Erweiterbarkeit durch die VSCode Extension API \cite{noauthor_vscode-extension-api_nodate} auch noch weitere Schnittstellen an, die zur Erweiterung und Anpassung von grundlegenden Funktionen genutzt werden können. Durch die Nutzung bereits vorhandener und etablierter Benutzerinterfaces und Erweiterungsmöglichkeiten kann die Entwicklung der IDE stark vereinfacht werden. Dabei ist zu beachten, dass ggf. nicht alle erwünschten Änderungen über die angebotenen Schnittstellen umgesetzt werden können. Zudem entsteht durch die Nutzung von tiefgreifenden Schnittstellen von Theia und OpenSumi eine Bindung an das jeweilige Framework, wodurch ein späterer Wechsel auf eine andere Plattform erschwert wird. Aus diesen Gründen wird eine Implementierung über die VSCode Extension API angestrebt. Diese wird von allen drei genannten Möglichkeiten unterstützt, wodurch die entwickelte Lösung auf alle anwendbar sein sollte. Während der prototypischen Implementierung wurde \ac{VSCode} als Code Editor für die IDE verwendet.
\input{content/06_implementierung/062_crosslab_kompatibilität.tex}
\section{Kollaboration}\label{section:prototypische-implementierung:kollaboration}

Für die prototypische Implementierung der Kollaborationsmechanismen wurde Yjs \cite{noauthor_yjs_nodate} verwendet. Yjs basiert auf dem Konzept von \acp{CRDT}. Alle Teilnehmer einer Kollaborationssitzung sind in Yjs gleichberechtig. Das bedeutet, dass alle Nutzer sowohl als Producer als auch als Consumer auftreten. Dementsprechend wurde ein Prosumer implementiert, der beide Rollen abdeckt. Die entwickelte Erweiterung bietet eine entsprechende Instanz des Prosumers an und erlaubt auch den direkten Zugriff auf diese von anderen Erweiterungen aus. Dadurch können auch andere Erweiterung den Prosumer nutzen um ihre Daten innerhalb eines Experiments zu synchronisieren.
\section{Dateisystem}\label{section:prototypische-implementierung:dateisystem}

Während der prototypische Implementierung wurde eine Erweiterung zur Bereitstellung eines integrierten Dateisystems entwickelt. Dafür wurde ein entsprechender \texttt{FileSystemProvider} implementiert. Dieser kann genutzt werden um Dateisystem Operationen für ein spezielles URL-Schema anzubieten. Der implementierte \texttt{FileSystemProvider} nutzt weitere Subprovider zur Bereitstellung der tatsächlichen Dateisystem Operationen. Dabei wird bei der Erstellung des \texttt{FileSystemProvider} ein entsprechender Provider angegeben, der zunächst alle Pfade abdeckt. Zusätzlich können dann für gewisse Pfade andere Provider registriert werden. Die Provider sind dann für die Ausführung aller Dateisystem-Operationen verantwortlich die innerhalb dieser Pfade ausgeführt werden. Im Rahmen der prototypischen Implementierung wurde ein In-Memory Provider sowie ein Indexed Database Provider implementiert.

Bei dem integrierten Dateisystem handelt es sich um ein projektbasiertes Dateisystem. Im Detail bedeutet dies, dass Nutzer Projekte erstellen können, welche innerhalb des Dateisystems unter dem Pfad \texttt{/projects} gespeichert werden. Dieser Pfad nutzt den Indexed Database Provider um die persistente Speicherung der Projekte des Nutzers sicherzustellen. Alle weiteren Funktionen der IDE werden dann im Rahmen dieser Projekte ausgeführt. So werden z.B. für die Kompilierung alle Dateien des aktuell geöffneten Projekts betrachtet. Die URL des aktuell geöffneten Projekts kann über eine entsprechende Funktion von anderen Erweiterungen abgefragt werden. Weiterhin bietet die Erweiterung die Möglichkeit Funktionen zur Ausführung bei einem Projektwechsel zu registrieren. Außerdem erlaubt es die Erweiterung auch anderen Erweiterungen das aktuelle Projekt zu wechseln.

Bei dem Öffnen eines neuen Ordners wird VSCode standardmäßig neugeladen. Dadurch werden auch alle Erweiterungen beendet und neugestartet, was dazu führt, dass alle Verbindungen eines laufenden Experiments geschlossen werden. Dies hat zur Folge, dass das Experiment beendet wird. Um dies zu vermeiden muss der Wechsel von Projekten über einen anderen Weg geschehen. Dazu wurde zunächst ein Pfad festgelegt, welcher standardmäßig von der IDE geöffnet wird. Im Falle der prototypischen Implementierung wurde der Pfad \texttt{/workspace} ausgewählt. Dieser Pfad nutzt standardmäßig ein In-Memory Dateisystem. Sollte der Nutzer nun ein Projekt öffnen wird von diesem Moment an der Pfad \texttt{/workspace} in allen URLs durch den Pfad des geöffneten Projektes ersetzt. Dadurch wird kein Neuladen der IDE ausgelöst aber Nutzer können dennoch zwischen ihren Projekten wechseln. Das Umschreiben der Pfade führt allerdings zu Problemen beim Kopieren, Ausschneiden und Einfügen von Ordnern und Dateien. Daher müssen die entsprechenden Kommandos überschrieben werden um die erwartete Funktionalität zu gewährleisten. Um die Benutzererfahrung weiter zu verbessern wurde auch ein \texttt{FileSearchProvider} sowie ein \texttt{TextSearchProvider} implementiert, wodurch Nutzern nach Dateien und Text innerhalb ihres aktuellen Projekts suchen können.

Um das Teilen von Projekten sowie das gleichzeitige Bearbeiten dieser zwischen Nutzern innerhalb eines Experiments zu ermöglichen wurde eine entsprechende Komponente implementiert. Diese nutzt den \texttt{FileSystemProvider} sowie den von der Kollaborationserweiterung bereitgestellten Prosumer. Zu Beginn wird kein Projekt geteilt. Sobald ein Nutzer ein Projekt teilt wird es zu dem geteilten Objekt hinzugefügt. Weiterhin werden auch Funktionen registriert, die auf Änderungen innerhalb des Projekts reagieren. Andere Nutzer die an der Kollaboration teilnehmen können dann das geteilte Projekt über das bereitgestellte Benutzerinterface aufrufen. Alle Änderungen an Dateien und Ordnern werden in der geteilten Version des Projekts synchronisiert. Weiterhin wird die aktuelle Position eines Nutzers innerhalb einer Datei über dessen Zustandsinformationen geteilt. Diese Position wird dann bei anderen Nutzern innerhalb der selben Datei markiert. Wenn der Besitzer des Projekts das Teilen beendet wird das Projekt für alle anderen Nutzer geschlossen. Geteilte Projekte besitzen Pfade der Form \texttt{/shared/\{\{user\_id\}\}/\{\{project\_name\}\}}, wobei der Pfad \texttt{/shared} ein In-Memory Dateisystem verwendet.
\section{Kompilierung}\label{section:prototypische-implementierung:kompilierung}
\section{Debugging}\label{section:prototypische-implementierung:debugging}

Für die Bereitstellung der Debug-Funktionen wurde in der prototypischen Implementierung der Debugger gdb \todoaddref[]{gdb} verwendet. Dieser erlaubt das Debuggen von Microcontrollern und implementiert zudem Teile des \ac{DAP}. Um gdb in Experimenten nutzen zu können wurde ein cloud-instanziierbares Laborgerät entwickelt. Dieses stellt einen Debugging Adapter Service Producer und einen Debugging Target Service Consumer bereit für die Kommunikation mit der IDE sowie der zu debuggenden Steuereinheit.

Wenn ein Nutzer eine Debug-Sitzung startet wird über den Debugging Adapter Service eine entsprechende Nachricht an das Laborgerät des Debuggers gesendet. Diese Nachricht beinhaltet das aktuelle Projekt des Nutzers. Dieses benötigt der Debugger während der Debug-Sitzung, weshalb es für die Dauer der Debug-Sitzung in einem entsprechenden Ordner auf dem Dateisystem hinterlegt wird. Weiterhin wird das Projekt an den Compiler gesendet, wobei für die Ermöglichung des Debuggens spezielle Einstellungen vorgenommen werden müssen. Das Ergebnis der Kompilierung wird dann über den Debugging Target Service an die Steuereinheit gesendet, wodurch gleichzeitig der Steuereinheit der Beginn einer Debug-Sitzung mitgeteilt wird. Weiterhin wird der Debugger selbst gestartet, wobei der tatsächliche Start der Debug-Sitzung erst durch die Nachrichten des \ac{DAP} geschieht. Nachdem alle Vorbereitungen getroffen wurden und eine Antwort von der Steuereinheit empfangen wurde, wird eine Antwort an die IDE gesendet. Die Antwort enthält den Kennzeichner der Debug-Sitzung sowie Einstellungen für diese. Im Falle der prototypischen Implementierung werden hierbei der Kennzeichner der Debug-Sitzung, das Debug-Ziel sowie der Pfad des kompilierten Programms als Einstellungen übergeben. Diese werden dann von der IDE beim Start des \ac{DAP} verwendet.

Damit das \ac{DAP} korrekt ausgeführt werden kann muss eine Umschreibung der Pfade vorgenommen werden, da die Dateien des Nutzers und des Debuggers in unterschiedlichen Pfaden liegen. Außerdem gibt es Dateien, die nur auf dem Dateisystem des Debuggers vorhanden sind, wie z.B. Bibliotheken. Alle Pfade, die von der IDE gesendet werden beginnen entweder mit \texttt{crosslabfs:/workspace} für Dateien innerhalb eines Projekts oder mit \texttt{crosslab-remote:} für Dateien innerhalb des Dateisystems des Debuggers. Bei Ersteren wird das genannte Präfix durch den lokalen Pfad des Projektes auf dem Dateisystem des Debuggers ersetzt, während bei Zweiteren das Präfix gelöscht wird. Bei Nachrichten vom Debugger an die IDE geschieht die Behandlung der Präfixe auf umgekehrte Art.

Um das kollaborative Debuggen innerhalb eines Experiments zu unterstützen müssen einige Nachrichten des \ac{DAP} speziell behandelt werden. Dazu gehören unter anderem die Initialisierung und das Beenden der Debug-Sitzung sowie die Behandlung von Breakpoints. \dots \todo{Umschreiben der Pfade} \todo{Behandlung von Start-/Stopp-Antworten}

Für die Einbindung in die IDE wurde eine Erweiterung entwickelt. Diese beinhaltet Implementierungen der Schnittstellen \texttt{DebugAdapter}, \texttt{DebugAdapterDescriptorFactory} und \texttt{DebugConfigurationProvider}. Mithilfe dieser kann bereits ein Großteil der Funktionalität implementiert werden. Weiterhin werden auch Bedienelemente zum Starten bzw. Beitreten einer Debug-Sitzung bereitgestellt. Nutzer können eine Debug-Sitzung über eine entsprechende Schaltfläche starten, falls keine aktive Debug-Sitzung bestehen sollte. Ansonsten können Nutzer über eine weitere Schaltfläche einer bestehenden Debug-Sitzung beitreten, falls sie Zugriff auf das dazugehörige Projekt haben. Um das Debuggen von Dateien zu ermöglichen, die nur im Dateisystem des Debuggers vorhanden sind wurde ein \texttt{TextDocumentContentProvider} implementiert. Dieser ermöglicht Lesezugriff auf die Dateien innerhalb des Dateisystems des Debuggers sowie das Setzen von Breakpoints innerhalb dieser.
\section{Testen}\label{section:prototypische-implementierung:testen}
\section{Language Server}\label{section:prototypische-implementierung:language-server}
\section{Simulation}\label{section:prototypische-implementierung:simulation}