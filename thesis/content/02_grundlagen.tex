\chapter{Grundlagen}\label{section:grundlagen}

In diesem Kapitel werden einige Grundlagen für diese Arbeit beschrieben. Zunächst werden integrierte Entwicklungsumgebungen in \autoref{section:grundlagen:integrierte-entwicklungsumgebung} erläutert. Daraufhin werden das Verbundprojekt CrossLab und die dazugehörige Architektur für online Labore in \autoref{section:grundlagen:crosslab} vorgestellt. Anschließend werden das online Labor GOLDi in \autoref{section:grundlagen:goldi} und die dazugehörige online IDE WIDE in \autoref{section:grundlagen:wide} beschrieben.

\section{Integrierte Entwicklungsumgebung}\label{section:grundlagen:integrierte-entwicklungsumgebung}
Eine integrierte Entwicklungsumgebung bzw. \ac{IDE} ist meistens auf einen speziellen Anwendungsfall ausgelegt und besteht aus einem Code Editor sowie weiteren benötigten Softwarewerkzeugen, wie z.B. Compiler, Debugger und Language Server \cite{noauthor_language-server-protocol_nodate}. Dabei werden oftmals alle diese Komponenten direkt mit der \ac{IDE} ausgeliefert, wodurch der Nutzer direkt mit der Programmierung beginnen kann. Somit besitzt eine \ac{IDE} mehr Features als ein Code Editor, da diese nur die Bearbeitung von Code erlauben, während \acp{IDE} u.a. auch die Kompilierung und das Debuggen von Programmen ermöglichen. Language Server erweitern die Funktionen eines Code Editors, indem sie u.a. Code-Vervollständigung, Code-Navigation und Refactoring ermöglichen.

\section{CrossLab}\label{section:grundlagen:crosslab}
CrossLab \cite{aubel_adaptable_2022} ist ein Verbundprojekt der Technischen Universität Bergakademie Freiberg, der Technischen Universität Dortmund, der NORDAKADEMIE und der Technischen Universität Ilmenau. Im Rahmen des Projekts wurde eine neue Architektur für online Labore erarbeitet \cite{nau_new_2022}. Diese basiert auf dem Konzept von sogenannten \textit{Laborgeräten}. Diese können verschiedene \textit{Services} anbieten bzw. konsumieren. Beispiele für Services sind z.B. der Austausch von Sensor- und Aktorwerten oder die Übertragung der Webcamaufnahme eines elektromechanischen Hardwaremodells. Durch die Verbindung der Services von verschiedenen Laborgeräten kann ein \textit{Experiment} erstellt werden. Ein Vorteil dieser Architektur ist die einfache Wiederverwendbarkeit und Ersetzbarkeit von einzelnen Laborgeräten in Experimenten. So können z.B. Laborgeräte mit gleichen Services in einer entsprechenden \textit{Gerätegruppe} hinterlegt werden, welche dann statt eines konkreten Laborgeräts zur Erstellung eines Experiments genutzt werden kann. Beim Start des Experiments wird dann ein verfügbares Laborgerät aus der Gerätegruppe ausgewählt. Weiterhin gibt es noch \textit{cloud-instanziierbare} und \textit{edge-instanziierbare} Laborgeräte. Beim Start eines Experiments mit cloud- oder edge-instanziierbaren Laborgeräten wird eine entsprechende Instanz des Laborgeräts erstellt. Dabei wird für cloud-instanziierbare Laborgeräte eine Nachricht an die hinterlegte Instanziierungs-URL geschickt, wodurch die Instanz erstellt wird. Für edge-instanziierbare Laborgeräte wird mithilfe der hinterlegten Code-URL eine URL erstellt, die vom Nutzer aufgerufen werden kann um die Instanz zu erstellen.

\section{GOLDi}\label{section:grundlagen:goldi}
Das \acl{GOLDi} (\acs{GOLDi}) \cite{sitepoint_goldi_nodate} ist ein hybrides online Labor. Das bedeutet, dass es sowohl reale als auch virtuelle/simulierte elektromechanische Hardwaremodelle unterstützt. Vor der CrossLab Architektur bestanden Experimente des GOLDi-Remotelab immer aus einem elektromechanischen Hardwaremodell und einer Steuereinheit. Beispiele für elektromechanische Hardwaremodelle sind das 3-Achsen-Portal und ein Aufzugmodell, während Beispiele für Steuereinheiten u.a. Microcontroller und \acp{FPGA} sind. Die neue CrossLab-Architektur ermöglicht nun auch die Zusammenstellung neuer Experimente mit mehreren verschiedenen Laborgeräten. Außerdem bietet das GOLDi-Remotelab auch weitere Werkzeuge wie den Schaltungseditor \acs{BEAST} (\aclu{BEAST}) und die online IDE \acs{WIDE} (\aclu{WIDE}) an. Alle Experimente und Werkzeuge des GOLDi-Remotelab sind kostenlos nutzbar.

\section{WIDE}\label{section:grundlagen:wide}
WIDE ist die online IDE des GOLDi-Remotelab. Beim Start eines Experiments mit einer entsprechenden Steuereinheit wird WIDE automatisch mit geladen. Nutzer können WIDE verwenden um Programme für die Steuereinheit zu erstellen. Dabei sind Programme an die entsprechende Steuereinheit gebunden. D.h. wenn der Nutzer ein Programm für einen Microcontroller erstellt hat wird dieses nur in entsprechenden Experimenten angezeigt. Weiterhin wird auch die Erstellung neuer Projekte entsprechend eingeschränkt je nachdem welche Steuereinheit im Experiment genutzt wird. WIDE bietet auch eine Vielzahl an Beispielen für Nutzer, die auch an das entsprechende Experiment angepasst werden. Die Projekte der Nutzer werden im Browser gespeichert. Nutzer können weiterhin ihre Programme kompilieren lassen und auf die Steuereinheit hochladen. WIDE kann auch Standalone genutzt werden, wobei die zuvor erwähnten Einschränkungen entfallen. Die Steuereinheit, die verwendete Programmiersprache und das elektromechanische Hardwaremodell werden bereits bei der Erstellung eines Projekts festgelegt. Basierend darauf wird bei der Kompilierung der entsprechende Compiler ausgewählt.