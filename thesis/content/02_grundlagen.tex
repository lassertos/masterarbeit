\chapter{Grundlagen}\label{section:grundlagen}

In diesem Kapitel werden zunächst integrierte Entwicklungsumgebungen in \autoref{section:grundlagen:integrierte-entwicklungsumgebung} erläutert. Daraufhin werden das Verbundprojekt CrossLab und die dazugehörige Architektur für online Labore in \autoref{section:grundlagen:crosslab} vorgestellt.

\section{Integrierte Entwicklungsumgebung}\label{section:grundlagen:integrierte-entwicklungsumgebung}
Eine integrierte Entwicklungsumgebung bzw. \ac{IDE} ist meistens auf einen speziellen Anwendungsfall ausgelegt und besteht aus einem Code Editor sowie weiteren benötigten Softwarewerkzeugen, wie z.B. Compiler, Debugger und Language Server \cite{noauthor_language-server-protocol_nodate}. Oftmals werden alle diese Komponenten direkt mit der \ac{IDE} ausgeliefert, wodurch der Nutzer direkt mit der Programmierung beginnen kann. Somit besitzen \acp{IDE} mehr Features als Code Editoren, da diese nur die Bearbeitung von Code erlauben, während \acp{IDE} u.a. auch die Kompilierung und das Debuggen von Programmen ermöglichen. Language Server erweitern die Funktionen eines Code Editors, indem sie u.a. Code-Vervollständigung, Code-Navigation und Refactoring ermöglichen.

\section{CrossLab}\label{section:grundlagen:crosslab}
CrossLab \cite{aubel_adaptable_2022} ist ein Verbundprojekt der Technischen Universität Bergakademie Freiberg, der Technischen Universität Dortmund, der NORDAKADEMIE und der Technischen Universität Ilmenau. Im Rahmen des Projekts wurde eine neue Architektur für online Labore erarbeitet \cite{nau_new_2022}.

Die CrossLab-Architektur basiert auf dem Konzept von sogenannten \textit{Laborgeräten}. Diese können verschiedene \textit{Services} anbieten bzw. konsumieren. Beispiele für Services sind der \textit{Electrical Connection Service}, welcher den Austausch von Sensor- und Aktorwerten ermöglicht, und der \textit{Webcam Service}, welcher bspw. die Übertragung der Webcamaufnahmen von einem elektromechanischen Hardwaremodell ermöglicht. Services besitzen immer einen \textit{Producer}, der die Funktionen des Services bereitstellt, sowie einen \textit{Consumer}, der diese nutzen kann. Dabei können auch sogenannte \textit{Prosumer} entwickelt werden, die beide Rollen erfüllen können. Durch die Verbindung der Services von verschiedenen Laborgeräten kann ein \textit{Experiment} erstellt werden. Ein Vorteil dieser Architektur ist die einfache Wiederverwendbarkeit und Austauschbarkeit von einzelnen Laborgeräten in Experimenten. So können z.B. Laborgeräte mit gleichen Services in einer entsprechenden \textit{Laborgerätegruppe} hinterlegt werden, welche dann statt eines konkreten Laborgeräts zur Erstellung eines Experiments genutzt werden kann. Beim Start des Experiments wird dann ein verfügbares Laborgerät aus der Laborgerätegruppe ausgewählt. Weiterhin gibt es noch \textit{cloud-instanziierbare} und \textit{edge-instanziierbare} Laborgeräte. Beim Start eines Experiments mit cloud- oder edge-instanziierbaren Laborgeräten wird eine entsprechende Instanz des Laborgeräts erstellt. Dabei wird für cloud-instanziierbare Laborgeräte eine Nachricht an die hinterlegte Instanziierungs-URL geschickt, wodurch die Instanz erstellt wird. Für edge-instanziierbare Laborgeräte wird mithilfe der hinterlegten Code-URL eine URL erstellt, die vom Nutzer aufgerufen werden muss, um die Instanz zu erstellen.

Das Backend von CrossLab besteht aus mehreren verschiedenen Diensten, welche zusammen eine \textit{CrossLab-Instanz} bilden. Diese Dienste sind im Folgenden aufgelistet:
\begin{itemize}
    \item \textbf{Authentication Service} \\ Dieser Dienst ist für die Authentifizierung der Nutzer verantwortlich.
    \item \textbf{Authorization Service} \\ Dieser Dienst ist für die Autorisierung der Nutzer verantwortlich.
    \item \textbf{Device Service} \\ Dieser Dienst verwaltet die Laborgeräte der CrossLab-Instanz.
    \item \textbf{Experiment Service} \\ Dieser Dienst ist für die Erstellung und Verwaltung der Experimente der CrossLab-Instanz verantwortlich.
    \item \textbf{Federation Service} \\ Dieser Dienst ist für das Teilen von Laborgeräten und Experimenten mit anderen CrossLab-Instanzen verantwortlich.
\end{itemize}
Um ein Experiment in einer CrossLab-Instanz starten zu können benötigen Nutzer ein entsprechendes Nutzerkonto für diese CrossLab-Instanz.