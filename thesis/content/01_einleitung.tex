\chapter{Einleitung}\label{section:einleitung}

% \begin{note}
%     \textbf{Notizen:}
%     \begin{itemize}
%         \item Remote Labore (z.B. GOLDi)
%         \item IDEs in Remote Laboren (z.B. GOLDi und WIDE)
%         \item Änderungen und Möglichkeiten durch CrossLab
%         \item Weiterer Verlauf der Arbeit
%     \end{itemize}
% \end{note}

Praktische Versuche sind in der Lehre verschiedenster Fachrichtungen, wie z.B. in der Informatik, Elektrotechnik und Chemie essenziell. Dabei können Studierende die aus der Vorlesung bekannten theoretischen Grundlagen in der Praxis anwenden und somit ein tieferes Verständnis für diese erlangen. Ein Beispiel für einen praktischen Versuch ist die Steuerung eines elektromechanischen Hardwaremodells über die Programmierung eines Microcontrollers oder das Erstellen einer entsprechenden Schaltung. Diese praktischen Versuche werden meistens in Laboren durchgeführt, welche die benötigte Hardware bereitstellen. Dabei gibt es auch sogenannte \textit{online Labore}. Diese erlauben die Durchführung der Versuche über eine entsprechende Webanwendung. Dadurch kann es Nutzern ermöglicht werden, die Versuche unabhängig von den Zugangszeiten eines normalen Labors durchzuführen.

Ein Beispiel für ein derartiges online Labor ist das an der Technischen Universität Ilmenau \cite{noauthor_tu-ilmenau_2025} entwickelte \ac{GOLDi} \cite{sitepoint_goldi_nodate}, welches im Folgenden als GOLDi-Remotelab bezeichnet wird. Es erlaubt Nutzern verschiedene Versuche, bestehend aus einem elektromechanischen Hardwaremodell, z.B. dem Modell eines 3-Achsen-Portalkrans oder eines Aufzugs, sowie einer Steuereinheit, z.B. einem Microcontroller oder einem \ac{FPGA}, zusammenzustellen. Hierbei besitzt das GOLDi-Remotelab die Besonderheit, dass für jedes reale elektromechanische Hardwaremodell auch eine Simulation dessen bereitgestellt wird, die statt dem realen Modell innerhalb eines Versuchs verwendet werden kann. Somit können Nutzer auch Versuche für ein entsprechendes Modell durchführen, selbst wenn das reale Modell nicht verfügbar ist. Daher wird das GOLDi-Remotelab auch als ein \textit{hybrides online Labor} bezeichnet.

Eine weitere zentrale Komponente des GOLDi-Remotelab ist die integrierte Entwicklungsumgebung, im Englischen \ac{IDE}, \ac{WIDE} \cite{henke_hidden_2021}. Dabei handelt es sich um eine sogenannte \textit{online IDE}, da WIDE im Browser des Nutzers ausgeführt wird. WIDE ermöglicht es Nutzern, ihre Programme für die in einem Versuch verwendete Steuereinheit direkt in dem bereitgestellten Web-Interface des Versuchs zu erstellen. Nutzer können zudem ihre verschiedenen Programme verwalten, diese kompilieren und das Kompilat für die Programmierung der verwendeten Steuereinheit nutzen.

Ein aktuelles Projekt, welches sich mit online Laboren befasst, ist das Verbundprojekt \textit{CrossLab} \cite{aubel_adaptable_2022} der Technischen Universität Bergakademie Freiberg \cite{noauthor_tu-freiberg_nodate}, der Technischen Universität Dortmund \cite{dortmund_tu-dortmund_nodate}, der NORDAKADEMIE \cite{noauthor_nordakademie_nodate} und der Technischen Universität Ilmenau mit der folgenden Zielsetzung:

\begin{quote}
    ,,CrossLab zielt auf die Etablierung eines hochschulübergreifenden, interdisziplinären Netzwerkes von digitalisierten Labormodulen, die vergleichbar mit den Konzepten der Industrie 4.0, bedarfsbezogen in einer Lernumgebung für studierenden-zentrierte Lehre kombiniert werden können. Dafür werden durch die Partner TU Bergakademie Freiberg, TU Ilmenau, TU Dortmund und der NORDAKADEMIE sowohl auf didaktischer, technischer und organisatorischer Ebene Lösungen entwickelt und evaluiert.`` \cite{noauthor_crosslab_nodate}
\end{quote}

Im Rahmen dieses Verbundprojekts wurde eine neue Architektur für online Labore entwickelt \cite{nau_new_2022}. Diese basiert auf dem Konzept sogenannter \textit{Laborgeräte}. Diese können verschiedene \textit{Services} anbieten und durch die Verbindung dieser zu einem \textit{Experiment} zusammengestellt werden. Dadurch wird die Wiederverwendbarkeit und Austauschbarkeit von Laborgeräten in unterschiedlichen Experimenten ermöglicht. Weiterhin erlaubt diese Architektur die gemeinsame Nutzung von Laborgeräten über Institutionsgrenzen hinweg. Somit können z.B. Studierende der NORDAKADEMIE auf die in Ilmenau vorhandenen elektromechanischen Hardwaremodelle zugreifen.

Um diese neuen Möglichkeiten nutzen zu können, wurde eine Umstellung des GOLDi-Remotelab auf diese neue Architektur vorgenommen. Um diese Umstellung abzuschließen, muss u.a. eine neue CrossLab-kompatible online IDE entwickelt werden. Dafür soll ein entsprechendes Laborgerät samt Services entwickelt werden. Dabei soll auch die durch die neue Architektur unterstützte Möglichkeit betrachtet werden, einzelne Funktionen mithilfe weiterer Laborgeräte zu implementieren. Ein Beispiel hierfür wäre die Auslagerung eines Compilers auf ein entsprechendes Laborgerät, das einen Service für die Nutzung dessen anbietet. Dieser Service könnte dann wiederum von dem Laborgerät der IDE verwendet werden, um die Kompilierung für Nutzer bereitzustellen. Die entwickelten Laborgeräte und Services sollen in allen online Laboren nutzbar sein, welche die CrossLab-Architektur verwenden.

Zunächst werden in \autoref{section:grundlagen} grundlegende Begriffe für diese Arbeit vorgestellt, darunter integrierte Entwicklungsumgebungen und die CrossLab-Architektur. Danach wird in \autoref{section:stand-der-technik} der aktuelle Stand der Technik im Bezug auf online IDEs erarbeitet und vorgestellt. Anschließend wird in \autoref{section:anforderungsanalyse} die Anforderungsanalyse für die zu entwickelnde online IDE dargelegt. Daraufhin werden in \autoref{section:konzeption} Konzepte für die Bereitstellung der verschiedenen Funktionen der zu entwickelnden IDE vorgestellt. Darauf aufbauend wird in \autoref{section:prototypische-implementierung} die prototypische Implementierung der IDE beschrieben. In \autoref{section:diskussion} werden die Erfüllung der gestellten Anforderungen sowie offene Probleme betrachtet. Abschließend wird in \autoref{section:zusammenfassung-und-ausblick} ein Fazit gegeben.
