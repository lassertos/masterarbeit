\chapter{Einleitung}\label{section:einleitung}

% Das Grid of Online Laboratory Devices Ilmenau (GOLDi) ist ein hybrides Online Labor. Es ermöglicht Studierenden reale Hardwaremodelle mit verschiedenen Kontrolleinheiten zu steuern. Vor dem CrossLab Projekt waren nur Experiment mit einem Hardwaremodell und einer Kontrolleinheit möglich. Aufgrund der neuen CrossLab Architektur ist es allerdings nun möglich Experimente freier zu konfigurieren. Dabei werden sogenannte Laborgeräte definiert, die gewisse Services anbieten. Über diese Services können verschiedene Laborgeräte zu einem Experiment zusammengefügt werden. Nehmen wir zum Beispiel ein 3-Achs-Portal sowie einen Microcontroller. Beide diese Geräte können einen Service anbieten, der ihre Pins definiert. Über diese Services können dann die Pins der beiden Geräte verbunden werden.

% WIDE ist eine eigens für das GOLDi Remotelab entwickelte integrierte Entwicklungsumgebung (IDE). Sie ermöglicht es Nutzern direkt im Browser Programme für die Kontrolleinheiten zu schreiben. Weiterhin wird auch die Kompilierung des Quellcodes sowie das Hochladen des Kompilats auf die Kontrolleinheit unterstützt. Allerdings ist WIDE auf die alte Architektur des GOLDi Remotelab ausgelegt, in welcher ein Experiment nur aus jeweils einem Hardwaremodell und einer Kontrolleinheit bestand. In der neuen CrossLab Architektur kann jedoch ein Experiment mehrere Hardwaremodelle und Kontrolleinheiten enthalten.

% Um die neuen Anforderungen, die aus der CrossLab Architektur hervorgehen, zu addressieren ist eine Anpassung bzw. Neukonzeption von WIDE nötig. Dafür werden in Kapitel \ref{grundlagen} einige grundlegende Begriffe eingeführt. Danach wird in Kapitel \ref{systematische_literaturrecherche} eine Systematische Literaturrecherche durchgeführt. In Kapitel \ref{anforderungsanalyse} werden die Anforderungen an eine IDE im Kontext des GOLDi Remotelab sowie der CrossLab Architektur gesammelt. In Kapitel \ref{stand_der_technik} werden bestehende Lösungen auf ihre Anwendbarkeit überprüft und verglichen. In Kapitel \ref{konzeption} wird die Lösung konzipiert. In Kapitel \ref{implementierung} wird die Implementierung der konzipierten Lösung beschrieben. In Kapitel \ref{auswertung} wird der Erfüllungsgrad der Anforderungen ausgewertet. Schließlich wird in Kapitel \ref{zusammenfassung_und_fazit} eine Zusammenfassung der Arbeit sowie ein Fazit gegeben.