\section{Language Server}\label{section:prototypische-implementierung:language-server}

% \begin{note}
%     \textbf{Notizen:}
%     \begin{itemize}
%         \item Begründung warum Arduino Language Server genutzt wurde
%         \item Anbindung des Arduino Language Server als Laborgerät
%         \item Beschreibung der implementierten VSCode Erweiterung
%         \item Beschreibung der aufgetretenen Probleme und deren Lösung
%     \end{itemize}
% \end{note}

Für die prototypische Implementierung wurde ein cloud-instanziierbares Laborgerät für die Bereitstellung des Arduino Language Servers \cite{noauthor_arduino-language-server_2025} implementiert. Dieses besitzt einen entsprechenden Language Server Service Producer. Wenn die IDE den Language Server startet schickt sie zunächst die entsprechende Initialisierungsnachricht mit dem aktuellen Projekt des Nutzers. Dieses wird auf der Seite des Language Server gespeichert. Danach wird der Language Server gestartet und eine Antwort an die IDE gesendet. Diese kann daraufhin mit der Ausführung des \ac{LSP} beginnen. Dabei muss auf der Seite des Language Servers eine Anpassung der Pfade für eingehende und ausgehende \ac{LSP} Nachrichten erfolgen.

Für die IDE wurde eine entsprechende Erweiterung entwickelt. Diese stellt einen Language Server Service Consumer bereit. Wenn eine Verbindung für diesen besteht, wird der entsprechende Language Server gestartet und über den VSCode Language Client angebunden. Weiterhin wurde ein \texttt{TextDocumentContentProvider} implementiert, der in Verbindung mit dem Language Server Service Consumer für den Lesezugriff auf die lokalen Dateien des Language Servers ermöglicht.

Das betrachtete Experiment wird um das neue Laborgerät für die Bereitstellung des Arduino Language Servers erweitert. Dieses wird über den Language Server Service mit den IDEs verbunden.