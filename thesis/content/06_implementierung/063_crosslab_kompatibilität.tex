\section{CrossLab Kompatibilität}\label{section:prototypische-implementierung:crosslab-kompatibilität}

% \begin{note}
%     \textbf{Notizen:}
%     \begin{itemize}
%         \item Beschreibung der Basis-Erweiterung
%         \item Beschreibung wie CrossLab-Services hinzugefügt werden können
%     \end{itemize}
% \end{note}

Für die Implementierung der in \autoref{section:konzeption:crosslab-kompatibilität} konzipierten Lösung wurde eine entsprechende Erweiterung entwickelt. Diese ist für die Verwaltung der angebotenen CrossLab-Services sowie für die Anmeldung der IDE als Laborgerät innerhalb einer CrossLab-Instanz verantwortlich.

\paragraph{Verwaltung der CrossLab-Services}
Die Erweiterung hat Zugriff auf alle anderen Erweiterungen und kann deren angebotenen CrossLab-Services einsehen. Dadurch können diese zu dem Laborgerät der IDE hinzugefügt werden. Weiterhin ist die Erweiterung in der Lage andere Erweiterungen zu aktivieren. Somit ist es möglich die zu ladenden Erweiterungen über die Experimentkonfiguration zu definieren. Dafür kann eine entsprechende Liste von Kennzeichnern der entsprechenden Erweiterungen bei der Konfiguration der IDE innerhalb eines Experiments angegeben werden. Diese werden dann aktiviert und deren CrossLab-Services zum Laborgerät der IDE hinzugefügt.

\paragraph{Anmeldung als Laborgerät}
Für die Anmeldung der IDE innerhalb einer CrossLab-Instanz wird die URL der erstellten Laborgerät-Instanz sowie ein entsprechender Token benötigt. Diese werden im Falle von edge-instanziierbaren Laborgeräten über die Query-Parameter der vom Nutzer aufzurufenden URL bereitgestellt. Die IDE muss diese entsprechend an die Erweiterung weiterleiten. Da Erweiterungen über ihre URLs hinterlegt werden und auch Zugriff auf diese besitzen können die Query-Parameter zu diesen hinzugefügt werden. Die Erweiterung kann dann die URL des Laborgeräts und den Token auslesen und damit die Anmeldung durchführen.