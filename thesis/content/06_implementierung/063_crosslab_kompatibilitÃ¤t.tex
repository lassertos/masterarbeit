\section{CrossLab Kompatibilität}\label{section:prototypische-implementierung:crosslab-kompatibilität}

\begin{note}
    \textbf{Notizen:}
    \begin{itemize}
        \item Beschreibung der Basis-Erweiterung
        \item Beschreibung wie CrossLab-Services hinzugefügt werden können
    \end{itemize}
\end{note}

Für die Implementierung der in \autoref{section:konzeption:crosslab-kompatibilität} konzipierten Lösung wurde eine entsprechende Erweiterung entwickelt. Diese ist für die Verwaltung der angebotenen CrossLab-Services sowie für die Anmeldung der IDE als Laborgerät innerhalb einer CrossLab-Instanz verantwortlich.

\paragraph{Verwaltung der CrossLab-Services}
Die Erweiterung hat Zugriff auf alle anderen Erweiterungen und kann deren angebotenen CrossLab-Services einsehen. Dadurch können diese zu dem Laborgerät der IDE hinzugefügt werden. Weiterhin ist die Erweiterung in der Lage andere Erweiterungen zu aktivieren. Somit ist es möglich die zu ladenden Erweiterungen über die Experimentkonfiguration zu definieren. Dafür kann eine entsprechende Liste von Kennzeichnern der entsprechenden Erweiterungen bei der Konfiguration der IDE innerhalb eines Experiments angegeben werden. Diese werden dann aktiviert und deren CrossLab-Services hinzugefügt.

\paragraph{Anmeldung als Laborgerät}
Für die Anmeldung der IDE innerhalb einer CrossLab-Instanz wird die URL des erstellten Laborgeräts sowie ein entsprechender Token benötigt. Diese werden im Falle von edge-instanziierbaren Laborgeräten über die Query-Parameter der aufzurufenden URL bereitgestellt. Die IDE muss diese entsprechend an die Erweiterung weiterleiten. Da Erweiterungen über ihre URLs hinterlegt werden und auch Zugriff auf diese besitzen können die Query-Parameter zu diesen hinzugefügt werden. Die Erweiterung kann dann die URL des Laborgeräts und den Token auslesen und damit die Anmeldung durchführen.

% Zur Herstellung der CrossLab-Kompatibilität wurde eine entsprechende Erweiterung implementiert. Diese ist für die Verwaltung eines sogenannten \texttt{DeviceHandler} zuständig. Dieser wird über den SOA-Client der CrossLab-Implementierung bereitgestellt. Der \texttt{DeviceHandler} erlaubt das Hinzufügen von Services zu einem Laborgerät. Die Erweiterung ist in der Lage alle anderen Erweiterungen sowie deren exportierten Objekte einzusehen. Sollte eine Erweiterung entsprechende CrossLab-Services anbieten können diese über den \texttt{DeviceHandler} hinzugefügt werden. Dabei besteht auch die Möglichkeit die betrachteten Erweiterungen für ein spezifischen Experiment einzuschränken indem bei der Konfiguration des Laborgeräts der IDE die Eigenschaft \texttt{extensions} auf ein Array gesetzt wird, dass die Kennzeichner der zu betrachtenden Erweiterungen beinhaltet. Weiterhin wird für die Anmeldung der IDE als Laborgerät bei einer CrossLab-Instanz die URL des Laborgeräts sowie ein dazugehöriger Token benötigt. Da es sich bei der IDE um ein edge-instanziierbares Laborgerät handelt werden diese Informationen über Query-Parameter innerhalb der vom Nutzer aufzurufenden URL hinterlegt. Die benötigten Parameter werden dann ausgelesen und wieder als Query-Parameter an die URL der Erweiterung angehangen. Dadurch können die Parameter wieder innerhalb der Erweiterung ausgelesen und dort für die Anmeldung des Laborgeräts bei der CrossLab-Instanz verwendet werden.