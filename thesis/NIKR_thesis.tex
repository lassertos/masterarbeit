% === Dokumentklasse ==========================================================
\documentclass[%
    a4paper,							% A4
    12pt,								% Schriftgröße 12
    twoside,							% zweiseitig
    fleqn								% Formeln linksbündig ausrichten
]{book}


% === Grundlegende Pakete =====================================================
\usepackage{etoolbox}					% portiert viele nützliche Sachen aus e-TeX (z.B. booleans, ifs)
\usepackage[							% erweiterte Angabe von Farben
    pdftex,								% Farbtreiber auswählen
    dvipsnames							% vordefinierte Farben laden
]{xcolor}								% erweiterte Angabe von Farben
\usepackage{xparse}						% high-level Interface für Dokumentbefehle wie \NewDocument(Command,Environment)

\usepackage[utf8]{inputenc}				% Kodierung der *.tex Dateien ist UTF-8 nicht ISO-8859-1
\usepackage{cmap}						% character map tables -> pdf Inhalt wird besser durchsuch- und kopierbar
\usepackage[T1]{fontenc}				% OT1 encoding für deutsche Sonderzeichen -> pdf Inhalt kann Sonderzeichen durchsucht werden
\usepackage{lmodern}					% Latin-Modern-Schriftart (Computer Modern in Verbindung mit OT1 führt zu Bitmap-Fonts auf Windows)
\usepackage[nopatch=eqnum]{microtype}					% mikrotypografische Einstellungen, die den gesetzten Text nochmals wesentlich verbessern (weniger badboxes)


% === Benutzerdefinierte Einstellungen ========================================
% === Art der Arbeit ==========================================================
% von den nachfolgenden Blöcken bitte den richtigen auswählen und die anderen auskommentieren/löschen

% --- Diplom ----------
%\newcommand{\settingsDegree}{{Diplom}}
%\newcommand{\settingsDegreeName}{{Diplominformatiker}}
%\newcommand{\settingsDegreeName}{{Diplomingenieur}}

% --- Bachelor ----------
% \newcommand{\settingsDegree}{{Bachelor}}
% \newcommand{\settingsDegreeName}{{Bachelor of Science}}

% --- Master ----------
\newcommand{\settingsDegree}{{Master}}
\newcommand{\settingsDegreeName}{{Master of Science}}

% === Name, Abgabedatum und Sprache der Arbeit ================================
\newcommand{\settingsName}{{Pierre Helbing}}
\newcommand{\settingsFinishDate}{{DD.MM.YYYY}}
\newcommand{\settingsLanguage}{german}   		% german / american

% === Weitere Einstellungen ===================================================
% --- Suchpfad (Unterverzeichnis) für eingebundene Grafiken ----------
\newcommand{\settingsGraphicsPath}{image/}

% --- Hinweiskapitel ----------
\newbool{settingsWithHints}
\setbool{settingsWithHints}{false}				% true / false

% --- Zeilennummern ----------
\newbool{settingsWithLineNumbers}
\setbool{settingsWithLineNumbers}{false}			% true / false

% --- Todos ----------
\newbool{settingsWithTodos}
\setbool{settingsWithTodos}{true}				% true / false

% --- Anzahl an Nummerierungsebenen im Text und Inhaltsverzeichnis ----------
% 1: \section
% 2: \section + \subsection
% Achtung: 3 oder 4 nur nach Absprache mit Betreuer !
% 3: \section + \subsection + \subsubsection
% 4: \section + \subsection + \subsubsection + \paragraph
\setcounter{secnumdepth}{2}
\setcounter{tocdepth}{2}

% --- Anforderungen ----------

\usepackage{tabularx}
\usepackage{multicol}
\newenvironment{myreq}[1]{%
    \setlist[description]{font=\normalfont\color{darkgray}}%
    \begin{tcolorbox}[colframe=black,colback=white, sharp corners, boxrule=1pt]%
        \bfseries\color{blue}%
        \begin{description}#1}%
            {\end{description}\end{tcolorbox}}

\newcommand{\threeinline}[3]{\begin{multicols}{3}#1 #2 #3\end{multicols}}
\newcommand{\twoinline}[2]{\begin{multicols}{2}#1 #2\end{multicols}}

\newcommand{\reqno}{\item[Requirement \#:]}
\newcommand{\reqtype}{\item[Requirement Type:]}
\newcommand{\reqevent}{\item[Event/BUC/PUC \#:]}
\newcommand{\reqdesc}{\item[Description:]}
\newcommand{\reqrat}{\item[Rationale:]}
\newcommand{\reqorig}{\item[Originator:]}
\newcommand{\reqfit}{\item[Fit Criterion:]}
\newcommand{\reqsatis}{\item[Customer Satisfaction:]}
\newcommand{\reqdissat}{\item[Customer Dissatisfaction:]}
\newcommand{\reqdep}{\item[Dependencies:]}
\newcommand{\reqconf}{\item[Conflicts:]}
\newcommand{\reqmater}{\item[Materials:]}
\newcommand{\reqhist}{\item[History:]}
\usepackage{multicol}
\usepackage{multirow}
\usepackage[dvipsnames]{xcolor}
\usepackage{enumitem}
\usepackage{tcolorbox}

% === Übersetzungen ===========================================================
% Definitionen je nach \mylanguage (siehe NIKR_settings.tex)
\ifdefstring{\settingsLanguage}{german}{%
    \newcommand{\acroname}{Abkürzungsverzeichnis}	% Name für Abkürzungsverzeichnis
    \newcommand{\todoname}{Todo Liste}	% Name für Todo-Liste
    \newcommand{\pagename}{Seite}		% Name für Seite
}%
{%
    \newcommand{\acroname}{Acronyms}	% Name für Abkürzungsverzeichnis
    \newcommand{\todoname}{Todo list}	% Name für Todo-Liste
    \newcommand{\pagename}{page}		% Name für Seite
}%


% === Wichtige Pakete und Einstellungen =======================================
\usepackage{calc}						% ermöglicht Arithmetik in den Argumenten von Befehlen

% Einstellungen je nach \mylanguage (siehe NIKR_settings.tex)
\ifdefstring{\settingsLanguage}{german}{%
    \usepackage[ngerman]{babel}			% Spracheinstellungen (für deutsch z.B. Contents -> Inhaltsverzeichnis, etc.)
    \usepackage{bibgerm}            	% Stylefile für deutsche Literaturstellenangabe
    \bibliographystyle{gerapali}		% Stil für Literaturangaben festlegen
}%
{%
    \usepackage[american]{babel}		% Spracheinstellungen (für deutsch z.B. Contents -> Inhaltsverzeichnis, etc.)
    \bibliographystyle{apalike}			% Stil für Literaturangaben festlegen
    \frenchspacing						% einfaches Leerzeiches nach Satzende (für deutsch bereits Standard)
}%
\usepackage[%
    noadjust							% noadjust verhindert automatische Leerzeichen um die Referenz, was am Zeilenanfang zu Problemen führt
]{cite}     							% erlaubt Zeilenumbruch innerhalb von Zitierungen

\usepackage{graphicx}					% erweiterte Argumente in \in­clude­graph­ics
\graphicspath{{\settingsGraphicsPath}}	% Standard-Pfad für Bilder siehe NIKR_settings.tex

\usepackage[bf]{caption}				% erlaubt erweiterte Formatierungen in \caption (siehe unten)
\usepackage{subcaption}					% mehrere Abbildungen nebeneinander

\usepackage{amsmath}					% ermöglicht \DeclareMathOperator
\usepackage{amssymb}					% mathmatische Symbole und Sonderzeichen
\usepackage{nicefrac}					% für \nicefrac
\usepackage{nccmath}					% für \mfrac

\usepackage{icomma}						% intelligentes Komma (macht Verwendung von {,} überflüssig)
\usepackage{siunitx}					% für einheitliche Angabe von Einheiten

\usepackage{fancyhdr}					% Kopf- und Fußzeilen (siehe unten)

\usepackage{hhline}						% erweitere Rahmengestaltung in Tabellen

\usepackage[%							% Einbettung von Links im Dokument und erlaubt die Nutzugn von \url
    hyperfootnotes=false,				% keine Fußnoten als Link im Dokument (geht nicht mit footmisc)
    pagebackref=true					% Backrefs im Literaturverzeichnis
]{hyperref}           					% Einbettung von Links im Dokument und erlaubt die Nutzugn von \url
\renewcommand*{\backref}[1]{\textit{(\pagename:~#1)}}   % Format für backrefs

\usepackage[%							% erlaubt erweiterte Formatierungen von Fußnoten (siehe unten)
    multiple,							% mehrere mit Komma abtrennen
    hang								% linksbündig, \footnotemargin entscheidet über Einrückung
]{footmisc}								% erlaubt erweiterte Formatierungen von Fußnoten (siehe unten)
\patchcmd{\footref}{\ref}{\ref*}{}{}	% Hyperlink in \footref entfernen

\usepackage[nohyperlinks]{acronym}		% Abkürzungsverzeichnis

\usepackage{setspace}					% für \setstretch (ändern Zeilenabstand, aber nicht floating Umgebungen)

\usepackage[%							% Todos
    colorinlistoftodos,					% farbige Markierungen in Todo-Liste
    prependcaption,						% caption=val
    textsize=tiny,						% Schriftgröße
    linecolor=red,						% Standard-Linienfarbe für \todo
    backgroundcolor=red!25,				% Standard-Hintergrundfarbe für \todo
    bordercolor=red,					% Standard-Rahmenfarbe für \todo
    textwidth=2cm,						% Standard-Breite für \todo
]{todonotes}							% Todos
\ifbool{settingsWithTodos}{%
    \setlength{\marginparwidth}{2cm}	% sonst werden Notes am Rand nicht richtig angezeigt
    \NewDocumentCommand{\todoaddref}{O{} m}{%
        \todo[linecolor=blue,backgroundcolor=blue!25,bordercolor=blue,#1]{#2}%
    }
    \NewDocumentCommand{\todouncertain}{O{} m}{%
        \todo[linecolor=green,backgroundcolor=green!25,bordercolor=green,#1]{#2}%
    }
    \NewDocumentCommand{\todooptional}{O{} m}{%
        \todo[linecolor=cyan,backgroundcolor=cyan!25,bordercolor=cyan,#1]{#2}%
    }
    \pretocmd{\mainmatter}{\listoftodos[\todoname]{\markboth{\MakeUppercase{\todoname}}{\MakeUppercase{\todoname}}}}{}{}
}{
    \presetkeys{todonotes}{disable}{}	% disable \todo
    \NewDocumentCommand{\todoaddref}{O{} m}{}
    \NewDocumentCommand{\todouncertain}{O{} m}{}
    \NewDocumentCommand{\todooptional}{O{} m}{}
}

\usepackage[switch*,pagewise]{lineno}	% Zeilennummern
\ifbool{settingsWithLineNumbers}{%
    \renewcommand\linenumberfont{\textbf\sffamily\color{black!50}\footnotesize}
    \apptocmd{\mainmatter}{\linenumbers}{}{}
    \pretocmd{\backmatter}{\nolinenumbers}{}{}
}{}

\usepackage{placeins}					% FloatBarriers

\usepackage{enumitem}					% erweiterte Formatierung von \enumerate, \itemize und \description

\usepackage[linewidth=0.5pt]{mdframed}	% für Boxen in Hinweisen

\usepackage[ddmmyyyy]{datetime}			% Datumsangabe
\renewcommand{\dateseparator}{.}		% Punkt als Trennzeichen in Datumsangabe


% === Längen und Abstände =====================================================
% horizontales Layout
\setlength{\oddsidemargin}{0.2in}
\setlength{\evensidemargin}{0.0in}
\setlength{\textwidth}{\paperwidth - 2.2in}

% vertikales Layout
%\setlength{\topskip}{0.0cm}
\setlength{\headheight}{15.1pt}
%\setlength{\headsep}{0.0cm}
\setlength{\topmargin}{0.0cm}
\setlength{\footskip}{0.6in}
\setlength{\textheight}{\paperheight - 2.0in}
\addtolength{\textheight}{-1.0\headheight}
\addtolength{\textheight}{-1.0\headsep}
\addtolength{\textheight}{-1.0\footskip}

% Zeilenabstand
\setstretch{1.3}
\AtBeginEnvironment{tabular}{\setstretch{1.3}}

% Einrückung von Formeln
\setlength{\mathindent}{1.0cm}

% Absätze
\setlength{\parindent}{0.0cm}

% Fußnoten
\renewcommand{\footnotelayout}{\setstretch{1.2}}
\setlength\footnotemargin{10pt}

% Listen (noitemsep, nosep, ...)
\setlist{noitemsep}

\setlength{\topsep}{0.3cm}


% === Bild- und Tabellenunterschrift ==========================================
\renewcommand{\captionfont}{\small \setstretch{1.3}}
\newcommand{\NIcaption}[2]{\caption[#1]{#1\protect\\ \emph{#2}}}
\setcaptionmargin{0.75cm}


% === Abkürzungsverzeichnis ===================================================
% Verwendung vor jedem Kapitel zurücksetzen
\pretocmd{\chapter}{\acresetall}{}{}


% === Seitenstil ==============================================================
% Pagestyle plain überschreiben
\pagestyle{fancy}
% Kapitel- und Abschnittangaben ohne Punkt
\renewcommand{\sectionmark}[1]{\markright{\uppercase{\thesection~~#1}}}
\renewcommand{\chaptermark}[1]{\markboth{\uppercase{\chaptername\ \thechapter~~#1}}{}}
\fancypagestyle{plain}{%
    \fancyhead[ER]{\itshape\leftmark}%
    \fancyhead[OL]{\itshape\rightmark}%
    \fancyhead[EL,OR]{\thepage}%
    \fancyfoot[EL,OL]{}%
    \fancyfoot[EC,OC]{}%
    \renewcommand{\headrulewidth}{0.4pt}%
    \renewcommand{\footrulewidth}{0.4pt}%
}


% === Verweise ================================================================
% Klammern in Formel-Referenzen entfernen
\makeatletter
\renewcommand\tagform@[1]{\maketag@@@{\ignorespaces#1\unskip\@@italiccorr}}
\makeatother


% === Angabe von Einheiten ====================================================
\ifdefstring{\settingsLanguage}{german}{%
    \sisetup{locale=DE}		% deutsche Lokalisierung (konvertiert 1.00 automatisch zu 1,00)
}%
{%
    \sisetup{locale=US}		% englische Lokalisierung (konvertiert 1,00 automatisch zu 1.00)
}%


% === Mathematische Definitionen ==============================================
% Darstellung von Vektoren und Matrizen
\renewcommand{\vec}[1]{\underline{\mathbf{\MakeLowercase{#1}}}}		% Vektoren
\newcommand{\veci}[1]{\underline{\MakeLowercase{#1}}}				% Vektoren als Indizes
\newcommand{\mat}[1]{\underline{\mathbf{\MakeUppercase{#1}}}}		% Matrizen
\newcommand{\mati}[1]{\underline{\MakeUppercase{#1}}}				% Matrizen als Indizes
% zusätzliche mathematische Operatoren (damit sie nicht als Formelzeichen geschrieben werden)
\DeclareMathOperator{\step}{step}									% Stufenfunktion
\DeclareMathOperator{\sign}{sign}									% Vorzeichen


% === Manuelle Definition von Worttrennungen ==================================
\hyphenation{
    Convolutional
    Neural
    Net-work
    Net-works
    Ko-ef-fi-zi-ent
    Ko-ef-fi-zi-ent-en
    Drop-out
    pixel-genaue
    Patch
    Patch-größen
}


% === Pseudocode-Darstellung ==================================================
% Import nicht oben, weil \parindent zuvor gesetzt werden muss!
% siehe: https://ctan.org/pkg/algorithm2e?lang=de
\ifdefstring{\settingsLanguage}{german}{%
    \usepackage[%						% Pseudocode
        linesnumbered,					% mit Zeilennummern
        noend,							% Ende von Befehlen, wie etwa While, unterdrücken
        ruled,							% Layout
        german,							% deutsche Bezeichnung und deutsches Verzeichnis
        %onelanguage,					% Keywords übersetzen
        algochapter						% Nummerierung analog zu Abbildungen
    ]{algorithm2e}						% Pseudocode
}{%
    \usepackage[%						% Pseudocode
        linesnumbered,					% mit Zeilennummern
        noend,							% Ende von Befehlen, wie etwa While, unterdrücken
        ruled,							% Layout
        algochapter						% Nummerierung analog zu Abbildungen
    ]{algorithm2e}						% Pseudocode
}%

\newenvironment{NIalgorithm}{%
    % Algorithmus um 1.5em einrücken, damit Zeilennummern nicht im Rand sind
    \setlength{\algomargin}{1.5em}%
    % Padding oben und unten für Caption
    \setlength{\interspacetitleruled}{\smallskipamount}%
    % Padding oben und unten für Algorithmus
    \SetAlgoInsideSkip{smallskip}%
    % Kommentarstyle ändern
    \newcommand\NIcommentstyle[1]{\ttfamily\textcolor{black!60}{##1}}
    \SetCommentSty{NIcommentstyle}%
    % zweiten Teil in NIcaption unterdrücken (falls NIcaption genutzt wird)
    \renewcommand{\NIcaption}[2]{\caption[##1]{##1}}%
    \begin{algorithm}%
        % Zeilenabstand minimal vergrößern
        \setstretch{1.1}%
        % kleine Schrift
        \small%
        % Semikolon unterdrücken
        \DontPrintSemicolon%
        % korrekt ausgerichtete mehrzeilige Input- bzw. Outputdefinitionen mittels \Input und \Output
        \SetKwInOut{Input}{Input}%
        \SetKwInOut{Output}{Output}%
        % Abschnitt für In- und Outputs zurückrücken
        \pretocmd{\Input}{\Indentp{-1.5em}}{}{}%
        \apptocmd{\Output}{\Indentp{1.5em}}{}{}%
        }{%
    \end{algorithm}%
}

\begin{document}
% --- Titelseite, Danksagung, Einverständniserklärung --------------------------
\pagestyle{empty}
% Buchstaben für Seitennummerierung verwenden 
% (verhindert "destination with the same identifier (name{page.X})" Warnung)
\pagenumbering{alph}

% === Titelblatt ==============================================================
\begin{titlepage}
	\hspace{0.2cm}
	\begin{minipage}{3.5cm}
		\includegraphics[width=0.8\textwidth]{images/logo}
	\end{minipage}
	\hspace{0.2cm}
	\begin{minipage}{11cm}
		\vspace{0.7cm}
		\large
		{\bf Technische Universität Ilmenau}\newline
		Fakultät für Informatik und Automatisierung\newline
		Fachgebiet Neuroinformatik und Kognitive Robotik
	\end{minipage}
	\begin{center}
		\vspace{0.8cm}
		{\Large\bfseries Entwicklung einer CrossLab-kompatiblen integrierten Entwicklungsumgebung für das GOLDi-Remotelab\\}
		\vspace{0.8cm}
		\settingsDegree arbeit zur Erlangung des akademischen Grades \settingsDegreeName\\[0.5cm]
		{\Large \bfseries \settingsName\\[1.0cm]}
		\begin{table}[ht]
			\centering
			\begin{tabular}{ll}
				Betreuer: & Dr. Detlef Streitferdt                    \\[2mm]
				\multicolumn{2}{l}{Verantwortlicher Hochschullehrer:} \\
				          & Prof. Dr.-Ing. habil. Daniel Ziener       \\[2cm]
				\multicolumn{2}{p{13cm}}{Die \settingsDegree arbeit wurde am \settingsFinishDate \ bei der Fakultät für Informatik und Automatisierung der Technischen Universität Ilmenau eingereicht.}
			\end{tabular}
		\end{table}
		% Hinweis für Entwurfsversion ausgeben (Variablen aus NIKR_settings.tex prüfen)
		\ifboolexpr{bool{settingsWithTodos} or bool{settingsWithLineNumbers} or bool{settingsWithHints}}{%
		{\color{red} Entwurf: \today \\[2mm]
		\textbf{Dies ist nicht die finale Druckvorlage.}\\[2mm]
		{\small%
		\ifbool{settingsWithHints}{%
			Hinweiskapitel aktiviert (siehe \texttt{settingsWithHints} in \texttt{NIKR\_settings.tex})\\%
		}{}%					
		\ifbool{settingsWithLineNumbers}{%
			Zeilennummern aktiviert (siehe \texttt{settingsWithLineNumbers} in \texttt{NIKR\_settings.tex})\\%
		}{}%
		\ifbool{settingsWithTodos}{%
			Todo-Markierungen aktiviert (siehe \texttt{settingsWithTodos} in \texttt{NIKR\_settings.tex})\\%
		}{}%
		}%
		}%
		}{}
	\end{center}
\end{titlepage}

\cleardoublepage

% === Danksagung ==============================================================
% Sollten Sie diesen Abschnitt nicht nutzen wollen, kommentieren Sie alle Zeilen
% bis zur Einverstaendniserklaerung (inkl. \cleardoublepage) aus
\vspace*{5cm}

Danksagung

Dieser Abschnitt {\bf kann} genutzt werden, um denjenigen Personen Dank auszusprechen, die Sie bei der Erstellung der Arbeit unterstützt haben.

\cleardoublepage

% === Einverstaendniserklaerung ================================================
\vspace*{16cm}

\begin{tabular}{lp{12.5cm}}
	{Erklärung:} & {"`Hiermit versichere ich, dass ich diese wissenschaftliche Arbeit selbständig verfasst und nur die angegebenen

			
			
			
			
			
			
			
			
			
			
			Quellen und Hilfsmittel verwendet habe. Alle von mir aus anderen
			Veröffentlichungen übernommenen Passagen sind als solche gekennzeichnet."'}
\end{tabular}
\vspace*{1.5cm}

\begin{tabular}{l}
	Ilmenau, \settingsFinishDate \\
	\\
\end{tabular}
\hfill
\begin{tabular}{c}
	{\makebox[6.0cm]{\dotfill}} \\
	\settingsName               \\
\end{tabular}

\cleardoublepage


% --- Inhalts- und Abkürzungsverzeichnis ---------------------------------------
\frontmatter
\pagestyle{plain}

% Abstract
\section*{Abstract}
\todo{Abstract Deutsch}
% \vspace*{\fill}
\section*{Abstract}
\todo{Abstract Englisch}

% Inhaltsverzeichnis
\tableofcontents
% Abkürzungsverzeichnis
\chapter*{\acroname}
\markboth{\MakeUppercase{\acroname}}{\MakeUppercase{\acroname}}
\acrodefplural{LMS}[LMS]{Lernmanagementsysteme}

\begin{acronym}[XXXXXX] % durch XXXXXX kann der Einzug bestimmt wird
	\setlength{\itemsep}{-\parsep}
	\acro{GOLDi}{Grid of Online Laboratory Devices Ilmenau}
	\acro{BEAST}{Block Diagram Editing and Simulating Tool}
	\acro{WIDE}{Web Integrated Development Environment}
	\acro{FPGA}{Field Programmable Gate Array}
	\acro{IDE}{Integrated Development Environment}
	\acro{LSP}{Language Server Protocol}
	\acro{DAP}{Debug Adapter Protocol}
	\acro{VSCode}{Visual Studio Code}
	\acro{OT}{Operational Transformation}
	\acro{CRDT}{Conflict-free Replicated Data Type}
	\acro{LMS}{Lernmanagementsystem}
	\acro{LTI}{Learning Tools Interoperability}
\end{acronym}



% --- Inhalt -------------------------------------------------------------------
\mainmatter

% Hinweiskapitel
\ifbool{settingsWithHints}{\input{content/hinweise}}{}

% Kapitel
\chapter{Einleitung}\label{section:einleitung}

% \begin{note}
%     \textbf{Notizen:}
%     \begin{itemize}
%         \item Remote Labore (z.B. GOLDi)
%         \item IDEs in Remote Laboren (z.B. GOLDi und WIDE)
%         \item Änderungen und Möglichkeiten durch CrossLab
%         \item Weiterer Verlauf der Arbeit
%     \end{itemize}
% \end{note}

Praktische Versuche sind in der Lehre verschiedenster Fachrichtungen, wie z.B. in der Informatik, Elektrotechnik und Chemie essentiell. Dabei können Studierende die aus der Vorlesung bekannten theoretischen Grundlagen in der Praxis anwenden und somit ein tieferes Verständnis für diese erlangen. Ein Beispiel für einen praktischen Versuch ist die Steuerung eines elektromechanischen Hardwaremodells über die Programmierung eines Microcontroller oder das Erstellen einer entsprechenden Schaltung. Diese praktischen Versuche werden meistens in Laboren durchgeführt, welche die benötigte Hardware bereitstellen. Dabei gibt es auch sogenannte \textit{online Labore}. Diese erlauben die Durchführung der Versuche über eine entsprechende Webanwendung. Dadurch kann es Nutzern ermöglicht werden die Versuche unabhängig von den Zugangszeiten eines normalen Labors durchzuführen.

Ein Beispiel für ein derartiges online Labor ist das an der Technischen Universität Ilmenau \cite{noauthor_tu-ilmenau_2025} entwickelte \ac{GOLDi} \cite{sitepoint_goldi_nodate}, welches im Folgenden als GOLDi-Remotelab bezeichnet wird. Dieses erlaubt es Nutzern verschiedene Versuche bestehend aus einem elektromechanischen Hardwaremodell, z.B. ein Modell eines 3-Achsen-Portalkrans oder eines Aufzugs, sowie einer Steuereinheit, z.B. ein Microcontroller oder ein \ac{FPGA}, zusammenzustellen. Hierbei besitzt das GOLDi-Remotelab die Besonderheit, dass für jedes reale elektromechanischen Hardwaremodelle auch eine Simulation dessen bereitgestellt wird, die statt dem realen Modell innerhalb eines Versuchs verwendet werden kann. Somit können Nutzer auch Versuche für ein entsprechendes Modell durchführen, selbst wenn das reale Modell nicht verfügbar ist. Daher wird das GOLDi-Remotelab auch als ein \textit{hybrides online Labor} bezeichnet.

Eine weitere zentrale Komponente des GOLDi-Remotelab ist die integrierte Entwicklungsumgebung, im Englischen \ac{IDE}, \ac{WIDE} \cite{henke_hidden_2021}. Dabei handelt es sich um eine sogenannte \textit{online IDE}, da WIDE im Browser des Nutzers ausgeführt wird. WIDE ermöglicht es Nutzern ihre Programme für die in einem Versuch verwendete Steuereinheit direkt in dem bereitgestellten Web-Interface des Versuchs zu erstellen. Dabei können Nutzer zudem ihre verschiedenen Programme verwalten, diese kompilieren und das Kompilat für die Programmierung der verwendeten Steuereinheit nutzen.

Ein aktuelles Projekt welches sich mit online Laboren befasst ist das Verbundprojekt \textit{CrossLab} \cite{aubel_adaptable_2022} der Technischen Universität Bergakademie Freiberg \cite{noauthor_tu-freiberg_nodate}, der Technischen Universität Dortmund \cite{dortmund_tu-dortmund_nodate}, der NORDAKADEMIE \cite{noauthor_nordakademie_nodate} und der Technischen Universität Ilmenau, mit der folgenden Zielsetzung:

\begin{quote}
    ,,CrossLab zielt auf die Etablierung eines hochschulübergreifenden, interdisziplinären Netzwerkes von digitalisierten Labormodulen, die vergleichbar mit den Konzepten der Industrie 4.0, bedarfsbezogen in einer Lernumgebung für studierenden-zentrierte Lehre kombiniert werden können. Dafür werden durch die Partner TU Bergakademie Freiberg, TU Ilmenau, TU Dortmund und der NORDAKADEMIE sowohl auf didaktischer, technischer und organisatorischer Ebene Lösungen entwickelt und evaluiert.`` \cite{noauthor_crosslab_nodate}
\end{quote}

Im Rahmen dieses Verbundprojekts wurde eine neue Architektur für online Labore entwickelt \cite{nau_new_2022}. Diese basiert auf dem Konzept sogenannter \textit{Laborgeräte}. Diese können verschiedene \textit{Services} anbieten und durch die Verbindung dieser zu einem \textit{Experiment} zusammengestellt werden. Dadurch wird die Wiederverwendbarkeit und Austauschbarkeit von Laborgeräten in unterschiedlichen Experimenten ermöglicht. Weiterhin erlaubt diese Architektur die gemeinsame Nutzung von Laborgeräten über Institutionsgrenzen hinweg. Somit können z.B. Studierende der NORDAKADEMIE auf die in Ilmenau vorhandenen elektromechanischen Hardwaremodelle zugreifen.

Um diese neuen Möglichkeiten nutzen zu können wurde eine Umstellung des GOLDi-Remotelab auf diese neue Architektur vorgenommen. Um diese Umstellung fertigzustellen muss u.a. eine neue CrossLab-kompatible online IDE entwickelt werden. Dafür soll ein entsprechendes Laborgerät samt Services entwickelt werden. Dabei soll auch die, durch die neue Architektur unterstützte, Möglichkeit betrachtet werden einzelne Funktionen mithilfe weiterer Laborgeräte zu implementieren. Ein Beispiel Hierfür wäre die Auslagerung eines Compilers auf ein entsprechendes Laborgerät, dass einen Service für die Nutzung dessen anbietet. Dieser Service könnte dann wiederum von dem Laborgerät der IDE verwendet werden um die Kompilierung für Nutzer bereitzustellen. Die entwickelten Laborgeräte und Services sollen in allen online Laboren nutzbar sein, welche die CrossLab-Architektur verwenden.

Zunächst werden in \autoref{section:grundlagen} grundlegende Begriffe für diese Arbeit vorgestellt, darunter integrierte Entwicklungsumgebungen und die CrossLab-Architektur. Danach wird in \autoref{section:stand-der-technik} der aktuelle Stand der Technik im Bezug auf online IDEs erarbeitet und vorgestellt. Anschließend wird in \autoref{section:anforderungsanalyse} die Anforderungsanalyse für die zu entwickelnde online IDE dargelegt. Daraufhin werden in \autoref{section:konzeption} Konzepte für die Bereitstellung der verschiedenen Funktionen der zu entwickelnden IDE vorgestellt. Darauf aufbauend wird in \autoref{section:prototypische-implementierung} die prototypische Implementierung der IDE beschrieben. In \autoref{section:diskussion} werden die Erfüllung der gestellten Anforderungen sowie offene Probleme betrachtet. Abschließend wird in \autoref{section:zusammenfassung-und-ausblick} ein Fazit gegeben.

\chapter{Grundlagen}\label{section:grundlagen}

In diesem Kapitel werden zunächst integrierte Entwicklungsumgebungen in \autoref{section:grundlagen:integrierte-entwicklungsumgebung} erläutert. Daraufhin werden das Verbundprojekt CrossLab und die dazugehörige Architektur für online Labore in \autoref{section:grundlagen:crosslab} vorgestellt.

\section{Integrierte Entwicklungsumgebung}\label{section:grundlagen:integrierte-entwicklungsumgebung}
Eine integrierte Entwicklungsumgebung bzw. \ac{IDE} ist meistens auf einen speziellen Anwendungsfall ausgelegt und besteht aus einem Code Editor sowie weiteren benötigten Softwarewerkzeugen, wie z.B. Compiler, Debugger und Language Server \cite{noauthor_language-server-protocol_nodate}. Oftmals werden alle diese Komponenten direkt mit der \ac{IDE} ausgeliefert, wodurch der Nutzer direkt mit der Programmierung beginnen kann. Somit besitzen \acp{IDE} mehr Features als Code Editoren, da diese nur die Bearbeitung von Code erlauben, während \acp{IDE} u.a. auch die Kompilierung und das Debuggen von Programmen ermöglichen. Language Server erweitern die Funktionen eines Code Editors, indem sie u.a. Code-Vervollständigung, Code-Navigation und Refactoring ermöglichen.

\section{CrossLab}\label{section:grundlagen:crosslab}
CrossLab \cite{aubel_adaptable_2022} ist ein Verbundprojekt der Technischen Universität Bergakademie Freiberg, der Technischen Universität Dortmund, der NORDAKADEMIE und der Technischen Universität Ilmenau. Im Rahmen des Projekts wurde eine neue Architektur für online Labore erarbeitet \cite{nau_new_2022}.

Die CrossLab-Architektur basiert auf dem Konzept von sogenannten \textit{Laborgeräten}. Diese können verschiedene \textit{Services} anbieten bzw. konsumieren. Beispiele für Services sind der \textit{Electrical Connection Service}, welcher den Austausch von Sensor- und Aktorwerten ermöglicht, und der \textit{Webcam Service}, welcher bspw. die Übertragung der Webcamaufnahmen von einem elektromechanischen Hardwaremodell ermöglicht. Services besitzen immer einen \textit{Producer}, der die Funktionen des Services bereitstellt, sowie einen \textit{Consumer}, der diese nutzen kann. Dabei können auch sogenannte \textit{Prosumer} entwickelt werden, die beide Rollen erfüllen können. Durch die Verbindung der Services von verschiedenen Laborgeräten kann ein \textit{Experiment} erstellt werden. Ein Vorteil dieser Architektur ist die einfache Wiederverwendbarkeit und Austauschbarkeit von einzelnen Laborgeräten in Experimenten. So können z.B. Laborgeräte mit gleichen Services in einer entsprechenden \textit{Laborgerätegruppe} hinterlegt werden, welche dann statt eines konkreten Laborgeräts zur Erstellung eines Experiments genutzt werden kann. Beim Start des Experiments wird dann ein verfügbares Laborgerät aus der Laborgerätegruppe ausgewählt. Weiterhin gibt es noch \textit{cloud-instanziierbare} und \textit{edge-instanziierbare} Laborgeräte. Beim Start eines Experiments mit cloud- oder edge-instanziierbaren Laborgeräten wird eine entsprechende Instanz des Laborgeräts erstellt. Dabei wird für cloud-instanziierbare Laborgeräte eine Nachricht an die hinterlegte Instanziierungs-URL geschickt, wodurch die Instanz erstellt wird. Für edge-instanziierbare Laborgeräte wird mithilfe der hinterlegten Code-URL eine URL erstellt, die vom Nutzer aufgerufen werden muss, um die Instanz zu erstellen.

Das Backend von CrossLab besteht aus mehreren verschiedenen Diensten, welche zusammen eine \textit{CrossLab-Instanz} bilden. Diese Dienste sind im Folgenden aufgelistet:
\begin{itemize}
    \item \textbf{Authentication Service} \\ Dieser Dienst ist für die Authentifizierung der Nutzer verantwortlich.
    \item \textbf{Authorization Service} \\ Dieser Dienst ist für die Autorisierung der Nutzer verantwortlich.
    \item \textbf{Device Service} \\ Dieser Dienst verwaltet die Laborgeräte der CrossLab-Instanz.
    \item \textbf{Experiment Service} \\ Dieser Dienst ist für die Erstellung und Verwaltung der Experimente der CrossLab-Instanz verantwortlich.
    \item \textbf{Federation Service} \\ Dieser Dienst ist für das Teilen von Laborgeräten und Experimenten mit anderen CrossLab-Instanzen verantwortlich.
\end{itemize}
Um ein Experiment in einer CrossLab-Instanz starten zu können benötigen Nutzer ein entsprechendes Nutzerkonto für diese CrossLab-Instanz.
\chapter{Stand der Technik}\label{section:stand-der-technik}

In diesem Kapitel wird der Stand der Technik dargestellt. Dazu wird zunächst in \autoref{section:stand-der-technik:literaturrecherche} eine systematische Literaturrecherche durchgeführt und die Ergebnisse dieser geschildert.

\section{Literaturrecherche}\label{section:stand-der-technik:literaturrecherche}

Um einen Überblick über den aktuellen Stand der Forschung zu bekommen, wird zunächst eine Literaturrecherche vorgenommen. Dabei sollen vorhandene online IDEs gefunden sowie die folgenden Fragen beantwortet werden:

\begin{enumerate}
    \item Welche Implementierungen von online IDEs gibt es? \hfill (\autoref{section:stand-der-technik:literaturrecherche:implementierungen})
    \item Welchen Architekturmustern folgen online IDEs? \hfill (\autoref{section:stand-der-technik:literaturrecherche:architekturmuster})
    \item Welche Vorteile haben online IDEs? \hfill (\autoref{section:stand-der-technik:literaturrecherche:vorteile})
    \item Welche Nachteile haben online IDEs? \hfill (\autoref{section:stand-der-technik:literaturrecherche:nachteile})
    \item Welche Anforderungen werden an online IDEs gestellt? \hfill (\autoref{section:stand-der-technik:literaturrecherche:anforderungen})
\end{enumerate}

Die folgenden Datenbanken wurden für die Literaturrecherche ausgewählt:

\begin{itemize}
    \item ACM Digital Library \cite{noauthor_acm_nodate}
    \item IEEE Xplore \cite{noauthor_ieee-xplore_nodate}
    \item Scopus \cite{noauthor_scopus_nodate}
    \item Web of Science \cite{noauthor_web-of-science_nodate}
\end{itemize}

Zunächst wurde eine allgemeine Suche nach online IDEs in den genannten Datenbanken vorgenommen. Dazu wurden zunächst die in \autoref{table:search-terms} genannten Stichwörter jeweils mit ihren Synonymen mit einer OR-Operation verknüpft. Danach wurden die daraus resultierenden Terme mit einer AND-Operation verbunden. Die so entstehende Suchanfrage wurde dann für die Suche in den Datenbanken verwendet. Dabei wurden die Titel, Abstracts und Keywords der Publikationen durchsucht.

In \autoref{figure:stand-der-technik:literaturrecherche:prisma-diagram} ist der komplette Ablauf der Literaturrecherche dargestellt. Um die Anzahl der zu betrachtenden Publikationen zu verringern, wurde eine weitere Filterung der Ergebnisse vorgenommen. Dafür wurden nur Publikationen betrachtet, die IDE oder ein entsprechendes Synonym in ihrem Titel oder ihren Keywords enthalten. Dadurch sinkt die Anzahl der Treffer auf insgesamt $1705$. Danach wurden alle exakten Duplikate über einen Vergleich der Titel und Links herausgefiltert, wodurch die Anzahl der Publikationen auf $1243$ sinkt. In einem weiteren Schritt wurden die Titel und Abstracts der Publikationen genauer betrachtet. Dabei wurden unter anderem Arbeiten herausgefiltert, deren Titel und Abstracts keinen Bezug zu den Forschungsfragen besitzen. Weiterhin wurden Publikationen bevorzugt, die sich zudem mit textbasierten Programmiersprachen, Kollaboration und Lehre auf Universitätsniveau befassen. Aus dieser Filterung resultieren $97$ Publikationen. Als letzte Filterung wurden Publikationen, welche vor $2019$ veröffentlicht wurden aussortiert, falls sie weniger als $10$ Zitationen haben sowie vor $2014$ veröffentlichte Publikationen mit weniger als $25$ Zitationen. Die Anzahl der Zitationen wurde mithilfe von Google Scholar \cite{noauthor_google-scholar_nodate} ermittelt. Dadurch ergibt sich die Anzahl von $64$ zu betrachtenden Publikationen.

Die Ergebnisse der Literaturrecherche werden im Hinblick auf die gestellten Forschungsfragen in den folgenden Unterabschnitten vorgestellt.

\begin{table}[tbp]
    \centering
    \begin{tabularx}{\textwidth}{>{\hsize=.6\hsize\linewidth=\hsize}X
            >{\hsize=1.4\hsize\linewidth=\hsize}X}
        \toprule
        Stichwort                           & Synonyme                                                                                                     \\
        \midrule
        integrated development environments & IDEs, code editors, development environments, development tools, programming tools, programming environments \\
        web                                 & browser, online, cloud                                                                                       \\
        \bottomrule
    \end{tabularx}
    \caption{Suchbegriffe}
    \label{table:search-terms}
\end{table}

\begin{figure}[htbp]
    \centering
    \includegraphics[width=\textwidth]{diagrams/PRISMA.pdf}
    \caption{PRISMA Diagramm}
    \label{figure:stand-der-technik:literaturrecherche:prisma-diagram}
\end{figure}

\subsection{Implementierungen}\label{section:stand-der-technik:literaturrecherche:implementierungen}

Es gibt eine Vielzahl an verschiedenen Implementierungen von online IDEs. Um einen guten Überblick über die unterschiedlichen Ansätze und die Änderungen über den Verlauf der Zeit zu bekommen werden im Folgenden einige ausgewählte IDEs vorgestellt. Dabei werden die IDEs in zeitlich sortiert vorgestellt.

\paragraph{Adinda}
Deursen et al. (2010) \cite{van_deursen_adinda_2010} beschreiben eine online IDE namens Adinda. Die grundlegende Idee von Adinda ist die Zerlegung der Funktionalität einer IDE in einen leichtgewichtigen Client sowie mehrere zusammenarbeitende (Web-)Services. Diese Services sollen dann bestimmte Aufgaben erfüllen, wie z.B. Kompilierung, Testen, kollaboratives Editieren und Datenerhebung. Es werden weiterhin verschiedenste Forschungsfragen aufgestellt, die für das vorgestellte System von Interesse sind. Die prototypische Implementierung von Adinda basiert auf WWWorkspace \cite{ryan_web_2007} und nutzt serverseitig die Eclipse IDE \cite{noauthor_eclipse_nodate}. Der Prototyp unterstützt das Erstellen von Nutzer-Workspaces, Java Projekten, Paketen und Klassendateien sowie Syntax Highlighting, Kompilierung und Code-Vervollständigung.

\paragraph{CEclipse}
Wu et al. (2011) \cite{wu_ceclipse_2011} stellen die online IDE Cloud Eclipse (CEclipse) vor. Die Ziele von CEclipse sind:
\begin{enumerate}
    \item die Bereitstellung von Funktionen der Eclipse IDE \cite{noauthor_eclipse_nodate}, wie zum Beispiel Code-Vervollständigung
    \item die Behandlung von den online IDE spezifischen Sicherheitsproblemen \quoted{\textit{Wrong file operations}}, \quoted{\textit{Banned operation calling}} und \quoted{\textit{Excessive resource consumption}}
    \item die Ausnutzung von Cloud Computing Möglichkeiten um Entwickler besser zu unterstützen.
\end{enumerate}
Für $1.$ wurde ein entsprechendes Protokoll entwickelt, was es ermöglicht die gewünschten Funktionen der Eclipse IDE aufzurufen und das Ergebnis im Browser darzustellen. Um die in $2.$ genannten Probleme Handhaben zu können wird ein \textit{Program Behavior Analysis Service} beschrieben. Durch die Einschränkung des Dateisystems auf einen speziellen Ordner kann das Problem \quoted{Wrong file operations} gelöst werden. Das Verbieten bzw. Erlauben von Methoden über eine Blacklist bzw. eine Whitelist kann zur Lösung des Problems \quoted{Banned operation calling} angewendet werden. Durch eine Zeitbegrenzung von laufenden Prozessen kann schließlich auch das Problem \quoted{Excessive resource consumption} behoben werden. Für $3.$ werden über den \textit{Program Behavior Mining Service} Daten über die Nutzung der IDE gesammelt werden. Diese Daten können dann z.B. dazu genutzt werden dem Entwickler häufig verwendete Befehle mit höherer Priorität vorzuschlagen.

\paragraph{Collabode}
Goldman et al. (2011) \cite{goldman_real-time_2011} beschreiben Collabode, eine kollaborative online IDE für Java. Collabode ermöglicht es mehreren Nutzern gleichzeitig Änderungen an Dateien vorzunehmen. Die Änderungen werden in Echtzeit zwischen den Nutzern synchronisiert. Dabei wird ein spezieller Algorithmus verwendet. Dieser Algorithmus sorgt dafür, dass nur Änderungen eingepflegt werden, die keinen syntaktischen Fehler beinhalten oder erzeugen. Dadurch wird sichergestellt, das Nutzer nur ihre eigenen Fehler sehen und das Programm unabhängig von den ggf. vorhandenen Fehlern ihrer Teammitglieder kompilieren können. Collabode nutzt EtherPad \cite{noauthor_etherpad_nodate} als Code Editor im Frontend und die Eclipse IDE \cite{noauthor_eclipse_nodate} für die Bereitstellung von Kompilierung, Syntax Highlighting, etc. im Backend.

\paragraph{CoRED}
Lautamäki et al. (2012) \cite{lautamaki_cored_2012} stellen den Collaborative Real-time Editor (CoRED) vor, einen online Code Editor für Java Programme. CoRED nutzt den ACE Editor \cite{noauthor_ace_nodate} im Frontend sowie das Java Development Kit (JDK) \cite{noauthor_jdk_nodate} zur Kompilierung im Backend. Fehlermeldungen während des Kompiliervorgangs werden an den Client zurückgesendet und dann im Frontend angezeigt. Zur Bereitstellung von Echtzeit Kollaboration wird der Algorithmus \textit{Differential Synchronization with shadows} \cite{fraser_differential_2009} von Neil Fraser eingesetzt. Ein weiteres Feature von CoRED ist das Sperren von Code Bereichen für andere Nutzer sowie die Möglichkeit Kommentare im Code zu hinterlassen. Auf diese Kommentare können dann andere Nutzer antworten, wodurch eine weitere Interaktionsmöglichkeit besteht. Weiterhin bietet CoRED auch Code-Vervollständigung.

\paragraph{IDEOL}
Tran et al. (2013) \cite{tran_interactive_2013} stellt die online IDE IDEOL vor. IDEOL erlaubt es Nutzern in Echtzeit miteinander zu kollaborieren. Dies beinhaltet das gleichzeitige Bearbeiten von Dateien mit Synchronisierung der Änderungen zwischen den Nutzern sowie ein Diskussionsforum mit einem Tagging-Mechanismus. Über diesen Mechanismus können Nutzer Codezeilen oder ganze Dateien in einer Diskussion taggen. Weiterhin können Nutzer auch Dateien an ihre Nachrichten anhängen. Zudem bietet IDEOL eine Übersicht über die Änderungen im Code. IDEOL bietet Support für die Entwicklung von C/C++ Programmen und erlaubt auch die Kompilierung, Ausführung sowie das Debuggen von diesen. Um die Echtzeit Kollaboration zu ermöglichen unterscheidet IDEOL zwischen exklusiven und nicht-exklusiven Operationen. Zu den exklusiven Operationen zählen die Kompilierung, Ausführung und das Debuggen eines Programms. Dateien werden über einen Server synchronisiert und dort gespeichert. Die Behandlung von gleichzeitigen Operationen wird mithilfe eines Operational Transformation Algorithmus \cite{sun_operational_1998} vorgenommen. Exklusive Operationen werden auf der im Server persistent gespeicherten Version ausgeführt, sodass die im Arbeitsspeicher des Servers vorhandene Version weiterhin zur Bearbeitung verwendet werden kann. Nguyen et al. (2016) \cite{nguyen_enhancing_2016} nutzen IDEOL für die web-basierte kollaborative Umgebung EduCo. Das Ziel von EduCo ist die Bereitstellung von Funktionen für Lehrende und Lernende besonders im Hinblick auf deren Interaktion und Kollaboration.

\paragraph{TouchDevelop}
Ball et al. (2015) \cite{ball_beyond_2015} beschreiben die online IDE TouchDevelop. Das Hauptfeature von TouchDevelop ist die Speicherung aller Programmänderungen, Versionen, Laufzeitinformationen, Bugs sowie Kommentare, Fragen und Feedback von Nutzern in einer zentralen Datenbank. Diese Daten können über entsprechende APIs abgefragt werden. Die Nutzeroberfläche von TouchDevelop unterscheidet sich stark von anderen textbasierten Editoren. So bekommt der Nutzer eine Auswahl an kontextabhängigen Optionen, z.B. if-Anweisungen, for-Schleifen oder verfügbare Variablen. Alle IDE Funktionen sind offline verfügbar, da sie komplett auf der Clientseite implementiert sind. Weiterhin nutzt TouchDevelop eine eigene Programmiersprache. Diese folgt dem imperativen Programmierparadigma, besitzt ein starkes Typsystem sowie eine Vielzahl an plattformübergreifenden APIs.

\paragraph{CodePilot}
Warner und Guo (2017) \cite{warner_codepilot_2017} stellen die online IDE CodePilot vor. Diese ermöglicht es Nutzern Webapplikationen mithilfe von HTML, CSS und JavaScript zu entwickeln. Zudem können Nutzer mithilfe von Firepad \cite{noauthor_firepad_nodate} gleichzeitig an Projekten arbeiten. Weiterhin bietet CodePilot eine GitHub \cite{noauthor_github_nodate} Integration, wodurch Nutzer ihre Projekte aus GitHub importieren können. Über diese Integration werden auch bei jedem Commit die entsprechenden Änderungen an GitHub zurückgesendet. Weiterhin können Nutzer über einen Aktivitätsfeed die aktuellen Ereignisse nachverfolgen und mithilfe eines integrierten Text-Chats miteinander kommunizieren. Weiterhin werden durch den Ace-Editor \cite{noauthor_ace_nodate} Syntax Highlighting und Code-Vervollständigung bereitgestellt. Zusätzlich bietet CodePilot die Möglichkeit Issues zu erstellen. Diese werden dann mit GitHub synchronisiert und es wird ein Snapshot des aktuellen Projekts auf GitHub hinterlegt. Bei der Erstellung von Issues können auch Screenshots angefügt werden. Jede Issue enthält die Referenz auf den entsprechenden Snapshot des Projekts.

\paragraph{RIDE}
Über mehrere Publikationen wird die Entwicklung der Reflex IDE (RIDE) beschrieben. Zunächst wird von Bastrykina et al. (2021) \cite{bastrykina_developing_2021} ein entsprechender Kernel mit dem Xtext Framework \cite{noauthor_xtext_nodate} entwickelt. Dieser kann in der Eclipse IDE verwendet werden, um Funktionen wie Code-Vervollständigung und Code-Generation für die domainspezifische Sprache Reflex verfügbar zu machen. Zudem wird auch eine Integration mit dem \ac{LSP} \cite{noauthor_language-server-protocol_nodate} erreicht, wodurch diese Features auch für andere Code Editoren nutzbar sind. Darauf aufbauend wird von Gornev und Liakh (2021) \cite{gornev_ride_2021} die Konzipierung und Implementierung einer auf Theia basierten Web-Variante von RIDE vorgestellt. Gornev et al. (2022) \cite{gornev_towards_2022} beschreiben ein System, welches Docker verwendet um die Web-Version von RIDE für mehrere simultane Nutzer bereitstellen zu können. Gornev und Bondarchuk (2023) \cite{gornev_towards_2023} stellen ein Framework vor welches Echtzeit-Kollaboration in RIDE ermöglicht. Kuznetsov und Zyubin (2024) \cite{kuznetsov_development_2024} beschreiben die Entwicklung eines Projektmanagement-Systems für RIDE.

\paragraph{PyodideU}
Jefferson et al. (2024) \cite{jefferson_pyodideu_2024} beschreiben eine IDE, die es Nutzern ermöglicht Python Code im Browser zu schreiben und auszuführen. Dabei wird das Programm des Nutzers lokal in dessen Browser durchgeführt. Dies wird durch den Einsatz von PyodideU erreicht, einer erweiterten Version der auf WebAssembly basierenden Python Distribution Pyodide \cite{noauthor_pyodide_nodate}. Zusätzlich wird den Nutzern auch eine Grafikbibliothek angeboten samt eines Debuggers, der es ermöglicht Zeile für Zeile und auch rückwärts durch das Programm zu gehen und die entsprechenden Änderungen an der Grafik zu sehen. Weiterhin wird durch PyodideU auch die synchrone Eingabe von Daten unterstützt, während Python im Main-Thread des Browsers läuft. Zudem wird auch ein Dateisystem bereitgestellt. Insgesamt wurde die IDE sowohl von Studenten als auch von Lehrenden als hilfreich wahrgenommen.

\paragraph{CS50}
Malan (2024) \cite{malan_containerizing_2024} beschreibt die verschiedenen Ansätze zur Bereitstellung einer integrierten Entwicklungsumgebung für die Teilnehmer des Einführungskurses in die Programmierung (CS50) an der Harvard University. Zunächst wurde seit 2007 ein On-Campus Cluster für die Studenten angeboten. Studenten konnten über SSH mit diesem verbinden und dort ihre Programme ausführen. Dieser Cluster wurde 2008 mithilfe von Amazon Web Services (AWS) \cite{noauthor_amazon_nodate} in die Cloud überführt. Diese cloud-basierte Lösung wurde 2011 durch Client-seitige virtuelle Maschinen ersetzt. In 2015 wurde eine auf Docker basierende Lösung erarbeitet, die zunächst die Cloud9 IDE \cite{noauthor_cloud_nodate} als Frontend nutzte. In 2021 wurde schließlich eine auf Github Codespaces aufbauende Lösung eingeführt, die \ac{VSCode} \cite{noauthor_vscode_nodate} als Code Editor verwendet.
\subsection{Architekturmuster}\label{section:stand-der-technik:literaturrecherche:architekturmuster}

Die Publikationen beschreiben eine Vielzahl an verschiedenen webbasierten integrierten Entwicklungsumgebungen. Dabei kann eine Unterteilung in die folgenden drei Kategorien erfolgen:

\begin{itemize}
    \item \textbf{Client-Server-basierte Lösungen} \\
          Systeme dieser Art zeichnen sich dadurch aus, dass sie eine Client-Server-Archi-tektur verwenden. Hierbei werden Features, die nicht innerhalb eines Browsers ausgeführt werden können (z.B. Kompilierung) über einen entsprechenden Server bereitgestellt.
    \item \textbf{Browser-basierte Lösungen} \\
          Systeme dieser Art zeichnen sich dadurch aus, dass alle Features im Browser des Nutzers ausgeführt werden können, ohne die Hilfe eines separaten Servers.
    \item \textbf{Cloud-basierte Lösungen} \\
          Systeme dieser Art zeichnen sich dadurch aus, dass sie als Cloud-Service angeboten werden können. Meist erhalten Nutzer eine komplett eigene Umgebung samt Dateisystem, Compiler, Debugger und weiteren Tools. Diese Lösungen nutzen oftmals Containerisierung.
\end{itemize}
\subsection{Vorteile}\label{section:stand-der-technik:literaturrecherche:vorteile}

Online IDEs bieten eine Vielzahl an Vorteilen gegenüber lokalen IDEs. Diese Vorteile werden im Folgenden erläutert.

\paragraph{Keine Installation}
Ein Vorteil von online IDEs ist die Tatsache, dass diese keine Installation benötigen \cite{srinivasa_bad_2022}\cite{tran_interactive_2013}\cite{yang_evaluations_2018}. Somit wird es Nutzern ermöglicht direkt mit der Programmierung zu beginnen, ohne zuvor die benötigte Software auf ihrem System installieren zu müssen. Dadurch ist es auch möglich, die IDE auf Systemen auszuführen, auf denen sie sonst nicht installierbar wäre, wie z.B. mobilen Endgeräten \cite{jefferson_pyodideu_2024}\cite{ball_beyond_2015}\cite{uehara_javascript_2019}. Im Bezug auf die Lehre erlaubt es den Lehrenden und Lernenden, sich besser auf die Inhalte der Lehrveranstaltung zu konzentrieren, da sie weniger Zeit damit verbringen müssen systemabhängige Probleme mit der Einrichtung der IDE zu lösen \cite{valez_student_2020}.

\paragraph{Zeit- und Ortsunabhängigkeit}
Nutzer können jederzeit und von überall auf online IDEs zugreifen, solange sie einen entsprechenden Browser sowie eine Internetverbindung haben. Weiterhin können einige komplett im Browser nutzbare IDEs nach dem erstmaligen Laden auch offline genutzt werden \cite{jefferson_pyodideu_2024}. Außerdem erlaubt eine serverseitige Speicherung von Daten einen systemunabhängigen Zugriff auf diese \cite{ball_beyond_2015}. Somit können Nutzer selbst wählen, wann und mit welchem Gerät sie die IDE nutzen möchten, ohne von Zugriffszeiten und spezieller Hardware abhängig zu sein.

\paragraph{Einheitliche Umgebung}
Nutzern kann unabhängig von ihrem System eine einheitliche Entwicklungsumgebung angeboten werden \cite{molnar_evaluation_2023}\cite{tran_interactive_2013}. Somit können systemabhängige bzw. versionsabhängige Probleme vermieden werden, was z.B. in der Lehre dazu führt, dass Lehrende und Lernende weniger Zeit für die Behebung derartiger Probleme aufbringen müssen \cite{valez_student_2020}. Außerdem können Client-Server sowie cloudbasierte online IDEs die Benutzererfahrung für Nutzer mit weniger performanten Systemen verbessern, da ein Teil der Berechnungen von einem externen Server durchgeführt wird.

\paragraph{Einbindbarkeit in Lernmanagementsysteme}
Ein weiterer Vorteil von online IDEs ist deren vereinfachte Einbindbarkeit in \acp{LMS}, wie z.B. Moodle \cite{noauthor_moodle_nodate}. Da online IDEs im Browser des Nutzers ausgeführt werden, können diese direkt in LMS eingebunden werden. Weiterhin können durch die Implementierung entsprechender Schnittstellen, wie z.B. \ac{LTI} \cite{noauthor_lti_nodate}, auch weitere Funktionen wie die automatische Bewertung der Lösungen von Lernenden ermöglicht werden. Diese Art der Integration kann die Benutzererfahrung der Lernenden verbessern.

\paragraph{Einfachere Datenerhebung}
Online IDEs können die Erhebung von Nutzerdaten vereinfachen \cite{efopoulos_wipe_2005}\cite{singh_pyguru_nodate}\cite{helminen_recording_2013}. So können u.a. Cursorbewegungen, Code-Änderungen und Zeitaufwand aufgenommen und später analysiert werden. Mithilfe der Nutzerdaten können z.B. Lehrende erkennen, bei welchen Aufgaben Studenten die meisten Probleme haben bzw. womit sie die meiste Zeit verbringen.

\paragraph{Skalierbarkeit}
Browser- sowie cloudbasierte IDEs besitzen eine hohe Skalierbarkeit. Browserbasierte IDEs können, nachdem sie im Browser des Nutzers geladen wurden, ohne bzw. mit geringen zusätzlichen Serverressourcen genutzt werden \cite{ball_beyond_2015}\cite{jefferson_pyodideu_2024}. Cloudbasierte IDEs können sich an dynamische Lastverhältnisse anpassen \cite{noauthor_azure-cloud-services_nodate}\cite{noauthor_ec2-autoscaling_nodate}. So kann für jeden Nutzer beim Starten der IDE eine entsprechende Instanz gestartet werden, die beim Verlassen der IDE wieder gestoppt wird. Dabei ist der Cloud-Anbieter für die Bereitstellung der entsprechenden Serverressourcen verantwortlich. Klassische Client-Server Implementierungen können auch die Skalierbarkeit erhöhen, wenn man die zuvor genannten Vorteile betrachtet.
\subsection{Nachteile}\label{section:stand-der-technik:literaturrecherche:nachteile}

Neben den zuvor beschriebenen Vorteilen besitzen online IDEs auch einige Nachteile, welche im Folgenden beschrieben werden.

\paragraph{Onlinezwang}
Ein Nachteil von online IDEs ist der oftmals damit verbundene Onlinezwang und die damit einhergehenden Probleme, die durch Latenzen, Instabilitäten und Ausfälle des Netzwerks ausgelöst werden \cite{kats_software_2012}\cite{leisner_good-bye_2019}. So können Latenzen zu einer schlechteren Nutzererfahrung führen, indem z.B. während einer Echtzeit Kollaboration die Änderungen anderer Nutzer stark verzögert ankommen. Instabilitäten können dazu führen, dass Änderungen verloren gehen bzw. der Nutzer auf eine stabilere Verbindung warten muss. Netwerkausfälle bedeuten in vielen Fällen, dass der Nutzer nicht mit der Programmierung fortfahren kann bis die Netzwerkverbindung wiederhergestellt wird. Außerdem kann es auch vorkommen, dass ein Fehler auf der Serverseite auftritt, wodurch Nutzer die online IDE nicht nutzen können. In diesem Fall müssen Nutzer darauf warten, bis das zugrundeliegende Problem behoben wird.

\paragraph{Abhängigkeit von Anbietern}
In den meisten Fällen besitzen Nutzer keine Kontrolle über die installierte Software \cite{kats_software_2012}. Sollte der Anbieter Updates durchführen kann dies dazu führen, dass zuvor funktionierende Programme des Nutzers nun versionsabhängige Fehler beinhalten. Weiterhin können die angebotenen IDEs auch komplett eingestellt werden, wodurch Nutzer keine Möglichkeit mehr besitzen diese zu nutzen. Zudem können Anbieter auch ihre Preise anpassen, wodurch die Kosten für die online IDE steigen können.

\paragraph{Implementierungs und Verwaltungsaufwand}
Der Implementierungs- und Verwaltungsaufwand von online IDEs kann sehr hoch sein \cite{malan_standardizing_2022} und somit einige der Vorteile von online IDEs aufwiegen. In gewissen Szenarien ist auch die Skalierbarkeit von online IDEs ein Problem, da diese oftmals mit erhöhten Kosten oder einem entsprechend höherem Verwaltungsaufwand verbunden ist. Diese Probleme sind entsprechend gravierender, wenn kein entsprechendes Personal für verfügbar ist und diese Aufgaben z.B. an Lehrende übertragen werden.

\paragraph{Sicherheit}
Anbieter von online IDEs müssen sicherstellen, dass Nutzer keine ungewünschen Aktionen ausführen können, die das System beeinträchtigen könnten. Derartige Aktionen sind z.B. \quoted{Wrong file operations}, \quoted{Banned operation calling}, \quoted{Excessive resource consumption} \cite{wu_ceclipse_2011} und \quoted{Arbitrary code execution} \cite{srinivasa_bad_2022}.

\paragraph{Erweiterbarkeit}
Zudem sind viele der während der Literaturrecherche gefundenen online IDEs sehr anwendungsspezifisch und somit in einigen Aspekten stark eingeschränkt, z.B. in den unterstützten Editoren, Programmiersprachen sowie den vorhandenen Einstellungs- und Erweiterungsmöglichkeiten. Weiterhin werden viele der gefundenen IDEs nicht mehr angeboten bzw. unterstützt und oftmals sind diese auch nicht quelloffen.


\subsection{Anforderungen}\label{section:stand-der-technik:literaturrecherche:anforderungen}

Die Anforderungen an online IDEs sind sehr anwendungsspezifisch. Allerdings gibt es dennoch einige allgemeinere Anforderungen, die im Folgenden erläutert werden.

% Anforderungen:
% \begin{itemize}
%     \item Angemessenes Nutzerinterface
%     \item Syntax-Highlighting
%     \item Code-Vervollständigung
%     \item Kompilierung
%     \item Debugging
%     \item Kollaboration
%     \item (Erweiterbarkeit)
%     \item Sehr anwendungsspezifisch (z.B. Flashing von Microcontrollern, Rollenwechsel in Kollaboration)
%     \item Kommunikationsmöglichkeiten
%     \item Projektmanagement Features
%     \item Versionskontrolle
%     \item Programmiersprachen Support
%     \item Programmausührung
%     \item Testen
%     \item Input / Output
%     \item Guides für Nutzer
%     \item Möglichkeiten zur Datenerhebung
%     \item Interfaces für Lehrende
% \end{itemize}

\paragraph{Interface}
Das Nutzerinterface sollte an die Zielgruppe angepasst sein \cite{malan_standardizing_2022}. Ein professioneller Programmierer erwartet ein anderes Interface als ein Schüler, der zum ersten Mal programmiert. Dementsprechend könnte für erstere z.B. ein bereits vorhandenes IDE Interface genutzt werden, während für zweitere eine vereinfachtes Interface angeboten werden sollte, dass z.B. nur die nötigsten Funktionen anbietet um die unerfahrenen Nutzer nicht zu überfordern.

\paragraph{Editor Funktionen}
Online IDEs sollten Editor Funktionen, wie z.B. Syntax Highlighting, Code-Vervollständigung, Code-Navigation und Refactoring anbieten. Dabei müssen diese nicht für alle Programmiersprachen angeboten werden, aber es ist vom Vorteil wenn diese über entsprechende Erweiterungsmechanismen hinzugefügt werden können.

\paragraph{Projektmanagement}
Es sollte Nutzern ermöglicht werden ihre Projekte in der online IDE speichern zu können. Dabei gibt es die Möglichkeiten der clientseitigen Speicherung, bei denen die Daten z.B. im Browser des Nutzers hinterlegt werden, und der serverseitigen Speicherung. Letztere ermöglicht wie zuvor erwähnt den systemunabhängigen Zugriff auf die Projekte des Nutzers.

\paragraph{Kollaboration}
Eine häufig genannte Anforderung ist die Bereitstellung von Kollaborationsmöglichkeiten für die Nutzer der online IDE. Dabei gibt es synchrone und asynchrone Kollaborationsmöglichkeiten. Erstere beinhalten z.B. die Möglichkeit mit anderen Nutzern gleichzeitig an einer geteilten Datei arbeiten zu können. Zweitere beinhalten z.B. Versionskontrollsysteme, Chats und Foren.

\paragraph{Erweiterbarkeit}
Eine erweiterbare online IDE hat den Vorteil, dass sie ggf. für mehrere verschiedene Anwendungsfälle genutzt werden kann. Eine bereits genannte Erweiterungsmöglichkeit ist z.B. die Unterstützung neuer Programmiersprachen über die Bereitstellung der entsprechenden Editor Funktionen.

\paragraph{Datenerhebung}
Wie bereits in den Vorteilen erwähnt ermöglicht die Datenerhebung ein besseres Verständnis für das Nutzerverhalten. Zudem können die erhobenen Daten genutzt werden um die Nutzererfahrung zu verbessern. Daher ist die Möglichkeit der Datenerhebung eine oftmals genannte Anforderung.

\section{Weitere Entwicklungen}\label{section:stand-der-technik:weitere-entwicklungen}

In diesem Kapitel werden weitere Entwicklungen beschrieben, die nicht im Fokus der Literaturrecherche standen, aber dennoch von Interesse für diese Arbeit sind. Diese umfassen die folgenden Kategorien:

\begin{itemize}
    \item \textbf{Standalone IDEs und Code Editoren} \\
          IDEs und Code Editoren, die lokal auf dem Rechner des Nutzers installiert und verwendet werden können.
    \item \textbf{IDE Frameworks} \\
          Frameworks, die zur Erstellung eigener IDEs verwendet werden können.
    \item \textbf{Remote Development Plattformen} \\
          Plattformen, die Nutzern vollständige IDEs online zur Verfügung stellen.
    \item \textbf{Educational IDE Plattformen} \\
          Plattformen, deren Zielgruppe Lehrende und Lernende sind.
\end{itemize}

In den folgenden Unterabschnitten werden ausgewählte Beispiele für die einzelnen Kategorien dargelegt.

\subsection{IDEs und Code Editoren}\label{section:stand-der-technik:weitere-entwicklungen:ides-und-code-editoren}

Dieser Unterabschnitt befasst sich mit IDEs und Code Editoren. Dabei werden im Folgenden Visual Studio Code \cite{noauthor_vscode_nodate}, JetBrains IDEs \cite{noauthor_jetbrains_nodate} und Stackblitz \cite{noauthor_stackblitz_nodate} vorgestellt.

\paragraph{Visual Studio Code}
\acf{VSCode} \cite{noauthor_vscode_nodate} ist ein von Microsoft \cite{noauthor_microsoft_nodate} entwickelter Code Editor. \ac{VSCode} baut auf dem ebenfalls von Microsoft entwickelten Monaco Editor \cite{noauthor_monaco_nodate} auf und bietet zusätzliche Features an, wie z.B. Workspace-Management, Nutzerinterfaces für Versionskontrolle und Debugging sowie die Möglichkeit Erweiterungen zu installieren. Diese Erweiterungen nutzen die VSCode Extension API \cite{noauthor_vscode-extension-api_nodate} und erlauben es, \ac{VSCode} um verschiedenste Funktionen zu erweitern. \ac{VSCode} ist frei verfügbar und kann sowohl als Desktop-, Browser- oder Cloudanwendung genutzt werden. Somit kann \ac{VSCode} für unterschiedlichste Anwendungsfälle verwendet werden.

\paragraph{JetBrains IDEs}
JetBrains \cite{noauthor_jetbrains_nodate} bietet eine Vielzahl an spezialisierten IDEs an. Diese sind meist auf eine spezifische Programmiersprache ausgelegt und beinhalten bereits die zur Programmierung benötigten Softwarewerkzeuge, wie z.B. Compiler, Interpreter, Debugger, Testframeworks und mehr. Zusätzlich könnnen die IDEs mit entsprechenden Plugins \cite{noauthor_jetbrains-plugins_nodate} erweitert werden. Der Großteil der von JetBrains angebotenen IDEs ist kostenpflichtig und als Desktop- sowie als Cloudanwendung nutzbar.

\paragraph{Stackblitz}
Stackblitz \cite{noauthor_stackblitz_nodate} ist eine online IDE für Projekte, die auf der Programmiersprache JavaScript basieren. Dabei wird sowohl die direkte Entwicklung für Browser als auch die Entwicklung für die JavaScript Runtime Node.js \cite{noauthor_nodejs_nodate} unterstützt. Die komplette IDE wird im Browser des Nutzers ausgeführt und bietet Funktionen, wie z.B. Debugging, Versionskontrolle über Git und die Möglichkeit auch kurzzeitig offline weiterarbeiten zu können, bis die Verbindung wiederhergestellt wird. Node.js wird mithilfe von WebContainern \cite{noauthor_webcontainer_nodate} im Browser des Nutzers ausgeführt. Stackblitz besitzt eine kostenlose Variante für die private und nichtkommerzielle Nutzung sowie verschiedene kostenpflichtige Varianten.
\subsection{IDE Frameworks}\label{section:stand-der-technik:weitere-entwicklungen:ide-frameworks}

IDE Frameworks ermöglichen die Implementierung eigener IDEs, basierend auf einem entsprechenden Grundgerüst. Dabei werden im Folgenden Eclipse Theia \cite{noauthor_theia_nodate} und OpenSumi \cite{noauthor_opensumi_nodate} vorgestellt.

\paragraph{Eclipse Theia}
Eclipse Theia \cite{noauthor_theia_nodate} ist ein Projekt der Eclipse Foundation \cite{milinkovich_eclipse-foundation_nodate} bestehend aus der Theia Platform, einem Framework für die Entwicklung von IDEs, sowie der darauf aufbauenden Theia IDE. Theia nutzt den Monaco Editor \cite{noauthor_monaco_nodate} als Code Editor. Eine mithilfe der Theia Platform erstellten IDE besteht aus sogenannten Theia Erweiterungen. Diese sind Teil der Kompilierung und können tiefgreifende Änderungen an der IDE vornehmen. Weiterhin können auch Erweiterungen genutzt werden, die mithilfe der VSCode Extension API \cite{noauthor_vscode-extension-api_nodate} erstellt wurden. Diese werden zur Laufzeit hinzugefügt und haben somit nur einen begrenzten Zugriff auf die internen Schnittstellen der IDE. Daher bieten Theia Erweiterungen die Möglichkeit, tiefergreifende Änderungen vorzunehmen, als es mit VSCode Erweiterungen möglich ist. Mit der Theia Platform erstellte IDEs bestehen aus einem Backend sowie einem Frontend. Allerdings kann das Frontend mit entsprechendem Funktionsverlust auch ohne das Backend verwendet werden. Dadurch können die IDEs als Desktop-, Browser- und Cloudanwendungen implementiert werden.

\paragraph{OpenSumi}
OpenSumi \cite{noauthor_opensumi_nodate} ist ein von Alibaba \cite{noauthor_alibaba_nodate} entwickeltes Framework zur Entwicklung von IDEs. OpenSumi verwendet den Monaco Editor \cite{noauthor_monaco_nodate} als Code Editor. Ähnlich wie bei Theia werden auch bei OpenSumi zwei verschiedene Arten von Erweiterungen unterstützt: OpenSumi Erweiterungen und VSCode Erweiterungen. Dabei bieten OpenSumi Erweiterungen mehr Anpassungsmöglichkeiten als VSCode Erweiterungen, da diese auf weitere Schnittstellen zugreifen können. OpenSumi ermöglicht die Entwicklung von IDEs als Desktop-, Browser- und Cloudanwendungen.

% Theia
% OpenSumi
\subsection{Remote Development Plattformen}\label{section:stand-der-technik:weitere-entwicklungen:remote-development-plattformen}

Remote Development Plattformen ermöglichen es Nutzern komplette Entwicklungsumgebungen online zu nutzen. Diese werden auf entsprechenden Servern gehostet. Dabei werden meist Container verwendet, welche die komplette benötigte Software für die Entwicklungsumgebung des Nutzers beinhalten und gleichzeitig von anderen Nutzern isoliert sind. Als Beispiele für Remote Development Plattformen werden im Folgenden Eclipse Che \todoaddref[]{Eclipse Che}, Github Codespaces \todoaddref[]{Github Codespaces} und

\paragraph{Eclipse Che} \dots

\paragraph{Github Codespaces} \dots

\paragraph{}

% Eclipse Che (https://eclipse.dev/che/)
% Coder (https://coder.com/)
% Gitpod (https://www.gitpod.io/)
% Replit (https://replit.com/)
% Codeanywhere (https://codeanywhere.com/)
% Github Codespaces (https://github.com/features/codespaces)
% Amazon Cloud9 (discontinued) (https://aws.amazon.com/de/cloud9/)
\subsection{Educational IDE Plattformen}\label{section:stand-der-technik:weitere-entwicklungen:educational-ide-plattformen}

\paragraph{CodeHS} \dots

\paragraph{Codio} \dots

% Codeboard (https://codeboard.io/)
% Coding Rooms (discontinued) (https://www.codingrooms.com/)
% CodeHS (https://codehs.com/)
% JDoodle? (https://www.jdoodle.com/)
% KIRA (https://www.kira-learning.com/)
% Codio (https://www.codio.com/)
% Online IDE (https://online-ide.de/)
% Codegrade (https://www.codegrade.com/)
% CodeOcean (https://codeocean.openhpi.de/)
\chapter{Anforderungsanalyse}\label{section:anforderungsanalyse}

In diesem Kapitel werden die Anforderung an die zu entwickelnde IDE dargelegt. Diese wurden in Gesprächen mit Experten des GOLDi-Remotelab sowie Experten des CrossLab Projekts erhoben. Um einen Kontext für die Anforderungen herzustellen, wird in \autoref{section:anforderungsanalyse:beispielszenario} ein entsprechendes Beispielszenario eines Praktikumsversuchs angeführt. Daraufhin werden in \autoref{section:anforderungsanalyse:anforderungen} die erhobenen Anforderungen beschrieben.

\section{Beispielszenario: Praktikumsversuch}\label{section:anforderungsanalyse:beispielszenario}

In diesem Abschnitt wird ein Beispielszenario für den Einsatz der zu entwickelnden integrierten Entwicklungsumgebung beschrieben werden. Bei dem Beispielszenario handelt es sich um einen Praktikumsversuch, bei dem die Studierenden in Zweiergruppen mehrere verschiedene Aufgaben lösen müssen. Diese Aufgaben nutzen unterschiedliche Experimentkonfigurationen.

\paragraph{Vorbereitung}
Für den Praktikumsversuch müssen ggf. neue CrossLab-kompatible Geräte implementiert und angebunden werden. Dafür können entweder bereits vorhandene Services genutzt oder neue Services entwickelt werden. Sollten neue Services entwickelt werden ist es ggf. notwendig bereits vorhandene Geräte mit diesen zu erweitern. Weiterhin werden Testfälle ausgearbeitet, welche die automatische Überprüfung von abgegebenen Lösungen ermöglichen.

\paragraph{Durchführung}
Nachdem der Praktikumsversuch fertiggestellt wurde kann er den Studierenden vorgestellt werden. Dabei zeigt ein Dozent den Studierenden die Aufgaben und wie diese zu bearbeiten sind. Danach können die Studierenden mit der Bearbeitung der Aufgaben beginnen. Dafür starten sie die entsprechenden Experimente über ein ihnen bereitgestelltes Interface, z.B. über ein vorhandenes \ac{LMS}. Die Studierenden sollen die gestellten Aufgaben gemeinsam bearbeiten. Sollten während der Bearbeitung der Aufgaben Probleme auftreten können sich die Studierenden an einen entsprechenden Betreuer wenden. Die Studierenden können ihre erarbeitete Lösung jederzeit mit den hinterlegten Testfällen überprüfen. Sobald alle Testfälle erfolgreich beendet werden ist die entsprechende Aufgabe erfüllt und die Ergebnisse werden entsprechend gespeichert.

\paragraph{Nachbereitung}
Während der Ausführung der Experimente können Daten erfasst werden, die es Lehrenden ermöglicht nachzuvollziehen bei welchen Aufgaben die Studierenden Probleme hatten bzw. wofür sie die meiste Zeit aufwenden mussten. Dazu kann z.B. die Code-Historie der Studierenden gespeichert werden sowie die in einem Experiment verbrachte Zeit. Durch die Analyse der entsprechenden Daten können Lehrende Änderungen am Praktikumsversuch vornehmen um diesen zu verbessern.

\section{Anforderungen}\label{section:anforderungsanalyse:anforderungen}

\newarray\Requirements
\expandarrayelementfalse
\newcounter{requirement}
\newenvironment{requirement}[1]
{
\def\reqlabel{requirement:#1}
\refstepcounter{requirement}
\label{\reqlabel}
\Requirements(\value{requirement})={\autoref{requirement:#1} & #1}
\tabularx{\textwidth}{p{2.5cm} X}
\toprule
\multicolumn{1}{p{3.25cm}}{\textbf{Anforderung \arabic{requirement}}} & \multicolumn{1}{c}{\textbf{#1}} \\
\midrule
}{
\bottomrule
\endtabularx
}
\newcommand{\reqdescription}{Beschreibung &}
\newcommand{\reqrationale}{Begründung &}
\def\requirementautorefname{Anforderung}
\def\replspaces#1#2{\expandafter\replspacesA\expandafter#2#1 \end}
\def\replspacesA#1#2 #3{#2\ifx\end#3\else#1\afterfi{\replspacesA#1#3}\fi}
\def\afterfi#1#2\fi{\fi#1}
\def\replace#1#2#3{%
    \def\tmp##1#2{##1#3\tmp}%
    \tmp#1\stopreplace#2\stopreplace}
\def\stopreplace#1\stopreplace{}

Im Folgenden werden die Anforderungen an die zu entwickelnde IDE beschrieben. Dabei wird für jede der Anforderungen eine entsprechende Beschreibung sowie Begründung angegeben. Eine Übersicht aller Anforderungen ist in \autoref{table:anforderungen} gegeben.

\begin{requirement}{CrossLab-Kompatibilität}
    \reqdescription Die zu entwickelnde IDE soll CrossLab-Services anbieten und konsumieren können. \\
    \reqrationale Durch die Verwendung von CrossLab-Services für die verschiedenen Funktionen der IDE kann diese in einem Experiment mit anderen Laborgeräten verbunden werden. Dadurch können Funktionen der IDE auf weitere Laborgeräte ausgelagert werden, wodurch die Erweiterbarkeit der IDE verbessert werden kann. \\
\end{requirement}

\begin{requirement}{Erweiterbarkeit}
    \reqdescription Die zu entwickelnde IDE soll Schnittstellen zum Hinzufügen von CrossLab-Services und Benutzerinterfaces besitzen. \\
    \reqrationale Um die Weiterentwicklung der IDE zu vereinfachen sollen entsprechende Schnittstellen zur Verfügung stehen. Dabei sollte mindestens das Hinzufügen neuer CrossLab-Services und Benutzerinterfaces möglich sein. \\
\end{requirement}

\begin{requirement}{Kostenlos nutzbar}
    \reqdescription Die zu entwickelnde IDE soll kostenlos nutzbar sein. Daher sollten auch die Kosten für die Implementierung und den Betrieb möglichst gering sein. \\
    \reqrationale Um die IDE in einer Vielzahl von verschiedenen Szenarien einsetzen zu können ist es vom Vorteil keine assoziierten Kosten für die Nutzung dieser zu haben. Somit kann sie z.B. auch im GOLDi-Remotelab und anderen CrossLab-kompatiblen online Laboren eingesetzt werden. \\
\end{requirement}

\begin{requirement}{Komplett im Browser nutzbar}
    \reqdescription Die zu entwickelnde IDE soll komplett im Browser nutzbar sein. \\
    \reqrationale Durch die Nutzbarkeit der kompletten IDE direkt im Browser des Nutzers wird die Verwendung dieser vereinfacht, da Nutzer keine weitere Software installieren müssen. \\
\end{requirement}

\begin{requirement}{Nur CrossLab-Nutzerkonto nötig}
    \reqdescription Die zu entwickelnde IDE soll nur ein CrossLab-Nutzerkonto zur Verwendung benötigen. \\
    \reqrationale Durch die Notwendigkeit eines zweiten Nutzerkontos könnte die IDE für gewisse Nutzergruppen uninteressant werden. Somit soll nur ein CrossLab-Nutzerkonto für die Nutzung der IDE vorausgesetzt werden. \\
\end{requirement}

\begin{requirement}{Standalone nutzbar}
    \reqdescription Die zu entwickelnde IDE soll standalone nutzbar sein. Das bedeutet, dass sie als einziges Laborgerät in einem Experiment verwendet werden kann. Dabei kann es zu Einschränkungen der angebotenen Funktionen kommen. Die Editierung von Quellcode soll in allen Fällen gewährleistet werden. \\
    \reqrationale Da Experimente in der CrossLab-Architektur meist aus mehreren verbundenen Laborgeräten bestehen kann es dazu kommen, dass manche dieser ggf. nicht immer verfügbar sind, da sie aktuell von anderen Nutzern verwendet werden. Wenn die IDE standalone nutzbar ist können Nutzer dennoch an ihren Programmen weiterarbeiten. \\
\end{requirement}

\begin{requirement}{Kollaboration}
    \reqdescription Die zu entwickelnde IDE soll Echtzeit-Kollaboration durch die Synchronisation von geteilten Daten und den Austausch von Zustandsinformationen ermöglichen. \\
    \reqrationale Durch die Ermöglichung der Synchronisation von Daten und dem Austausch von Zustandsinformationen zwischen Nutzern innerhalb eines Experiments kann die Zusammenarbeit dieser gefördert werden. \\
\end{requirement}

\begin{requirement}{Kollaboration: CrossLab-Kompatibilität}
    \reqdescription Um die Echtzeit-Kollaboration von Laborgeräten innerhalb eines Experiments zu ermöglichen, sollen entsprechende CrossLab-Services entwickelt werden. Diese sollen die Synchronisation von geteilten Daten sowie den Austausch von Zustandsinformationen erlauben. Die zu entwickelnde IDE soll diese CrossLab-Services für die Echtzeit-Kollaboration mit anderen Laborgeräten nutzen. \\
    \reqrationale Die Bereitstellung von CrossLab-Services für Echtzeit-Kollaboration vereinfacht die Entwicklung neuer kollaborativer Laborgeräte. \\
\end{requirement}

\begin{requirement}{Dateisystem}
    \reqdescription Die zu entwickelnde IDE soll ein integriertes Dateisystem mit einem entsprechenden Benutzerinterface besitzen, dass die Erstellung, Bearbeitung, Verschiebung, Löschung und persistente Speicherung von Dateien und Ordnern ermöglicht. \\
    \reqrationale Ein in der IDE integriertes Dateisystem vereinfacht die Nutzung der IDE, da kein externes Dateisystem benötigt wird. \\
\end{requirement}

\begin{requirement}{Dateisystem: CrossLab-Kompatibilität}
    \reqdescription Für die Bereitstellung und Nutzung von Dateisystemen sollen entsprechende CrossLab-Services konzipiert werden. Die zu entwickelnde IDE soll diese CrossLab-Services für die Anbindung weiterer Dateisysteme verwenden. \\
    \reqrationale Die Möglichkeit weitere Dateisysteme über CrossLab-Services hinzuzufügen kann z.B. den Zugriff auf lokale Dateien des Nutzers oder eine geräteunabhängige Speicherung von Dateien auf einem Server ermöglichen. \\
\end{requirement}

\begin{requirement}{Dateisystem: Kollaboration}
    \reqdescription Die zu entwickelnde IDE soll das Teilen von Ordnern und den darin enthaltenen Dateien zwischen Nutzern innerhalb eines Experiments ermöglichen. Änderungen innerhalb geteilter Ordner sollen zwischen allen teilnehmenden Nutzern synchronisiert werden. Geteilte Ordner können nur von ihrem Besitzer gelöscht, verschoben oder umbenannt werden. Das Teilen von Ordnern soll auch beendet werden können. \\
    \reqrationale Durch das Teilen von Ordnern und den enthaltenen Dateien können Nutzer gemeinsam an diesen arbeiten. Dadurch können z.B. Gruppenarbeiten im Rahmen eines Praktikumsversuch effizienter durchgeführt werden, während die Lernenden gleichzeitig ihre Teamfähigkeit verbessern können. \\
\end{requirement}

\begin{requirement}{Kompilierung}
    \reqdescription Die zu entwickelnde IDE soll die Kompilierung von Quellcode unterstützen. Dabei sollen entsprechende Bedienelemente für die Kompilierung bereitgestellt werden.  \\
    \reqrationale Die Kompilierung des Quellcodes ist in vielen Programmiersprachen ein wichtiger Schritt um das Programm auf dem Zielsystem ausführen zu können. Dementsprechend sollte die IDE bei Vorhandensein eines Compilers entsprechende Bedienelemente zur Kompilierung des Quellcodes bereitstellen. \\
\end{requirement}

\begin{requirement}{Kompilierung: CrossLab-Kompatibilität}
    \reqdescription Für die Bereitstellung und Nutzung von Compilern sollen entsprechende CrossLab-Services konzipiert werden. Die zu entwickelnde IDE soll für die Anbindung von Compilern diese CrossLab-Services verwenden. \\
    \reqrationale Die Möglichkeit weitere Compiler über CrossLab-Services an die IDE anschließen zu können erlaubt die Unterstützung vieler verschiedener Steuereinheiten innerhalb eines Experiments. \\
\end{requirement}

\begin{requirement}{Debuggen}
    \reqdescription Die zu entwickelnde IDE soll das Debuggen von Programmen der Nutzer ermöglichen. Dabei sollen entsprechende Bedienelemente für das Debuggen bereitgestellt werden. Beim Start des Debuggens soll die neueste Version des aktuellen Programms auf die entsprechende Steuereinheit geladen werden. \\
    \reqrationale Nutzer können durch das Debuggen ihrer Programme schneller Fehler in diesen finden und beheben. Weiterhin erlaubt das Debuggen eines Programms einen besseren Einblick in dessen Laufzeitverhalten. \\
\end{requirement}

\begin{requirement}{Debuggen: CrossLab-Kompatibilität}
    \reqdescription Für die Bereitstellung und Nutzung von Debuggern sollen entsprechende CrossLab-Services konzipiert werden. Die zu entwickelnde IDE soll für die Anbindung von Debuggern diese CrossLab-Services verwenden. Weiterhin sollen auch CrossLab-Services für die Kommunikation zwischen einem Debugger und einer zu debuggenden Steuereinheit konzipiert werden. \\
    \reqrationale Die Möglichkeit weitere Debugger über entsprechende CrossLab-Services mit der IDE und Steuereinheiten verbinden zu können erlaubt eine größere Anzahl an verschiedenen Experimenten. \\
\end{requirement}

\begin{requirement}{Debuggen: Kollaboration}
    \reqdescription Die zu entwickelnde IDE soll es Nutzern ermöglichen gleichzeitig an einer Debug-Sitzung teilzunehmen, falls dies mit dem verwendeten Debugger möglich ist. Dabei müssen beide Nutzer Zugriff auf die gleichen Dateien besitzen. Weiterhin sollen die Breakpoints aller an der Debug-Sitzung teilnehmenden Nutzer synchronisiert werden. Pro Steuereinheit soll nur eine Debug-Sitzung gestartet werden können. Nur der Ersteller der Debug-Sitzung kann diese beenden. \\
    \reqrationale Durch das kollaborative Debuggen können Nutzer gemeinsam einen Einblick in das Laufzeitverhalten des Programs erlangen. Dadurch kann auch die Suche nach Fehlern sowie deren Behebung effizienter erfolgen. \\
\end{requirement}

\begin{requirement}{Language Server}
    \reqdescription Die zu entwickelnde IDE soll die Anbindung von Language Servern unterstützen. \\
    \reqrationale Durch die Anbindung von Language Servern können Editorfunktionen wie Code-Vervollständigung, Code-Navigation und Refactoring ermöglicht werden. Diese können die Benutzererfahrung verbessern. \\
\end{requirement}

\begin{requirement}{Language Server: CrossLab-Kompatibilität}
    \reqdescription Für die Bereitstellung und Nutzung von Language Servern sollen entsprechende CrossLab-Services konzipiert werden. Die zu entwickelnde IDE soll für die Anbindung von Language Servern diese CrossLab-Services verwenden. \\
    \reqrationale Durch die Anbindbarkeit von Language Servern über CrossLab-Services können diese von anderen Laborgeräten bereitgestellt und in Experimenten von der IDE genutzt werden. \\
\end{requirement}

\begin{requirement}{Hochladen von Programmen auf Steuereinheiten}
    \reqdescription Die zu entwickelnde IDE soll das Hochladen von Programmen auf Steuereinheiten unterstützen und entsprechende Bedienelemente dafür besitzen. Die Kompilierung und das Hochladen von Programmen kann in einen Schritt zusammengefasst werden um den Ablauf effizienter zu gestalten. \\
    \reqrationale Nutzer können innerhalb eines Experiments Programme für Steuereinheiten schreiben. Zur Ausführung müssen die Programme auf die entsprechende Steuereinheit hochgeladen werden.  \\
\end{requirement}

\begin{requirement}{Testen}
    \reqdescription Die zu entwickelnde IDE soll es ermöglichen Testfälle für ein Experiment zu konfigurieren und die Ausführung dieser während des Experiments unterstützen. Die Testfälle sollen über ein entsprechendes Benutzerinterface eingesehen und ausgeführt werden können. Testfälle sollen die Interaktionen zwischen verschiedenen Laborgeräten innerhalb eines Experiments überprüfen können. \\
    \reqrationale Die Möglichkeit Testfälle für ein Experiment zu konfigurieren erlaubt es z.B. Lehrenden die Ziele ihrer Lehrveranstaltung im Vorhinein festzulegen. Während des Experiments können die Lernenden dann ihre erstellte Lösung überprüfen. \\
\end{requirement}

\begin{requirement}{Testen: CrossLab-Kompatibilität}
    \reqdescription Für die Konfiguration und Ausführung von Testfällen sollen entsprechende CrossLab-Services konzipiert werden. Dabei müssen Laborgeräte die Möglichkeit besitzen, Funktionen bereitzustellen, die in den Testfällen verwendet werden können. Die zu entwickelnde IDE soll für die Konfiguration und Ausführung von Testfällen die konzipierten CrossLab-Services verwenden. \\
    \reqrationale Um die Konfiguration und Ausführung von Testfällen zu ermöglichen benötigt die zu entwickelnde IDE ggf. Zugriff auf andere Laborgeräte. Dies kann seitens der Laborgeräte durch die Bereitstellung von Funktionen zur Nutzung in Testfällen ermöglicht werden. Durch die Konzeption entsprechender CrossLab-Service kann die Bereitstellung der Funktionen sowie die Konfiguration und Ausführung der Testfälle innerhalb eines Experiments unterstützt werden. \\
\end{requirement}

\begin{table}[t]
    \centering
    \begin{tabular}{l l}
        \toprule
        \Requirements(1)  \\
        \Requirements(2)  \\
        \Requirements(3)  \\
        \Requirements(4)  \\
        \Requirements(5)  \\
        \Requirements(6)  \\
        \Requirements(7)  \\
        \Requirements(8)  \\
        \Requirements(9)  \\
        \Requirements(10) \\
        \Requirements(11) \\
        \Requirements(12) \\
        \Requirements(13) \\
        \Requirements(14) \\
        \Requirements(15) \\
        \Requirements(16) \\
        \Requirements(17) \\
        \Requirements(18) \\
        \Requirements(19) \\
        \Requirements(20) \\
        \Requirements(21) \\
        \bottomrule
    \end{tabular}
    \caption{Übersicht der Anforderungen}
    \label{table:anforderungen}
\end{table}

\chapter{Konzeption} \label{konzeption}

Im Fokus dieser Arbeit liegt die Programmierung von Mikrocontrollern im Rahmen des GOLDi Remotelab. Bei den verwendeten Microcontrollern handelt es sich um ATmega2560. Damit diese mit der CrossLab Infrastruktur kommunizieren können sind sie über einen FPGA mit einem Raspberry Pi Compute Module 4 verbunden. Dabei übernimmt der FPGA die Kommunikation zwischen dem CM und dem Microcontroller, während das CM die Kommunikationsschnittstelle zur CrossLab Infrastruktur übernimmt. Als Beispiel für ein steuerbares elektromechanisches Modell wird das 3-Achs-Portal verwendet. Neben den realen Systemen sollen auch die virtuellen Versionen in Betracht gezogen werden. Daraus folgen vier verschiedene Experiment-Konfigurationen, die für die Konzeption betrachtet werden:

\begin{enumerate}
    \item Realer Microcontroller und reales 3-Achs-Portal
    \item Realer Microcontroller und virtuelles 3-Achs-Portal
    \item Virtueller Microcontroller und reales 3-Achs-Portal
    \item Virtueller Microcontroller und virtuelles 3-Achs-Portal
\end{enumerate}

Weiterhin ist zu beachten, dass auch mehrere Steuereinheiten und Modelle in einem Experiment enthalten sein können.

In \autoref{konzeption:crosslab-kompatibilität} wird zunächst dargelegt wie die CrossLab-Kompatibilität erreicht werden kann. Danach wird in \autoref{konzeption:kollaboration} der Ansatz für die Bereitstellung von Kollaborationsmöglichkeiten beschrieben. In \autoref{konzeption:datenspeicherung} wird das Konzept für die Datenspeicherung erläutert. Darauf folgend wird in \autoref{konzeption:kompilierung} die Funktionsweise der Kompilierung dargelegt. Danach wird in \autoref{konzeption:debugging} das Konzept zur Ermöglichung von Debugging beschrieben. In \autoref{konzeption:testen} wird die Vorgehensweise für die Ausführung von Tests erläutert. Weiterhin wird in \autoref{konzeption:language_server} die Einbindung von Language Servern dargelegt. Schließlich wird in \autoref{konzeption:simulation} ein Konzept für die Simulation von Steuereinheiten beschrieben.

\section{CrossLab-Kompatibilität}\label{section:konzeption:crosslab-kompatibilität}

Die Sicherstellung der CrossLab-Kompatibilität für die verschiedenen Features der zu entwickelnden IDE ist in \autoref{requirement:CrossLab-Kompatibilität} festgelegt, mit spezifischeren Forderungen für das Dateisystem in \autoref{requirement:Dateisystem: CrossLab-Kompatibilität}, die Kompilierung in \autoref{requirement:Kompilierung: CrossLab-Kompatibilität}, das Debuggen in \autoref{requirement:Debuggen: CrossLab-Kompatibilität}, das Testen in \autoref{requirement:Testen: CrossLab-Kompatibilität} und für Language Server in \autoref{requirement:Language Server: CrossLab-Kompatibilität}. Die Funktionsweise von Experimenten innerhalb der CrossLab-Architektur wird in \autoref{section:grundlagen:crosslab} dargestellt, daher folgt nur eine kurze Wiederholung der wichtigsten Begriffe. Die CrossLab-Architektur ermöglicht die Definition von sogenannten \emph{Services}. Diese Services können als \emph{Consumer}, \emph{Producer} oder \emph{Prosumer} implementiert werden. Innerhalb eines Experiments können dann Consumer und Producer miteinander verbunden werden. Prosumer implementieren sowohl einen Consumer als auch einen Producer und können dementsprechend mit allen Varianten eine Verbindung aufbauen.

\autoref{requirement:Erweiterbarkeit} verlangt die Erweiterbarkeit der IDE um zusätzliche CrossLab-Services. Um dies zu erreichen gibt es verschiedene Möglichkeiten. So könnte eine zentrale Komponente genutzt werden, um alle vorhandenen CrossLab-Services zu verwalten. Diese zentrale Komponente könnte entweder selbst in der Lage sein Services, die von anderen Komponenten bereitgestellt werden, zum Laborgerät hinzuzufügen oder sie könnte eine entsprechende Schnittstelle bereitstellen, die es anderen Komponenten ermöglicht das Laborgerät mit ihren angebotenen Services zu erweitern.

Weiterhin besteht die Frage welche Art eines Laborgeräts für die Einbindung der IDE in die CrossLab-Architektur am besten geeignet ist. Dabei ist zu beachten, dass die IDE von mehreren Nutzern gleichzeitig und auch standalone in Experimenten verwendet werden soll (sh. \autoref{requirement:Kollaboration} und \autoref{requirement:Standalone nutzbar}). Daher kommt nur die Einbindung als cloud- oder edge-instanziierbares Gerät in Frage. Die Instanzen von cloud-instanziierbare Laborgeräten werden auf Servern ausgeführt und benötigen dementsprechende Ressourcen. Aufgrund dieser Tatsache kann es ggf. dazu kommen, dass Nutzer warten müssen bis die entsprechenden Serverkapazitäten vorhanden sind. Dies könnte die Benutzererfahrung verschlechtern. Eine Einbindung der IDE als edge-instanziierbares Laborgerät kann dieses Problem umgehen, da die Instanzen auf der Seite des Nutzers ausgeführt werden. Allerdings muss dabei beachtet werden, dass für eine Implementierung der IDE als edge-instanziierbares Gerät die grundlegenden Funktionen dieser komplett im Browser des Nutzers ausgeführt werden können müssen. Zu den grundlegenden Funktionen gehören dabei ein Dateisystem für die Bearbeitung und persistente Speicherung von Dateien und Ordnern sowie der Code Editor zum Editieren von Dateien.

\section{Kollaboration}\label{section:konzeption:kollaboration}

Es gibt viele verschiedene Methoden zur Synchronisierung von Daten zwischen mehreren Teilnehmern. Beispiele derartiger Methoden sind \emph{\ac{OT}} \cite{sun_operational_1998}, \emph{Differential Synchronization} \cite{fraser_differential_2009} und \emph{\ac{CRDTs}} \cite{shapiro_conflict-free_2011}. Aufgrund der Tatsache, dass jede dieser Methoden ihre Vor- und Nachteile besitzt, sollte die entwickelte Lösung unabhängig von dem zugrunde liegenden Synchronisationsalgorithmus sein. Der Kollaborationsdienst ist als Consumer, Producer und Prosumer nutzbar. Bei Experimenten mit einem zentralen Synchronisationspunkt (z.B. bei der Verwendung von \ac{OT}) bietet dieser einen Producer an während die restlichen Geräte, die an der Kollaboration teilnehmen, einen Consumer nutzen. Dahingegen nutzen bei Experimenten ohne einen zentralen Synchronisationspunkt (z.B. bei der Verwendung von \ac{CRDTs}) alle Geräte, die an der Kollaboration teilnehmen, einen Prosumer. In der Experimentbeschreibung sollten bei den Konfigurationen der Verbindungen von Kollaborationsdiensten stets die Synchronisationsmethode sowie die sogenannten \emph{Räume} angegeben werden, die in der Verbindung genutzt werden sollen. Räume besitzen einen eindeutigen Namen und ein JSON-Objekt, das zwischen allen Teilnehmern innerhalb des Raums synchronisiert wird. Das synchronisierte JSON-Objekt kann z.B. Ordner oder Dateien abbilden, die von den Teilnehmern geteilt werden. Jeder Raum besitzt einen sogenannten \emph{Provider}. Dieser nutzt die in der Konfiguration der Verbindung angegebene Synchronisationsmethode um den Inhalt des Raums zu synchronisieren. Weiterhin bietet der Provider eine Schnittstelle um Statusinformationen auszutauschen. Diese Informationen sind teilnehmerspezifisch, d.h. sie können nur von dem jeweiligen Teilnehmer aktualisiert werden. Ein Beispiel für derartige Statusinformationen ist z.B. die aktuelle Position eines Teilnehmers innerhalb einer Datei.

\begin{figure}[htbp]
    \centering
    \begin{sequencediagram}
        \newthread{consumer}{Consumer}
        \newthreadShift{producer}{Producer}{4cm}

        \begin{call}{consumer}{erstelle Räume}{consumer}{}
        \end{call}

        \prelevel\prelevel

        \begin{call}{producer}{erstelle Räume}{producer}{}
        \end{call}

        \postlevel

        \begin{call}{consumer}{sende ID}{producer}{}
            \begin{call}{producer}{registriere Consumer}{producer}{}
            \end{call}
        \end{call}

        \postlevel

        \begin{call}{consumer}{starte Synchronisation}{producer}{}
        \end{call}
    \end{sequencediagram}
    \caption{Initialisierung Kollaboration}\label{abbildung:initialisierung-kollaboration}
\end{figure}

Die Kommunikation zwischen den Kollaborationsteilnehmern erfolgt über ein entsprechendes Nachrichtenprotokoll. In \autoref{abbildung:initialisierung-kollaboration} ist der Verbindungsaufbau zwischen einem Consumer und einem Producer dargestellt. Zunächst erstellen beide die in der Verbindungskonfiguration festgelegten Räume. Dabei verknüpft der Consumer den Raum direkt mit der Verbindung. Der Producer hingegen wartet auf die Initialisierungsnachricht des Consumer, welche dessen ID beinhaltet. Die ID kann dann genutzt werden um den Consumer dem entsprechenden Räumen zuzuweisen. Sobald der Producer das erfolgreiche Ende der Initialisierung an den Consumer meldet beginnt dieser mit der Synchronisation. Da die verschiedenen Synchronisationsmethoden ggf. unterschiedliche Nachrichtenformate besitzen wird eine allgemeine Nachricht definiert, die dann die spezifischen Informationen für die zugrundeliegende Methode beinhalten. Daraus folgt auch, dass es nicht möglich ist einen Consumer mit einem Producer zu verbinden, der eine andere Synchronisationsmethode verwendet. Weiterhin ist darauf zu achten, dass ggf. mehrere Provider für eine Synchronisationsmethode benötigt werden. Dies ist z.B. der Fall bei Methoden mit einem zentralen Synchronisationspunkt.

Während die Behandlung von Aktualisierungen der Räume durch das Protokoll der zugrundeliegenden Synchronisationsmethode erfolgt, wird für die Behandlung von Statusaktualisierungen der Teilnehmer ein allgemeines Protokoll eingeführt. Dabei ist der Status eines Teilnehmers immer als ein JSON-Objekt darstellbar, wobei ein Wert von \texttt{null} angibt, dass der Teilnehmer nicht mehr erreichbar ist. Zu Beginn der Synchronisation schicken Consumer ihren aktuellen Status an den Producer. Dieser speichert den aktuellen Status und sendet ihn an die restlichen Consumer. Wenn sich der Status eines Consumer kann er entweder den kompletten Status an den Producer senden oder nur die vorgenommenen Änderungen. Der Producer aktualisiert seine gespeicherten Statusinformationen für den Consumer und leitet die Änderungen an die restlichen Consumer weiter. Diese aktualisieren ebenfalls ihre lokalen Statusinformationen und können dann auf die vorgenommenen Änderungen reagieren. Sollte ein Consumer nicht innerhalb eines vordefinierten Zeitraums seinen Status aktualisieren wird dieser auf \texttt{null} gesetzt und die Änderung an die restlichen Consumer weitergeleitet.
\section{Datenspeicherung} \label{konzeption:datenspeicherung}

Das bisherige WIDE System nutzt ein projektbasiertes Dateisystem. In diesem muss der Name eines Projektes einzigartig sein. Weiterhin können nicht mehrere Projekte gleichzeitig geöffnet werden. Zudem werden Metadaten zu einem Projekt gespeichert. Diese umfassen das elektromechanische Modell, die Steuereinheit sowie die Programmiersprache, die bei der Erstellung des Projekts genutzt bzw. ausgewählt wurden. Dadurch ist es möglich dem Nutzer nur die Projekte anzuzeigen, die in dem aktuellen Experiment von Interesse sein könnten. In der neuen CrossLab Architektur ist es nun allerdings möglich mehrere Steuereinheiten und elektromechanische Modelle sowie weitere Laborgeräte zu einem Experiment zusammenzustellen. Deshalb sollte es dem Nutzer ermöglicht werden mehrere Projekte gleichzeitig öffnen und bearbeiten zu können. Ein weiteres Feature von WIDE ist die Bereitstellung von Beispielprojekten. Diese können von Nutzern verwendet werden um einen Einblick in die Programmierung einer gegebenen Steuereinheit zu bekommen. Um dieses Feature weiterhin unterstützen zu können sollte eine Konfigurationsmöglichkeit gegeben werden, welche die Bereitstellung derartiger Beispiele ermöglicht.
\input{content/05_konzeption/054_kompilierung.tex}
\section{Debugging} \label{konzeption:debugging}

% Notiz: Debugging Server benötigt Zugriff auf zur Kompilierung verwendete Dateien (sollte wahrscheinlich zusammen mit Compiler auf einem System laufen)

% \begin{itemize}
%     \item Debugging von realem Microcontroller über RPi (avr-gdb)
%     \item Debugging von virtuellem Microcontroller über Cloud-instanziierbares Gerät oder über avr-gcc im Browser
% \end{itemize}

% Das Debuggen von Programmen auf den vorhandenen Microcontrollern gestaltet sich schwierig. Eine Möglichkeit ist die Nutzung der Bibliothek avr\_debug. Diese wird zusammen mit dem Programm kompiliert und auf den Microcontroller hochgeladen. Dort erstellt sie ein Interface für den Debugger gdb. Dieses Interface nutzt die Serielle Schnittstelle des Microcontrollers zur Kommunikation mit gdb. Das CM agiert in diesem Szenario als Schnittstelle zwischen gdb und unserer IDE. Ein Nachteil dieses Vorgehens ist der hohe Speicherverbrauch der Bibliothek, welcher die Anzahl möglicher Programme einschränkt. Allerdings ist ein Vorteil dieses Ansatzes, dass keine zusätzlichen Kosten durch die Anschaffung externe Debugger entstehen.

% Ein weiteres Problem, was beim Debuggen eines laufenden Experimentes beachtet werden muss, ist die fortlaufende Ansteuerung von weiteren Geräten. Nehmen wir als Beispiel ein einfaches Experiment bestehend aus einem Microcontroller und einem 3-Achs-Portal. Wenn wir das Program des Microcontrollers unterbrechen, während dieser den Portalkran aktiv nach rechts bewegt, so wird diese Bewegung nicht unterbrochen. Um sicherzustellen, dass die Signale von Aktoren während eines Breakpoints nicht an andere Geräte weitergeleitet werden müssen die anderen Geräte entsprechend benachrichtigt werden.
\section{Testen} \label{konzeption-testen}
\input{content/05_konzeption/057_language_server.tex}
\section{Simulation}\label{konzeption:simulation}
\chapter{Prototypische Implementierung}\label{section:prototypische-implementierung}

% TODO: schauen, was in Kapitel 6 verwendet werden kann
% Im Fokus dieser Arbeit liegt die Programmierung von Mikrocontrollern im Rahmen des GOLDi Remotelab. Bei den verwendeten Microcontrollern handelt es sich um ATmega2560. Damit diese mit der CrossLab Infrastruktur kommunizieren können sind sie über einen FPGA mit einem Raspberry Pi Compute Module 4 verbunden. Dabei übernimmt der FPGA die Kommunikation zwischen dem CM und dem Microcontroller, während das CM die Kommunikationsschnittstelle zur CrossLab Infrastruktur übernimmt. Als Beispiel für ein steuerbares elektromechanisches Modell wird das 3-Achs-Portal verwendet. Neben den realen Systemen sollen auch die virtuellen Versionen in Betracht gezogen werden. Daraus folgen vier verschiedene Experiment-Konfigurationen, die für die Konzeption betrachtet werden:

% \begin{enumerate}
%     \item Realer Microcontroller und reales 3-Achs-Portal
%     \item Realer Microcontroller und virtuelles 3-Achs-Portal
%     \item Virtueller Microcontroller und reales 3-Achs-Portal
%     \item Virtueller Microcontroller und virtuelles 3-Achs-Portal
% \end{enumerate}

% Weiterhin ist zu beachten, dass auch mehrere Steuereinheiten und Modelle in einem Experiment enthalten sein können.

Für die Prototypische Implementierung der IDE wurde ein Fokus auf die Programmierung von AVR Microcontrollern gelegt. Hierbei wurde für Beispielzwecke der ATmega2560 genutzt. Dies ermöglicht die Betrachtung aller in den Anforderungen beschriebenen Funktionen. In \autoref{section:prototypische-implementierung:code-editor} wird zunächst die Auswahl des zugrunde liegenden Code Editors beschrieben. Danach wird in \autoref{section:prototypische-implementierung:crosslab-kompatibilität} die Herstellung der CrossLab-Kompatibilität erläutert. Darauf aufbauend wird in \autoref{section:prototypische-implementierung:kollaboration} die Implementierung der Kollaborationsmechanismen dargelegt. In \autoref{section:prototypische-implementierung:dateisystem} wird die Implementierung des Dateisystems beschrieben. \autoref{section:prototypische-implementierung:kompilierung} befasst sich mit der Einbindung der eines Compilers für die AVR Microcontroller. Daraufhin wird in \autoref{section:prototypische-implementierung:debugging} die Einbindung eines entsprechenden Debuggers erläutert. Danach wird in \autoref{section:prototypische-implementierung:testen} die Implementierung der Konfiguration und Ausführung von Testfällen beschrieben. Schließlich wird in \autoref{section:prototypische-implementierung:language-server} die Anbindung eines passenden Language Servers dargelegt.

\section{Code Editor}\label{implementierung:code-editor}
\section{Experimentkonfiguration}\label{section:prototypische-implementierung:experimentkonfiguration}

\begin{note}
    \textbf{Notizen:}
    \begin{itemize}
        \item Beschreibung der Experimentkonfiguration
        \item Beschreibung des virtuellen Modells
        \item Beschreibung der Microcontroller Simulation
    \end{itemize}
\end{note}
\section{CrossLab Kompatibilität}\label{section:prototypische-implementierung:crosslab-kompatibilität}

\begin{note}
    \textbf{Notizen:}
    \begin{itemize}
        \item Beschreibung der Basis-Erweiterung
        \item Beschreibung wie CrossLab-Services hinzugefügt werden können
    \end{itemize}
\end{note}

Für die Implementierung der in \autoref{section:konzeption:crosslab-kompatibilität} konzipierten Lösung wurde eine entsprechende Erweiterung entwickelt. Diese ist für die Verwaltung der angebotenen CrossLab-Services sowie für die Anmeldung der IDE als Laborgerät innerhalb einer CrossLab-Instanz verantwortlich.

\paragraph{Verwaltung der CrossLab-Services}
Die Erweiterung hat Zugriff auf alle anderen Erweiterungen und kann deren angebotenen CrossLab-Services einsehen. Dadurch können diese zu dem Laborgerät der IDE hinzugefügt werden. Weiterhin ist die Erweiterung in der Lage andere Erweiterungen zu aktivieren. Somit ist es möglich die zu ladenden Erweiterungen über die Experimentkonfiguration zu definieren. Dafür kann eine entsprechende Liste von Kennzeichnern der entsprechenden Erweiterungen bei der Konfiguration der IDE innerhalb eines Experiments angegeben werden. Diese werden dann aktiviert und deren CrossLab-Services hinzugefügt.

\paragraph{Anmeldung als Laborgerät}
Für die Anmeldung der IDE innerhalb einer CrossLab-Instanz wird die URL des erstellten Laborgeräts sowie ein entsprechender Token benötigt. Diese werden im Falle von edge-instanziierbaren Laborgeräten über die Query-Parameter der aufzurufenden URL bereitgestellt. Die IDE muss diese entsprechend an die Erweiterung weiterleiten. Da Erweiterungen über ihre URLs hinterlegt werden und auch Zugriff auf diese besitzen können die Query-Parameter zu diesen hinzugefügt werden. Die Erweiterung kann dann die URL des Laborgeräts und den Token auslesen und damit die Anmeldung durchführen.

% Zur Herstellung der CrossLab-Kompatibilität wurde eine entsprechende Erweiterung implementiert. Diese ist für die Verwaltung eines sogenannten \texttt{DeviceHandler} zuständig. Dieser wird über den SOA-Client der CrossLab-Implementierung bereitgestellt. Der \texttt{DeviceHandler} erlaubt das Hinzufügen von Services zu einem Laborgerät. Die Erweiterung ist in der Lage alle anderen Erweiterungen sowie deren exportierten Objekte einzusehen. Sollte eine Erweiterung entsprechende CrossLab-Services anbieten können diese über den \texttt{DeviceHandler} hinzugefügt werden. Dabei besteht auch die Möglichkeit die betrachteten Erweiterungen für ein spezifischen Experiment einzuschränken indem bei der Konfiguration des Laborgeräts der IDE die Eigenschaft \texttt{extensions} auf ein Array gesetzt wird, dass die Kennzeichner der zu betrachtenden Erweiterungen beinhaltet. Weiterhin wird für die Anmeldung der IDE als Laborgerät bei einer CrossLab-Instanz die URL des Laborgeräts sowie ein dazugehöriger Token benötigt. Da es sich bei der IDE um ein edge-instanziierbares Laborgerät handelt werden diese Informationen über Query-Parameter innerhalb der vom Nutzer aufzurufenden URL hinterlegt. Die benötigten Parameter werden dann ausgelesen und wieder als Query-Parameter an die URL der Erweiterung angehangen. Dadurch können die Parameter wieder innerhalb der Erweiterung ausgelesen und dort für die Anmeldung des Laborgeräts bei der CrossLab-Instanz verwendet werden.
\section{Kollaboration}\label{section:prototypische-implementierung:kollaboration}

% \begin{note}
%     \textbf{Notizen:}
%     \begin{itemize}
%         \item Kurze Begründung warum Yjs für die Implementierung ausgewählt wurde
%         \item Beschreibung der Implementierung mit Yjs
%               \begin{itemize}
%                   \item CollaborationProvider
%                   \item CollaborationTypes
%               \end{itemize}
%     \end{itemize}
% \end{note}

Um das in \autoref{section:konzeption:kollaboration} beschriebene Konzept zur Bereitstellung von Echtzeit-Kollaboration innerhalb von Experimenten umzusetzen, wurden der Collaboration Service und eine Erweiterung für die IDE implementiert. Diese Erweiterung wird im Folgenden als \textit{Collaboration Erweiterung} bezeichnet.

Die Implementierung des Collaboration Service ist nach dem in \autoref{section:konzeption:kollaboration} vorgestellten Konzept erfolgt. Als Synchronisationsmethode wurde die Bibliothek Yjs \cite{noauthor_yjs_nodate} verwendet, welche auf dem Konzept von \acp{CRDT} basiert\todo{Erklärungssatz}. Alle Teilnehmer einer Kollaborationssitzung sind in Yjs gleichberechtig. Das bedeutet, dass alle Nutzer sowohl als Producer als auch als Consumer auftreten. Dementsprechend wurde zunächst nur ein Collaboration Service Prosumer implementiert, der beide Rollen abdeckt. Für die Implementierung der verschiedenen kollaborativen Datentypen wurden die von Yjs angebotenen Datentypen \texttt{Map}, \texttt{Array} und \texttt{Text} verwendet. Dabei können diese direkt auf Objekte, Arrays und Strings abgebildet werden. Für die Implementierung der restlichen kollaborativen Datentypen wurde \texttt{Text} verwendet, wobei der entsprechende Datentyp über ein zusätzliches Attribut festgelegt wird, um die Zuordnung zu ermöglichen. Die Möglichkeit Attribute an den Datentyp \texttt{Text} anzufügen, welche entsprechend synchronisiert werden, wird bereits von Yjs unterstützt.

Die Collaboration Erweiterung bietet einen Collaboration Service Prosumer an. Dieser kann von anderen Erweiterungen verwendet werden, um entsprechende Räumen beizutreten. Dadurch kann die Implementierung von spezifischen kollaborativen Funktionen durch die entsprechenden Erweiterungen vorgenommen werden. Aktuell wird nur Yjs als Synchronisationsmethode unterstützt.

In der betrachteten Experimentkonfiguration wird zur Veranschaulichung der Kollaboration eine weitere IDE hinzugefügt. Beide IDEs erhalten einen Collaboration Service Prosumer. Es wird eine Verbindung zwischen den beiden IDEs über den Collaboration Service hinzugefügt.
\section{Dateisystem}\label{section:prototypische-implementierung:dateisystem}

\begin{figure}[tbp]
    \centering
    \begin{tikzpicture}
        \begin{class}[text width=6cm]{FilesystemServiceProducer}{0,0}
            \operation{+ onCreateDirectory()}
            \operation{+ onDelete()}
            \operation{+ onMove()}
            \operation{+ onCopy()}
            \operation{+ onExists()}
            \operation{+ onStat()}
            \operation{+ onReadDirectory()}
            \operation{+ onReadFile()}
            \operation{+ onWriteFile()}
            \operation{+ onRegisterWatcher()}
            \operation{+ onUnregisterWatcher()}
        \end{class}
        \begin{class}[text width=6cm]{FilesystemServiceConsumer}{7,0}
            \operation{+ createDirectory()}
            \operation{+ delete()}
            \operation{+ move()}
            \operation{+ copy()}
            \operation{+ exists()}
            \operation{+ stat()}
            \operation{+ readDirectory()}
            \operation{+ readFile()}
            \operation{+ writeFile()}
            \operation{+ registerWatcher()}
            \operation{+ unregisterWatcher()}
            \operation{+ onWatcherEvent()}
        \end{class}
    \end{tikzpicture}
    \caption{Klassendiagramm Filesystem Service}
    \label{figure:klassendiagramm-dateisystem-service}
\end{figure}

\begin{figure}[tbp]
    \centering
    \includegraphics[trim={0 3px 0 0},clip,width=\textwidth]{images/projects-shared.png}
    \caption{Benutzerinterface der Projektverwaltung}
    \label{figure:benutzerinterface:dateisystem}
\end{figure}

Um die in \autoref{section:konzeption:dateisystem} beschriebenen Konzepte für die Bereitstellung und Nutzung von Dateisystemen umzusetzen, wurden der Filesystem Service sowie eine entsprechende Erweiterung für die IDE entwickelt. Diese Erweiterung wird im Folgenden als \textit{Filesystem Erweiterung} bezeichnet.

In \autoref{figure:klassendiagramm-dateisystem-service} ist ein Klassendiagramm für den Filesystem Service dargestellt. Der Filesystem Service Consumer besitzt die in \autoref{section:konzeption:dateisystem} genannten Funktionen zur Interaktion mit dem vom Filesystem Service Producer angebotenen Dateisystem. Der Filesystem Service Producer ermöglicht die Registrierung von Event-Handlern um auf die verschiedenen Anfragen reagieren zu können. Dadurch kann die Implementierung an das angebotene Dateisystem angepasst werden.

Die Filesystem Erweiterung ist für die Bereitstellung des integrierten Dateisystems und die Ermöglichung des kollaborativen Arbeitens an Programmen verwantwortlich. Für das integrierte Dateisystem wurde ein projektbasierter Ansatz gewählt. Dabei besitzen Nutzer mehrere verschiedene \textit{Projektordner}, deren direkte Unterordner als \textit{Projekte} bezeichnet werden. In der aktuellen Implementierung gibt es standardmäßig den Projektordner \texttt{/projects}. Die Projektordner sowie die enthaltenen Projekte werden dem Nutzer über ein entsprechendes Benutzerinterface angezeigt. Über dieses kann der Nutzer Projekte erstellen, öffnen, umbenennen und löschen. Für die Einbindung des Dateisystems wurde das Interface \texttt{FileSystemProvider} der VSCode Extension API implementiert. Für die Bereitstellung der Dateisystem-Funktionen können unterschiedliche \textit{Subprovider} verwendet werden. In der aktuellen Implementierung stehen dabei Subprovider für In-Memory, die Indexed Database API sowie den Filesystem Service zur Verfügung. Dabei werden standardmäßig der In-Memory Subprovider und der Indexed Database Subprovider für die persistente Speicherung der Projekte im Pfad \texttt{/projects} verwendet\todo{Umformulieren damit Aussage klarer wird}. Allerdings können der standardmäßig benutzte Subprovider sowie die Subprovider für bestimmte Pfade angepasst und neue Subprovider hinzugefügt werden. Weiterhin wurden auch die Schnittstellen \texttt{FileSearchProvider} und \texttt{TextSeachProvider} der VSCode Extension API implementiert, um es Nutzern zu ermöglichen, nach Dateien und Texten innerhalb des aktuellen Projekts zu suchen. Für das Benutzerinterface wurde eine \texttt{TreeView} verwendet, deren Daten über einen entsprechenden \texttt{TreeDataProvider} bereitgestellt werden. Das von der Filesystem Erweiterung bereitgestellte Benutzerinterface zur Verwaltung der Projekte ist in \autoref{figure:benutzerinterface:dateisystem} dargestellt. Dieses enthält auch bereits ein geteiltes Projekt, um die Darstellung unterschiedlicher Projektordner hervorzuheben.

Beim Öffnen eines neuen Ordners wird VSCode standardmäßig neugeladen. Dadurch werden auch alle Erweiterungen beendet und neugestartet, was dazu führt, dass das laufende Experiment beendet wird. Um dies zu vermeiden, muss der Wechsel von Projekten über einen anderen Mechanismus geschehen. Dazu wurde zunächst ein Pfad festgelegt, welcher standardmäßig von der IDE geöffnet wird. Im Falle der prototypischen Implementierung wurde der Pfad \texttt{/workspace} ausgewählt. Dieser Pfad nutzt standardmäßig einen In-Memory Subprovider. Sollte der Nutzer nun ein Projekt öffnen, wird von diesem Moment an der Pfad \texttt{/workspace} in allen URLs durch den Pfad des geöffneten Projektes ersetzt. Dadurch wird kein Neuladen der IDE ausgelöst und Nutzer können zwischen ihren Projekten wechseln. Das Umschreiben der Pfade führt allerdings zu Problemen beim Kopieren, Ausschneiden und Einfügen von Ordnern und Dateien. Daher müssen die entsprechenden Kommandos überschrieben werden, um die erwartete Funktionalität zu gewährleisten.

Um das Teilen von Projekten sowie das gleichzeitige Bearbeiten dieser zwischen Nutzern innerhalb eines Experiments zu ermöglichen, wurde eine entsprechende Komponente implementiert. Diese nutzt den \texttt{FileSystemProvider} sowie den von der Collaboration Erweiterung bereitgestellten Prosumer. Zu Beginn wird kein Projekt geteilt. Sobald ein Nutzer ein Projekt teilt, wird es zu dem geteilten Objekt hinzugefügt\todo{ggf. Referenz zu Collaboration Service}. Weiterhin werden auch Funktionen registriert, die auf Änderungen innerhalb des Projekts reagieren. Andere Nutzer die an der Kollaboration teilnehmen, können dann das geteilte Projekt über das bereitgestellte Benutzerinterface aufrufen. Alle Änderungen an Dateien und Ordnern werden zwischen den Nutzern synchronisiert. Weiterhin wird die aktuelle Position eines Nutzers innerhalb einer Datei über dessen Zustandsinformationen geteilt. Diese Position wird dann bei anderen Nutzern innerhalb derselben Datei markiert (sh. \autoref{figure:benutzerinterface:dateisystem}). Wenn der Besitzer des Projekts das Teilen beendet, wird das Projekt für alle anderen Nutzer geschlossen. Geteilte Projekte besitzen Pfade der Form \texttt{/shared/\{\{user\_id\}\}/\{\{project\_name\}\}}, wobei der Pfad \texttt{/shared} ein In-Memory Dateisystem verwendet. Die Projektordner \texttt{/shared/\{\{user\_id\}\}} werden automatisch erstellt, wenn ein Nutzer dem entsprechenden Raum beitritt.

Der betrachteten Experimentkonfiguration werden keine neuen Laborgeräte hinzugefügt. Die IDEs werden um einen Filesystem Service Consumer erweitert, der die Anbindung von weiteren Dateisystemen ermöglicht. Außerdem bieten die IDEs auch einen Filesystem Service Producer an, um die Nutzung des integrierten Dateisystems durch andere Laborgeräte zu ermöglichen (sh. \autoref{figure:experimentkonfiguration:dateisystem}). Zudem wird ein neuer Raum für die Verbindungen der Collaboration Service Prosumer hinzugefügt, um das Teilen von Projekten zu ermöglichen.
\section{Kompilierung}\label{section:prototypische-implementierung:kompilierung}

\begin{note}
    \textbf{Notizen:}
    \begin{itemize}
        \item Begründung warum Arduino CLI für die Implementierung genutzt wurde
        \item Beschreibung der Anbindung der Arduino CLI als Laborgerät
        \item (Angabe der Rückgabeformate?)
        \item Ggf. Beschreibung der Benutzerinterfaces + Screenshots
    \end{itemize}
\end{note}

Für die Kompilierung wurde in der prototypischen Implementierung das Arduino Command Line Interface \cite{noauthor_arduino-cli_nodate} verwendet. Dieses nutzt intern den Compiler \ac{GCC} \cite{noauthor_gcc_nodate}. Neben der Kompilierung des Quellcodes werden auch Arduino spezifische Vorverarbeitungsschritte durchgeführt. Dadurch können Nutzer auch ggf. ihnen bereits bekannte Arduino Funktionen wie z.B. \texttt{digitalWrite} und \texttt{digitalRead} nutzen. Im Allgemeinen sollte dadurch die Programmierung der Microcontroller für die Nutzer vereinfacht werden. Um die Arduino-cli zur Kompilierung innerhalb eines Experiments nutzen zu können muss diese als ein entsprechendes Laborgerät bereitgestellt werden. Dafür wurde ein cloud-instanziierbares Laborgerät entwickelt. Dieses bietet einen entsprechenden Kompilierungs Service Producer an, welcher während einem Experiment mit der IDE verbunden werden kann. Das instanziierte Laborgerät nimmt die entsprechenden Kompilieranfragen entgegen und bearbeitet diese. Sollte die Kompilierung erfolgreich sein wird eine entsprechende Antwort mit dem Ergebnis der Kompilierung an die IDE gesendet. Im Fehlerfall wird die Fehlermeldung in der Antwort mitgesendet.

Um die Kompilierung aus der IDE starten zu können wurde eine entsprechende Erweiterung entwickelt. Diese fügt zwei Bedienelemente hinzu. Eine Schaltfläche zur Kompilierung des aktuellen Projekts sowie eine Taste zum Kompilieren und Hochladen des aktuellen Projekts. Die erste Schaltfläche kann dazu genutzt werden um zu überprüfen ob das aktuelle Projekt kompiliert werden kann und um die entsprechenden Mitteilungen vom Compiler zu erhalten. Die zweite Schaltfläche führt auch eine Kompilierung des aktuellen Projekts durch und zeigt entsprechende Rückmeldungen an. Sollte die Kompilierung erfolgreich sein wird anschließend das Ergebnis dieser an die zu programmierende Steuereinheit gesendet. In einer kollaborativen Sitzung ist das Kompilieren von Projekten stets erlaubt. Allerdings wird das Hochladen von Projekten deaktiviert falls ein anderer Nutzer aktuell ein Projekt auf die Steuereinheit hochlädt oder falls die Steuereinheit in einer Debug-Sitzung verwendet wird.
\section{Debugging}\label{section:prototypische-implementierung:debugging}

% \begin{note}
%     \textbf{Notizen:}
%     \begin{itemize}
%         \item Begründung warum gdb für die Implementierung genutzt wurde
%         \item Beschreibung der Anbindung von gdb als Laborgerät
%         \item Erwähnung der verwendeten VSCode APIs
%               \begin{itemize}
%                   \item \dots
%               \end{itemize}
%         \item Beschreibung der Kollaboration
%               \begin{itemize}
%                   \item Behandlung von Breakpoints
%                   \item Behandlung von Start- und Stoppanfragen
%                   \item Behandlung von Neustarts
%                   \item Behandlung von Beenden der Sitzung
%                   \item Behandlung von Initialisierungsnachrichten
%               \end{itemize}
%         \item Eigenheiten der gdb DAP-Implementierung (keine new-events)
%         \item Beschreibung der Einbindung externer Dateien
%         \item Ggf. Beschreibung der Benutzerinterfaces + Screenshots
%     \end{itemize}
% \end{note}

Für die Bereitstellung der Debug-Funktionen wurde in der prototypischen Implementierung der Debugger \ac{GDB} \cite{noauthor_gdb_nodate} verwendet. Dieser erlaubt das Debuggen von Microcontrollern und besitzt zudem einen integrierten Debug Adapter. Um \ac{GDB} in Experimenten nutzen zu können wurde ein cloud-instanziierbares Laborgerät entwickelt. Dieses stellt einen Debugging Adapter Service Producer und einen Debugging Target Service Consumer für die Kommunikation mit der IDE sowie der zu debuggenden Steuereinheit bereit.

Wenn ein Nutzer eine Debug-Sitzung startet wird über den Debugging Adapter Service eine entsprechende Nachricht an das Laborgerät des Debuggers gesendet. Diese Nachricht beinhaltet das aktuelle Projekt des Nutzers. Dieses benötigt der Debugger während der Debug-Sitzung, weshalb es für die Dauer der Debug-Sitzung in einem entsprechenden Ordner auf dem Dateisystem hinterlegt wird. Weiterhin wird das Projekt an den Compiler gesendet, wobei für die Ermöglichung des Debuggens spezielle Einstellungen vorgenommen werden müssen. Das Ergebnis der Kompilierung wird dann über den Debugging Target Service an die Steuereinheit gesendet, wodurch der Steuereinheit gleichzeitig der Beginn einer Debug-Sitzung mitgeteilt wird. Weiterhin wird der Debugger selbst gestartet, wobei der tatsächliche Start der Debug-Sitzung erst durch die Nachrichten des \ac{DAP} geschieht. Nachdem alle Vorbereitungen getroffen wurden und eine Antwort von der Steuereinheit empfangen wurde, wird eine Antwort an die IDE gesendet. Die Antwort enthält den Kennzeichner der Debug-Sitzung sowie Einstellungen für diese. Im Falle der prototypischen Implementierung werden hierbei der Kennzeichner der Debug-Sitzung, das Debug-Ziel sowie der Pfad des kompilierten Programms als Einstellungen übergeben. Diese werden dann von der IDE beim Start des \ac{DAP} verwendet.

Damit das \ac{DAP} korrekt ausgeführt werden kann muss eine Umschreibung der Pfade vorgenommen werden, da die Dateien des Nutzers und des Debuggers in unterschiedlichen Pfaden liegen. Außerdem gibt es Dateien, die nur auf dem Dateisystem des Debuggers vorhanden sind, wie z.B. Bibliotheken. Alle Pfade, die von der IDE gesendet werden beginnen entweder mit \texttt{crosslabfs:/workspace} für Dateien innerhalb eines Projekts oder mit \texttt{crosslab-remote:} für Dateien innerhalb des Dateisystems des Debuggers. Bei Ersteren wird das genannte Präfix durch den lokalen Pfad des Projektes auf dem Dateisystem des Debuggers ersetzt, während bei Zweiteren das Präfix gelöscht wird. Bei Nachrichten vom Debugger an die IDE geschieht die Behandlung der jeweiligen Präfixe auf umgekehrte Art.

Um das kollaborative Debuggen innerhalb eines Experiments zu unterstützen müssen einige Nachrichten des \ac{DAP} speziell behandelt werden. Dazu gehören die Nachrichten für Breakpoints, Stacktraces, das Starten und Stoppen des Programms sowie das Beenden der Debug-Sitzung. Wenn Nutzer einer Debug-Sitzung beitreten schicken sie zunächst nur ihre lokalen Breakpoints über eine \texttt{SetBreakpoints}-Anfrage. Damit nicht die Breakpoints der anderen Nutzer gelöscht werden müssen diese entsprechend zu dieser Nachricht hinzugefügt werden. Dafür werden die Breakpoints aller Nutzer separat verwaltet. Außerdem müssen alle Events gespeichert werden, um diese beitretenden Nutzern schicken zu können, damit diese den aktuellen Zustand der Debug-Sitzung herstellen können. Allgemein werden Events, die vom Debug Adapter ausgegeben werden, an alle Nutzer einer Debug-Sitzung gesendet. Wenn ein Nutzer das Programm startet oder fortsetzt muss ein \texttt{Continued}-Event an die anderen Nutzer gesendet werden. Bei Breakpoints und Stacktraces muss darauf geachtet werden, dass die URLs für Dateien des Debuggers entsprechend angepasst werden, damit die IDE diese öffnen kann. Wenn der Ersteller der Debug-Sitzung diese beendet, so wird sie für alle Nutzer beendet. Sollte ein anderer Nutzer die Debug-Sitzung beenden, so wird sie nur für den Nutzer selbst beendet.

Für die Einbindung in die IDE wurde eine entsprechende Erweiterung entwickelt. Diese fügt einen Debugging Adapter Service Consumer hinzu und beinhaltet Implementierungen der Schnittstellen \texttt{DebugAdapter}, \texttt{DebugAdapterDescriptorFactory} und \texttt{DebugConfigurationProvider}. Mithilfe dieser Schnittstellen kann bereits ein Großteil der Funktionalität implementiert werden. Weiterhin werden auch Bedienelemente zum Starten bzw. Beitreten einer Debug-Sitzung bereitgestellt. Nutzer können eine Debug-Sitzung über eine entsprechende Schaltfläche starten, falls keine aktive Debug-Sitzung bestehen sollte. Ansonsten können Nutzer über eine weitere Schaltfläche einer bestehenden Debug-Sitzung beitreten, falls sie Zugriff auf das dazugehörige Projekt haben. Um das Debuggen von Dateien zu ermöglichen, die nur im Dateisystem des Debuggers vorhanden sind wurde ein \texttt{TextDocumentContentProvider} implementiert. Dieser ermöglicht Lesezugriff auf die Dateien innerhalb des Dateisystems des Debuggers sowie das Setzen von Breakpoints innerhalb dieser.

Das betrachtete Experiment wird um das neue Laborgerät für die Bereitstellung von GDB erweitert. Dieses wird über den Debugging Adapter Service mit den IDEs verbunden. Außerdem wird die Steuereinheit um einen Debugging Target Service Producer erweitert. Der Debugger wird über den Debugging Target Service mit der Steuereinheit verbunden.
\section{Testen}\label{section:prototypische-implementierung:testen}

\begin{note}
    \textbf{Notizen:}
    \begin{itemize}
        \item Beschreibung der bereitgestellten Funktionen der Simulation
        \item Erwähnung der verwendeten VSCode APIs
              \begin{itemize}
                  \item \dots
              \end{itemize}
        \item Ggf. Beschreibung der Benutzerinterfaces + Screenshots
    \end{itemize}
\end{note}
\section{Language Server}\label{section:prototypische-implementierung:language-server}

% \begin{note}
%     \textbf{Notizen:}
%     \begin{itemize}
%         \item Begründung warum Arduino Language Server genutzt wurde
%         \item Anbindung des Arduino Language Server als Laborgerät
%         \item Beschreibung der implementierten VSCode Erweiterung
%         \item Beschreibung der aufgetretenen Probleme und deren Lösung
%     \end{itemize}
% \end{note}

Für die prototypische Implementierung wurde ein cloud-instanziierbares Laborgerät für die Bereitstellung des \textit{Arduino Language Servers} \cite{noauthor_arduino-language-server_2025} implementiert. Dieses besitzt einen entsprechenden Language Server Service Producer. Wenn die IDE den Language Server startet schickt sie zunächst die entsprechende Initialisierungsnachricht mit dem aktuellen Projekt des Nutzers. Dieses wird auf der Seite des Language Server gespeichert. Danach wird der Language Server gestartet und eine Antwort an die IDE gesendet. Diese kann daraufhin mit der Ausführung des \ac{LSP} beginnen. Dabei muss auf der Seite des Language Servers eine Anpassung der URIs für eingehende und ausgehende \ac{LSP} Nachrichten erfolgen. Dies kann auf eine ähnliche Weise erfolgen wie es bereits in \autoref{section:prototypische-implementierung:debugging} für das \ac{DAP} beschrieben wurde.

Für die IDE wurde eine entsprechende Erweiterung entwickelt. Diese wird im Folgenden als \textit{Language Server Erweiterung} bezeichnet und stellt einen Language Server Service Consumer bereit. Wenn eine Verbindung für diesen besteht, wird der entsprechende Language Server gestartet und über den VSCode Language Client angebunden. Weiterhin wurde ein \texttt{TextDocumentContentProvider} implementiert, der in Verbindung mit dem Language Server Service Consumer für den Lesezugriff auf die lokalen Dateien des Language Servers ermöglicht.

Die betrachtete Experimentkonfiguration wird um das neue Laborgerät für die Bereitstellung des Arduino Language Servers erweitert. Dieses wird über den Language Server Service mit den IDEs verbunden.
\chapter{Diskussion}\label{section:diskussion}
\chapter{Zusammenfassung und Ausblick}\label{section:zusammenfassung-und-ausblick}

% \begin{note}
%     \textbf{Notizen:}
%     \begin{itemize}
%         \item Zusammenfassung der vorherigen Kapitel
%         \item Ausblick (Weiterentwicklung, Evaluation, ...)
%     \end{itemize}
% \end{note}

Im Verlauf dieser Arbeit wurde die Konzeption und Implementierung einer CrossLab-kompatiblen IDE beschrieben. Dafür wurde in \autoref{section:stand-der-technik} zunächst der aktuelle Stand der Technik ermittelt. Dafür wurde zunächst eine Literaturrecherche durchgeführt um die folgenden Forschungsfragen zu beantworten:

\begin{enumerate}
    \item Welche Implementierungen von online IDEs gibt es?
    \item Welchen Architekturmustern folgen online IDEs?
    \item Welche Vorteile haben online IDEs?
    \item Welche Nachteile haben online IDEs?
    \item Welche Anforderungen werden an online IDEs gestellt?
\end{enumerate}

Zusätzlich wurden auch weitere Entwicklungen abseits der betrachteten Literatur betrachtet. Auf Basis der gewonnenen Erkenntnisse wurden in \autoref{section:anforderungsanalyse} Anforderungen für die zu entwickelnde IDE erhoben. Darauf aufbauend wurden in \autoref{section:konzeption} Konzepte für die einzelnen Funktionen der IDE entwickelt. Dabei wurde zunächst ein Konzept für die Herstellung der CrossLab-Kompatibilität der IDE vorgestellt. In den nachfolgenden Kapiteln wurden CrossLab-Services für die verschiedenen Funktionen der IDE konzipiert. Diese umfassen die folgenden Services:

\begin{itemize}
    \item \textbf{Collaboration Service} \\ Dieser Service ermöglicht die Echtzeit-Kollaboration zwischen Nutzern innerhalb eines Experiments. Darunter z.B. das Teilen von Projekten und Debug-Sitzungen.
          \newpage
    \item \textbf{Filesystem Service} \\ Dieser Service ermöglicht die Bereitstellung und Nutzung von Dateisystemen.
    \item \textbf{Compilation Service} \\ Dieser Service ermöglicht die Bereitstellung und Nutzung von Compilern.
    \item \textbf{Programming Service} \\ Dieser Service ermöglicht die Programmierung von Steuereinheiten.
    \item \textbf{Debugging Adapter und Debugging Target Service} \\ Diese Services ermöglichen die Bereitstellung und Nutzung von Debuggern. Dabei ist der Debugging Adapter Service für die Kommunikation zwischen der IDE und einem Debug Adapter zuständig, während der Debugging Target Service die Kommunikation zwischen dem Debugger und der Steuereinheit ermöglicht.
    \item \textbf{Testing Service} \\ Dieser Service ermöglicht die Erstellung und Ausführung von Testfällen.
    \item \textbf{Language Server Service} \\ Dieser Service ermöglicht die Bereitstellung und Nutzung von Language Servern.
\end{itemize}

In \autoref{section:prototypische-implementierung} wurde die prototypische Implementierung der IDE vorgestellt. Diese wurde auf Basis des Code Editors \ac{VSCode} mithilfe der VSCode Extension API vorgenommen. Die IDE selbst ist als ein edge-instanziierbares Laborgerät implementiert. Es wurden die folgenden Erweiterungen für die IDE implementiert:

\begin{itemize}
    \item \textbf{Basis Erweiterung} \\ Diese Erweiterung ist verantwortlich für die Verwaltung der CrossLab-Services der IDE und für die Anmeldung der IDE als Laborgerät bei der entsprechenden CrossLab-Instanz.
    \item \textbf{Collaboration Erweiterung} \\ Diese Erweiterung fügt einen Collaboration Service Prosumer zur IDE hinzu. Dieser kann von anderen Erweiterungen verwendet werden, um kollaborative Funktionen anzubieten.
    \item \textbf{Filesystem Erweiterung} \\ Diese Erweiterung beinhaltet das integrierte Dateisystem der IDE. Zusätzlich fügt sie einen Filesystem Service Consumer sowie einen Filesystem Service Producer hinzu. Der Consumer kann für die Anbindung weiterer Dateisysteme verwendet werden, während der Producer anderen Laborgeräten den Zugriff auf das integrierte Dateisystem ermöglicht. Zusätzlich wird das Teilen von Projekten zwischen Nutzern innerhalb eines Experiments unterstützt.
    \item \textbf{Compilation Erweiterung} \\ Diese Erweiterung fügt einen Compilation Service Consumer sowie einen Programming Service Consumer zur IDE hinzu. Diese werden für die Anbindung von Compilern sowie für die Programmierung von Steuereinheiten verwendet.
    \item \textbf{Debugging Erweiterung} \\ Diese Erweiterung fügt einen Debugging Adapter Service Consumer zur IDE hinzu. Dieser wird für die Anbindung von Debuggern verwendet. Zusätzlich wird das Teilen von Debug-Sitzungen zwischen Nutzern innerhalb eines Experiments unterstützt.
    \item \textbf{Testing Erweiterung} \\ Diese Erweiterung fügt einen Testing Service Consumer zur IDE hinzu. Dieser wird für die Ausführung von Testfällen innerhalb eines Experiments verwendet.
    \item \textbf{Language Server Erweiterung} \\ Diese Erweiterung fügt einen Language Server Service Consumer zur IDE hinzu. Dieser wird für die Anbindung von Language Servern verwendet.
\end{itemize}

Zusätzlich zu den CrossLab-Services und Erweiterungen wurden auch neue Laborgeräte implementiert. Darunter cloud-instanziierbare Laborgeräte zur Bereitstellung des \ac{Arduino CLI}, des \ac{GDB}, des Arduino Language Servers und einer simulierten Microcontroller Steuereinheit. Weiterhin wurde ein edge-instanziierbares steuerbares Laborgerät implementiert.

Schließlich wurden in \autoref{section:diskussion} die Ergebnisse dieser Arbeit betrachtet und diskutiert. Dabei wurden verschiedene alternative Lösungsansätze, Erweiterungsmöglichkeiten und offene Aufgaben betrachtet. So könnten z.B. gewisse Nutzerdaten für Analysezwecke gesammelt werden. Dadurch könnten u.a. Lehrende erkennen für welche Aufgaben die Lernenden am meisten Zeit aufwenden und können diese falls benötigt entsprechend anpassen. Weiterhin sollte die Komplexität der konzipierten und implementierten Lösungen im Hinblick auf deren Implementierung und Anwendung verringert werden. So sollten entsprechende Bibliotheken und Werkzeuge entwickelt werden um die Implementierung neuer Laborgeräte und die Konfiguration von Experimenten zu vereinfachen. Als Beispiel könnte ein Experimentkonfigurator konzipiert werden. Dieser könnte Laborgeräte und ihre angebotenen Services visualisieren und deren Verbindung zu einem Experiment ermöglichen. Weiterhin sollten weitere Laborgeräte implementiert werden, um eine größere Anzahl verschiedener Experimente unterstützen zu können. Somit kann auch der Einstieg für z.B. Lehrende erleichtert werden. Außerdem sollte auch die Sicherheit der Lösungen genauer betrachtet werden. Dies lag nicht im Fokus dieser Arbeit, ist aber für eine finale Version von hoher Bedeutung. Schließlich steht auch die Evaluation der konzipierten und implementierten Lösungen noch aus. Dabei könnte z.B. der Ressourcenverbrauch analysiert und mit alternativen Lösungen verglichen werden. Zudem könnte ein Vergleich mit WIDE erfolgen, um zu erfahren, wie die neuen Funktionen der IDE von den Studierenden angenommen werden. Außerdem kann für weitere Entwicklungen WebAssembly in Betracht gezogen werden, um Funktionen der IDE wie z.B. die Kompilierung der Programme direkt im Browser des Nutzers durchführen zu können. Dadurch können ggf. die benötigten Serverressourcen weiter reduziert werden.

% opt: zusätzliche Anhänge
\begin{appendix}
    % \input{content/anhang1}
\end{appendix}

% --- Abbildungs-, Tabellen-, Algorithmen- und Literaturverzeichnis ------------
\backmatter
% Abbildungsverzeichnis
\listoffigures
% Tabellenverzeichnis
\listoftables
% Algorithmenverzeichnis
% \listofalgorithms
% Literaturverzeichnis
\cleardoublepage	% Literaturverzeichnis immer auf ungerader Seite
\phantomsection		% Anker für Sprungmarke im Inhaltsverzeichnis korrigieren
\addcontentsline{toc}{chapter}{\bibname}
\printbibliography
\end{document}
